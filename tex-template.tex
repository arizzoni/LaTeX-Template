\documentclass[12pt, titlepage]{article}
\newcommand{\theauthor}{Alessandro Rizzoni}
\newcommand{\thedoctitle}{\LaTeX\ Template}

\usepackage{hyperref}

% Code Execution and Typesetting
\usepackage[%
    pygopt={
        style=bw
        },
    upquote=true
    ]{pythontex}
\setpythontexfv{%
    breaklines,
    breakanywhere,
    frame=leftline,
    framerule=0.8pt,
    firstnumber=last,
    numbers=left,
    numbersep=0pt
    }
\renewcommand{\FancyVerbFormatLine}[1]{\small #1}
\renewcommand{\theFancyVerbLine}{%
    \tiny\rmfamily{\arabic{FancyVerbLine}\,}
    }

% Page Layout
\usepackage[%
    letterpaper, % 8.5" x 11" paper size
    margin=1in
    ]{geometry}

% Mathematics
\usepackage{mathtools} % Mathematics typesetting
\usepackage[% Mathematics with modern fonts
    warnings-off={% Turn off default always-on warnings
        mathtools-colon,
        mathtools-overbracket
        }
    ]{unicode-math} 
\usepackage{lualatex-math} % Math fixes for LuaLaTeX
\usepackage{siunitx} % Handy typesetting of units

% Font
\usepackage{fontspec} % Modern .ttc, .ttf, and .otf fonts
\setmathfont{STIXTwoMath-Regular.otf}
\setmainfont{STIXTwoText-Regular.otf}[%
    Ligatures=TeX,
    ItalicFont=STIXTwoText-Italic.otf,
    BoldFont=STIXTwoText-Bold.otf,
    BoldItalicFont=STIXTwoText-BoldItalic.otf,
    ]
\setsansfont{Iosevka Aile}[%
    Ligatures=TeX,
    Scale=MatchLowercase
    ]
\setmonofont{Iosevka Fixed}[%
    Ligatures=TeX,
    Scale=MatchLowercase
    ]

% Localization
\usepackage[% American English localization
    USenglish
    ]{babel} 
\usepackage[% Localized context-sensitive quotes
    autostyle=true
    ]{csquotes}

% Headers
\usepackage{fancyhdr}
\renewcommand{\sectionmark}[1]{\markright{#1}}
\setlength{\headheight}{24pt}
\fancypagestyle{fancy}{
    \fancyhf{}
    \renewcommand{\headrulewidth}{0.8pt}
    \renewcommand{\footrulewidth}{0pt}
    \fancyhead[L]{\sffamily\thedoctitle}
    \fancyhead[R]{\sffamily\small\rightmark}
    \fancyfoot[C]{\sffamily\small\thepage}
}
\pagestyle{fancy}

% Section Title Formatting
\usepackage{titlesec}
\titleformat{\section}{\Large\sffamily}{\thesection}{1em}{\Large\sffamily}
\titleformat{\subsection}{\large\sffamily}{\thesubsection}{1em}{\large\sffamily}
\titleformat{\subsubsection}{\normalsize\sffamily}{\thesubsubsection}{1em}{\normalsize\sffamily}
% \AddToHook{cmd/section/before}{\clearpage}

% Table of Contents Formatting
\usepackage{titletoc}
\dottedcontents{section}[1em]{}{1em}{0.5em}
\dottedcontents{subsection}[1em]{}{1em}{0.5em}
\dottedcontents{subsubsection}[1em]{}{1em}{0.5em}

% Tables and Lists
\usepackage{tabularray} % Tables and Arrays
\usepackage{enumitem} % Reformat Enumerated Lists

% Graphics
\usepackage{pgf} % Base graphics package
\usepackage{blox} % System block diagrams
\usepackage{tikz-timing} % Timing diagram graphics
\usepackage[siunitx]{circuitikz} % Electrical circuit graphics
\ctikzloadstyle{romano} % Better styling for CircuiTikz
\tikzset{romano circuit style} % Apply the style

%% Pythontex 
% Get lengths (in LaTeX points!)
\newlength\inlength \inlength=1in
\newlength\cmlength \cmlength=1cm
\newlength\mmlength \mmlength=1mm
\newlength\emlength \emlength=1em
\newlength\exlength \exlength=1ex
\newlength\bplength \bplength=1bp
\newlength\ddlength \ddlength=1dd
\newlength\pclength \pclength=1pc

% Get font parameters
\makeatletter

% Font Sizes
\tiny\xdef\thetinysize{\f@size}
\scriptsize\xdef\thescriptsize{\f@size}
\footnotesize\xdef\thefootnotesize{\f@size}
\small\xdef\thesmallsize{\f@size}
\normalsize\xdef\thenormalsize{\f@size}
\large\xdef\thelargesize{\f@size}
\Large\xdef\theLargesize{\f@size}
\LARGE\xdef\theLARGEsize{\f@size}
\huge\xdef\thehugesize{\f@size}
\Huge\xdef\theHugesize{\f@size}

% Fonts 
\sffamily\mdseries\upshape\xdef\thesffont{\expandafter\string\the\font}
\ttfamily\mdseries\upshape\xdef\thettfont{\expandafter\string\the\font}
\rmfamily\mdseries\upshape\xdef\thermfont{\expandafter\string\the\font}
\sffamily\bfseries\upshape\xdef\theboldsffont{\expandafter\string\the\font}
\ttfamily\bfseries\upshape\xdef\theboldttfont{\expandafter\string\the\font}
\rmfamily\bfseries\upshape\xdef\theboldrmfont{\expandafter\string\the\font}
\sffamily\mdseries\itshape\xdef\theitalicsffont{\expandafter\string\the\font}
\ttfamily\mdseries\itshape\xdef\theitalicttfont{\expandafter\string\the\font}
\rmfamily\mdseries\itshape\xdef\theitalicrmfont{\expandafter\string\the\font}
\sffamily\bfseries\itshape\xdef\thebolditalicsffont{\expandafter\string\the\font}
\ttfamily\bfseries\itshape\xdef\thebolditalicttfont{\expandafter\string\the\font}
\rmfamily\bfseries\itshape\xdef\thebolditalicrmfont{\expandafter\string\the\font}
\sffamily\mdseries\scshape\xdef\thesmallcapssffont{\expandafter\string\the\font}
\ttfamily\mdseries\scshape\xdef\thesmallcapsttfont{\expandafter\string\the\font}
\rmfamily\mdseries\scshape\xdef\thesmallcapsrmfont{\expandafter\string\the\font}
\sffamily\bfseries\scshape\xdef\theboldsmallcapssffont{\expandafter\string\the\font}
\ttfamily\bfseries\scshape\xdef\theboldsmallcapsttfont{\expandafter\string\the\font}
\rmfamily\bfseries\scshape\xdef\theboldsmallcapsrmfont{\expandafter\string\the\font}
\sffamily\mdseries\slshape\xdef\theslantedsffont{\expandafter\string\the\font}
\ttfamily\mdseries\slshape\xdef\theslantedttfont{\expandafter\string\the\font}
\rmfamily\mdseries\slshape\xdef\theslantedrmfont{\expandafter\string\the\font}
\sffamily\bfseries\slshape\xdef\theboldslantedsffont{\expandafter\string\the\font}
\ttfamily\bfseries\slshape\xdef\theboldslantedttfont{\expandafter\string\the\font}
\rmfamily\bfseries\slshape\xdef\theboldslantedrmfont{\expandafter\string\the\font}

% Math Fonts
\xdef\themathfont{\expandafter\string\the\textfont0}
\xdef\themathfontone{\expandafter\string\the\textfont1}
\xdef\themathfonttwo{\expandafter\string\the\textfont2}
\xdef\themathfontthree{\expandafter\string\the\textfont3}

\makeatother

% Define Python Context
\setpythontexcontext{%
    romanfont=\thermfont,
    romanboldfont=\theboldrmfont,
    romanitalicfont=\theitalicrmfont,
    romanbolditalicfont=\thebolditalicrmfont,
    sansfont=\thesffont,
    sansboldfont=\theboldsffont,
    sansitalicfont=\theitalicsffont,
    sansbolditalicfont=\thebolditalicsffont,
    monofont=\thettfont,
    monoboldfont=\theboldttfont,
    monoitalicfont=\theitalicttfont,
    monobolditalicfont=\thebolditalicttfont,
    mathfont=\themathfont,
    mathfontone=\themathfontone,
    mathfonttwo=\themathfonttwo,
    mathfontthree=\themathfontthree,
    scsansfont = \thesmallcapssffont,
    scromanfont = \thesmallcapsrmfont,
    scmonofont = \thesmallcapsttfont,
    scsansboldfont = \theboldsmallcapssffont,
    scromanboldfont = \theboldsmallcapsrmfont,
    scmonoboldfont = \theboldsmallcapsttfont,
    slantedsansfont = \theslantedsffont,
    slantedromanfont = \theslantedrmfont,
    slantedmonofont = \theslantedttfont,
    slantedsansboldfont = \theboldslantedsffont,
    slantedromanboldfont = \theboldslantedrmfont,
    slantedmonoboldfont = \theboldslantedttfont,
    tiny=\thetinysize,
    scriptsize=\thescriptsize,
    footnotesize=\thefootnotesize,
    small=\thesmallsize,
    normalsize=\thenormalsize,
    large=\thelargesize,
    Large=\theLargesize,
    LARGE=\theLARGEsize,
    huge=\thehugesize,
    Huge=\theHugesize,
    baselineskip=\the\baselineskip,
    % baselinestretch=\the\baselinestretch,
    columnsep=\the\columnsep,
    columnwidth=\the\columnwidth,
    evensidemargin=\the\evensidemargin,
    linewidth=\the\linewidth,
    oddsidemargin=\the\oddsidemargin,
    paperwidth=\the\paperwidth,
    paperheight=\the\paperheight,
    parindent=\the\parindent,
    parskip=\the\parskip,
    textheight=\the\textheight,
    textwidth=\the\textwidth,
    topmargin=\the\topmargin,
    unitlength=\the\unitlength,
    in=\the\inlength,
    cm=\the\cmlength,
    mm=\the\mmlength,
    em=\the\emlength,
    ex=\the\exlength,
    bp=\the\bplength,
    dd=\the\ddlength,
    pc=\the\pclength
}

\AtBeginDocument{
    \renewcommand{\abstractname}{Introduction}
    \addtocontents{toc}{\protect\thispagestyle{empty}}
}

\begin{pythontexcustomcode}{py}
import os

import numpy as np # Python numerical computing library
import matplotlib as mpl

from cycler import cycler # Property cycler utilities
from matplotlib import pyplot as plt # Pyplot API
from matplotlib import rcParams as rc # Matplotlib plot styling
from matplotlib.ticker import EngFormatter # Plot tick formatting

class Environment():

    build_directory = os.path.abspath(os.getcwd())
    output_directory = os.sep.join(build_directory.split(os.sep)[:-1])
    figures_directory = os.path.join(output_directory, 'figures')
    pgf_directory = os.path.join(figures_directory, 'pgf')
    pdf_directory = os.path.join(figures_directory, 'pdf')

    def __init__(self):
        figure_directories = [ self.figures_directory, self.pgf_directory, self.pdf_directory ]
        for directory in figure_directories:
            if not os.path.exists(directory):
                os.mkdir(directory)

environment = Environment()

def metallic_ratio(n):
    return 0.5 * ( n + np.sqrt(n**2 + 4) )
    
def eng_format(arg: str):
    return EngFormatter(unit=arg, sep=r'\,')

def save_pythontex_figure(figure, figure_name):
    if type(figure) == mpl.figure.Figure:
        figure.savefig(os.path.join(environment.pgf_directory, f'{figure_name}.pgf'))
        figure.savefig(os.path.join(environment.pdf_directory, f'{figure_name}.pdf'))
    return figure

class Geometry():

    # Convert from (LaTeX!) points to inches 
    # There is ~ 0.14 micron floating point error here
    in_length = float(pytex.context['in'][:-2]) / 72.27
    cm_length = float(pytex.context['cm'][:-2]) / 72.27
    mm_length = float(pytex.context['mm'][:-2]) / 72.27
    em_length = float(pytex.context['em'][:-2]) / 72.27
    ex_length = float(pytex.context['ex'][:-2]) / 72.27
    bp_length = float(pytex.context['bp'][:-2]) / 72.27
    dd_length = float(pytex.context['dd'][:-2]) / 72.27
    pc_length = float(pytex.context['pc'][:-2]) / 72.27

    column_width = float(pytex.context['columnwidth'][:-2]) / 72.27
    text_width = float(pytex.context['textwidth'][:-2]) / 72.27
    text_height = float(pytex.context['textheight'][:-2]) / 72.27
    figure_width = column_width - 2 * em_length
    figure_height = figure_width / metallic_ratio(1) # Define figure height as a function of the figure width and the golden ratio

    axis_dimensions = (0, 0, 1, 1) # 0 lr margin, 0 tb margin, 100% figure size

geometry = Geometry()

# Determine font parameters
class Font():
    # Font faces
    roman = pytex.context['romanfont'].split('/')[1][:-3]
    bold_roman = pytex.context['romanboldfont'].split('/')[1][:-3]
    bold_italic_roman = pytex.context['romanbolditalicfont'].split('/')[1][:-3]
    italic_roman = pytex.context['romanitalicfont'].split('/')[1][:-3]
    sans = pytex.context['sansfont'].split('/')[1][:-3]
    mono = pytex.context['monofont'].split('/')[1][:-3]
    math = pytex.context['mathfont'].split('/')[1][:-3]
    
    # Font sizes
    tiny = pytex.context['tiny']
    script_size = pytex.context['scriptsize']
    footnote_size = pytex.context['footnotesize']
    small = pytex.context['small']
    normal_size = pytex.context['normalsize']
    large = pytex.context['large']
    llarge = pytex.context['Large']
    lllarge = pytex.context['LARGE']
    huge = pytex.context['huge']
    hhuge = pytex.context['Huge']

font = Font() # Instantiate the class for use

# figure settings
cmap = plt.get_cmap('grey') # Select colormap
num_plot_styles = 4 # number of colors for plotting

# Initialize empty lists for plot colors and styles
plot_colors = []
line_styles = []
for i in range(num_plot_styles): # Populate the color and style lists
    plot_colors.append(cmap(1.0 * i/num_plot_styles))
    line_styles.append((0, (i+1, i)))

# Define the main cycler with the two component lists 
style_cycler = cycler(color=plot_colors, linestyle=line_styles)

# Document-wide Matplotlib Configuration
rc.update({
        'backend': 'pgf',
        'lines.linewidth': 1,
        'font.family': 'serif',
        'font.size': font.footnote_size,
        'text.usetex': True,
        'axes.prop_cycle': style_cycler,
        'axes.labelsize': font.footnote_size,
        'axes.linewidth': 0.8,
        'xtick.direction': 'in',
        'xtick.top': True,
        'xtick.bottom': True,
        'xtick.minor.visible': True,
        'ytick.direction': 'in',
        'ytick.left': True,
        'ytick.right': True,
        'ytick.minor.visible': True,
        'legend.fontsize': font.footnote_size,
        'legend.fancybox': False,
        'figure.figsize': (geometry.figure_width, geometry.figure_height),
        'figure.dpi': 600,
	'figure.constrained_layout.use': True,
        'figure.constrained_layout.hspace': 0,
        'figure.constrained_layout.wspace': 0,
        'figure.constrained_layout.w_pad': 0,
        'figure.constrained_layout.h_pad': 0,
        'savefig.format': 'pgf',
        'savefig.bbox': 'tight',
        'savefig.transparent': True,
        'pgf.rcfonts': False,
	'pgf.preamble': '\n'.join([
                r'\usepackage{mathtools}',
                r'\usepackage[warnings-off={mathtools-colon, mathtools-overbracket}]{unicode-math}',
                r'\usepackage{lualatex-math}',
                r'\usepackage{siunitx}',
                r'\usepackage{fontspec}',
                r'\setmainfont{%s}[Ligatures=TeX, ItalicFont=%s, BoldFont=%s, BoldItalicFont=%s]' %(font.roman, font.italic_roman, font.bold_roman, font.bold_italic_roman),
                r'\setmathfont{%s}' %(font.math),
                r'\setsansfont{%s}[Ligatures=TeX, Scale=MatchLowercase]' %(font.sans),
                r'\setmonofont{%s}[Ligatures=TeX, Scale=MatchLowercase]' %(font.mono),
		r'\usepackage[USenglish]{babel}',
                r'\usepackage[autostyle=true]{csquotes}'
	]),
        'pgf.texsystem': 'lualatex', # default is xetex, but lualatex is preferred
})

\end{pythontexcustomcode}

\begin{document}

\title{\sffamily\thedoctitle}
\author{\sffamily\theauthor}
\date{\sffamily\today}
\maketitle

\begin{abstract}
    This is a document template using Python\TeX\ and some other packages to simplify the creation of scientific and engineering documents.
\end{abstract}

\tableofcontents
\clearpage

\section{Example Plot}

This plot is required to trigger the Python\TeX\ compilation dependency.

\begin{pycode}

fig, ax = plt.subplots()

x = np.linspace(0, 1, 1000)
for frequency in range(1, num_plot_styles + 1, 1):
    _ = ax.plot(
	x,
	np.sin(2*np.pi*frequency*x),
        label=r'\SI{%.1f}{\hertz}' % frequency
        )

ax.xaxis.set_major_formatter(
    eng_format(r'\unit{\second}')
    )

ax.yaxis.set_major_formatter(
    eng_format(r'\unit{\volt}')
    )

_ = ax.legend()

figure_name = 'example_figure'
save_pythontex_figure(fig, figure_name)

\end{pycode}

\begin{figure}[h!]
%% Creator: Matplotlib, PGF backend
%%
%% To include the figure in your LaTeX document, write
%%   \input{<filename>.pgf}
%%
%% Make sure the required packages are loaded in your preamble
%%   \usepackage{pgf}
%%
%% Also ensure that all the required font packages are loaded; for instance,
%% the lmodern package is sometimes necessary when using math font.
%%   \usepackage{lmodern}
%%
%% Figures using additional raster images can only be included by \input if
%% they are in the same directory as the main LaTeX file. For loading figures
%% from other directories you can use the `import` package
%%   \usepackage{import}
%%
%% and then include the figures with
%%   \import{<path to file>}{<filename>.pgf}
%%
%% Matplotlib used the following preamble
%%   \def\mathdefault#1{#1}
%%   \everymath=\expandafter{\the\everymath\displaystyle}
%%   \usepackage{mathtools}
%%   \usepackage[warnings-off={mathtools-colon, mathtools-overbracket}]{unicode-math}
%%   \usepackage{lualatex-math}
%%   \usepackage{siunitx}
%%   \usepackage{fontspec}
%%   \setmainfont{STIXTwoText-Regular.otf}[Ligatures=TeX, ItalicFont=STIXTwoText-Regular.otf, BoldFont=STIXTwoText-Regular.otf, BoldItalicFont=STIXTwoText-Regular.otf]
%%   \setmathfont{STIXTwoMath-Regular.otf}
%%   \setsansfont{IosevkaAile}[Ligatures=TeX, Scale=MatchLowercase]
%%   \setmonofont{IosevkaFixed}[Ligatures=TeX, Scale=MatchLowercase]
%%   \usepackage[USenglish]{babel}
%%   \usepackage[autostyle=true]{csquotes}
%%   \usepackage{fontspec}
%%   \makeatletter\@ifpackageloaded{underscore}{}{\usepackage[strings]{underscore}}\makeatother
%%
\begingroup%
\makeatletter%
\begin{pgfpicture}%
\pgfpathrectangle{\pgfpointorigin}{\pgfqpoint{5.264342in}{3.329936in}}%
\pgfusepath{use as bounding box, clip}%
\begin{pgfscope}%
\pgfsetbuttcap%
\pgfsetmiterjoin%
\pgfsetlinewidth{0.000000pt}%
\definecolor{currentstroke}{rgb}{0.000000,0.000000,0.000000}%
\pgfsetstrokecolor{currentstroke}%
\pgfsetstrokeopacity{0.000000}%
\pgfsetdash{}{0pt}%
\pgfpathmoveto{\pgfqpoint{0.000000in}{-0.000000in}}%
\pgfpathlineto{\pgfqpoint{5.264342in}{-0.000000in}}%
\pgfpathlineto{\pgfqpoint{5.264342in}{3.329936in}}%
\pgfpathlineto{\pgfqpoint{0.000000in}{3.329936in}}%
\pgfpathlineto{\pgfqpoint{0.000000in}{-0.000000in}}%
\pgfpathclose%
\pgfusepath{}%
\end{pgfscope}%
\begin{pgfscope}%
\pgfsetbuttcap%
\pgfsetmiterjoin%
\pgfsetlinewidth{0.000000pt}%
\definecolor{currentstroke}{rgb}{0.000000,0.000000,0.000000}%
\pgfsetstrokecolor{currentstroke}%
\pgfsetstrokeopacity{0.000000}%
\pgfsetdash{}{0pt}%
\pgfpathmoveto{\pgfqpoint{0.690792in}{0.277222in}}%
\pgfpathlineto{\pgfqpoint{5.164342in}{0.277222in}}%
\pgfpathlineto{\pgfqpoint{5.164342in}{3.229936in}}%
\pgfpathlineto{\pgfqpoint{0.690792in}{3.229936in}}%
\pgfpathlineto{\pgfqpoint{0.690792in}{0.277222in}}%
\pgfpathclose%
\pgfusepath{}%
\end{pgfscope}%
\begin{pgfscope}%
\pgfsetbuttcap%
\pgfsetroundjoin%
\definecolor{currentfill}{rgb}{0.000000,0.000000,0.000000}%
\pgfsetfillcolor{currentfill}%
\pgfsetlinewidth{0.803000pt}%
\definecolor{currentstroke}{rgb}{0.000000,0.000000,0.000000}%
\pgfsetstrokecolor{currentstroke}%
\pgfsetdash{}{0pt}%
\pgfsys@defobject{currentmarker}{\pgfqpoint{0.000000in}{0.000000in}}{\pgfqpoint{0.000000in}{0.048611in}}{%
\pgfpathmoveto{\pgfqpoint{0.000000in}{0.000000in}}%
\pgfpathlineto{\pgfqpoint{0.000000in}{0.048611in}}%
\pgfusepath{stroke,fill}%
}%
\begin{pgfscope}%
\pgfsys@transformshift{0.894135in}{0.277222in}%
\pgfsys@useobject{currentmarker}{}%
\end{pgfscope}%
\end{pgfscope}%
\begin{pgfscope}%
\pgfsetbuttcap%
\pgfsetroundjoin%
\definecolor{currentfill}{rgb}{0.000000,0.000000,0.000000}%
\pgfsetfillcolor{currentfill}%
\pgfsetlinewidth{0.803000pt}%
\definecolor{currentstroke}{rgb}{0.000000,0.000000,0.000000}%
\pgfsetstrokecolor{currentstroke}%
\pgfsetdash{}{0pt}%
\pgfsys@defobject{currentmarker}{\pgfqpoint{0.000000in}{-0.048611in}}{\pgfqpoint{0.000000in}{0.000000in}}{%
\pgfpathmoveto{\pgfqpoint{0.000000in}{0.000000in}}%
\pgfpathlineto{\pgfqpoint{0.000000in}{-0.048611in}}%
\pgfusepath{stroke,fill}%
}%
\begin{pgfscope}%
\pgfsys@transformshift{0.894135in}{3.229936in}%
\pgfsys@useobject{currentmarker}{}%
\end{pgfscope}%
\end{pgfscope}%
\begin{pgfscope}%
\definecolor{textcolor}{rgb}{0.000000,0.000000,0.000000}%
\pgfsetstrokecolor{textcolor}%
\pgfsetfillcolor{textcolor}%
\pgftext[x=0.894135in,y=0.228611in,,top]{\color{textcolor}{\rmfamily\fontsize{10.000000}{12.000000}\selectfont\catcode`\^=\active\def^{\ifmmode\sp\else\^{}\fi}\catcode`\%=\active\def%{\%}$0$\,\unit{\second}}}%
\end{pgfscope}%
\begin{pgfscope}%
\pgfsetbuttcap%
\pgfsetroundjoin%
\definecolor{currentfill}{rgb}{0.000000,0.000000,0.000000}%
\pgfsetfillcolor{currentfill}%
\pgfsetlinewidth{0.803000pt}%
\definecolor{currentstroke}{rgb}{0.000000,0.000000,0.000000}%
\pgfsetstrokecolor{currentstroke}%
\pgfsetdash{}{0pt}%
\pgfsys@defobject{currentmarker}{\pgfqpoint{0.000000in}{0.000000in}}{\pgfqpoint{0.000000in}{0.048611in}}{%
\pgfpathmoveto{\pgfqpoint{0.000000in}{0.000000in}}%
\pgfpathlineto{\pgfqpoint{0.000000in}{0.048611in}}%
\pgfusepath{stroke,fill}%
}%
\begin{pgfscope}%
\pgfsys@transformshift{1.707508in}{0.277222in}%
\pgfsys@useobject{currentmarker}{}%
\end{pgfscope}%
\end{pgfscope}%
\begin{pgfscope}%
\pgfsetbuttcap%
\pgfsetroundjoin%
\definecolor{currentfill}{rgb}{0.000000,0.000000,0.000000}%
\pgfsetfillcolor{currentfill}%
\pgfsetlinewidth{0.803000pt}%
\definecolor{currentstroke}{rgb}{0.000000,0.000000,0.000000}%
\pgfsetstrokecolor{currentstroke}%
\pgfsetdash{}{0pt}%
\pgfsys@defobject{currentmarker}{\pgfqpoint{0.000000in}{-0.048611in}}{\pgfqpoint{0.000000in}{0.000000in}}{%
\pgfpathmoveto{\pgfqpoint{0.000000in}{0.000000in}}%
\pgfpathlineto{\pgfqpoint{0.000000in}{-0.048611in}}%
\pgfusepath{stroke,fill}%
}%
\begin{pgfscope}%
\pgfsys@transformshift{1.707508in}{3.229936in}%
\pgfsys@useobject{currentmarker}{}%
\end{pgfscope}%
\end{pgfscope}%
\begin{pgfscope}%
\definecolor{textcolor}{rgb}{0.000000,0.000000,0.000000}%
\pgfsetstrokecolor{textcolor}%
\pgfsetfillcolor{textcolor}%
\pgftext[x=1.707508in,y=0.228611in,,top]{\color{textcolor}{\rmfamily\fontsize{10.000000}{12.000000}\selectfont\catcode`\^=\active\def^{\ifmmode\sp\else\^{}\fi}\catcode`\%=\active\def%{\%}$200$\,m\unit{\second}}}%
\end{pgfscope}%
\begin{pgfscope}%
\pgfsetbuttcap%
\pgfsetroundjoin%
\definecolor{currentfill}{rgb}{0.000000,0.000000,0.000000}%
\pgfsetfillcolor{currentfill}%
\pgfsetlinewidth{0.803000pt}%
\definecolor{currentstroke}{rgb}{0.000000,0.000000,0.000000}%
\pgfsetstrokecolor{currentstroke}%
\pgfsetdash{}{0pt}%
\pgfsys@defobject{currentmarker}{\pgfqpoint{0.000000in}{0.000000in}}{\pgfqpoint{0.000000in}{0.048611in}}{%
\pgfpathmoveto{\pgfqpoint{0.000000in}{0.000000in}}%
\pgfpathlineto{\pgfqpoint{0.000000in}{0.048611in}}%
\pgfusepath{stroke,fill}%
}%
\begin{pgfscope}%
\pgfsys@transformshift{2.520881in}{0.277222in}%
\pgfsys@useobject{currentmarker}{}%
\end{pgfscope}%
\end{pgfscope}%
\begin{pgfscope}%
\pgfsetbuttcap%
\pgfsetroundjoin%
\definecolor{currentfill}{rgb}{0.000000,0.000000,0.000000}%
\pgfsetfillcolor{currentfill}%
\pgfsetlinewidth{0.803000pt}%
\definecolor{currentstroke}{rgb}{0.000000,0.000000,0.000000}%
\pgfsetstrokecolor{currentstroke}%
\pgfsetdash{}{0pt}%
\pgfsys@defobject{currentmarker}{\pgfqpoint{0.000000in}{-0.048611in}}{\pgfqpoint{0.000000in}{0.000000in}}{%
\pgfpathmoveto{\pgfqpoint{0.000000in}{0.000000in}}%
\pgfpathlineto{\pgfqpoint{0.000000in}{-0.048611in}}%
\pgfusepath{stroke,fill}%
}%
\begin{pgfscope}%
\pgfsys@transformshift{2.520881in}{3.229936in}%
\pgfsys@useobject{currentmarker}{}%
\end{pgfscope}%
\end{pgfscope}%
\begin{pgfscope}%
\definecolor{textcolor}{rgb}{0.000000,0.000000,0.000000}%
\pgfsetstrokecolor{textcolor}%
\pgfsetfillcolor{textcolor}%
\pgftext[x=2.520881in,y=0.228611in,,top]{\color{textcolor}{\rmfamily\fontsize{10.000000}{12.000000}\selectfont\catcode`\^=\active\def^{\ifmmode\sp\else\^{}\fi}\catcode`\%=\active\def%{\%}$400$\,m\unit{\second}}}%
\end{pgfscope}%
\begin{pgfscope}%
\pgfsetbuttcap%
\pgfsetroundjoin%
\definecolor{currentfill}{rgb}{0.000000,0.000000,0.000000}%
\pgfsetfillcolor{currentfill}%
\pgfsetlinewidth{0.803000pt}%
\definecolor{currentstroke}{rgb}{0.000000,0.000000,0.000000}%
\pgfsetstrokecolor{currentstroke}%
\pgfsetdash{}{0pt}%
\pgfsys@defobject{currentmarker}{\pgfqpoint{0.000000in}{0.000000in}}{\pgfqpoint{0.000000in}{0.048611in}}{%
\pgfpathmoveto{\pgfqpoint{0.000000in}{0.000000in}}%
\pgfpathlineto{\pgfqpoint{0.000000in}{0.048611in}}%
\pgfusepath{stroke,fill}%
}%
\begin{pgfscope}%
\pgfsys@transformshift{3.334253in}{0.277222in}%
\pgfsys@useobject{currentmarker}{}%
\end{pgfscope}%
\end{pgfscope}%
\begin{pgfscope}%
\pgfsetbuttcap%
\pgfsetroundjoin%
\definecolor{currentfill}{rgb}{0.000000,0.000000,0.000000}%
\pgfsetfillcolor{currentfill}%
\pgfsetlinewidth{0.803000pt}%
\definecolor{currentstroke}{rgb}{0.000000,0.000000,0.000000}%
\pgfsetstrokecolor{currentstroke}%
\pgfsetdash{}{0pt}%
\pgfsys@defobject{currentmarker}{\pgfqpoint{0.000000in}{-0.048611in}}{\pgfqpoint{0.000000in}{0.000000in}}{%
\pgfpathmoveto{\pgfqpoint{0.000000in}{0.000000in}}%
\pgfpathlineto{\pgfqpoint{0.000000in}{-0.048611in}}%
\pgfusepath{stroke,fill}%
}%
\begin{pgfscope}%
\pgfsys@transformshift{3.334253in}{3.229936in}%
\pgfsys@useobject{currentmarker}{}%
\end{pgfscope}%
\end{pgfscope}%
\begin{pgfscope}%
\definecolor{textcolor}{rgb}{0.000000,0.000000,0.000000}%
\pgfsetstrokecolor{textcolor}%
\pgfsetfillcolor{textcolor}%
\pgftext[x=3.334253in,y=0.228611in,,top]{\color{textcolor}{\rmfamily\fontsize{10.000000}{12.000000}\selectfont\catcode`\^=\active\def^{\ifmmode\sp\else\^{}\fi}\catcode`\%=\active\def%{\%}$600$\,m\unit{\second}}}%
\end{pgfscope}%
\begin{pgfscope}%
\pgfsetbuttcap%
\pgfsetroundjoin%
\definecolor{currentfill}{rgb}{0.000000,0.000000,0.000000}%
\pgfsetfillcolor{currentfill}%
\pgfsetlinewidth{0.803000pt}%
\definecolor{currentstroke}{rgb}{0.000000,0.000000,0.000000}%
\pgfsetstrokecolor{currentstroke}%
\pgfsetdash{}{0pt}%
\pgfsys@defobject{currentmarker}{\pgfqpoint{0.000000in}{0.000000in}}{\pgfqpoint{0.000000in}{0.048611in}}{%
\pgfpathmoveto{\pgfqpoint{0.000000in}{0.000000in}}%
\pgfpathlineto{\pgfqpoint{0.000000in}{0.048611in}}%
\pgfusepath{stroke,fill}%
}%
\begin{pgfscope}%
\pgfsys@transformshift{4.147626in}{0.277222in}%
\pgfsys@useobject{currentmarker}{}%
\end{pgfscope}%
\end{pgfscope}%
\begin{pgfscope}%
\pgfsetbuttcap%
\pgfsetroundjoin%
\definecolor{currentfill}{rgb}{0.000000,0.000000,0.000000}%
\pgfsetfillcolor{currentfill}%
\pgfsetlinewidth{0.803000pt}%
\definecolor{currentstroke}{rgb}{0.000000,0.000000,0.000000}%
\pgfsetstrokecolor{currentstroke}%
\pgfsetdash{}{0pt}%
\pgfsys@defobject{currentmarker}{\pgfqpoint{0.000000in}{-0.048611in}}{\pgfqpoint{0.000000in}{0.000000in}}{%
\pgfpathmoveto{\pgfqpoint{0.000000in}{0.000000in}}%
\pgfpathlineto{\pgfqpoint{0.000000in}{-0.048611in}}%
\pgfusepath{stroke,fill}%
}%
\begin{pgfscope}%
\pgfsys@transformshift{4.147626in}{3.229936in}%
\pgfsys@useobject{currentmarker}{}%
\end{pgfscope}%
\end{pgfscope}%
\begin{pgfscope}%
\definecolor{textcolor}{rgb}{0.000000,0.000000,0.000000}%
\pgfsetstrokecolor{textcolor}%
\pgfsetfillcolor{textcolor}%
\pgftext[x=4.147626in,y=0.228611in,,top]{\color{textcolor}{\rmfamily\fontsize{10.000000}{12.000000}\selectfont\catcode`\^=\active\def^{\ifmmode\sp\else\^{}\fi}\catcode`\%=\active\def%{\%}$800$\,m\unit{\second}}}%
\end{pgfscope}%
\begin{pgfscope}%
\pgfsetbuttcap%
\pgfsetroundjoin%
\definecolor{currentfill}{rgb}{0.000000,0.000000,0.000000}%
\pgfsetfillcolor{currentfill}%
\pgfsetlinewidth{0.803000pt}%
\definecolor{currentstroke}{rgb}{0.000000,0.000000,0.000000}%
\pgfsetstrokecolor{currentstroke}%
\pgfsetdash{}{0pt}%
\pgfsys@defobject{currentmarker}{\pgfqpoint{0.000000in}{0.000000in}}{\pgfqpoint{0.000000in}{0.048611in}}{%
\pgfpathmoveto{\pgfqpoint{0.000000in}{0.000000in}}%
\pgfpathlineto{\pgfqpoint{0.000000in}{0.048611in}}%
\pgfusepath{stroke,fill}%
}%
\begin{pgfscope}%
\pgfsys@transformshift{4.960999in}{0.277222in}%
\pgfsys@useobject{currentmarker}{}%
\end{pgfscope}%
\end{pgfscope}%
\begin{pgfscope}%
\pgfsetbuttcap%
\pgfsetroundjoin%
\definecolor{currentfill}{rgb}{0.000000,0.000000,0.000000}%
\pgfsetfillcolor{currentfill}%
\pgfsetlinewidth{0.803000pt}%
\definecolor{currentstroke}{rgb}{0.000000,0.000000,0.000000}%
\pgfsetstrokecolor{currentstroke}%
\pgfsetdash{}{0pt}%
\pgfsys@defobject{currentmarker}{\pgfqpoint{0.000000in}{-0.048611in}}{\pgfqpoint{0.000000in}{0.000000in}}{%
\pgfpathmoveto{\pgfqpoint{0.000000in}{0.000000in}}%
\pgfpathlineto{\pgfqpoint{0.000000in}{-0.048611in}}%
\pgfusepath{stroke,fill}%
}%
\begin{pgfscope}%
\pgfsys@transformshift{4.960999in}{3.229936in}%
\pgfsys@useobject{currentmarker}{}%
\end{pgfscope}%
\end{pgfscope}%
\begin{pgfscope}%
\definecolor{textcolor}{rgb}{0.000000,0.000000,0.000000}%
\pgfsetstrokecolor{textcolor}%
\pgfsetfillcolor{textcolor}%
\pgftext[x=4.960999in,y=0.228611in,,top]{\color{textcolor}{\rmfamily\fontsize{10.000000}{12.000000}\selectfont\catcode`\^=\active\def^{\ifmmode\sp\else\^{}\fi}\catcode`\%=\active\def%{\%}$1$\,\unit{\second}}}%
\end{pgfscope}%
\begin{pgfscope}%
\pgfsetbuttcap%
\pgfsetroundjoin%
\definecolor{currentfill}{rgb}{0.000000,0.000000,0.000000}%
\pgfsetfillcolor{currentfill}%
\pgfsetlinewidth{0.602250pt}%
\definecolor{currentstroke}{rgb}{0.000000,0.000000,0.000000}%
\pgfsetstrokecolor{currentstroke}%
\pgfsetdash{}{0pt}%
\pgfsys@defobject{currentmarker}{\pgfqpoint{0.000000in}{0.000000in}}{\pgfqpoint{0.000000in}{0.027778in}}{%
\pgfpathmoveto{\pgfqpoint{0.000000in}{0.000000in}}%
\pgfpathlineto{\pgfqpoint{0.000000in}{0.027778in}}%
\pgfusepath{stroke,fill}%
}%
\begin{pgfscope}%
\pgfsys@transformshift{0.690792in}{0.277222in}%
\pgfsys@useobject{currentmarker}{}%
\end{pgfscope}%
\end{pgfscope}%
\begin{pgfscope}%
\pgfsetbuttcap%
\pgfsetroundjoin%
\definecolor{currentfill}{rgb}{0.000000,0.000000,0.000000}%
\pgfsetfillcolor{currentfill}%
\pgfsetlinewidth{0.602250pt}%
\definecolor{currentstroke}{rgb}{0.000000,0.000000,0.000000}%
\pgfsetstrokecolor{currentstroke}%
\pgfsetdash{}{0pt}%
\pgfsys@defobject{currentmarker}{\pgfqpoint{0.000000in}{-0.027778in}}{\pgfqpoint{0.000000in}{0.000000in}}{%
\pgfpathmoveto{\pgfqpoint{0.000000in}{0.000000in}}%
\pgfpathlineto{\pgfqpoint{0.000000in}{-0.027778in}}%
\pgfusepath{stroke,fill}%
}%
\begin{pgfscope}%
\pgfsys@transformshift{0.690792in}{3.229936in}%
\pgfsys@useobject{currentmarker}{}%
\end{pgfscope}%
\end{pgfscope}%
\begin{pgfscope}%
\pgfsetbuttcap%
\pgfsetroundjoin%
\definecolor{currentfill}{rgb}{0.000000,0.000000,0.000000}%
\pgfsetfillcolor{currentfill}%
\pgfsetlinewidth{0.602250pt}%
\definecolor{currentstroke}{rgb}{0.000000,0.000000,0.000000}%
\pgfsetstrokecolor{currentstroke}%
\pgfsetdash{}{0pt}%
\pgfsys@defobject{currentmarker}{\pgfqpoint{0.000000in}{0.000000in}}{\pgfqpoint{0.000000in}{0.027778in}}{%
\pgfpathmoveto{\pgfqpoint{0.000000in}{0.000000in}}%
\pgfpathlineto{\pgfqpoint{0.000000in}{0.027778in}}%
\pgfusepath{stroke,fill}%
}%
\begin{pgfscope}%
\pgfsys@transformshift{1.097478in}{0.277222in}%
\pgfsys@useobject{currentmarker}{}%
\end{pgfscope}%
\end{pgfscope}%
\begin{pgfscope}%
\pgfsetbuttcap%
\pgfsetroundjoin%
\definecolor{currentfill}{rgb}{0.000000,0.000000,0.000000}%
\pgfsetfillcolor{currentfill}%
\pgfsetlinewidth{0.602250pt}%
\definecolor{currentstroke}{rgb}{0.000000,0.000000,0.000000}%
\pgfsetstrokecolor{currentstroke}%
\pgfsetdash{}{0pt}%
\pgfsys@defobject{currentmarker}{\pgfqpoint{0.000000in}{-0.027778in}}{\pgfqpoint{0.000000in}{0.000000in}}{%
\pgfpathmoveto{\pgfqpoint{0.000000in}{0.000000in}}%
\pgfpathlineto{\pgfqpoint{0.000000in}{-0.027778in}}%
\pgfusepath{stroke,fill}%
}%
\begin{pgfscope}%
\pgfsys@transformshift{1.097478in}{3.229936in}%
\pgfsys@useobject{currentmarker}{}%
\end{pgfscope}%
\end{pgfscope}%
\begin{pgfscope}%
\pgfsetbuttcap%
\pgfsetroundjoin%
\definecolor{currentfill}{rgb}{0.000000,0.000000,0.000000}%
\pgfsetfillcolor{currentfill}%
\pgfsetlinewidth{0.602250pt}%
\definecolor{currentstroke}{rgb}{0.000000,0.000000,0.000000}%
\pgfsetstrokecolor{currentstroke}%
\pgfsetdash{}{0pt}%
\pgfsys@defobject{currentmarker}{\pgfqpoint{0.000000in}{0.000000in}}{\pgfqpoint{0.000000in}{0.027778in}}{%
\pgfpathmoveto{\pgfqpoint{0.000000in}{0.000000in}}%
\pgfpathlineto{\pgfqpoint{0.000000in}{0.027778in}}%
\pgfusepath{stroke,fill}%
}%
\begin{pgfscope}%
\pgfsys@transformshift{1.300822in}{0.277222in}%
\pgfsys@useobject{currentmarker}{}%
\end{pgfscope}%
\end{pgfscope}%
\begin{pgfscope}%
\pgfsetbuttcap%
\pgfsetroundjoin%
\definecolor{currentfill}{rgb}{0.000000,0.000000,0.000000}%
\pgfsetfillcolor{currentfill}%
\pgfsetlinewidth{0.602250pt}%
\definecolor{currentstroke}{rgb}{0.000000,0.000000,0.000000}%
\pgfsetstrokecolor{currentstroke}%
\pgfsetdash{}{0pt}%
\pgfsys@defobject{currentmarker}{\pgfqpoint{0.000000in}{-0.027778in}}{\pgfqpoint{0.000000in}{0.000000in}}{%
\pgfpathmoveto{\pgfqpoint{0.000000in}{0.000000in}}%
\pgfpathlineto{\pgfqpoint{0.000000in}{-0.027778in}}%
\pgfusepath{stroke,fill}%
}%
\begin{pgfscope}%
\pgfsys@transformshift{1.300822in}{3.229936in}%
\pgfsys@useobject{currentmarker}{}%
\end{pgfscope}%
\end{pgfscope}%
\begin{pgfscope}%
\pgfsetbuttcap%
\pgfsetroundjoin%
\definecolor{currentfill}{rgb}{0.000000,0.000000,0.000000}%
\pgfsetfillcolor{currentfill}%
\pgfsetlinewidth{0.602250pt}%
\definecolor{currentstroke}{rgb}{0.000000,0.000000,0.000000}%
\pgfsetstrokecolor{currentstroke}%
\pgfsetdash{}{0pt}%
\pgfsys@defobject{currentmarker}{\pgfqpoint{0.000000in}{0.000000in}}{\pgfqpoint{0.000000in}{0.027778in}}{%
\pgfpathmoveto{\pgfqpoint{0.000000in}{0.000000in}}%
\pgfpathlineto{\pgfqpoint{0.000000in}{0.027778in}}%
\pgfusepath{stroke,fill}%
}%
\begin{pgfscope}%
\pgfsys@transformshift{1.504165in}{0.277222in}%
\pgfsys@useobject{currentmarker}{}%
\end{pgfscope}%
\end{pgfscope}%
\begin{pgfscope}%
\pgfsetbuttcap%
\pgfsetroundjoin%
\definecolor{currentfill}{rgb}{0.000000,0.000000,0.000000}%
\pgfsetfillcolor{currentfill}%
\pgfsetlinewidth{0.602250pt}%
\definecolor{currentstroke}{rgb}{0.000000,0.000000,0.000000}%
\pgfsetstrokecolor{currentstroke}%
\pgfsetdash{}{0pt}%
\pgfsys@defobject{currentmarker}{\pgfqpoint{0.000000in}{-0.027778in}}{\pgfqpoint{0.000000in}{0.000000in}}{%
\pgfpathmoveto{\pgfqpoint{0.000000in}{0.000000in}}%
\pgfpathlineto{\pgfqpoint{0.000000in}{-0.027778in}}%
\pgfusepath{stroke,fill}%
}%
\begin{pgfscope}%
\pgfsys@transformshift{1.504165in}{3.229936in}%
\pgfsys@useobject{currentmarker}{}%
\end{pgfscope}%
\end{pgfscope}%
\begin{pgfscope}%
\pgfsetbuttcap%
\pgfsetroundjoin%
\definecolor{currentfill}{rgb}{0.000000,0.000000,0.000000}%
\pgfsetfillcolor{currentfill}%
\pgfsetlinewidth{0.602250pt}%
\definecolor{currentstroke}{rgb}{0.000000,0.000000,0.000000}%
\pgfsetstrokecolor{currentstroke}%
\pgfsetdash{}{0pt}%
\pgfsys@defobject{currentmarker}{\pgfqpoint{0.000000in}{0.000000in}}{\pgfqpoint{0.000000in}{0.027778in}}{%
\pgfpathmoveto{\pgfqpoint{0.000000in}{0.000000in}}%
\pgfpathlineto{\pgfqpoint{0.000000in}{0.027778in}}%
\pgfusepath{stroke,fill}%
}%
\begin{pgfscope}%
\pgfsys@transformshift{1.910851in}{0.277222in}%
\pgfsys@useobject{currentmarker}{}%
\end{pgfscope}%
\end{pgfscope}%
\begin{pgfscope}%
\pgfsetbuttcap%
\pgfsetroundjoin%
\definecolor{currentfill}{rgb}{0.000000,0.000000,0.000000}%
\pgfsetfillcolor{currentfill}%
\pgfsetlinewidth{0.602250pt}%
\definecolor{currentstroke}{rgb}{0.000000,0.000000,0.000000}%
\pgfsetstrokecolor{currentstroke}%
\pgfsetdash{}{0pt}%
\pgfsys@defobject{currentmarker}{\pgfqpoint{0.000000in}{-0.027778in}}{\pgfqpoint{0.000000in}{0.000000in}}{%
\pgfpathmoveto{\pgfqpoint{0.000000in}{0.000000in}}%
\pgfpathlineto{\pgfqpoint{0.000000in}{-0.027778in}}%
\pgfusepath{stroke,fill}%
}%
\begin{pgfscope}%
\pgfsys@transformshift{1.910851in}{3.229936in}%
\pgfsys@useobject{currentmarker}{}%
\end{pgfscope}%
\end{pgfscope}%
\begin{pgfscope}%
\pgfsetbuttcap%
\pgfsetroundjoin%
\definecolor{currentfill}{rgb}{0.000000,0.000000,0.000000}%
\pgfsetfillcolor{currentfill}%
\pgfsetlinewidth{0.602250pt}%
\definecolor{currentstroke}{rgb}{0.000000,0.000000,0.000000}%
\pgfsetstrokecolor{currentstroke}%
\pgfsetdash{}{0pt}%
\pgfsys@defobject{currentmarker}{\pgfqpoint{0.000000in}{0.000000in}}{\pgfqpoint{0.000000in}{0.027778in}}{%
\pgfpathmoveto{\pgfqpoint{0.000000in}{0.000000in}}%
\pgfpathlineto{\pgfqpoint{0.000000in}{0.027778in}}%
\pgfusepath{stroke,fill}%
}%
\begin{pgfscope}%
\pgfsys@transformshift{2.114194in}{0.277222in}%
\pgfsys@useobject{currentmarker}{}%
\end{pgfscope}%
\end{pgfscope}%
\begin{pgfscope}%
\pgfsetbuttcap%
\pgfsetroundjoin%
\definecolor{currentfill}{rgb}{0.000000,0.000000,0.000000}%
\pgfsetfillcolor{currentfill}%
\pgfsetlinewidth{0.602250pt}%
\definecolor{currentstroke}{rgb}{0.000000,0.000000,0.000000}%
\pgfsetstrokecolor{currentstroke}%
\pgfsetdash{}{0pt}%
\pgfsys@defobject{currentmarker}{\pgfqpoint{0.000000in}{-0.027778in}}{\pgfqpoint{0.000000in}{0.000000in}}{%
\pgfpathmoveto{\pgfqpoint{0.000000in}{0.000000in}}%
\pgfpathlineto{\pgfqpoint{0.000000in}{-0.027778in}}%
\pgfusepath{stroke,fill}%
}%
\begin{pgfscope}%
\pgfsys@transformshift{2.114194in}{3.229936in}%
\pgfsys@useobject{currentmarker}{}%
\end{pgfscope}%
\end{pgfscope}%
\begin{pgfscope}%
\pgfsetbuttcap%
\pgfsetroundjoin%
\definecolor{currentfill}{rgb}{0.000000,0.000000,0.000000}%
\pgfsetfillcolor{currentfill}%
\pgfsetlinewidth{0.602250pt}%
\definecolor{currentstroke}{rgb}{0.000000,0.000000,0.000000}%
\pgfsetstrokecolor{currentstroke}%
\pgfsetdash{}{0pt}%
\pgfsys@defobject{currentmarker}{\pgfqpoint{0.000000in}{0.000000in}}{\pgfqpoint{0.000000in}{0.027778in}}{%
\pgfpathmoveto{\pgfqpoint{0.000000in}{0.000000in}}%
\pgfpathlineto{\pgfqpoint{0.000000in}{0.027778in}}%
\pgfusepath{stroke,fill}%
}%
\begin{pgfscope}%
\pgfsys@transformshift{2.317537in}{0.277222in}%
\pgfsys@useobject{currentmarker}{}%
\end{pgfscope}%
\end{pgfscope}%
\begin{pgfscope}%
\pgfsetbuttcap%
\pgfsetroundjoin%
\definecolor{currentfill}{rgb}{0.000000,0.000000,0.000000}%
\pgfsetfillcolor{currentfill}%
\pgfsetlinewidth{0.602250pt}%
\definecolor{currentstroke}{rgb}{0.000000,0.000000,0.000000}%
\pgfsetstrokecolor{currentstroke}%
\pgfsetdash{}{0pt}%
\pgfsys@defobject{currentmarker}{\pgfqpoint{0.000000in}{-0.027778in}}{\pgfqpoint{0.000000in}{0.000000in}}{%
\pgfpathmoveto{\pgfqpoint{0.000000in}{0.000000in}}%
\pgfpathlineto{\pgfqpoint{0.000000in}{-0.027778in}}%
\pgfusepath{stroke,fill}%
}%
\begin{pgfscope}%
\pgfsys@transformshift{2.317537in}{3.229936in}%
\pgfsys@useobject{currentmarker}{}%
\end{pgfscope}%
\end{pgfscope}%
\begin{pgfscope}%
\pgfsetbuttcap%
\pgfsetroundjoin%
\definecolor{currentfill}{rgb}{0.000000,0.000000,0.000000}%
\pgfsetfillcolor{currentfill}%
\pgfsetlinewidth{0.602250pt}%
\definecolor{currentstroke}{rgb}{0.000000,0.000000,0.000000}%
\pgfsetstrokecolor{currentstroke}%
\pgfsetdash{}{0pt}%
\pgfsys@defobject{currentmarker}{\pgfqpoint{0.000000in}{0.000000in}}{\pgfqpoint{0.000000in}{0.027778in}}{%
\pgfpathmoveto{\pgfqpoint{0.000000in}{0.000000in}}%
\pgfpathlineto{\pgfqpoint{0.000000in}{0.027778in}}%
\pgfusepath{stroke,fill}%
}%
\begin{pgfscope}%
\pgfsys@transformshift{2.724224in}{0.277222in}%
\pgfsys@useobject{currentmarker}{}%
\end{pgfscope}%
\end{pgfscope}%
\begin{pgfscope}%
\pgfsetbuttcap%
\pgfsetroundjoin%
\definecolor{currentfill}{rgb}{0.000000,0.000000,0.000000}%
\pgfsetfillcolor{currentfill}%
\pgfsetlinewidth{0.602250pt}%
\definecolor{currentstroke}{rgb}{0.000000,0.000000,0.000000}%
\pgfsetstrokecolor{currentstroke}%
\pgfsetdash{}{0pt}%
\pgfsys@defobject{currentmarker}{\pgfqpoint{0.000000in}{-0.027778in}}{\pgfqpoint{0.000000in}{0.000000in}}{%
\pgfpathmoveto{\pgfqpoint{0.000000in}{0.000000in}}%
\pgfpathlineto{\pgfqpoint{0.000000in}{-0.027778in}}%
\pgfusepath{stroke,fill}%
}%
\begin{pgfscope}%
\pgfsys@transformshift{2.724224in}{3.229936in}%
\pgfsys@useobject{currentmarker}{}%
\end{pgfscope}%
\end{pgfscope}%
\begin{pgfscope}%
\pgfsetbuttcap%
\pgfsetroundjoin%
\definecolor{currentfill}{rgb}{0.000000,0.000000,0.000000}%
\pgfsetfillcolor{currentfill}%
\pgfsetlinewidth{0.602250pt}%
\definecolor{currentstroke}{rgb}{0.000000,0.000000,0.000000}%
\pgfsetstrokecolor{currentstroke}%
\pgfsetdash{}{0pt}%
\pgfsys@defobject{currentmarker}{\pgfqpoint{0.000000in}{0.000000in}}{\pgfqpoint{0.000000in}{0.027778in}}{%
\pgfpathmoveto{\pgfqpoint{0.000000in}{0.000000in}}%
\pgfpathlineto{\pgfqpoint{0.000000in}{0.027778in}}%
\pgfusepath{stroke,fill}%
}%
\begin{pgfscope}%
\pgfsys@transformshift{2.927567in}{0.277222in}%
\pgfsys@useobject{currentmarker}{}%
\end{pgfscope}%
\end{pgfscope}%
\begin{pgfscope}%
\pgfsetbuttcap%
\pgfsetroundjoin%
\definecolor{currentfill}{rgb}{0.000000,0.000000,0.000000}%
\pgfsetfillcolor{currentfill}%
\pgfsetlinewidth{0.602250pt}%
\definecolor{currentstroke}{rgb}{0.000000,0.000000,0.000000}%
\pgfsetstrokecolor{currentstroke}%
\pgfsetdash{}{0pt}%
\pgfsys@defobject{currentmarker}{\pgfqpoint{0.000000in}{-0.027778in}}{\pgfqpoint{0.000000in}{0.000000in}}{%
\pgfpathmoveto{\pgfqpoint{0.000000in}{0.000000in}}%
\pgfpathlineto{\pgfqpoint{0.000000in}{-0.027778in}}%
\pgfusepath{stroke,fill}%
}%
\begin{pgfscope}%
\pgfsys@transformshift{2.927567in}{3.229936in}%
\pgfsys@useobject{currentmarker}{}%
\end{pgfscope}%
\end{pgfscope}%
\begin{pgfscope}%
\pgfsetbuttcap%
\pgfsetroundjoin%
\definecolor{currentfill}{rgb}{0.000000,0.000000,0.000000}%
\pgfsetfillcolor{currentfill}%
\pgfsetlinewidth{0.602250pt}%
\definecolor{currentstroke}{rgb}{0.000000,0.000000,0.000000}%
\pgfsetstrokecolor{currentstroke}%
\pgfsetdash{}{0pt}%
\pgfsys@defobject{currentmarker}{\pgfqpoint{0.000000in}{0.000000in}}{\pgfqpoint{0.000000in}{0.027778in}}{%
\pgfpathmoveto{\pgfqpoint{0.000000in}{0.000000in}}%
\pgfpathlineto{\pgfqpoint{0.000000in}{0.027778in}}%
\pgfusepath{stroke,fill}%
}%
\begin{pgfscope}%
\pgfsys@transformshift{3.130910in}{0.277222in}%
\pgfsys@useobject{currentmarker}{}%
\end{pgfscope}%
\end{pgfscope}%
\begin{pgfscope}%
\pgfsetbuttcap%
\pgfsetroundjoin%
\definecolor{currentfill}{rgb}{0.000000,0.000000,0.000000}%
\pgfsetfillcolor{currentfill}%
\pgfsetlinewidth{0.602250pt}%
\definecolor{currentstroke}{rgb}{0.000000,0.000000,0.000000}%
\pgfsetstrokecolor{currentstroke}%
\pgfsetdash{}{0pt}%
\pgfsys@defobject{currentmarker}{\pgfqpoint{0.000000in}{-0.027778in}}{\pgfqpoint{0.000000in}{0.000000in}}{%
\pgfpathmoveto{\pgfqpoint{0.000000in}{0.000000in}}%
\pgfpathlineto{\pgfqpoint{0.000000in}{-0.027778in}}%
\pgfusepath{stroke,fill}%
}%
\begin{pgfscope}%
\pgfsys@transformshift{3.130910in}{3.229936in}%
\pgfsys@useobject{currentmarker}{}%
\end{pgfscope}%
\end{pgfscope}%
\begin{pgfscope}%
\pgfsetbuttcap%
\pgfsetroundjoin%
\definecolor{currentfill}{rgb}{0.000000,0.000000,0.000000}%
\pgfsetfillcolor{currentfill}%
\pgfsetlinewidth{0.602250pt}%
\definecolor{currentstroke}{rgb}{0.000000,0.000000,0.000000}%
\pgfsetstrokecolor{currentstroke}%
\pgfsetdash{}{0pt}%
\pgfsys@defobject{currentmarker}{\pgfqpoint{0.000000in}{0.000000in}}{\pgfqpoint{0.000000in}{0.027778in}}{%
\pgfpathmoveto{\pgfqpoint{0.000000in}{0.000000in}}%
\pgfpathlineto{\pgfqpoint{0.000000in}{0.027778in}}%
\pgfusepath{stroke,fill}%
}%
\begin{pgfscope}%
\pgfsys@transformshift{3.537597in}{0.277222in}%
\pgfsys@useobject{currentmarker}{}%
\end{pgfscope}%
\end{pgfscope}%
\begin{pgfscope}%
\pgfsetbuttcap%
\pgfsetroundjoin%
\definecolor{currentfill}{rgb}{0.000000,0.000000,0.000000}%
\pgfsetfillcolor{currentfill}%
\pgfsetlinewidth{0.602250pt}%
\definecolor{currentstroke}{rgb}{0.000000,0.000000,0.000000}%
\pgfsetstrokecolor{currentstroke}%
\pgfsetdash{}{0pt}%
\pgfsys@defobject{currentmarker}{\pgfqpoint{0.000000in}{-0.027778in}}{\pgfqpoint{0.000000in}{0.000000in}}{%
\pgfpathmoveto{\pgfqpoint{0.000000in}{0.000000in}}%
\pgfpathlineto{\pgfqpoint{0.000000in}{-0.027778in}}%
\pgfusepath{stroke,fill}%
}%
\begin{pgfscope}%
\pgfsys@transformshift{3.537597in}{3.229936in}%
\pgfsys@useobject{currentmarker}{}%
\end{pgfscope}%
\end{pgfscope}%
\begin{pgfscope}%
\pgfsetbuttcap%
\pgfsetroundjoin%
\definecolor{currentfill}{rgb}{0.000000,0.000000,0.000000}%
\pgfsetfillcolor{currentfill}%
\pgfsetlinewidth{0.602250pt}%
\definecolor{currentstroke}{rgb}{0.000000,0.000000,0.000000}%
\pgfsetstrokecolor{currentstroke}%
\pgfsetdash{}{0pt}%
\pgfsys@defobject{currentmarker}{\pgfqpoint{0.000000in}{0.000000in}}{\pgfqpoint{0.000000in}{0.027778in}}{%
\pgfpathmoveto{\pgfqpoint{0.000000in}{0.000000in}}%
\pgfpathlineto{\pgfqpoint{0.000000in}{0.027778in}}%
\pgfusepath{stroke,fill}%
}%
\begin{pgfscope}%
\pgfsys@transformshift{3.740940in}{0.277222in}%
\pgfsys@useobject{currentmarker}{}%
\end{pgfscope}%
\end{pgfscope}%
\begin{pgfscope}%
\pgfsetbuttcap%
\pgfsetroundjoin%
\definecolor{currentfill}{rgb}{0.000000,0.000000,0.000000}%
\pgfsetfillcolor{currentfill}%
\pgfsetlinewidth{0.602250pt}%
\definecolor{currentstroke}{rgb}{0.000000,0.000000,0.000000}%
\pgfsetstrokecolor{currentstroke}%
\pgfsetdash{}{0pt}%
\pgfsys@defobject{currentmarker}{\pgfqpoint{0.000000in}{-0.027778in}}{\pgfqpoint{0.000000in}{0.000000in}}{%
\pgfpathmoveto{\pgfqpoint{0.000000in}{0.000000in}}%
\pgfpathlineto{\pgfqpoint{0.000000in}{-0.027778in}}%
\pgfusepath{stroke,fill}%
}%
\begin{pgfscope}%
\pgfsys@transformshift{3.740940in}{3.229936in}%
\pgfsys@useobject{currentmarker}{}%
\end{pgfscope}%
\end{pgfscope}%
\begin{pgfscope}%
\pgfsetbuttcap%
\pgfsetroundjoin%
\definecolor{currentfill}{rgb}{0.000000,0.000000,0.000000}%
\pgfsetfillcolor{currentfill}%
\pgfsetlinewidth{0.602250pt}%
\definecolor{currentstroke}{rgb}{0.000000,0.000000,0.000000}%
\pgfsetstrokecolor{currentstroke}%
\pgfsetdash{}{0pt}%
\pgfsys@defobject{currentmarker}{\pgfqpoint{0.000000in}{0.000000in}}{\pgfqpoint{0.000000in}{0.027778in}}{%
\pgfpathmoveto{\pgfqpoint{0.000000in}{0.000000in}}%
\pgfpathlineto{\pgfqpoint{0.000000in}{0.027778in}}%
\pgfusepath{stroke,fill}%
}%
\begin{pgfscope}%
\pgfsys@transformshift{3.944283in}{0.277222in}%
\pgfsys@useobject{currentmarker}{}%
\end{pgfscope}%
\end{pgfscope}%
\begin{pgfscope}%
\pgfsetbuttcap%
\pgfsetroundjoin%
\definecolor{currentfill}{rgb}{0.000000,0.000000,0.000000}%
\pgfsetfillcolor{currentfill}%
\pgfsetlinewidth{0.602250pt}%
\definecolor{currentstroke}{rgb}{0.000000,0.000000,0.000000}%
\pgfsetstrokecolor{currentstroke}%
\pgfsetdash{}{0pt}%
\pgfsys@defobject{currentmarker}{\pgfqpoint{0.000000in}{-0.027778in}}{\pgfqpoint{0.000000in}{0.000000in}}{%
\pgfpathmoveto{\pgfqpoint{0.000000in}{0.000000in}}%
\pgfpathlineto{\pgfqpoint{0.000000in}{-0.027778in}}%
\pgfusepath{stroke,fill}%
}%
\begin{pgfscope}%
\pgfsys@transformshift{3.944283in}{3.229936in}%
\pgfsys@useobject{currentmarker}{}%
\end{pgfscope}%
\end{pgfscope}%
\begin{pgfscope}%
\pgfsetbuttcap%
\pgfsetroundjoin%
\definecolor{currentfill}{rgb}{0.000000,0.000000,0.000000}%
\pgfsetfillcolor{currentfill}%
\pgfsetlinewidth{0.602250pt}%
\definecolor{currentstroke}{rgb}{0.000000,0.000000,0.000000}%
\pgfsetstrokecolor{currentstroke}%
\pgfsetdash{}{0pt}%
\pgfsys@defobject{currentmarker}{\pgfqpoint{0.000000in}{0.000000in}}{\pgfqpoint{0.000000in}{0.027778in}}{%
\pgfpathmoveto{\pgfqpoint{0.000000in}{0.000000in}}%
\pgfpathlineto{\pgfqpoint{0.000000in}{0.027778in}}%
\pgfusepath{stroke,fill}%
}%
\begin{pgfscope}%
\pgfsys@transformshift{4.350969in}{0.277222in}%
\pgfsys@useobject{currentmarker}{}%
\end{pgfscope}%
\end{pgfscope}%
\begin{pgfscope}%
\pgfsetbuttcap%
\pgfsetroundjoin%
\definecolor{currentfill}{rgb}{0.000000,0.000000,0.000000}%
\pgfsetfillcolor{currentfill}%
\pgfsetlinewidth{0.602250pt}%
\definecolor{currentstroke}{rgb}{0.000000,0.000000,0.000000}%
\pgfsetstrokecolor{currentstroke}%
\pgfsetdash{}{0pt}%
\pgfsys@defobject{currentmarker}{\pgfqpoint{0.000000in}{-0.027778in}}{\pgfqpoint{0.000000in}{0.000000in}}{%
\pgfpathmoveto{\pgfqpoint{0.000000in}{0.000000in}}%
\pgfpathlineto{\pgfqpoint{0.000000in}{-0.027778in}}%
\pgfusepath{stroke,fill}%
}%
\begin{pgfscope}%
\pgfsys@transformshift{4.350969in}{3.229936in}%
\pgfsys@useobject{currentmarker}{}%
\end{pgfscope}%
\end{pgfscope}%
\begin{pgfscope}%
\pgfsetbuttcap%
\pgfsetroundjoin%
\definecolor{currentfill}{rgb}{0.000000,0.000000,0.000000}%
\pgfsetfillcolor{currentfill}%
\pgfsetlinewidth{0.602250pt}%
\definecolor{currentstroke}{rgb}{0.000000,0.000000,0.000000}%
\pgfsetstrokecolor{currentstroke}%
\pgfsetdash{}{0pt}%
\pgfsys@defobject{currentmarker}{\pgfqpoint{0.000000in}{0.000000in}}{\pgfqpoint{0.000000in}{0.027778in}}{%
\pgfpathmoveto{\pgfqpoint{0.000000in}{0.000000in}}%
\pgfpathlineto{\pgfqpoint{0.000000in}{0.027778in}}%
\pgfusepath{stroke,fill}%
}%
\begin{pgfscope}%
\pgfsys@transformshift{4.554312in}{0.277222in}%
\pgfsys@useobject{currentmarker}{}%
\end{pgfscope}%
\end{pgfscope}%
\begin{pgfscope}%
\pgfsetbuttcap%
\pgfsetroundjoin%
\definecolor{currentfill}{rgb}{0.000000,0.000000,0.000000}%
\pgfsetfillcolor{currentfill}%
\pgfsetlinewidth{0.602250pt}%
\definecolor{currentstroke}{rgb}{0.000000,0.000000,0.000000}%
\pgfsetstrokecolor{currentstroke}%
\pgfsetdash{}{0pt}%
\pgfsys@defobject{currentmarker}{\pgfqpoint{0.000000in}{-0.027778in}}{\pgfqpoint{0.000000in}{0.000000in}}{%
\pgfpathmoveto{\pgfqpoint{0.000000in}{0.000000in}}%
\pgfpathlineto{\pgfqpoint{0.000000in}{-0.027778in}}%
\pgfusepath{stroke,fill}%
}%
\begin{pgfscope}%
\pgfsys@transformshift{4.554312in}{3.229936in}%
\pgfsys@useobject{currentmarker}{}%
\end{pgfscope}%
\end{pgfscope}%
\begin{pgfscope}%
\pgfsetbuttcap%
\pgfsetroundjoin%
\definecolor{currentfill}{rgb}{0.000000,0.000000,0.000000}%
\pgfsetfillcolor{currentfill}%
\pgfsetlinewidth{0.602250pt}%
\definecolor{currentstroke}{rgb}{0.000000,0.000000,0.000000}%
\pgfsetstrokecolor{currentstroke}%
\pgfsetdash{}{0pt}%
\pgfsys@defobject{currentmarker}{\pgfqpoint{0.000000in}{0.000000in}}{\pgfqpoint{0.000000in}{0.027778in}}{%
\pgfpathmoveto{\pgfqpoint{0.000000in}{0.000000in}}%
\pgfpathlineto{\pgfqpoint{0.000000in}{0.027778in}}%
\pgfusepath{stroke,fill}%
}%
\begin{pgfscope}%
\pgfsys@transformshift{4.757656in}{0.277222in}%
\pgfsys@useobject{currentmarker}{}%
\end{pgfscope}%
\end{pgfscope}%
\begin{pgfscope}%
\pgfsetbuttcap%
\pgfsetroundjoin%
\definecolor{currentfill}{rgb}{0.000000,0.000000,0.000000}%
\pgfsetfillcolor{currentfill}%
\pgfsetlinewidth{0.602250pt}%
\definecolor{currentstroke}{rgb}{0.000000,0.000000,0.000000}%
\pgfsetstrokecolor{currentstroke}%
\pgfsetdash{}{0pt}%
\pgfsys@defobject{currentmarker}{\pgfqpoint{0.000000in}{-0.027778in}}{\pgfqpoint{0.000000in}{0.000000in}}{%
\pgfpathmoveto{\pgfqpoint{0.000000in}{0.000000in}}%
\pgfpathlineto{\pgfqpoint{0.000000in}{-0.027778in}}%
\pgfusepath{stroke,fill}%
}%
\begin{pgfscope}%
\pgfsys@transformshift{4.757656in}{3.229936in}%
\pgfsys@useobject{currentmarker}{}%
\end{pgfscope}%
\end{pgfscope}%
\begin{pgfscope}%
\pgfsetbuttcap%
\pgfsetroundjoin%
\definecolor{currentfill}{rgb}{0.000000,0.000000,0.000000}%
\pgfsetfillcolor{currentfill}%
\pgfsetlinewidth{0.602250pt}%
\definecolor{currentstroke}{rgb}{0.000000,0.000000,0.000000}%
\pgfsetstrokecolor{currentstroke}%
\pgfsetdash{}{0pt}%
\pgfsys@defobject{currentmarker}{\pgfqpoint{0.000000in}{0.000000in}}{\pgfqpoint{0.000000in}{0.027778in}}{%
\pgfpathmoveto{\pgfqpoint{0.000000in}{0.000000in}}%
\pgfpathlineto{\pgfqpoint{0.000000in}{0.027778in}}%
\pgfusepath{stroke,fill}%
}%
\begin{pgfscope}%
\pgfsys@transformshift{5.164342in}{0.277222in}%
\pgfsys@useobject{currentmarker}{}%
\end{pgfscope}%
\end{pgfscope}%
\begin{pgfscope}%
\pgfsetbuttcap%
\pgfsetroundjoin%
\definecolor{currentfill}{rgb}{0.000000,0.000000,0.000000}%
\pgfsetfillcolor{currentfill}%
\pgfsetlinewidth{0.602250pt}%
\definecolor{currentstroke}{rgb}{0.000000,0.000000,0.000000}%
\pgfsetstrokecolor{currentstroke}%
\pgfsetdash{}{0pt}%
\pgfsys@defobject{currentmarker}{\pgfqpoint{0.000000in}{-0.027778in}}{\pgfqpoint{0.000000in}{0.000000in}}{%
\pgfpathmoveto{\pgfqpoint{0.000000in}{0.000000in}}%
\pgfpathlineto{\pgfqpoint{0.000000in}{-0.027778in}}%
\pgfusepath{stroke,fill}%
}%
\begin{pgfscope}%
\pgfsys@transformshift{5.164342in}{3.229936in}%
\pgfsys@useobject{currentmarker}{}%
\end{pgfscope}%
\end{pgfscope}%
\begin{pgfscope}%
\pgfsetbuttcap%
\pgfsetroundjoin%
\definecolor{currentfill}{rgb}{0.000000,0.000000,0.000000}%
\pgfsetfillcolor{currentfill}%
\pgfsetlinewidth{0.803000pt}%
\definecolor{currentstroke}{rgb}{0.000000,0.000000,0.000000}%
\pgfsetstrokecolor{currentstroke}%
\pgfsetdash{}{0pt}%
\pgfsys@defobject{currentmarker}{\pgfqpoint{0.000000in}{0.000000in}}{\pgfqpoint{0.048611in}{0.000000in}}{%
\pgfpathmoveto{\pgfqpoint{0.000000in}{0.000000in}}%
\pgfpathlineto{\pgfqpoint{0.048611in}{0.000000in}}%
\pgfusepath{stroke,fill}%
}%
\begin{pgfscope}%
\pgfsys@transformshift{0.690792in}{0.411435in}%
\pgfsys@useobject{currentmarker}{}%
\end{pgfscope}%
\end{pgfscope}%
\begin{pgfscope}%
\pgfsetbuttcap%
\pgfsetroundjoin%
\definecolor{currentfill}{rgb}{0.000000,0.000000,0.000000}%
\pgfsetfillcolor{currentfill}%
\pgfsetlinewidth{0.803000pt}%
\definecolor{currentstroke}{rgb}{0.000000,0.000000,0.000000}%
\pgfsetstrokecolor{currentstroke}%
\pgfsetdash{}{0pt}%
\pgfsys@defobject{currentmarker}{\pgfqpoint{-0.048611in}{0.000000in}}{\pgfqpoint{-0.000000in}{0.000000in}}{%
\pgfpathmoveto{\pgfqpoint{-0.000000in}{0.000000in}}%
\pgfpathlineto{\pgfqpoint{-0.048611in}{0.000000in}}%
\pgfusepath{stroke,fill}%
}%
\begin{pgfscope}%
\pgfsys@transformshift{5.164342in}{0.411435in}%
\pgfsys@useobject{currentmarker}{}%
\end{pgfscope}%
\end{pgfscope}%
\begin{pgfscope}%
\definecolor{textcolor}{rgb}{0.000000,0.000000,0.000000}%
\pgfsetstrokecolor{textcolor}%
\pgfsetfillcolor{textcolor}%
\pgftext[x=0.355833in, y=0.362407in, left, base]{\color{textcolor}{\rmfamily\fontsize{10.000000}{12.000000}\selectfont\catcode`\^=\active\def^{\ifmmode\sp\else\^{}\fi}\catcode`\%=\active\def%{\%}$\ensuremath{-}1$\,\unit{\volt}}}%
\end{pgfscope}%
\begin{pgfscope}%
\pgfsetbuttcap%
\pgfsetroundjoin%
\definecolor{currentfill}{rgb}{0.000000,0.000000,0.000000}%
\pgfsetfillcolor{currentfill}%
\pgfsetlinewidth{0.803000pt}%
\definecolor{currentstroke}{rgb}{0.000000,0.000000,0.000000}%
\pgfsetstrokecolor{currentstroke}%
\pgfsetdash{}{0pt}%
\pgfsys@defobject{currentmarker}{\pgfqpoint{0.000000in}{0.000000in}}{\pgfqpoint{0.048611in}{0.000000in}}{%
\pgfpathmoveto{\pgfqpoint{0.000000in}{0.000000in}}%
\pgfpathlineto{\pgfqpoint{0.048611in}{0.000000in}}%
\pgfusepath{stroke,fill}%
}%
\begin{pgfscope}%
\pgfsys@transformshift{0.690792in}{0.746971in}%
\pgfsys@useobject{currentmarker}{}%
\end{pgfscope}%
\end{pgfscope}%
\begin{pgfscope}%
\pgfsetbuttcap%
\pgfsetroundjoin%
\definecolor{currentfill}{rgb}{0.000000,0.000000,0.000000}%
\pgfsetfillcolor{currentfill}%
\pgfsetlinewidth{0.803000pt}%
\definecolor{currentstroke}{rgb}{0.000000,0.000000,0.000000}%
\pgfsetstrokecolor{currentstroke}%
\pgfsetdash{}{0pt}%
\pgfsys@defobject{currentmarker}{\pgfqpoint{-0.048611in}{0.000000in}}{\pgfqpoint{-0.000000in}{0.000000in}}{%
\pgfpathmoveto{\pgfqpoint{-0.000000in}{0.000000in}}%
\pgfpathlineto{\pgfqpoint{-0.048611in}{0.000000in}}%
\pgfusepath{stroke,fill}%
}%
\begin{pgfscope}%
\pgfsys@transformshift{5.164342in}{0.746971in}%
\pgfsys@useobject{currentmarker}{}%
\end{pgfscope}%
\end{pgfscope}%
\begin{pgfscope}%
\definecolor{textcolor}{rgb}{0.000000,0.000000,0.000000}%
\pgfsetstrokecolor{textcolor}%
\pgfsetfillcolor{textcolor}%
\pgftext[x=0.100000in, y=0.697943in, left, base]{\color{textcolor}{\rmfamily\fontsize{10.000000}{12.000000}\selectfont\catcode`\^=\active\def^{\ifmmode\sp\else\^{}\fi}\catcode`\%=\active\def%{\%}$\ensuremath{-}750$\,m\unit{\volt}}}%
\end{pgfscope}%
\begin{pgfscope}%
\pgfsetbuttcap%
\pgfsetroundjoin%
\definecolor{currentfill}{rgb}{0.000000,0.000000,0.000000}%
\pgfsetfillcolor{currentfill}%
\pgfsetlinewidth{0.803000pt}%
\definecolor{currentstroke}{rgb}{0.000000,0.000000,0.000000}%
\pgfsetstrokecolor{currentstroke}%
\pgfsetdash{}{0pt}%
\pgfsys@defobject{currentmarker}{\pgfqpoint{0.000000in}{0.000000in}}{\pgfqpoint{0.048611in}{0.000000in}}{%
\pgfpathmoveto{\pgfqpoint{0.000000in}{0.000000in}}%
\pgfpathlineto{\pgfqpoint{0.048611in}{0.000000in}}%
\pgfusepath{stroke,fill}%
}%
\begin{pgfscope}%
\pgfsys@transformshift{0.690792in}{1.082507in}%
\pgfsys@useobject{currentmarker}{}%
\end{pgfscope}%
\end{pgfscope}%
\begin{pgfscope}%
\pgfsetbuttcap%
\pgfsetroundjoin%
\definecolor{currentfill}{rgb}{0.000000,0.000000,0.000000}%
\pgfsetfillcolor{currentfill}%
\pgfsetlinewidth{0.803000pt}%
\definecolor{currentstroke}{rgb}{0.000000,0.000000,0.000000}%
\pgfsetstrokecolor{currentstroke}%
\pgfsetdash{}{0pt}%
\pgfsys@defobject{currentmarker}{\pgfqpoint{-0.048611in}{0.000000in}}{\pgfqpoint{-0.000000in}{0.000000in}}{%
\pgfpathmoveto{\pgfqpoint{-0.000000in}{0.000000in}}%
\pgfpathlineto{\pgfqpoint{-0.048611in}{0.000000in}}%
\pgfusepath{stroke,fill}%
}%
\begin{pgfscope}%
\pgfsys@transformshift{5.164342in}{1.082507in}%
\pgfsys@useobject{currentmarker}{}%
\end{pgfscope}%
\end{pgfscope}%
\begin{pgfscope}%
\definecolor{textcolor}{rgb}{0.000000,0.000000,0.000000}%
\pgfsetstrokecolor{textcolor}%
\pgfsetfillcolor{textcolor}%
\pgftext[x=0.100000in, y=1.033479in, left, base]{\color{textcolor}{\rmfamily\fontsize{10.000000}{12.000000}\selectfont\catcode`\^=\active\def^{\ifmmode\sp\else\^{}\fi}\catcode`\%=\active\def%{\%}$\ensuremath{-}500$\,m\unit{\volt}}}%
\end{pgfscope}%
\begin{pgfscope}%
\pgfsetbuttcap%
\pgfsetroundjoin%
\definecolor{currentfill}{rgb}{0.000000,0.000000,0.000000}%
\pgfsetfillcolor{currentfill}%
\pgfsetlinewidth{0.803000pt}%
\definecolor{currentstroke}{rgb}{0.000000,0.000000,0.000000}%
\pgfsetstrokecolor{currentstroke}%
\pgfsetdash{}{0pt}%
\pgfsys@defobject{currentmarker}{\pgfqpoint{0.000000in}{0.000000in}}{\pgfqpoint{0.048611in}{0.000000in}}{%
\pgfpathmoveto{\pgfqpoint{0.000000in}{0.000000in}}%
\pgfpathlineto{\pgfqpoint{0.048611in}{0.000000in}}%
\pgfusepath{stroke,fill}%
}%
\begin{pgfscope}%
\pgfsys@transformshift{0.690792in}{1.418043in}%
\pgfsys@useobject{currentmarker}{}%
\end{pgfscope}%
\end{pgfscope}%
\begin{pgfscope}%
\pgfsetbuttcap%
\pgfsetroundjoin%
\definecolor{currentfill}{rgb}{0.000000,0.000000,0.000000}%
\pgfsetfillcolor{currentfill}%
\pgfsetlinewidth{0.803000pt}%
\definecolor{currentstroke}{rgb}{0.000000,0.000000,0.000000}%
\pgfsetstrokecolor{currentstroke}%
\pgfsetdash{}{0pt}%
\pgfsys@defobject{currentmarker}{\pgfqpoint{-0.048611in}{0.000000in}}{\pgfqpoint{-0.000000in}{0.000000in}}{%
\pgfpathmoveto{\pgfqpoint{-0.000000in}{0.000000in}}%
\pgfpathlineto{\pgfqpoint{-0.048611in}{0.000000in}}%
\pgfusepath{stroke,fill}%
}%
\begin{pgfscope}%
\pgfsys@transformshift{5.164342in}{1.418043in}%
\pgfsys@useobject{currentmarker}{}%
\end{pgfscope}%
\end{pgfscope}%
\begin{pgfscope}%
\definecolor{textcolor}{rgb}{0.000000,0.000000,0.000000}%
\pgfsetstrokecolor{textcolor}%
\pgfsetfillcolor{textcolor}%
\pgftext[x=0.100000in, y=1.369015in, left, base]{\color{textcolor}{\rmfamily\fontsize{10.000000}{12.000000}\selectfont\catcode`\^=\active\def^{\ifmmode\sp\else\^{}\fi}\catcode`\%=\active\def%{\%}$\ensuremath{-}250$\,m\unit{\volt}}}%
\end{pgfscope}%
\begin{pgfscope}%
\pgfsetbuttcap%
\pgfsetroundjoin%
\definecolor{currentfill}{rgb}{0.000000,0.000000,0.000000}%
\pgfsetfillcolor{currentfill}%
\pgfsetlinewidth{0.803000pt}%
\definecolor{currentstroke}{rgb}{0.000000,0.000000,0.000000}%
\pgfsetstrokecolor{currentstroke}%
\pgfsetdash{}{0pt}%
\pgfsys@defobject{currentmarker}{\pgfqpoint{0.000000in}{0.000000in}}{\pgfqpoint{0.048611in}{0.000000in}}{%
\pgfpathmoveto{\pgfqpoint{0.000000in}{0.000000in}}%
\pgfpathlineto{\pgfqpoint{0.048611in}{0.000000in}}%
\pgfusepath{stroke,fill}%
}%
\begin{pgfscope}%
\pgfsys@transformshift{0.690792in}{1.753579in}%
\pgfsys@useobject{currentmarker}{}%
\end{pgfscope}%
\end{pgfscope}%
\begin{pgfscope}%
\pgfsetbuttcap%
\pgfsetroundjoin%
\definecolor{currentfill}{rgb}{0.000000,0.000000,0.000000}%
\pgfsetfillcolor{currentfill}%
\pgfsetlinewidth{0.803000pt}%
\definecolor{currentstroke}{rgb}{0.000000,0.000000,0.000000}%
\pgfsetstrokecolor{currentstroke}%
\pgfsetdash{}{0pt}%
\pgfsys@defobject{currentmarker}{\pgfqpoint{-0.048611in}{0.000000in}}{\pgfqpoint{-0.000000in}{0.000000in}}{%
\pgfpathmoveto{\pgfqpoint{-0.000000in}{0.000000in}}%
\pgfpathlineto{\pgfqpoint{-0.048611in}{0.000000in}}%
\pgfusepath{stroke,fill}%
}%
\begin{pgfscope}%
\pgfsys@transformshift{5.164342in}{1.753579in}%
\pgfsys@useobject{currentmarker}{}%
\end{pgfscope}%
\end{pgfscope}%
\begin{pgfscope}%
\definecolor{textcolor}{rgb}{0.000000,0.000000,0.000000}%
\pgfsetstrokecolor{textcolor}%
\pgfsetfillcolor{textcolor}%
\pgftext[x=0.455833in, y=1.704551in, left, base]{\color{textcolor}{\rmfamily\fontsize{10.000000}{12.000000}\selectfont\catcode`\^=\active\def^{\ifmmode\sp\else\^{}\fi}\catcode`\%=\active\def%{\%}$0$\,\unit{\volt}}}%
\end{pgfscope}%
\begin{pgfscope}%
\pgfsetbuttcap%
\pgfsetroundjoin%
\definecolor{currentfill}{rgb}{0.000000,0.000000,0.000000}%
\pgfsetfillcolor{currentfill}%
\pgfsetlinewidth{0.803000pt}%
\definecolor{currentstroke}{rgb}{0.000000,0.000000,0.000000}%
\pgfsetstrokecolor{currentstroke}%
\pgfsetdash{}{0pt}%
\pgfsys@defobject{currentmarker}{\pgfqpoint{0.000000in}{0.000000in}}{\pgfqpoint{0.048611in}{0.000000in}}{%
\pgfpathmoveto{\pgfqpoint{0.000000in}{0.000000in}}%
\pgfpathlineto{\pgfqpoint{0.048611in}{0.000000in}}%
\pgfusepath{stroke,fill}%
}%
\begin{pgfscope}%
\pgfsys@transformshift{0.690792in}{2.089115in}%
\pgfsys@useobject{currentmarker}{}%
\end{pgfscope}%
\end{pgfscope}%
\begin{pgfscope}%
\pgfsetbuttcap%
\pgfsetroundjoin%
\definecolor{currentfill}{rgb}{0.000000,0.000000,0.000000}%
\pgfsetfillcolor{currentfill}%
\pgfsetlinewidth{0.803000pt}%
\definecolor{currentstroke}{rgb}{0.000000,0.000000,0.000000}%
\pgfsetstrokecolor{currentstroke}%
\pgfsetdash{}{0pt}%
\pgfsys@defobject{currentmarker}{\pgfqpoint{-0.048611in}{0.000000in}}{\pgfqpoint{-0.000000in}{0.000000in}}{%
\pgfpathmoveto{\pgfqpoint{-0.000000in}{0.000000in}}%
\pgfpathlineto{\pgfqpoint{-0.048611in}{0.000000in}}%
\pgfusepath{stroke,fill}%
}%
\begin{pgfscope}%
\pgfsys@transformshift{5.164342in}{2.089115in}%
\pgfsys@useobject{currentmarker}{}%
\end{pgfscope}%
\end{pgfscope}%
\begin{pgfscope}%
\definecolor{textcolor}{rgb}{0.000000,0.000000,0.000000}%
\pgfsetstrokecolor{textcolor}%
\pgfsetfillcolor{textcolor}%
\pgftext[x=0.200000in, y=2.040087in, left, base]{\color{textcolor}{\rmfamily\fontsize{10.000000}{12.000000}\selectfont\catcode`\^=\active\def^{\ifmmode\sp\else\^{}\fi}\catcode`\%=\active\def%{\%}$250$\,m\unit{\volt}}}%
\end{pgfscope}%
\begin{pgfscope}%
\pgfsetbuttcap%
\pgfsetroundjoin%
\definecolor{currentfill}{rgb}{0.000000,0.000000,0.000000}%
\pgfsetfillcolor{currentfill}%
\pgfsetlinewidth{0.803000pt}%
\definecolor{currentstroke}{rgb}{0.000000,0.000000,0.000000}%
\pgfsetstrokecolor{currentstroke}%
\pgfsetdash{}{0pt}%
\pgfsys@defobject{currentmarker}{\pgfqpoint{0.000000in}{0.000000in}}{\pgfqpoint{0.048611in}{0.000000in}}{%
\pgfpathmoveto{\pgfqpoint{0.000000in}{0.000000in}}%
\pgfpathlineto{\pgfqpoint{0.048611in}{0.000000in}}%
\pgfusepath{stroke,fill}%
}%
\begin{pgfscope}%
\pgfsys@transformshift{0.690792in}{2.424651in}%
\pgfsys@useobject{currentmarker}{}%
\end{pgfscope}%
\end{pgfscope}%
\begin{pgfscope}%
\pgfsetbuttcap%
\pgfsetroundjoin%
\definecolor{currentfill}{rgb}{0.000000,0.000000,0.000000}%
\pgfsetfillcolor{currentfill}%
\pgfsetlinewidth{0.803000pt}%
\definecolor{currentstroke}{rgb}{0.000000,0.000000,0.000000}%
\pgfsetstrokecolor{currentstroke}%
\pgfsetdash{}{0pt}%
\pgfsys@defobject{currentmarker}{\pgfqpoint{-0.048611in}{0.000000in}}{\pgfqpoint{-0.000000in}{0.000000in}}{%
\pgfpathmoveto{\pgfqpoint{-0.000000in}{0.000000in}}%
\pgfpathlineto{\pgfqpoint{-0.048611in}{0.000000in}}%
\pgfusepath{stroke,fill}%
}%
\begin{pgfscope}%
\pgfsys@transformshift{5.164342in}{2.424651in}%
\pgfsys@useobject{currentmarker}{}%
\end{pgfscope}%
\end{pgfscope}%
\begin{pgfscope}%
\definecolor{textcolor}{rgb}{0.000000,0.000000,0.000000}%
\pgfsetstrokecolor{textcolor}%
\pgfsetfillcolor{textcolor}%
\pgftext[x=0.200000in, y=2.375623in, left, base]{\color{textcolor}{\rmfamily\fontsize{10.000000}{12.000000}\selectfont\catcode`\^=\active\def^{\ifmmode\sp\else\^{}\fi}\catcode`\%=\active\def%{\%}$500$\,m\unit{\volt}}}%
\end{pgfscope}%
\begin{pgfscope}%
\pgfsetbuttcap%
\pgfsetroundjoin%
\definecolor{currentfill}{rgb}{0.000000,0.000000,0.000000}%
\pgfsetfillcolor{currentfill}%
\pgfsetlinewidth{0.803000pt}%
\definecolor{currentstroke}{rgb}{0.000000,0.000000,0.000000}%
\pgfsetstrokecolor{currentstroke}%
\pgfsetdash{}{0pt}%
\pgfsys@defobject{currentmarker}{\pgfqpoint{0.000000in}{0.000000in}}{\pgfqpoint{0.048611in}{0.000000in}}{%
\pgfpathmoveto{\pgfqpoint{0.000000in}{0.000000in}}%
\pgfpathlineto{\pgfqpoint{0.048611in}{0.000000in}}%
\pgfusepath{stroke,fill}%
}%
\begin{pgfscope}%
\pgfsys@transformshift{0.690792in}{2.760187in}%
\pgfsys@useobject{currentmarker}{}%
\end{pgfscope}%
\end{pgfscope}%
\begin{pgfscope}%
\pgfsetbuttcap%
\pgfsetroundjoin%
\definecolor{currentfill}{rgb}{0.000000,0.000000,0.000000}%
\pgfsetfillcolor{currentfill}%
\pgfsetlinewidth{0.803000pt}%
\definecolor{currentstroke}{rgb}{0.000000,0.000000,0.000000}%
\pgfsetstrokecolor{currentstroke}%
\pgfsetdash{}{0pt}%
\pgfsys@defobject{currentmarker}{\pgfqpoint{-0.048611in}{0.000000in}}{\pgfqpoint{-0.000000in}{0.000000in}}{%
\pgfpathmoveto{\pgfqpoint{-0.000000in}{0.000000in}}%
\pgfpathlineto{\pgfqpoint{-0.048611in}{0.000000in}}%
\pgfusepath{stroke,fill}%
}%
\begin{pgfscope}%
\pgfsys@transformshift{5.164342in}{2.760187in}%
\pgfsys@useobject{currentmarker}{}%
\end{pgfscope}%
\end{pgfscope}%
\begin{pgfscope}%
\definecolor{textcolor}{rgb}{0.000000,0.000000,0.000000}%
\pgfsetstrokecolor{textcolor}%
\pgfsetfillcolor{textcolor}%
\pgftext[x=0.200000in, y=2.711159in, left, base]{\color{textcolor}{\rmfamily\fontsize{10.000000}{12.000000}\selectfont\catcode`\^=\active\def^{\ifmmode\sp\else\^{}\fi}\catcode`\%=\active\def%{\%}$750$\,m\unit{\volt}}}%
\end{pgfscope}%
\begin{pgfscope}%
\pgfsetbuttcap%
\pgfsetroundjoin%
\definecolor{currentfill}{rgb}{0.000000,0.000000,0.000000}%
\pgfsetfillcolor{currentfill}%
\pgfsetlinewidth{0.803000pt}%
\definecolor{currentstroke}{rgb}{0.000000,0.000000,0.000000}%
\pgfsetstrokecolor{currentstroke}%
\pgfsetdash{}{0pt}%
\pgfsys@defobject{currentmarker}{\pgfqpoint{0.000000in}{0.000000in}}{\pgfqpoint{0.048611in}{0.000000in}}{%
\pgfpathmoveto{\pgfqpoint{0.000000in}{0.000000in}}%
\pgfpathlineto{\pgfqpoint{0.048611in}{0.000000in}}%
\pgfusepath{stroke,fill}%
}%
\begin{pgfscope}%
\pgfsys@transformshift{0.690792in}{3.095723in}%
\pgfsys@useobject{currentmarker}{}%
\end{pgfscope}%
\end{pgfscope}%
\begin{pgfscope}%
\pgfsetbuttcap%
\pgfsetroundjoin%
\definecolor{currentfill}{rgb}{0.000000,0.000000,0.000000}%
\pgfsetfillcolor{currentfill}%
\pgfsetlinewidth{0.803000pt}%
\definecolor{currentstroke}{rgb}{0.000000,0.000000,0.000000}%
\pgfsetstrokecolor{currentstroke}%
\pgfsetdash{}{0pt}%
\pgfsys@defobject{currentmarker}{\pgfqpoint{-0.048611in}{0.000000in}}{\pgfqpoint{-0.000000in}{0.000000in}}{%
\pgfpathmoveto{\pgfqpoint{-0.000000in}{0.000000in}}%
\pgfpathlineto{\pgfqpoint{-0.048611in}{0.000000in}}%
\pgfusepath{stroke,fill}%
}%
\begin{pgfscope}%
\pgfsys@transformshift{5.164342in}{3.095723in}%
\pgfsys@useobject{currentmarker}{}%
\end{pgfscope}%
\end{pgfscope}%
\begin{pgfscope}%
\definecolor{textcolor}{rgb}{0.000000,0.000000,0.000000}%
\pgfsetstrokecolor{textcolor}%
\pgfsetfillcolor{textcolor}%
\pgftext[x=0.455833in, y=3.046695in, left, base]{\color{textcolor}{\rmfamily\fontsize{10.000000}{12.000000}\selectfont\catcode`\^=\active\def^{\ifmmode\sp\else\^{}\fi}\catcode`\%=\active\def%{\%}$1$\,\unit{\volt}}}%
\end{pgfscope}%
\begin{pgfscope}%
\pgfsetbuttcap%
\pgfsetroundjoin%
\definecolor{currentfill}{rgb}{0.000000,0.000000,0.000000}%
\pgfsetfillcolor{currentfill}%
\pgfsetlinewidth{0.602250pt}%
\definecolor{currentstroke}{rgb}{0.000000,0.000000,0.000000}%
\pgfsetstrokecolor{currentstroke}%
\pgfsetdash{}{0pt}%
\pgfsys@defobject{currentmarker}{\pgfqpoint{0.000000in}{0.000000in}}{\pgfqpoint{0.027778in}{0.000000in}}{%
\pgfpathmoveto{\pgfqpoint{0.000000in}{0.000000in}}%
\pgfpathlineto{\pgfqpoint{0.027778in}{0.000000in}}%
\pgfusepath{stroke,fill}%
}%
\begin{pgfscope}%
\pgfsys@transformshift{0.690792in}{0.344328in}%
\pgfsys@useobject{currentmarker}{}%
\end{pgfscope}%
\end{pgfscope}%
\begin{pgfscope}%
\pgfsetbuttcap%
\pgfsetroundjoin%
\definecolor{currentfill}{rgb}{0.000000,0.000000,0.000000}%
\pgfsetfillcolor{currentfill}%
\pgfsetlinewidth{0.602250pt}%
\definecolor{currentstroke}{rgb}{0.000000,0.000000,0.000000}%
\pgfsetstrokecolor{currentstroke}%
\pgfsetdash{}{0pt}%
\pgfsys@defobject{currentmarker}{\pgfqpoint{-0.027778in}{0.000000in}}{\pgfqpoint{-0.000000in}{0.000000in}}{%
\pgfpathmoveto{\pgfqpoint{-0.000000in}{0.000000in}}%
\pgfpathlineto{\pgfqpoint{-0.027778in}{0.000000in}}%
\pgfusepath{stroke,fill}%
}%
\begin{pgfscope}%
\pgfsys@transformshift{5.164342in}{0.344328in}%
\pgfsys@useobject{currentmarker}{}%
\end{pgfscope}%
\end{pgfscope}%
\begin{pgfscope}%
\pgfsetbuttcap%
\pgfsetroundjoin%
\definecolor{currentfill}{rgb}{0.000000,0.000000,0.000000}%
\pgfsetfillcolor{currentfill}%
\pgfsetlinewidth{0.602250pt}%
\definecolor{currentstroke}{rgb}{0.000000,0.000000,0.000000}%
\pgfsetstrokecolor{currentstroke}%
\pgfsetdash{}{0pt}%
\pgfsys@defobject{currentmarker}{\pgfqpoint{0.000000in}{0.000000in}}{\pgfqpoint{0.027778in}{0.000000in}}{%
\pgfpathmoveto{\pgfqpoint{0.000000in}{0.000000in}}%
\pgfpathlineto{\pgfqpoint{0.027778in}{0.000000in}}%
\pgfusepath{stroke,fill}%
}%
\begin{pgfscope}%
\pgfsys@transformshift{0.690792in}{0.478542in}%
\pgfsys@useobject{currentmarker}{}%
\end{pgfscope}%
\end{pgfscope}%
\begin{pgfscope}%
\pgfsetbuttcap%
\pgfsetroundjoin%
\definecolor{currentfill}{rgb}{0.000000,0.000000,0.000000}%
\pgfsetfillcolor{currentfill}%
\pgfsetlinewidth{0.602250pt}%
\definecolor{currentstroke}{rgb}{0.000000,0.000000,0.000000}%
\pgfsetstrokecolor{currentstroke}%
\pgfsetdash{}{0pt}%
\pgfsys@defobject{currentmarker}{\pgfqpoint{-0.027778in}{0.000000in}}{\pgfqpoint{-0.000000in}{0.000000in}}{%
\pgfpathmoveto{\pgfqpoint{-0.000000in}{0.000000in}}%
\pgfpathlineto{\pgfqpoint{-0.027778in}{0.000000in}}%
\pgfusepath{stroke,fill}%
}%
\begin{pgfscope}%
\pgfsys@transformshift{5.164342in}{0.478542in}%
\pgfsys@useobject{currentmarker}{}%
\end{pgfscope}%
\end{pgfscope}%
\begin{pgfscope}%
\pgfsetbuttcap%
\pgfsetroundjoin%
\definecolor{currentfill}{rgb}{0.000000,0.000000,0.000000}%
\pgfsetfillcolor{currentfill}%
\pgfsetlinewidth{0.602250pt}%
\definecolor{currentstroke}{rgb}{0.000000,0.000000,0.000000}%
\pgfsetstrokecolor{currentstroke}%
\pgfsetdash{}{0pt}%
\pgfsys@defobject{currentmarker}{\pgfqpoint{0.000000in}{0.000000in}}{\pgfqpoint{0.027778in}{0.000000in}}{%
\pgfpathmoveto{\pgfqpoint{0.000000in}{0.000000in}}%
\pgfpathlineto{\pgfqpoint{0.027778in}{0.000000in}}%
\pgfusepath{stroke,fill}%
}%
\begin{pgfscope}%
\pgfsys@transformshift{0.690792in}{0.545649in}%
\pgfsys@useobject{currentmarker}{}%
\end{pgfscope}%
\end{pgfscope}%
\begin{pgfscope}%
\pgfsetbuttcap%
\pgfsetroundjoin%
\definecolor{currentfill}{rgb}{0.000000,0.000000,0.000000}%
\pgfsetfillcolor{currentfill}%
\pgfsetlinewidth{0.602250pt}%
\definecolor{currentstroke}{rgb}{0.000000,0.000000,0.000000}%
\pgfsetstrokecolor{currentstroke}%
\pgfsetdash{}{0pt}%
\pgfsys@defobject{currentmarker}{\pgfqpoint{-0.027778in}{0.000000in}}{\pgfqpoint{-0.000000in}{0.000000in}}{%
\pgfpathmoveto{\pgfqpoint{-0.000000in}{0.000000in}}%
\pgfpathlineto{\pgfqpoint{-0.027778in}{0.000000in}}%
\pgfusepath{stroke,fill}%
}%
\begin{pgfscope}%
\pgfsys@transformshift{5.164342in}{0.545649in}%
\pgfsys@useobject{currentmarker}{}%
\end{pgfscope}%
\end{pgfscope}%
\begin{pgfscope}%
\pgfsetbuttcap%
\pgfsetroundjoin%
\definecolor{currentfill}{rgb}{0.000000,0.000000,0.000000}%
\pgfsetfillcolor{currentfill}%
\pgfsetlinewidth{0.602250pt}%
\definecolor{currentstroke}{rgb}{0.000000,0.000000,0.000000}%
\pgfsetstrokecolor{currentstroke}%
\pgfsetdash{}{0pt}%
\pgfsys@defobject{currentmarker}{\pgfqpoint{0.000000in}{0.000000in}}{\pgfqpoint{0.027778in}{0.000000in}}{%
\pgfpathmoveto{\pgfqpoint{0.000000in}{0.000000in}}%
\pgfpathlineto{\pgfqpoint{0.027778in}{0.000000in}}%
\pgfusepath{stroke,fill}%
}%
\begin{pgfscope}%
\pgfsys@transformshift{0.690792in}{0.612756in}%
\pgfsys@useobject{currentmarker}{}%
\end{pgfscope}%
\end{pgfscope}%
\begin{pgfscope}%
\pgfsetbuttcap%
\pgfsetroundjoin%
\definecolor{currentfill}{rgb}{0.000000,0.000000,0.000000}%
\pgfsetfillcolor{currentfill}%
\pgfsetlinewidth{0.602250pt}%
\definecolor{currentstroke}{rgb}{0.000000,0.000000,0.000000}%
\pgfsetstrokecolor{currentstroke}%
\pgfsetdash{}{0pt}%
\pgfsys@defobject{currentmarker}{\pgfqpoint{-0.027778in}{0.000000in}}{\pgfqpoint{-0.000000in}{0.000000in}}{%
\pgfpathmoveto{\pgfqpoint{-0.000000in}{0.000000in}}%
\pgfpathlineto{\pgfqpoint{-0.027778in}{0.000000in}}%
\pgfusepath{stroke,fill}%
}%
\begin{pgfscope}%
\pgfsys@transformshift{5.164342in}{0.612756in}%
\pgfsys@useobject{currentmarker}{}%
\end{pgfscope}%
\end{pgfscope}%
\begin{pgfscope}%
\pgfsetbuttcap%
\pgfsetroundjoin%
\definecolor{currentfill}{rgb}{0.000000,0.000000,0.000000}%
\pgfsetfillcolor{currentfill}%
\pgfsetlinewidth{0.602250pt}%
\definecolor{currentstroke}{rgb}{0.000000,0.000000,0.000000}%
\pgfsetstrokecolor{currentstroke}%
\pgfsetdash{}{0pt}%
\pgfsys@defobject{currentmarker}{\pgfqpoint{0.000000in}{0.000000in}}{\pgfqpoint{0.027778in}{0.000000in}}{%
\pgfpathmoveto{\pgfqpoint{0.000000in}{0.000000in}}%
\pgfpathlineto{\pgfqpoint{0.027778in}{0.000000in}}%
\pgfusepath{stroke,fill}%
}%
\begin{pgfscope}%
\pgfsys@transformshift{0.690792in}{0.679864in}%
\pgfsys@useobject{currentmarker}{}%
\end{pgfscope}%
\end{pgfscope}%
\begin{pgfscope}%
\pgfsetbuttcap%
\pgfsetroundjoin%
\definecolor{currentfill}{rgb}{0.000000,0.000000,0.000000}%
\pgfsetfillcolor{currentfill}%
\pgfsetlinewidth{0.602250pt}%
\definecolor{currentstroke}{rgb}{0.000000,0.000000,0.000000}%
\pgfsetstrokecolor{currentstroke}%
\pgfsetdash{}{0pt}%
\pgfsys@defobject{currentmarker}{\pgfqpoint{-0.027778in}{0.000000in}}{\pgfqpoint{-0.000000in}{0.000000in}}{%
\pgfpathmoveto{\pgfqpoint{-0.000000in}{0.000000in}}%
\pgfpathlineto{\pgfqpoint{-0.027778in}{0.000000in}}%
\pgfusepath{stroke,fill}%
}%
\begin{pgfscope}%
\pgfsys@transformshift{5.164342in}{0.679864in}%
\pgfsys@useobject{currentmarker}{}%
\end{pgfscope}%
\end{pgfscope}%
\begin{pgfscope}%
\pgfsetbuttcap%
\pgfsetroundjoin%
\definecolor{currentfill}{rgb}{0.000000,0.000000,0.000000}%
\pgfsetfillcolor{currentfill}%
\pgfsetlinewidth{0.602250pt}%
\definecolor{currentstroke}{rgb}{0.000000,0.000000,0.000000}%
\pgfsetstrokecolor{currentstroke}%
\pgfsetdash{}{0pt}%
\pgfsys@defobject{currentmarker}{\pgfqpoint{0.000000in}{0.000000in}}{\pgfqpoint{0.027778in}{0.000000in}}{%
\pgfpathmoveto{\pgfqpoint{0.000000in}{0.000000in}}%
\pgfpathlineto{\pgfqpoint{0.027778in}{0.000000in}}%
\pgfusepath{stroke,fill}%
}%
\begin{pgfscope}%
\pgfsys@transformshift{0.690792in}{0.814078in}%
\pgfsys@useobject{currentmarker}{}%
\end{pgfscope}%
\end{pgfscope}%
\begin{pgfscope}%
\pgfsetbuttcap%
\pgfsetroundjoin%
\definecolor{currentfill}{rgb}{0.000000,0.000000,0.000000}%
\pgfsetfillcolor{currentfill}%
\pgfsetlinewidth{0.602250pt}%
\definecolor{currentstroke}{rgb}{0.000000,0.000000,0.000000}%
\pgfsetstrokecolor{currentstroke}%
\pgfsetdash{}{0pt}%
\pgfsys@defobject{currentmarker}{\pgfqpoint{-0.027778in}{0.000000in}}{\pgfqpoint{-0.000000in}{0.000000in}}{%
\pgfpathmoveto{\pgfqpoint{-0.000000in}{0.000000in}}%
\pgfpathlineto{\pgfqpoint{-0.027778in}{0.000000in}}%
\pgfusepath{stroke,fill}%
}%
\begin{pgfscope}%
\pgfsys@transformshift{5.164342in}{0.814078in}%
\pgfsys@useobject{currentmarker}{}%
\end{pgfscope}%
\end{pgfscope}%
\begin{pgfscope}%
\pgfsetbuttcap%
\pgfsetroundjoin%
\definecolor{currentfill}{rgb}{0.000000,0.000000,0.000000}%
\pgfsetfillcolor{currentfill}%
\pgfsetlinewidth{0.602250pt}%
\definecolor{currentstroke}{rgb}{0.000000,0.000000,0.000000}%
\pgfsetstrokecolor{currentstroke}%
\pgfsetdash{}{0pt}%
\pgfsys@defobject{currentmarker}{\pgfqpoint{0.000000in}{0.000000in}}{\pgfqpoint{0.027778in}{0.000000in}}{%
\pgfpathmoveto{\pgfqpoint{0.000000in}{0.000000in}}%
\pgfpathlineto{\pgfqpoint{0.027778in}{0.000000in}}%
\pgfusepath{stroke,fill}%
}%
\begin{pgfscope}%
\pgfsys@transformshift{0.690792in}{0.881185in}%
\pgfsys@useobject{currentmarker}{}%
\end{pgfscope}%
\end{pgfscope}%
\begin{pgfscope}%
\pgfsetbuttcap%
\pgfsetroundjoin%
\definecolor{currentfill}{rgb}{0.000000,0.000000,0.000000}%
\pgfsetfillcolor{currentfill}%
\pgfsetlinewidth{0.602250pt}%
\definecolor{currentstroke}{rgb}{0.000000,0.000000,0.000000}%
\pgfsetstrokecolor{currentstroke}%
\pgfsetdash{}{0pt}%
\pgfsys@defobject{currentmarker}{\pgfqpoint{-0.027778in}{0.000000in}}{\pgfqpoint{-0.000000in}{0.000000in}}{%
\pgfpathmoveto{\pgfqpoint{-0.000000in}{0.000000in}}%
\pgfpathlineto{\pgfqpoint{-0.027778in}{0.000000in}}%
\pgfusepath{stroke,fill}%
}%
\begin{pgfscope}%
\pgfsys@transformshift{5.164342in}{0.881185in}%
\pgfsys@useobject{currentmarker}{}%
\end{pgfscope}%
\end{pgfscope}%
\begin{pgfscope}%
\pgfsetbuttcap%
\pgfsetroundjoin%
\definecolor{currentfill}{rgb}{0.000000,0.000000,0.000000}%
\pgfsetfillcolor{currentfill}%
\pgfsetlinewidth{0.602250pt}%
\definecolor{currentstroke}{rgb}{0.000000,0.000000,0.000000}%
\pgfsetstrokecolor{currentstroke}%
\pgfsetdash{}{0pt}%
\pgfsys@defobject{currentmarker}{\pgfqpoint{0.000000in}{0.000000in}}{\pgfqpoint{0.027778in}{0.000000in}}{%
\pgfpathmoveto{\pgfqpoint{0.000000in}{0.000000in}}%
\pgfpathlineto{\pgfqpoint{0.027778in}{0.000000in}}%
\pgfusepath{stroke,fill}%
}%
\begin{pgfscope}%
\pgfsys@transformshift{0.690792in}{0.948292in}%
\pgfsys@useobject{currentmarker}{}%
\end{pgfscope}%
\end{pgfscope}%
\begin{pgfscope}%
\pgfsetbuttcap%
\pgfsetroundjoin%
\definecolor{currentfill}{rgb}{0.000000,0.000000,0.000000}%
\pgfsetfillcolor{currentfill}%
\pgfsetlinewidth{0.602250pt}%
\definecolor{currentstroke}{rgb}{0.000000,0.000000,0.000000}%
\pgfsetstrokecolor{currentstroke}%
\pgfsetdash{}{0pt}%
\pgfsys@defobject{currentmarker}{\pgfqpoint{-0.027778in}{0.000000in}}{\pgfqpoint{-0.000000in}{0.000000in}}{%
\pgfpathmoveto{\pgfqpoint{-0.000000in}{0.000000in}}%
\pgfpathlineto{\pgfqpoint{-0.027778in}{0.000000in}}%
\pgfusepath{stroke,fill}%
}%
\begin{pgfscope}%
\pgfsys@transformshift{5.164342in}{0.948292in}%
\pgfsys@useobject{currentmarker}{}%
\end{pgfscope}%
\end{pgfscope}%
\begin{pgfscope}%
\pgfsetbuttcap%
\pgfsetroundjoin%
\definecolor{currentfill}{rgb}{0.000000,0.000000,0.000000}%
\pgfsetfillcolor{currentfill}%
\pgfsetlinewidth{0.602250pt}%
\definecolor{currentstroke}{rgb}{0.000000,0.000000,0.000000}%
\pgfsetstrokecolor{currentstroke}%
\pgfsetdash{}{0pt}%
\pgfsys@defobject{currentmarker}{\pgfqpoint{0.000000in}{0.000000in}}{\pgfqpoint{0.027778in}{0.000000in}}{%
\pgfpathmoveto{\pgfqpoint{0.000000in}{0.000000in}}%
\pgfpathlineto{\pgfqpoint{0.027778in}{0.000000in}}%
\pgfusepath{stroke,fill}%
}%
\begin{pgfscope}%
\pgfsys@transformshift{0.690792in}{1.015400in}%
\pgfsys@useobject{currentmarker}{}%
\end{pgfscope}%
\end{pgfscope}%
\begin{pgfscope}%
\pgfsetbuttcap%
\pgfsetroundjoin%
\definecolor{currentfill}{rgb}{0.000000,0.000000,0.000000}%
\pgfsetfillcolor{currentfill}%
\pgfsetlinewidth{0.602250pt}%
\definecolor{currentstroke}{rgb}{0.000000,0.000000,0.000000}%
\pgfsetstrokecolor{currentstroke}%
\pgfsetdash{}{0pt}%
\pgfsys@defobject{currentmarker}{\pgfqpoint{-0.027778in}{0.000000in}}{\pgfqpoint{-0.000000in}{0.000000in}}{%
\pgfpathmoveto{\pgfqpoint{-0.000000in}{0.000000in}}%
\pgfpathlineto{\pgfqpoint{-0.027778in}{0.000000in}}%
\pgfusepath{stroke,fill}%
}%
\begin{pgfscope}%
\pgfsys@transformshift{5.164342in}{1.015400in}%
\pgfsys@useobject{currentmarker}{}%
\end{pgfscope}%
\end{pgfscope}%
\begin{pgfscope}%
\pgfsetbuttcap%
\pgfsetroundjoin%
\definecolor{currentfill}{rgb}{0.000000,0.000000,0.000000}%
\pgfsetfillcolor{currentfill}%
\pgfsetlinewidth{0.602250pt}%
\definecolor{currentstroke}{rgb}{0.000000,0.000000,0.000000}%
\pgfsetstrokecolor{currentstroke}%
\pgfsetdash{}{0pt}%
\pgfsys@defobject{currentmarker}{\pgfqpoint{0.000000in}{0.000000in}}{\pgfqpoint{0.027778in}{0.000000in}}{%
\pgfpathmoveto{\pgfqpoint{0.000000in}{0.000000in}}%
\pgfpathlineto{\pgfqpoint{0.027778in}{0.000000in}}%
\pgfusepath{stroke,fill}%
}%
\begin{pgfscope}%
\pgfsys@transformshift{0.690792in}{1.149614in}%
\pgfsys@useobject{currentmarker}{}%
\end{pgfscope}%
\end{pgfscope}%
\begin{pgfscope}%
\pgfsetbuttcap%
\pgfsetroundjoin%
\definecolor{currentfill}{rgb}{0.000000,0.000000,0.000000}%
\pgfsetfillcolor{currentfill}%
\pgfsetlinewidth{0.602250pt}%
\definecolor{currentstroke}{rgb}{0.000000,0.000000,0.000000}%
\pgfsetstrokecolor{currentstroke}%
\pgfsetdash{}{0pt}%
\pgfsys@defobject{currentmarker}{\pgfqpoint{-0.027778in}{0.000000in}}{\pgfqpoint{-0.000000in}{0.000000in}}{%
\pgfpathmoveto{\pgfqpoint{-0.000000in}{0.000000in}}%
\pgfpathlineto{\pgfqpoint{-0.027778in}{0.000000in}}%
\pgfusepath{stroke,fill}%
}%
\begin{pgfscope}%
\pgfsys@transformshift{5.164342in}{1.149614in}%
\pgfsys@useobject{currentmarker}{}%
\end{pgfscope}%
\end{pgfscope}%
\begin{pgfscope}%
\pgfsetbuttcap%
\pgfsetroundjoin%
\definecolor{currentfill}{rgb}{0.000000,0.000000,0.000000}%
\pgfsetfillcolor{currentfill}%
\pgfsetlinewidth{0.602250pt}%
\definecolor{currentstroke}{rgb}{0.000000,0.000000,0.000000}%
\pgfsetstrokecolor{currentstroke}%
\pgfsetdash{}{0pt}%
\pgfsys@defobject{currentmarker}{\pgfqpoint{0.000000in}{0.000000in}}{\pgfqpoint{0.027778in}{0.000000in}}{%
\pgfpathmoveto{\pgfqpoint{0.000000in}{0.000000in}}%
\pgfpathlineto{\pgfqpoint{0.027778in}{0.000000in}}%
\pgfusepath{stroke,fill}%
}%
\begin{pgfscope}%
\pgfsys@transformshift{0.690792in}{1.216721in}%
\pgfsys@useobject{currentmarker}{}%
\end{pgfscope}%
\end{pgfscope}%
\begin{pgfscope}%
\pgfsetbuttcap%
\pgfsetroundjoin%
\definecolor{currentfill}{rgb}{0.000000,0.000000,0.000000}%
\pgfsetfillcolor{currentfill}%
\pgfsetlinewidth{0.602250pt}%
\definecolor{currentstroke}{rgb}{0.000000,0.000000,0.000000}%
\pgfsetstrokecolor{currentstroke}%
\pgfsetdash{}{0pt}%
\pgfsys@defobject{currentmarker}{\pgfqpoint{-0.027778in}{0.000000in}}{\pgfqpoint{-0.000000in}{0.000000in}}{%
\pgfpathmoveto{\pgfqpoint{-0.000000in}{0.000000in}}%
\pgfpathlineto{\pgfqpoint{-0.027778in}{0.000000in}}%
\pgfusepath{stroke,fill}%
}%
\begin{pgfscope}%
\pgfsys@transformshift{5.164342in}{1.216721in}%
\pgfsys@useobject{currentmarker}{}%
\end{pgfscope}%
\end{pgfscope}%
\begin{pgfscope}%
\pgfsetbuttcap%
\pgfsetroundjoin%
\definecolor{currentfill}{rgb}{0.000000,0.000000,0.000000}%
\pgfsetfillcolor{currentfill}%
\pgfsetlinewidth{0.602250pt}%
\definecolor{currentstroke}{rgb}{0.000000,0.000000,0.000000}%
\pgfsetstrokecolor{currentstroke}%
\pgfsetdash{}{0pt}%
\pgfsys@defobject{currentmarker}{\pgfqpoint{0.000000in}{0.000000in}}{\pgfqpoint{0.027778in}{0.000000in}}{%
\pgfpathmoveto{\pgfqpoint{0.000000in}{0.000000in}}%
\pgfpathlineto{\pgfqpoint{0.027778in}{0.000000in}}%
\pgfusepath{stroke,fill}%
}%
\begin{pgfscope}%
\pgfsys@transformshift{0.690792in}{1.283828in}%
\pgfsys@useobject{currentmarker}{}%
\end{pgfscope}%
\end{pgfscope}%
\begin{pgfscope}%
\pgfsetbuttcap%
\pgfsetroundjoin%
\definecolor{currentfill}{rgb}{0.000000,0.000000,0.000000}%
\pgfsetfillcolor{currentfill}%
\pgfsetlinewidth{0.602250pt}%
\definecolor{currentstroke}{rgb}{0.000000,0.000000,0.000000}%
\pgfsetstrokecolor{currentstroke}%
\pgfsetdash{}{0pt}%
\pgfsys@defobject{currentmarker}{\pgfqpoint{-0.027778in}{0.000000in}}{\pgfqpoint{-0.000000in}{0.000000in}}{%
\pgfpathmoveto{\pgfqpoint{-0.000000in}{0.000000in}}%
\pgfpathlineto{\pgfqpoint{-0.027778in}{0.000000in}}%
\pgfusepath{stroke,fill}%
}%
\begin{pgfscope}%
\pgfsys@transformshift{5.164342in}{1.283828in}%
\pgfsys@useobject{currentmarker}{}%
\end{pgfscope}%
\end{pgfscope}%
\begin{pgfscope}%
\pgfsetbuttcap%
\pgfsetroundjoin%
\definecolor{currentfill}{rgb}{0.000000,0.000000,0.000000}%
\pgfsetfillcolor{currentfill}%
\pgfsetlinewidth{0.602250pt}%
\definecolor{currentstroke}{rgb}{0.000000,0.000000,0.000000}%
\pgfsetstrokecolor{currentstroke}%
\pgfsetdash{}{0pt}%
\pgfsys@defobject{currentmarker}{\pgfqpoint{0.000000in}{0.000000in}}{\pgfqpoint{0.027778in}{0.000000in}}{%
\pgfpathmoveto{\pgfqpoint{0.000000in}{0.000000in}}%
\pgfpathlineto{\pgfqpoint{0.027778in}{0.000000in}}%
\pgfusepath{stroke,fill}%
}%
\begin{pgfscope}%
\pgfsys@transformshift{0.690792in}{1.350936in}%
\pgfsys@useobject{currentmarker}{}%
\end{pgfscope}%
\end{pgfscope}%
\begin{pgfscope}%
\pgfsetbuttcap%
\pgfsetroundjoin%
\definecolor{currentfill}{rgb}{0.000000,0.000000,0.000000}%
\pgfsetfillcolor{currentfill}%
\pgfsetlinewidth{0.602250pt}%
\definecolor{currentstroke}{rgb}{0.000000,0.000000,0.000000}%
\pgfsetstrokecolor{currentstroke}%
\pgfsetdash{}{0pt}%
\pgfsys@defobject{currentmarker}{\pgfqpoint{-0.027778in}{0.000000in}}{\pgfqpoint{-0.000000in}{0.000000in}}{%
\pgfpathmoveto{\pgfqpoint{-0.000000in}{0.000000in}}%
\pgfpathlineto{\pgfqpoint{-0.027778in}{0.000000in}}%
\pgfusepath{stroke,fill}%
}%
\begin{pgfscope}%
\pgfsys@transformshift{5.164342in}{1.350936in}%
\pgfsys@useobject{currentmarker}{}%
\end{pgfscope}%
\end{pgfscope}%
\begin{pgfscope}%
\pgfsetbuttcap%
\pgfsetroundjoin%
\definecolor{currentfill}{rgb}{0.000000,0.000000,0.000000}%
\pgfsetfillcolor{currentfill}%
\pgfsetlinewidth{0.602250pt}%
\definecolor{currentstroke}{rgb}{0.000000,0.000000,0.000000}%
\pgfsetstrokecolor{currentstroke}%
\pgfsetdash{}{0pt}%
\pgfsys@defobject{currentmarker}{\pgfqpoint{0.000000in}{0.000000in}}{\pgfqpoint{0.027778in}{0.000000in}}{%
\pgfpathmoveto{\pgfqpoint{0.000000in}{0.000000in}}%
\pgfpathlineto{\pgfqpoint{0.027778in}{0.000000in}}%
\pgfusepath{stroke,fill}%
}%
\begin{pgfscope}%
\pgfsys@transformshift{0.690792in}{1.485150in}%
\pgfsys@useobject{currentmarker}{}%
\end{pgfscope}%
\end{pgfscope}%
\begin{pgfscope}%
\pgfsetbuttcap%
\pgfsetroundjoin%
\definecolor{currentfill}{rgb}{0.000000,0.000000,0.000000}%
\pgfsetfillcolor{currentfill}%
\pgfsetlinewidth{0.602250pt}%
\definecolor{currentstroke}{rgb}{0.000000,0.000000,0.000000}%
\pgfsetstrokecolor{currentstroke}%
\pgfsetdash{}{0pt}%
\pgfsys@defobject{currentmarker}{\pgfqpoint{-0.027778in}{0.000000in}}{\pgfqpoint{-0.000000in}{0.000000in}}{%
\pgfpathmoveto{\pgfqpoint{-0.000000in}{0.000000in}}%
\pgfpathlineto{\pgfqpoint{-0.027778in}{0.000000in}}%
\pgfusepath{stroke,fill}%
}%
\begin{pgfscope}%
\pgfsys@transformshift{5.164342in}{1.485150in}%
\pgfsys@useobject{currentmarker}{}%
\end{pgfscope}%
\end{pgfscope}%
\begin{pgfscope}%
\pgfsetbuttcap%
\pgfsetroundjoin%
\definecolor{currentfill}{rgb}{0.000000,0.000000,0.000000}%
\pgfsetfillcolor{currentfill}%
\pgfsetlinewidth{0.602250pt}%
\definecolor{currentstroke}{rgb}{0.000000,0.000000,0.000000}%
\pgfsetstrokecolor{currentstroke}%
\pgfsetdash{}{0pt}%
\pgfsys@defobject{currentmarker}{\pgfqpoint{0.000000in}{0.000000in}}{\pgfqpoint{0.027778in}{0.000000in}}{%
\pgfpathmoveto{\pgfqpoint{0.000000in}{0.000000in}}%
\pgfpathlineto{\pgfqpoint{0.027778in}{0.000000in}}%
\pgfusepath{stroke,fill}%
}%
\begin{pgfscope}%
\pgfsys@transformshift{0.690792in}{1.552257in}%
\pgfsys@useobject{currentmarker}{}%
\end{pgfscope}%
\end{pgfscope}%
\begin{pgfscope}%
\pgfsetbuttcap%
\pgfsetroundjoin%
\definecolor{currentfill}{rgb}{0.000000,0.000000,0.000000}%
\pgfsetfillcolor{currentfill}%
\pgfsetlinewidth{0.602250pt}%
\definecolor{currentstroke}{rgb}{0.000000,0.000000,0.000000}%
\pgfsetstrokecolor{currentstroke}%
\pgfsetdash{}{0pt}%
\pgfsys@defobject{currentmarker}{\pgfqpoint{-0.027778in}{0.000000in}}{\pgfqpoint{-0.000000in}{0.000000in}}{%
\pgfpathmoveto{\pgfqpoint{-0.000000in}{0.000000in}}%
\pgfpathlineto{\pgfqpoint{-0.027778in}{0.000000in}}%
\pgfusepath{stroke,fill}%
}%
\begin{pgfscope}%
\pgfsys@transformshift{5.164342in}{1.552257in}%
\pgfsys@useobject{currentmarker}{}%
\end{pgfscope}%
\end{pgfscope}%
\begin{pgfscope}%
\pgfsetbuttcap%
\pgfsetroundjoin%
\definecolor{currentfill}{rgb}{0.000000,0.000000,0.000000}%
\pgfsetfillcolor{currentfill}%
\pgfsetlinewidth{0.602250pt}%
\definecolor{currentstroke}{rgb}{0.000000,0.000000,0.000000}%
\pgfsetstrokecolor{currentstroke}%
\pgfsetdash{}{0pt}%
\pgfsys@defobject{currentmarker}{\pgfqpoint{0.000000in}{0.000000in}}{\pgfqpoint{0.027778in}{0.000000in}}{%
\pgfpathmoveto{\pgfqpoint{0.000000in}{0.000000in}}%
\pgfpathlineto{\pgfqpoint{0.027778in}{0.000000in}}%
\pgfusepath{stroke,fill}%
}%
\begin{pgfscope}%
\pgfsys@transformshift{0.690792in}{1.619364in}%
\pgfsys@useobject{currentmarker}{}%
\end{pgfscope}%
\end{pgfscope}%
\begin{pgfscope}%
\pgfsetbuttcap%
\pgfsetroundjoin%
\definecolor{currentfill}{rgb}{0.000000,0.000000,0.000000}%
\pgfsetfillcolor{currentfill}%
\pgfsetlinewidth{0.602250pt}%
\definecolor{currentstroke}{rgb}{0.000000,0.000000,0.000000}%
\pgfsetstrokecolor{currentstroke}%
\pgfsetdash{}{0pt}%
\pgfsys@defobject{currentmarker}{\pgfqpoint{-0.027778in}{0.000000in}}{\pgfqpoint{-0.000000in}{0.000000in}}{%
\pgfpathmoveto{\pgfqpoint{-0.000000in}{0.000000in}}%
\pgfpathlineto{\pgfqpoint{-0.027778in}{0.000000in}}%
\pgfusepath{stroke,fill}%
}%
\begin{pgfscope}%
\pgfsys@transformshift{5.164342in}{1.619364in}%
\pgfsys@useobject{currentmarker}{}%
\end{pgfscope}%
\end{pgfscope}%
\begin{pgfscope}%
\pgfsetbuttcap%
\pgfsetroundjoin%
\definecolor{currentfill}{rgb}{0.000000,0.000000,0.000000}%
\pgfsetfillcolor{currentfill}%
\pgfsetlinewidth{0.602250pt}%
\definecolor{currentstroke}{rgb}{0.000000,0.000000,0.000000}%
\pgfsetstrokecolor{currentstroke}%
\pgfsetdash{}{0pt}%
\pgfsys@defobject{currentmarker}{\pgfqpoint{0.000000in}{0.000000in}}{\pgfqpoint{0.027778in}{0.000000in}}{%
\pgfpathmoveto{\pgfqpoint{0.000000in}{0.000000in}}%
\pgfpathlineto{\pgfqpoint{0.027778in}{0.000000in}}%
\pgfusepath{stroke,fill}%
}%
\begin{pgfscope}%
\pgfsys@transformshift{0.690792in}{1.686472in}%
\pgfsys@useobject{currentmarker}{}%
\end{pgfscope}%
\end{pgfscope}%
\begin{pgfscope}%
\pgfsetbuttcap%
\pgfsetroundjoin%
\definecolor{currentfill}{rgb}{0.000000,0.000000,0.000000}%
\pgfsetfillcolor{currentfill}%
\pgfsetlinewidth{0.602250pt}%
\definecolor{currentstroke}{rgb}{0.000000,0.000000,0.000000}%
\pgfsetstrokecolor{currentstroke}%
\pgfsetdash{}{0pt}%
\pgfsys@defobject{currentmarker}{\pgfqpoint{-0.027778in}{0.000000in}}{\pgfqpoint{-0.000000in}{0.000000in}}{%
\pgfpathmoveto{\pgfqpoint{-0.000000in}{0.000000in}}%
\pgfpathlineto{\pgfqpoint{-0.027778in}{0.000000in}}%
\pgfusepath{stroke,fill}%
}%
\begin{pgfscope}%
\pgfsys@transformshift{5.164342in}{1.686472in}%
\pgfsys@useobject{currentmarker}{}%
\end{pgfscope}%
\end{pgfscope}%
\begin{pgfscope}%
\pgfsetbuttcap%
\pgfsetroundjoin%
\definecolor{currentfill}{rgb}{0.000000,0.000000,0.000000}%
\pgfsetfillcolor{currentfill}%
\pgfsetlinewidth{0.602250pt}%
\definecolor{currentstroke}{rgb}{0.000000,0.000000,0.000000}%
\pgfsetstrokecolor{currentstroke}%
\pgfsetdash{}{0pt}%
\pgfsys@defobject{currentmarker}{\pgfqpoint{0.000000in}{0.000000in}}{\pgfqpoint{0.027778in}{0.000000in}}{%
\pgfpathmoveto{\pgfqpoint{0.000000in}{0.000000in}}%
\pgfpathlineto{\pgfqpoint{0.027778in}{0.000000in}}%
\pgfusepath{stroke,fill}%
}%
\begin{pgfscope}%
\pgfsys@transformshift{0.690792in}{1.820686in}%
\pgfsys@useobject{currentmarker}{}%
\end{pgfscope}%
\end{pgfscope}%
\begin{pgfscope}%
\pgfsetbuttcap%
\pgfsetroundjoin%
\definecolor{currentfill}{rgb}{0.000000,0.000000,0.000000}%
\pgfsetfillcolor{currentfill}%
\pgfsetlinewidth{0.602250pt}%
\definecolor{currentstroke}{rgb}{0.000000,0.000000,0.000000}%
\pgfsetstrokecolor{currentstroke}%
\pgfsetdash{}{0pt}%
\pgfsys@defobject{currentmarker}{\pgfqpoint{-0.027778in}{0.000000in}}{\pgfqpoint{-0.000000in}{0.000000in}}{%
\pgfpathmoveto{\pgfqpoint{-0.000000in}{0.000000in}}%
\pgfpathlineto{\pgfqpoint{-0.027778in}{0.000000in}}%
\pgfusepath{stroke,fill}%
}%
\begin{pgfscope}%
\pgfsys@transformshift{5.164342in}{1.820686in}%
\pgfsys@useobject{currentmarker}{}%
\end{pgfscope}%
\end{pgfscope}%
\begin{pgfscope}%
\pgfsetbuttcap%
\pgfsetroundjoin%
\definecolor{currentfill}{rgb}{0.000000,0.000000,0.000000}%
\pgfsetfillcolor{currentfill}%
\pgfsetlinewidth{0.602250pt}%
\definecolor{currentstroke}{rgb}{0.000000,0.000000,0.000000}%
\pgfsetstrokecolor{currentstroke}%
\pgfsetdash{}{0pt}%
\pgfsys@defobject{currentmarker}{\pgfqpoint{0.000000in}{0.000000in}}{\pgfqpoint{0.027778in}{0.000000in}}{%
\pgfpathmoveto{\pgfqpoint{0.000000in}{0.000000in}}%
\pgfpathlineto{\pgfqpoint{0.027778in}{0.000000in}}%
\pgfusepath{stroke,fill}%
}%
\begin{pgfscope}%
\pgfsys@transformshift{0.690792in}{1.887793in}%
\pgfsys@useobject{currentmarker}{}%
\end{pgfscope}%
\end{pgfscope}%
\begin{pgfscope}%
\pgfsetbuttcap%
\pgfsetroundjoin%
\definecolor{currentfill}{rgb}{0.000000,0.000000,0.000000}%
\pgfsetfillcolor{currentfill}%
\pgfsetlinewidth{0.602250pt}%
\definecolor{currentstroke}{rgb}{0.000000,0.000000,0.000000}%
\pgfsetstrokecolor{currentstroke}%
\pgfsetdash{}{0pt}%
\pgfsys@defobject{currentmarker}{\pgfqpoint{-0.027778in}{0.000000in}}{\pgfqpoint{-0.000000in}{0.000000in}}{%
\pgfpathmoveto{\pgfqpoint{-0.000000in}{0.000000in}}%
\pgfpathlineto{\pgfqpoint{-0.027778in}{0.000000in}}%
\pgfusepath{stroke,fill}%
}%
\begin{pgfscope}%
\pgfsys@transformshift{5.164342in}{1.887793in}%
\pgfsys@useobject{currentmarker}{}%
\end{pgfscope}%
\end{pgfscope}%
\begin{pgfscope}%
\pgfsetbuttcap%
\pgfsetroundjoin%
\definecolor{currentfill}{rgb}{0.000000,0.000000,0.000000}%
\pgfsetfillcolor{currentfill}%
\pgfsetlinewidth{0.602250pt}%
\definecolor{currentstroke}{rgb}{0.000000,0.000000,0.000000}%
\pgfsetstrokecolor{currentstroke}%
\pgfsetdash{}{0pt}%
\pgfsys@defobject{currentmarker}{\pgfqpoint{0.000000in}{0.000000in}}{\pgfqpoint{0.027778in}{0.000000in}}{%
\pgfpathmoveto{\pgfqpoint{0.000000in}{0.000000in}}%
\pgfpathlineto{\pgfqpoint{0.027778in}{0.000000in}}%
\pgfusepath{stroke,fill}%
}%
\begin{pgfscope}%
\pgfsys@transformshift{0.690792in}{1.954900in}%
\pgfsys@useobject{currentmarker}{}%
\end{pgfscope}%
\end{pgfscope}%
\begin{pgfscope}%
\pgfsetbuttcap%
\pgfsetroundjoin%
\definecolor{currentfill}{rgb}{0.000000,0.000000,0.000000}%
\pgfsetfillcolor{currentfill}%
\pgfsetlinewidth{0.602250pt}%
\definecolor{currentstroke}{rgb}{0.000000,0.000000,0.000000}%
\pgfsetstrokecolor{currentstroke}%
\pgfsetdash{}{0pt}%
\pgfsys@defobject{currentmarker}{\pgfqpoint{-0.027778in}{0.000000in}}{\pgfqpoint{-0.000000in}{0.000000in}}{%
\pgfpathmoveto{\pgfqpoint{-0.000000in}{0.000000in}}%
\pgfpathlineto{\pgfqpoint{-0.027778in}{0.000000in}}%
\pgfusepath{stroke,fill}%
}%
\begin{pgfscope}%
\pgfsys@transformshift{5.164342in}{1.954900in}%
\pgfsys@useobject{currentmarker}{}%
\end{pgfscope}%
\end{pgfscope}%
\begin{pgfscope}%
\pgfsetbuttcap%
\pgfsetroundjoin%
\definecolor{currentfill}{rgb}{0.000000,0.000000,0.000000}%
\pgfsetfillcolor{currentfill}%
\pgfsetlinewidth{0.602250pt}%
\definecolor{currentstroke}{rgb}{0.000000,0.000000,0.000000}%
\pgfsetstrokecolor{currentstroke}%
\pgfsetdash{}{0pt}%
\pgfsys@defobject{currentmarker}{\pgfqpoint{0.000000in}{0.000000in}}{\pgfqpoint{0.027778in}{0.000000in}}{%
\pgfpathmoveto{\pgfqpoint{0.000000in}{0.000000in}}%
\pgfpathlineto{\pgfqpoint{0.027778in}{0.000000in}}%
\pgfusepath{stroke,fill}%
}%
\begin{pgfscope}%
\pgfsys@transformshift{0.690792in}{2.022008in}%
\pgfsys@useobject{currentmarker}{}%
\end{pgfscope}%
\end{pgfscope}%
\begin{pgfscope}%
\pgfsetbuttcap%
\pgfsetroundjoin%
\definecolor{currentfill}{rgb}{0.000000,0.000000,0.000000}%
\pgfsetfillcolor{currentfill}%
\pgfsetlinewidth{0.602250pt}%
\definecolor{currentstroke}{rgb}{0.000000,0.000000,0.000000}%
\pgfsetstrokecolor{currentstroke}%
\pgfsetdash{}{0pt}%
\pgfsys@defobject{currentmarker}{\pgfqpoint{-0.027778in}{0.000000in}}{\pgfqpoint{-0.000000in}{0.000000in}}{%
\pgfpathmoveto{\pgfqpoint{-0.000000in}{0.000000in}}%
\pgfpathlineto{\pgfqpoint{-0.027778in}{0.000000in}}%
\pgfusepath{stroke,fill}%
}%
\begin{pgfscope}%
\pgfsys@transformshift{5.164342in}{2.022008in}%
\pgfsys@useobject{currentmarker}{}%
\end{pgfscope}%
\end{pgfscope}%
\begin{pgfscope}%
\pgfsetbuttcap%
\pgfsetroundjoin%
\definecolor{currentfill}{rgb}{0.000000,0.000000,0.000000}%
\pgfsetfillcolor{currentfill}%
\pgfsetlinewidth{0.602250pt}%
\definecolor{currentstroke}{rgb}{0.000000,0.000000,0.000000}%
\pgfsetstrokecolor{currentstroke}%
\pgfsetdash{}{0pt}%
\pgfsys@defobject{currentmarker}{\pgfqpoint{0.000000in}{0.000000in}}{\pgfqpoint{0.027778in}{0.000000in}}{%
\pgfpathmoveto{\pgfqpoint{0.000000in}{0.000000in}}%
\pgfpathlineto{\pgfqpoint{0.027778in}{0.000000in}}%
\pgfusepath{stroke,fill}%
}%
\begin{pgfscope}%
\pgfsys@transformshift{0.690792in}{2.156222in}%
\pgfsys@useobject{currentmarker}{}%
\end{pgfscope}%
\end{pgfscope}%
\begin{pgfscope}%
\pgfsetbuttcap%
\pgfsetroundjoin%
\definecolor{currentfill}{rgb}{0.000000,0.000000,0.000000}%
\pgfsetfillcolor{currentfill}%
\pgfsetlinewidth{0.602250pt}%
\definecolor{currentstroke}{rgb}{0.000000,0.000000,0.000000}%
\pgfsetstrokecolor{currentstroke}%
\pgfsetdash{}{0pt}%
\pgfsys@defobject{currentmarker}{\pgfqpoint{-0.027778in}{0.000000in}}{\pgfqpoint{-0.000000in}{0.000000in}}{%
\pgfpathmoveto{\pgfqpoint{-0.000000in}{0.000000in}}%
\pgfpathlineto{\pgfqpoint{-0.027778in}{0.000000in}}%
\pgfusepath{stroke,fill}%
}%
\begin{pgfscope}%
\pgfsys@transformshift{5.164342in}{2.156222in}%
\pgfsys@useobject{currentmarker}{}%
\end{pgfscope}%
\end{pgfscope}%
\begin{pgfscope}%
\pgfsetbuttcap%
\pgfsetroundjoin%
\definecolor{currentfill}{rgb}{0.000000,0.000000,0.000000}%
\pgfsetfillcolor{currentfill}%
\pgfsetlinewidth{0.602250pt}%
\definecolor{currentstroke}{rgb}{0.000000,0.000000,0.000000}%
\pgfsetstrokecolor{currentstroke}%
\pgfsetdash{}{0pt}%
\pgfsys@defobject{currentmarker}{\pgfqpoint{0.000000in}{0.000000in}}{\pgfqpoint{0.027778in}{0.000000in}}{%
\pgfpathmoveto{\pgfqpoint{0.000000in}{0.000000in}}%
\pgfpathlineto{\pgfqpoint{0.027778in}{0.000000in}}%
\pgfusepath{stroke,fill}%
}%
\begin{pgfscope}%
\pgfsys@transformshift{0.690792in}{2.223329in}%
\pgfsys@useobject{currentmarker}{}%
\end{pgfscope}%
\end{pgfscope}%
\begin{pgfscope}%
\pgfsetbuttcap%
\pgfsetroundjoin%
\definecolor{currentfill}{rgb}{0.000000,0.000000,0.000000}%
\pgfsetfillcolor{currentfill}%
\pgfsetlinewidth{0.602250pt}%
\definecolor{currentstroke}{rgb}{0.000000,0.000000,0.000000}%
\pgfsetstrokecolor{currentstroke}%
\pgfsetdash{}{0pt}%
\pgfsys@defobject{currentmarker}{\pgfqpoint{-0.027778in}{0.000000in}}{\pgfqpoint{-0.000000in}{0.000000in}}{%
\pgfpathmoveto{\pgfqpoint{-0.000000in}{0.000000in}}%
\pgfpathlineto{\pgfqpoint{-0.027778in}{0.000000in}}%
\pgfusepath{stroke,fill}%
}%
\begin{pgfscope}%
\pgfsys@transformshift{5.164342in}{2.223329in}%
\pgfsys@useobject{currentmarker}{}%
\end{pgfscope}%
\end{pgfscope}%
\begin{pgfscope}%
\pgfsetbuttcap%
\pgfsetroundjoin%
\definecolor{currentfill}{rgb}{0.000000,0.000000,0.000000}%
\pgfsetfillcolor{currentfill}%
\pgfsetlinewidth{0.602250pt}%
\definecolor{currentstroke}{rgb}{0.000000,0.000000,0.000000}%
\pgfsetstrokecolor{currentstroke}%
\pgfsetdash{}{0pt}%
\pgfsys@defobject{currentmarker}{\pgfqpoint{0.000000in}{0.000000in}}{\pgfqpoint{0.027778in}{0.000000in}}{%
\pgfpathmoveto{\pgfqpoint{0.000000in}{0.000000in}}%
\pgfpathlineto{\pgfqpoint{0.027778in}{0.000000in}}%
\pgfusepath{stroke,fill}%
}%
\begin{pgfscope}%
\pgfsys@transformshift{0.690792in}{2.290436in}%
\pgfsys@useobject{currentmarker}{}%
\end{pgfscope}%
\end{pgfscope}%
\begin{pgfscope}%
\pgfsetbuttcap%
\pgfsetroundjoin%
\definecolor{currentfill}{rgb}{0.000000,0.000000,0.000000}%
\pgfsetfillcolor{currentfill}%
\pgfsetlinewidth{0.602250pt}%
\definecolor{currentstroke}{rgb}{0.000000,0.000000,0.000000}%
\pgfsetstrokecolor{currentstroke}%
\pgfsetdash{}{0pt}%
\pgfsys@defobject{currentmarker}{\pgfqpoint{-0.027778in}{0.000000in}}{\pgfqpoint{-0.000000in}{0.000000in}}{%
\pgfpathmoveto{\pgfqpoint{-0.000000in}{0.000000in}}%
\pgfpathlineto{\pgfqpoint{-0.027778in}{0.000000in}}%
\pgfusepath{stroke,fill}%
}%
\begin{pgfscope}%
\pgfsys@transformshift{5.164342in}{2.290436in}%
\pgfsys@useobject{currentmarker}{}%
\end{pgfscope}%
\end{pgfscope}%
\begin{pgfscope}%
\pgfsetbuttcap%
\pgfsetroundjoin%
\definecolor{currentfill}{rgb}{0.000000,0.000000,0.000000}%
\pgfsetfillcolor{currentfill}%
\pgfsetlinewidth{0.602250pt}%
\definecolor{currentstroke}{rgb}{0.000000,0.000000,0.000000}%
\pgfsetstrokecolor{currentstroke}%
\pgfsetdash{}{0pt}%
\pgfsys@defobject{currentmarker}{\pgfqpoint{0.000000in}{0.000000in}}{\pgfqpoint{0.027778in}{0.000000in}}{%
\pgfpathmoveto{\pgfqpoint{0.000000in}{0.000000in}}%
\pgfpathlineto{\pgfqpoint{0.027778in}{0.000000in}}%
\pgfusepath{stroke,fill}%
}%
\begin{pgfscope}%
\pgfsys@transformshift{0.690792in}{2.357544in}%
\pgfsys@useobject{currentmarker}{}%
\end{pgfscope}%
\end{pgfscope}%
\begin{pgfscope}%
\pgfsetbuttcap%
\pgfsetroundjoin%
\definecolor{currentfill}{rgb}{0.000000,0.000000,0.000000}%
\pgfsetfillcolor{currentfill}%
\pgfsetlinewidth{0.602250pt}%
\definecolor{currentstroke}{rgb}{0.000000,0.000000,0.000000}%
\pgfsetstrokecolor{currentstroke}%
\pgfsetdash{}{0pt}%
\pgfsys@defobject{currentmarker}{\pgfqpoint{-0.027778in}{0.000000in}}{\pgfqpoint{-0.000000in}{0.000000in}}{%
\pgfpathmoveto{\pgfqpoint{-0.000000in}{0.000000in}}%
\pgfpathlineto{\pgfqpoint{-0.027778in}{0.000000in}}%
\pgfusepath{stroke,fill}%
}%
\begin{pgfscope}%
\pgfsys@transformshift{5.164342in}{2.357544in}%
\pgfsys@useobject{currentmarker}{}%
\end{pgfscope}%
\end{pgfscope}%
\begin{pgfscope}%
\pgfsetbuttcap%
\pgfsetroundjoin%
\definecolor{currentfill}{rgb}{0.000000,0.000000,0.000000}%
\pgfsetfillcolor{currentfill}%
\pgfsetlinewidth{0.602250pt}%
\definecolor{currentstroke}{rgb}{0.000000,0.000000,0.000000}%
\pgfsetstrokecolor{currentstroke}%
\pgfsetdash{}{0pt}%
\pgfsys@defobject{currentmarker}{\pgfqpoint{0.000000in}{0.000000in}}{\pgfqpoint{0.027778in}{0.000000in}}{%
\pgfpathmoveto{\pgfqpoint{0.000000in}{0.000000in}}%
\pgfpathlineto{\pgfqpoint{0.027778in}{0.000000in}}%
\pgfusepath{stroke,fill}%
}%
\begin{pgfscope}%
\pgfsys@transformshift{0.690792in}{2.491758in}%
\pgfsys@useobject{currentmarker}{}%
\end{pgfscope}%
\end{pgfscope}%
\begin{pgfscope}%
\pgfsetbuttcap%
\pgfsetroundjoin%
\definecolor{currentfill}{rgb}{0.000000,0.000000,0.000000}%
\pgfsetfillcolor{currentfill}%
\pgfsetlinewidth{0.602250pt}%
\definecolor{currentstroke}{rgb}{0.000000,0.000000,0.000000}%
\pgfsetstrokecolor{currentstroke}%
\pgfsetdash{}{0pt}%
\pgfsys@defobject{currentmarker}{\pgfqpoint{-0.027778in}{0.000000in}}{\pgfqpoint{-0.000000in}{0.000000in}}{%
\pgfpathmoveto{\pgfqpoint{-0.000000in}{0.000000in}}%
\pgfpathlineto{\pgfqpoint{-0.027778in}{0.000000in}}%
\pgfusepath{stroke,fill}%
}%
\begin{pgfscope}%
\pgfsys@transformshift{5.164342in}{2.491758in}%
\pgfsys@useobject{currentmarker}{}%
\end{pgfscope}%
\end{pgfscope}%
\begin{pgfscope}%
\pgfsetbuttcap%
\pgfsetroundjoin%
\definecolor{currentfill}{rgb}{0.000000,0.000000,0.000000}%
\pgfsetfillcolor{currentfill}%
\pgfsetlinewidth{0.602250pt}%
\definecolor{currentstroke}{rgb}{0.000000,0.000000,0.000000}%
\pgfsetstrokecolor{currentstroke}%
\pgfsetdash{}{0pt}%
\pgfsys@defobject{currentmarker}{\pgfqpoint{0.000000in}{0.000000in}}{\pgfqpoint{0.027778in}{0.000000in}}{%
\pgfpathmoveto{\pgfqpoint{0.000000in}{0.000000in}}%
\pgfpathlineto{\pgfqpoint{0.027778in}{0.000000in}}%
\pgfusepath{stroke,fill}%
}%
\begin{pgfscope}%
\pgfsys@transformshift{0.690792in}{2.558865in}%
\pgfsys@useobject{currentmarker}{}%
\end{pgfscope}%
\end{pgfscope}%
\begin{pgfscope}%
\pgfsetbuttcap%
\pgfsetroundjoin%
\definecolor{currentfill}{rgb}{0.000000,0.000000,0.000000}%
\pgfsetfillcolor{currentfill}%
\pgfsetlinewidth{0.602250pt}%
\definecolor{currentstroke}{rgb}{0.000000,0.000000,0.000000}%
\pgfsetstrokecolor{currentstroke}%
\pgfsetdash{}{0pt}%
\pgfsys@defobject{currentmarker}{\pgfqpoint{-0.027778in}{0.000000in}}{\pgfqpoint{-0.000000in}{0.000000in}}{%
\pgfpathmoveto{\pgfqpoint{-0.000000in}{0.000000in}}%
\pgfpathlineto{\pgfqpoint{-0.027778in}{0.000000in}}%
\pgfusepath{stroke,fill}%
}%
\begin{pgfscope}%
\pgfsys@transformshift{5.164342in}{2.558865in}%
\pgfsys@useobject{currentmarker}{}%
\end{pgfscope}%
\end{pgfscope}%
\begin{pgfscope}%
\pgfsetbuttcap%
\pgfsetroundjoin%
\definecolor{currentfill}{rgb}{0.000000,0.000000,0.000000}%
\pgfsetfillcolor{currentfill}%
\pgfsetlinewidth{0.602250pt}%
\definecolor{currentstroke}{rgb}{0.000000,0.000000,0.000000}%
\pgfsetstrokecolor{currentstroke}%
\pgfsetdash{}{0pt}%
\pgfsys@defobject{currentmarker}{\pgfqpoint{0.000000in}{0.000000in}}{\pgfqpoint{0.027778in}{0.000000in}}{%
\pgfpathmoveto{\pgfqpoint{0.000000in}{0.000000in}}%
\pgfpathlineto{\pgfqpoint{0.027778in}{0.000000in}}%
\pgfusepath{stroke,fill}%
}%
\begin{pgfscope}%
\pgfsys@transformshift{0.690792in}{2.625973in}%
\pgfsys@useobject{currentmarker}{}%
\end{pgfscope}%
\end{pgfscope}%
\begin{pgfscope}%
\pgfsetbuttcap%
\pgfsetroundjoin%
\definecolor{currentfill}{rgb}{0.000000,0.000000,0.000000}%
\pgfsetfillcolor{currentfill}%
\pgfsetlinewidth{0.602250pt}%
\definecolor{currentstroke}{rgb}{0.000000,0.000000,0.000000}%
\pgfsetstrokecolor{currentstroke}%
\pgfsetdash{}{0pt}%
\pgfsys@defobject{currentmarker}{\pgfqpoint{-0.027778in}{0.000000in}}{\pgfqpoint{-0.000000in}{0.000000in}}{%
\pgfpathmoveto{\pgfqpoint{-0.000000in}{0.000000in}}%
\pgfpathlineto{\pgfqpoint{-0.027778in}{0.000000in}}%
\pgfusepath{stroke,fill}%
}%
\begin{pgfscope}%
\pgfsys@transformshift{5.164342in}{2.625973in}%
\pgfsys@useobject{currentmarker}{}%
\end{pgfscope}%
\end{pgfscope}%
\begin{pgfscope}%
\pgfsetbuttcap%
\pgfsetroundjoin%
\definecolor{currentfill}{rgb}{0.000000,0.000000,0.000000}%
\pgfsetfillcolor{currentfill}%
\pgfsetlinewidth{0.602250pt}%
\definecolor{currentstroke}{rgb}{0.000000,0.000000,0.000000}%
\pgfsetstrokecolor{currentstroke}%
\pgfsetdash{}{0pt}%
\pgfsys@defobject{currentmarker}{\pgfqpoint{0.000000in}{0.000000in}}{\pgfqpoint{0.027778in}{0.000000in}}{%
\pgfpathmoveto{\pgfqpoint{0.000000in}{0.000000in}}%
\pgfpathlineto{\pgfqpoint{0.027778in}{0.000000in}}%
\pgfusepath{stroke,fill}%
}%
\begin{pgfscope}%
\pgfsys@transformshift{0.690792in}{2.693080in}%
\pgfsys@useobject{currentmarker}{}%
\end{pgfscope}%
\end{pgfscope}%
\begin{pgfscope}%
\pgfsetbuttcap%
\pgfsetroundjoin%
\definecolor{currentfill}{rgb}{0.000000,0.000000,0.000000}%
\pgfsetfillcolor{currentfill}%
\pgfsetlinewidth{0.602250pt}%
\definecolor{currentstroke}{rgb}{0.000000,0.000000,0.000000}%
\pgfsetstrokecolor{currentstroke}%
\pgfsetdash{}{0pt}%
\pgfsys@defobject{currentmarker}{\pgfqpoint{-0.027778in}{0.000000in}}{\pgfqpoint{-0.000000in}{0.000000in}}{%
\pgfpathmoveto{\pgfqpoint{-0.000000in}{0.000000in}}%
\pgfpathlineto{\pgfqpoint{-0.027778in}{0.000000in}}%
\pgfusepath{stroke,fill}%
}%
\begin{pgfscope}%
\pgfsys@transformshift{5.164342in}{2.693080in}%
\pgfsys@useobject{currentmarker}{}%
\end{pgfscope}%
\end{pgfscope}%
\begin{pgfscope}%
\pgfsetbuttcap%
\pgfsetroundjoin%
\definecolor{currentfill}{rgb}{0.000000,0.000000,0.000000}%
\pgfsetfillcolor{currentfill}%
\pgfsetlinewidth{0.602250pt}%
\definecolor{currentstroke}{rgb}{0.000000,0.000000,0.000000}%
\pgfsetstrokecolor{currentstroke}%
\pgfsetdash{}{0pt}%
\pgfsys@defobject{currentmarker}{\pgfqpoint{0.000000in}{0.000000in}}{\pgfqpoint{0.027778in}{0.000000in}}{%
\pgfpathmoveto{\pgfqpoint{0.000000in}{0.000000in}}%
\pgfpathlineto{\pgfqpoint{0.027778in}{0.000000in}}%
\pgfusepath{stroke,fill}%
}%
\begin{pgfscope}%
\pgfsys@transformshift{0.690792in}{2.827294in}%
\pgfsys@useobject{currentmarker}{}%
\end{pgfscope}%
\end{pgfscope}%
\begin{pgfscope}%
\pgfsetbuttcap%
\pgfsetroundjoin%
\definecolor{currentfill}{rgb}{0.000000,0.000000,0.000000}%
\pgfsetfillcolor{currentfill}%
\pgfsetlinewidth{0.602250pt}%
\definecolor{currentstroke}{rgb}{0.000000,0.000000,0.000000}%
\pgfsetstrokecolor{currentstroke}%
\pgfsetdash{}{0pt}%
\pgfsys@defobject{currentmarker}{\pgfqpoint{-0.027778in}{0.000000in}}{\pgfqpoint{-0.000000in}{0.000000in}}{%
\pgfpathmoveto{\pgfqpoint{-0.000000in}{0.000000in}}%
\pgfpathlineto{\pgfqpoint{-0.027778in}{0.000000in}}%
\pgfusepath{stroke,fill}%
}%
\begin{pgfscope}%
\pgfsys@transformshift{5.164342in}{2.827294in}%
\pgfsys@useobject{currentmarker}{}%
\end{pgfscope}%
\end{pgfscope}%
\begin{pgfscope}%
\pgfsetbuttcap%
\pgfsetroundjoin%
\definecolor{currentfill}{rgb}{0.000000,0.000000,0.000000}%
\pgfsetfillcolor{currentfill}%
\pgfsetlinewidth{0.602250pt}%
\definecolor{currentstroke}{rgb}{0.000000,0.000000,0.000000}%
\pgfsetstrokecolor{currentstroke}%
\pgfsetdash{}{0pt}%
\pgfsys@defobject{currentmarker}{\pgfqpoint{0.000000in}{0.000000in}}{\pgfqpoint{0.027778in}{0.000000in}}{%
\pgfpathmoveto{\pgfqpoint{0.000000in}{0.000000in}}%
\pgfpathlineto{\pgfqpoint{0.027778in}{0.000000in}}%
\pgfusepath{stroke,fill}%
}%
\begin{pgfscope}%
\pgfsys@transformshift{0.690792in}{2.894401in}%
\pgfsys@useobject{currentmarker}{}%
\end{pgfscope}%
\end{pgfscope}%
\begin{pgfscope}%
\pgfsetbuttcap%
\pgfsetroundjoin%
\definecolor{currentfill}{rgb}{0.000000,0.000000,0.000000}%
\pgfsetfillcolor{currentfill}%
\pgfsetlinewidth{0.602250pt}%
\definecolor{currentstroke}{rgb}{0.000000,0.000000,0.000000}%
\pgfsetstrokecolor{currentstroke}%
\pgfsetdash{}{0pt}%
\pgfsys@defobject{currentmarker}{\pgfqpoint{-0.027778in}{0.000000in}}{\pgfqpoint{-0.000000in}{0.000000in}}{%
\pgfpathmoveto{\pgfqpoint{-0.000000in}{0.000000in}}%
\pgfpathlineto{\pgfqpoint{-0.027778in}{0.000000in}}%
\pgfusepath{stroke,fill}%
}%
\begin{pgfscope}%
\pgfsys@transformshift{5.164342in}{2.894401in}%
\pgfsys@useobject{currentmarker}{}%
\end{pgfscope}%
\end{pgfscope}%
\begin{pgfscope}%
\pgfsetbuttcap%
\pgfsetroundjoin%
\definecolor{currentfill}{rgb}{0.000000,0.000000,0.000000}%
\pgfsetfillcolor{currentfill}%
\pgfsetlinewidth{0.602250pt}%
\definecolor{currentstroke}{rgb}{0.000000,0.000000,0.000000}%
\pgfsetstrokecolor{currentstroke}%
\pgfsetdash{}{0pt}%
\pgfsys@defobject{currentmarker}{\pgfqpoint{0.000000in}{0.000000in}}{\pgfqpoint{0.027778in}{0.000000in}}{%
\pgfpathmoveto{\pgfqpoint{0.000000in}{0.000000in}}%
\pgfpathlineto{\pgfqpoint{0.027778in}{0.000000in}}%
\pgfusepath{stroke,fill}%
}%
\begin{pgfscope}%
\pgfsys@transformshift{0.690792in}{2.961509in}%
\pgfsys@useobject{currentmarker}{}%
\end{pgfscope}%
\end{pgfscope}%
\begin{pgfscope}%
\pgfsetbuttcap%
\pgfsetroundjoin%
\definecolor{currentfill}{rgb}{0.000000,0.000000,0.000000}%
\pgfsetfillcolor{currentfill}%
\pgfsetlinewidth{0.602250pt}%
\definecolor{currentstroke}{rgb}{0.000000,0.000000,0.000000}%
\pgfsetstrokecolor{currentstroke}%
\pgfsetdash{}{0pt}%
\pgfsys@defobject{currentmarker}{\pgfqpoint{-0.027778in}{0.000000in}}{\pgfqpoint{-0.000000in}{0.000000in}}{%
\pgfpathmoveto{\pgfqpoint{-0.000000in}{0.000000in}}%
\pgfpathlineto{\pgfqpoint{-0.027778in}{0.000000in}}%
\pgfusepath{stroke,fill}%
}%
\begin{pgfscope}%
\pgfsys@transformshift{5.164342in}{2.961509in}%
\pgfsys@useobject{currentmarker}{}%
\end{pgfscope}%
\end{pgfscope}%
\begin{pgfscope}%
\pgfsetbuttcap%
\pgfsetroundjoin%
\definecolor{currentfill}{rgb}{0.000000,0.000000,0.000000}%
\pgfsetfillcolor{currentfill}%
\pgfsetlinewidth{0.602250pt}%
\definecolor{currentstroke}{rgb}{0.000000,0.000000,0.000000}%
\pgfsetstrokecolor{currentstroke}%
\pgfsetdash{}{0pt}%
\pgfsys@defobject{currentmarker}{\pgfqpoint{0.000000in}{0.000000in}}{\pgfqpoint{0.027778in}{0.000000in}}{%
\pgfpathmoveto{\pgfqpoint{0.000000in}{0.000000in}}%
\pgfpathlineto{\pgfqpoint{0.027778in}{0.000000in}}%
\pgfusepath{stroke,fill}%
}%
\begin{pgfscope}%
\pgfsys@transformshift{0.690792in}{3.028616in}%
\pgfsys@useobject{currentmarker}{}%
\end{pgfscope}%
\end{pgfscope}%
\begin{pgfscope}%
\pgfsetbuttcap%
\pgfsetroundjoin%
\definecolor{currentfill}{rgb}{0.000000,0.000000,0.000000}%
\pgfsetfillcolor{currentfill}%
\pgfsetlinewidth{0.602250pt}%
\definecolor{currentstroke}{rgb}{0.000000,0.000000,0.000000}%
\pgfsetstrokecolor{currentstroke}%
\pgfsetdash{}{0pt}%
\pgfsys@defobject{currentmarker}{\pgfqpoint{-0.027778in}{0.000000in}}{\pgfqpoint{-0.000000in}{0.000000in}}{%
\pgfpathmoveto{\pgfqpoint{-0.000000in}{0.000000in}}%
\pgfpathlineto{\pgfqpoint{-0.027778in}{0.000000in}}%
\pgfusepath{stroke,fill}%
}%
\begin{pgfscope}%
\pgfsys@transformshift{5.164342in}{3.028616in}%
\pgfsys@useobject{currentmarker}{}%
\end{pgfscope}%
\end{pgfscope}%
\begin{pgfscope}%
\pgfsetbuttcap%
\pgfsetroundjoin%
\definecolor{currentfill}{rgb}{0.000000,0.000000,0.000000}%
\pgfsetfillcolor{currentfill}%
\pgfsetlinewidth{0.602250pt}%
\definecolor{currentstroke}{rgb}{0.000000,0.000000,0.000000}%
\pgfsetstrokecolor{currentstroke}%
\pgfsetdash{}{0pt}%
\pgfsys@defobject{currentmarker}{\pgfqpoint{0.000000in}{0.000000in}}{\pgfqpoint{0.027778in}{0.000000in}}{%
\pgfpathmoveto{\pgfqpoint{0.000000in}{0.000000in}}%
\pgfpathlineto{\pgfqpoint{0.027778in}{0.000000in}}%
\pgfusepath{stroke,fill}%
}%
\begin{pgfscope}%
\pgfsys@transformshift{0.690792in}{3.162830in}%
\pgfsys@useobject{currentmarker}{}%
\end{pgfscope}%
\end{pgfscope}%
\begin{pgfscope}%
\pgfsetbuttcap%
\pgfsetroundjoin%
\definecolor{currentfill}{rgb}{0.000000,0.000000,0.000000}%
\pgfsetfillcolor{currentfill}%
\pgfsetlinewidth{0.602250pt}%
\definecolor{currentstroke}{rgb}{0.000000,0.000000,0.000000}%
\pgfsetstrokecolor{currentstroke}%
\pgfsetdash{}{0pt}%
\pgfsys@defobject{currentmarker}{\pgfqpoint{-0.027778in}{0.000000in}}{\pgfqpoint{-0.000000in}{0.000000in}}{%
\pgfpathmoveto{\pgfqpoint{-0.000000in}{0.000000in}}%
\pgfpathlineto{\pgfqpoint{-0.027778in}{0.000000in}}%
\pgfusepath{stroke,fill}%
}%
\begin{pgfscope}%
\pgfsys@transformshift{5.164342in}{3.162830in}%
\pgfsys@useobject{currentmarker}{}%
\end{pgfscope}%
\end{pgfscope}%
\begin{pgfscope}%
\pgfpathrectangle{\pgfqpoint{0.690792in}{0.277222in}}{\pgfqpoint{4.473550in}{2.952713in}}%
\pgfusepath{clip}%
\pgfsetbuttcap%
\pgfsetroundjoin%
\pgfsetlinewidth{1.003750pt}%
\definecolor{currentstroke}{rgb}{0.000000,0.000000,0.000000}%
\pgfsetstrokecolor{currentstroke}%
\pgfsetdash{{1.000000pt}{0.000000pt}}{0.000000pt}%
\pgfpathmoveto{\pgfqpoint{0.894135in}{1.753579in}}%
\pgfpathlineto{\pgfqpoint{0.971483in}{1.913584in}}%
\pgfpathlineto{\pgfqpoint{1.016263in}{2.005320in}}%
\pgfpathlineto{\pgfqpoint{1.052902in}{2.079501in}}%
\pgfpathlineto{\pgfqpoint{1.085469in}{2.144571in}}%
\pgfpathlineto{\pgfqpoint{1.113966in}{2.200700in}}%
\pgfpathlineto{\pgfqpoint{1.142462in}{2.255964in}}%
\pgfpathlineto{\pgfqpoint{1.166888in}{2.302561in}}%
\pgfpathlineto{\pgfqpoint{1.191313in}{2.348377in}}%
\pgfpathlineto{\pgfqpoint{1.215739in}{2.393345in}}%
\pgfpathlineto{\pgfqpoint{1.236094in}{2.430126in}}%
\pgfpathlineto{\pgfqpoint{1.256448in}{2.466238in}}%
\pgfpathlineto{\pgfqpoint{1.276803in}{2.501645in}}%
\pgfpathlineto{\pgfqpoint{1.297158in}{2.536312in}}%
\pgfpathlineto{\pgfqpoint{1.317512in}{2.570206in}}%
\pgfpathlineto{\pgfqpoint{1.337867in}{2.603292in}}%
\pgfpathlineto{\pgfqpoint{1.358222in}{2.635537in}}%
\pgfpathlineto{\pgfqpoint{1.374505in}{2.660707in}}%
\pgfpathlineto{\pgfqpoint{1.390789in}{2.685303in}}%
\pgfpathlineto{\pgfqpoint{1.407073in}{2.709309in}}%
\pgfpathlineto{\pgfqpoint{1.423357in}{2.732711in}}%
\pgfpathlineto{\pgfqpoint{1.439640in}{2.755492in}}%
\pgfpathlineto{\pgfqpoint{1.455924in}{2.777640in}}%
\pgfpathlineto{\pgfqpoint{1.472208in}{2.799139in}}%
\pgfpathlineto{\pgfqpoint{1.488492in}{2.819977in}}%
\pgfpathlineto{\pgfqpoint{1.504775in}{2.840140in}}%
\pgfpathlineto{\pgfqpoint{1.521059in}{2.859615in}}%
\pgfpathlineto{\pgfqpoint{1.537343in}{2.878390in}}%
\pgfpathlineto{\pgfqpoint{1.553627in}{2.896454in}}%
\pgfpathlineto{\pgfqpoint{1.565839in}{2.909527in}}%
\pgfpathlineto{\pgfqpoint{1.578052in}{2.922189in}}%
\pgfpathlineto{\pgfqpoint{1.590265in}{2.934435in}}%
\pgfpathlineto{\pgfqpoint{1.602478in}{2.946260in}}%
\pgfpathlineto{\pgfqpoint{1.614691in}{2.957661in}}%
\pgfpathlineto{\pgfqpoint{1.626903in}{2.968633in}}%
\pgfpathlineto{\pgfqpoint{1.639116in}{2.979173in}}%
\pgfpathlineto{\pgfqpoint{1.651329in}{2.989276in}}%
\pgfpathlineto{\pgfqpoint{1.663542in}{2.998940in}}%
\pgfpathlineto{\pgfqpoint{1.675755in}{3.008160in}}%
\pgfpathlineto{\pgfqpoint{1.687967in}{3.016933in}}%
\pgfpathlineto{\pgfqpoint{1.700180in}{3.025257in}}%
\pgfpathlineto{\pgfqpoint{1.712393in}{3.033128in}}%
\pgfpathlineto{\pgfqpoint{1.724606in}{3.040543in}}%
\pgfpathlineto{\pgfqpoint{1.736819in}{3.047500in}}%
\pgfpathlineto{\pgfqpoint{1.749031in}{3.053997in}}%
\pgfpathlineto{\pgfqpoint{1.761244in}{3.060030in}}%
\pgfpathlineto{\pgfqpoint{1.773457in}{3.065599in}}%
\pgfpathlineto{\pgfqpoint{1.785670in}{3.070700in}}%
\pgfpathlineto{\pgfqpoint{1.797883in}{3.075333in}}%
\pgfpathlineto{\pgfqpoint{1.810095in}{3.079495in}}%
\pgfpathlineto{\pgfqpoint{1.822308in}{3.083185in}}%
\pgfpathlineto{\pgfqpoint{1.834521in}{3.086401in}}%
\pgfpathlineto{\pgfqpoint{1.846734in}{3.089143in}}%
\pgfpathlineto{\pgfqpoint{1.858947in}{3.091410in}}%
\pgfpathlineto{\pgfqpoint{1.871159in}{3.093200in}}%
\pgfpathlineto{\pgfqpoint{1.883372in}{3.094514in}}%
\pgfpathlineto{\pgfqpoint{1.895585in}{3.095350in}}%
\pgfpathlineto{\pgfqpoint{1.907798in}{3.095708in}}%
\pgfpathlineto{\pgfqpoint{1.920011in}{3.095589in}}%
\pgfpathlineto{\pgfqpoint{1.932223in}{3.094991in}}%
\pgfpathlineto{\pgfqpoint{1.944436in}{3.093917in}}%
\pgfpathlineto{\pgfqpoint{1.956649in}{3.092365in}}%
\pgfpathlineto{\pgfqpoint{1.968862in}{3.090336in}}%
\pgfpathlineto{\pgfqpoint{1.981075in}{3.087832in}}%
\pgfpathlineto{\pgfqpoint{1.993287in}{3.084852in}}%
\pgfpathlineto{\pgfqpoint{2.005500in}{3.081399in}}%
\pgfpathlineto{\pgfqpoint{2.017713in}{3.077473in}}%
\pgfpathlineto{\pgfqpoint{2.029926in}{3.073075in}}%
\pgfpathlineto{\pgfqpoint{2.042139in}{3.068208in}}%
\pgfpathlineto{\pgfqpoint{2.054352in}{3.062873in}}%
\pgfpathlineto{\pgfqpoint{2.066564in}{3.057072in}}%
\pgfpathlineto{\pgfqpoint{2.078777in}{3.050806in}}%
\pgfpathlineto{\pgfqpoint{2.090990in}{3.044079in}}%
\pgfpathlineto{\pgfqpoint{2.103203in}{3.036892in}}%
\pgfpathlineto{\pgfqpoint{2.115416in}{3.029249in}}%
\pgfpathlineto{\pgfqpoint{2.127628in}{3.021151in}}%
\pgfpathlineto{\pgfqpoint{2.139841in}{3.012602in}}%
\pgfpathlineto{\pgfqpoint{2.152054in}{3.003605in}}%
\pgfpathlineto{\pgfqpoint{2.164267in}{2.994163in}}%
\pgfpathlineto{\pgfqpoint{2.176480in}{2.984279in}}%
\pgfpathlineto{\pgfqpoint{2.188692in}{2.973957in}}%
\pgfpathlineto{\pgfqpoint{2.200905in}{2.963201in}}%
\pgfpathlineto{\pgfqpoint{2.213118in}{2.952014in}}%
\pgfpathlineto{\pgfqpoint{2.225331in}{2.940400in}}%
\pgfpathlineto{\pgfqpoint{2.237544in}{2.928364in}}%
\pgfpathlineto{\pgfqpoint{2.249756in}{2.915910in}}%
\pgfpathlineto{\pgfqpoint{2.261969in}{2.903042in}}%
\pgfpathlineto{\pgfqpoint{2.278253in}{2.885248in}}%
\pgfpathlineto{\pgfqpoint{2.294537in}{2.866739in}}%
\pgfpathlineto{\pgfqpoint{2.310820in}{2.847524in}}%
\pgfpathlineto{\pgfqpoint{2.327104in}{2.827618in}}%
\pgfpathlineto{\pgfqpoint{2.343388in}{2.807032in}}%
\pgfpathlineto{\pgfqpoint{2.359672in}{2.785779in}}%
\pgfpathlineto{\pgfqpoint{2.375955in}{2.763873in}}%
\pgfpathlineto{\pgfqpoint{2.392239in}{2.741327in}}%
\pgfpathlineto{\pgfqpoint{2.408523in}{2.718157in}}%
\pgfpathlineto{\pgfqpoint{2.424807in}{2.694375in}}%
\pgfpathlineto{\pgfqpoint{2.441090in}{2.669999in}}%
\pgfpathlineto{\pgfqpoint{2.457374in}{2.645042in}}%
\pgfpathlineto{\pgfqpoint{2.473658in}{2.619521in}}%
\pgfpathlineto{\pgfqpoint{2.494012in}{2.586852in}}%
\pgfpathlineto{\pgfqpoint{2.514367in}{2.553358in}}%
\pgfpathlineto{\pgfqpoint{2.534722in}{2.519073in}}%
\pgfpathlineto{\pgfqpoint{2.555076in}{2.484032in}}%
\pgfpathlineto{\pgfqpoint{2.575431in}{2.448268in}}%
\pgfpathlineto{\pgfqpoint{2.595786in}{2.411817in}}%
\pgfpathlineto{\pgfqpoint{2.620211in}{2.367220in}}%
\pgfpathlineto{\pgfqpoint{2.644637in}{2.321750in}}%
\pgfpathlineto{\pgfqpoint{2.669063in}{2.275470in}}%
\pgfpathlineto{\pgfqpoint{2.693488in}{2.228447in}}%
\pgfpathlineto{\pgfqpoint{2.721985in}{2.172737in}}%
\pgfpathlineto{\pgfqpoint{2.750481in}{2.116215in}}%
\pgfpathlineto{\pgfqpoint{2.783049in}{2.050764in}}%
\pgfpathlineto{\pgfqpoint{2.819687in}{1.976241in}}%
\pgfpathlineto{\pgfqpoint{2.864468in}{1.884213in}}%
\pgfpathlineto{\pgfqpoint{2.929602in}{1.749358in}}%
\pgfpathlineto{\pgfqpoint{3.006950in}{1.589384in}}%
\pgfpathlineto{\pgfqpoint{3.047660in}{1.505984in}}%
\pgfpathlineto{\pgfqpoint{3.084298in}{1.431752in}}%
\pgfpathlineto{\pgfqpoint{3.116865in}{1.366626in}}%
\pgfpathlineto{\pgfqpoint{3.145362in}{1.310439in}}%
\pgfpathlineto{\pgfqpoint{3.173859in}{1.255111in}}%
\pgfpathlineto{\pgfqpoint{3.198284in}{1.208451in}}%
\pgfpathlineto{\pgfqpoint{3.222710in}{1.162568in}}%
\pgfpathlineto{\pgfqpoint{3.247135in}{1.117526in}}%
\pgfpathlineto{\pgfqpoint{3.267490in}{1.080680in}}%
\pgfpathlineto{\pgfqpoint{3.287845in}{1.044500in}}%
\pgfpathlineto{\pgfqpoint{3.308199in}{1.009021in}}%
\pgfpathlineto{\pgfqpoint{3.328554in}{0.974278in}}%
\pgfpathlineto{\pgfqpoint{3.348909in}{0.940305in}}%
\pgfpathlineto{\pgfqpoint{3.369263in}{0.907137in}}%
\pgfpathlineto{\pgfqpoint{3.389618in}{0.874806in}}%
\pgfpathlineto{\pgfqpoint{3.405902in}{0.849566in}}%
\pgfpathlineto{\pgfqpoint{3.422186in}{0.824897in}}%
\pgfpathlineto{\pgfqpoint{3.438469in}{0.800816in}}%
\pgfpathlineto{\pgfqpoint{3.454753in}{0.777339in}}%
\pgfpathlineto{\pgfqpoint{3.471037in}{0.754479in}}%
\pgfpathlineto{\pgfqpoint{3.487321in}{0.732251in}}%
\pgfpathlineto{\pgfqpoint{3.503604in}{0.710670in}}%
\pgfpathlineto{\pgfqpoint{3.519888in}{0.689749in}}%
\pgfpathlineto{\pgfqpoint{3.536172in}{0.669501in}}%
\pgfpathlineto{\pgfqpoint{3.552455in}{0.649939in}}%
\pgfpathlineto{\pgfqpoint{3.568739in}{0.631076in}}%
\pgfpathlineto{\pgfqpoint{3.585023in}{0.612923in}}%
\pgfpathlineto{\pgfqpoint{3.601307in}{0.595492in}}%
\pgfpathlineto{\pgfqpoint{3.613519in}{0.582899in}}%
\pgfpathlineto{\pgfqpoint{3.625732in}{0.570723in}}%
\pgfpathlineto{\pgfqpoint{3.637945in}{0.558968in}}%
\pgfpathlineto{\pgfqpoint{3.650158in}{0.547638in}}%
\pgfpathlineto{\pgfqpoint{3.662371in}{0.536738in}}%
\pgfpathlineto{\pgfqpoint{3.674584in}{0.526271in}}%
\pgfpathlineto{\pgfqpoint{3.686796in}{0.516240in}}%
\pgfpathlineto{\pgfqpoint{3.699009in}{0.506651in}}%
\pgfpathlineto{\pgfqpoint{3.711222in}{0.497505in}}%
\pgfpathlineto{\pgfqpoint{3.723435in}{0.488806in}}%
\pgfpathlineto{\pgfqpoint{3.735648in}{0.480558in}}%
\pgfpathlineto{\pgfqpoint{3.747860in}{0.472763in}}%
\pgfpathlineto{\pgfqpoint{3.760073in}{0.465423in}}%
\pgfpathlineto{\pgfqpoint{3.772286in}{0.458543in}}%
\pgfpathlineto{\pgfqpoint{3.784499in}{0.452123in}}%
\pgfpathlineto{\pgfqpoint{3.796712in}{0.446167in}}%
\pgfpathlineto{\pgfqpoint{3.808924in}{0.440676in}}%
\pgfpathlineto{\pgfqpoint{3.821137in}{0.435653in}}%
\pgfpathlineto{\pgfqpoint{3.833350in}{0.431099in}}%
\pgfpathlineto{\pgfqpoint{3.845563in}{0.427015in}}%
\pgfpathlineto{\pgfqpoint{3.857776in}{0.423404in}}%
\pgfpathlineto{\pgfqpoint{3.869988in}{0.420267in}}%
\pgfpathlineto{\pgfqpoint{3.882201in}{0.417604in}}%
\pgfpathlineto{\pgfqpoint{3.894414in}{0.415416in}}%
\pgfpathlineto{\pgfqpoint{3.906627in}{0.413705in}}%
\pgfpathlineto{\pgfqpoint{3.918840in}{0.412472in}}%
\pgfpathlineto{\pgfqpoint{3.931052in}{0.411715in}}%
\pgfpathlineto{\pgfqpoint{3.943265in}{0.411436in}}%
\pgfpathlineto{\pgfqpoint{3.955478in}{0.411636in}}%
\pgfpathlineto{\pgfqpoint{3.967691in}{0.412312in}}%
\pgfpathlineto{\pgfqpoint{3.979904in}{0.413467in}}%
\pgfpathlineto{\pgfqpoint{3.992116in}{0.415098in}}%
\pgfpathlineto{\pgfqpoint{4.004329in}{0.417206in}}%
\pgfpathlineto{\pgfqpoint{4.016542in}{0.419790in}}%
\pgfpathlineto{\pgfqpoint{4.028755in}{0.422848in}}%
\pgfpathlineto{\pgfqpoint{4.040968in}{0.426381in}}%
\pgfpathlineto{\pgfqpoint{4.053180in}{0.430385in}}%
\pgfpathlineto{\pgfqpoint{4.065393in}{0.434861in}}%
\pgfpathlineto{\pgfqpoint{4.077606in}{0.439806in}}%
\pgfpathlineto{\pgfqpoint{4.089819in}{0.445219in}}%
\pgfpathlineto{\pgfqpoint{4.102032in}{0.451098in}}%
\pgfpathlineto{\pgfqpoint{4.114244in}{0.457441in}}%
\pgfpathlineto{\pgfqpoint{4.126457in}{0.464245in}}%
\pgfpathlineto{\pgfqpoint{4.138670in}{0.471508in}}%
\pgfpathlineto{\pgfqpoint{4.150883in}{0.479227in}}%
\pgfpathlineto{\pgfqpoint{4.163096in}{0.487400in}}%
\pgfpathlineto{\pgfqpoint{4.175308in}{0.496024in}}%
\pgfpathlineto{\pgfqpoint{4.187521in}{0.505095in}}%
\pgfpathlineto{\pgfqpoint{4.199734in}{0.514611in}}%
\pgfpathlineto{\pgfqpoint{4.211947in}{0.524568in}}%
\pgfpathlineto{\pgfqpoint{4.224160in}{0.534963in}}%
\pgfpathlineto{\pgfqpoint{4.236372in}{0.545791in}}%
\pgfpathlineto{\pgfqpoint{4.248585in}{0.557050in}}%
\pgfpathlineto{\pgfqpoint{4.260798in}{0.568734in}}%
\pgfpathlineto{\pgfqpoint{4.273011in}{0.580840in}}%
\pgfpathlineto{\pgfqpoint{4.285224in}{0.593364in}}%
\pgfpathlineto{\pgfqpoint{4.297437in}{0.606301in}}%
\pgfpathlineto{\pgfqpoint{4.313720in}{0.624184in}}%
\pgfpathlineto{\pgfqpoint{4.330004in}{0.642783in}}%
\pgfpathlineto{\pgfqpoint{4.346288in}{0.662084in}}%
\pgfpathlineto{\pgfqpoint{4.362571in}{0.682076in}}%
\pgfpathlineto{\pgfqpoint{4.378855in}{0.702746in}}%
\pgfpathlineto{\pgfqpoint{4.395139in}{0.724082in}}%
\pgfpathlineto{\pgfqpoint{4.411423in}{0.746068in}}%
\pgfpathlineto{\pgfqpoint{4.427706in}{0.768693in}}%
\pgfpathlineto{\pgfqpoint{4.443990in}{0.791941in}}%
\pgfpathlineto{\pgfqpoint{4.460274in}{0.815797in}}%
\pgfpathlineto{\pgfqpoint{4.476558in}{0.840247in}}%
\pgfpathlineto{\pgfqpoint{4.492841in}{0.865275in}}%
\pgfpathlineto{\pgfqpoint{4.509125in}{0.890865in}}%
\pgfpathlineto{\pgfqpoint{4.529480in}{0.923619in}}%
\pgfpathlineto{\pgfqpoint{4.549834in}{0.957193in}}%
\pgfpathlineto{\pgfqpoint{4.570189in}{0.991555in}}%
\pgfpathlineto{\pgfqpoint{4.590544in}{1.026670in}}%
\pgfpathlineto{\pgfqpoint{4.610898in}{1.062504in}}%
\pgfpathlineto{\pgfqpoint{4.631253in}{1.099022in}}%
\pgfpathlineto{\pgfqpoint{4.655679in}{1.143694in}}%
\pgfpathlineto{\pgfqpoint{4.680104in}{1.189235in}}%
\pgfpathlineto{\pgfqpoint{4.704530in}{1.235579in}}%
\pgfpathlineto{\pgfqpoint{4.728956in}{1.282661in}}%
\pgfpathlineto{\pgfqpoint{4.757452in}{1.338432in}}%
\pgfpathlineto{\pgfqpoint{4.785949in}{1.395008in}}%
\pgfpathlineto{\pgfqpoint{4.818516in}{1.460511in}}%
\pgfpathlineto{\pgfqpoint{4.855155in}{1.535080in}}%
\pgfpathlineto{\pgfqpoint{4.899935in}{1.627146in}}%
\pgfpathlineto{\pgfqpoint{4.960999in}{1.753579in}}%
\pgfpathlineto{\pgfqpoint{4.960999in}{1.753579in}}%
\pgfusepath{stroke}%
\end{pgfscope}%
\begin{pgfscope}%
\pgfpathrectangle{\pgfqpoint{0.690792in}{0.277222in}}{\pgfqpoint{4.473550in}{2.952713in}}%
\pgfusepath{clip}%
\pgfsetbuttcap%
\pgfsetroundjoin%
\pgfsetlinewidth{1.003750pt}%
\definecolor{currentstroke}{rgb}{0.250980,0.250980,0.250980}%
\pgfsetstrokecolor{currentstroke}%
\pgfsetdash{{2.000000pt}{1.000000pt}}{0.000000pt}%
\pgfpathmoveto{\pgfqpoint{0.894135in}{1.753579in}}%
\pgfpathlineto{\pgfqpoint{0.942986in}{1.955404in}}%
\pgfpathlineto{\pgfqpoint{0.967412in}{2.054879in}}%
\pgfpathlineto{\pgfqpoint{0.987767in}{2.136488in}}%
\pgfpathlineto{\pgfqpoint{1.008121in}{2.216583in}}%
\pgfpathlineto{\pgfqpoint{1.024405in}{2.279356in}}%
\pgfpathlineto{\pgfqpoint{1.040689in}{2.340798in}}%
\pgfpathlineto{\pgfqpoint{1.056973in}{2.400753in}}%
\pgfpathlineto{\pgfqpoint{1.073256in}{2.459071in}}%
\pgfpathlineto{\pgfqpoint{1.085469in}{2.501645in}}%
\pgfpathlineto{\pgfqpoint{1.097682in}{2.543154in}}%
\pgfpathlineto{\pgfqpoint{1.109895in}{2.583539in}}%
\pgfpathlineto{\pgfqpoint{1.122107in}{2.622742in}}%
\pgfpathlineto{\pgfqpoint{1.134320in}{2.660707in}}%
\pgfpathlineto{\pgfqpoint{1.146533in}{2.697381in}}%
\pgfpathlineto{\pgfqpoint{1.158746in}{2.732711in}}%
\pgfpathlineto{\pgfqpoint{1.170959in}{2.766646in}}%
\pgfpathlineto{\pgfqpoint{1.183171in}{2.799139in}}%
\pgfpathlineto{\pgfqpoint{1.195384in}{2.830144in}}%
\pgfpathlineto{\pgfqpoint{1.203526in}{2.849964in}}%
\pgfpathlineto{\pgfqpoint{1.211668in}{2.869091in}}%
\pgfpathlineto{\pgfqpoint{1.219810in}{2.887512in}}%
\pgfpathlineto{\pgfqpoint{1.227952in}{2.905215in}}%
\pgfpathlineto{\pgfqpoint{1.236094in}{2.922189in}}%
\pgfpathlineto{\pgfqpoint{1.244236in}{2.938424in}}%
\pgfpathlineto{\pgfqpoint{1.252377in}{2.953908in}}%
\pgfpathlineto{\pgfqpoint{1.260519in}{2.968633in}}%
\pgfpathlineto{\pgfqpoint{1.268661in}{2.982589in}}%
\pgfpathlineto{\pgfqpoint{1.276803in}{2.995768in}}%
\pgfpathlineto{\pgfqpoint{1.284945in}{3.008160in}}%
\pgfpathlineto{\pgfqpoint{1.293087in}{3.019758in}}%
\pgfpathlineto{\pgfqpoint{1.301229in}{3.030554in}}%
\pgfpathlineto{\pgfqpoint{1.309370in}{3.040543in}}%
\pgfpathlineto{\pgfqpoint{1.317512in}{3.049717in}}%
\pgfpathlineto{\pgfqpoint{1.325654in}{3.058071in}}%
\pgfpathlineto{\pgfqpoint{1.333796in}{3.065599in}}%
\pgfpathlineto{\pgfqpoint{1.341938in}{3.072297in}}%
\pgfpathlineto{\pgfqpoint{1.350080in}{3.078160in}}%
\pgfpathlineto{\pgfqpoint{1.358222in}{3.083185in}}%
\pgfpathlineto{\pgfqpoint{1.362293in}{3.085382in}}%
\pgfpathlineto{\pgfqpoint{1.366364in}{3.087368in}}%
\pgfpathlineto{\pgfqpoint{1.370434in}{3.089143in}}%
\pgfpathlineto{\pgfqpoint{1.374505in}{3.090707in}}%
\pgfpathlineto{\pgfqpoint{1.378576in}{3.092060in}}%
\pgfpathlineto{\pgfqpoint{1.382647in}{3.093200in}}%
\pgfpathlineto{\pgfqpoint{1.386718in}{3.094129in}}%
\pgfpathlineto{\pgfqpoint{1.390789in}{3.094845in}}%
\pgfpathlineto{\pgfqpoint{1.394860in}{3.095350in}}%
\pgfpathlineto{\pgfqpoint{1.398931in}{3.095642in}}%
\pgfpathlineto{\pgfqpoint{1.403002in}{3.095721in}}%
\pgfpathlineto{\pgfqpoint{1.407073in}{3.095589in}}%
\pgfpathlineto{\pgfqpoint{1.411144in}{3.095243in}}%
\pgfpathlineto{\pgfqpoint{1.415215in}{3.094686in}}%
\pgfpathlineto{\pgfqpoint{1.419286in}{3.093917in}}%
\pgfpathlineto{\pgfqpoint{1.423357in}{3.092935in}}%
\pgfpathlineto{\pgfqpoint{1.427428in}{3.091741in}}%
\pgfpathlineto{\pgfqpoint{1.431499in}{3.090336in}}%
\pgfpathlineto{\pgfqpoint{1.435569in}{3.088719in}}%
\pgfpathlineto{\pgfqpoint{1.439640in}{3.086891in}}%
\pgfpathlineto{\pgfqpoint{1.443711in}{3.084852in}}%
\pgfpathlineto{\pgfqpoint{1.451853in}{3.080143in}}%
\pgfpathlineto{\pgfqpoint{1.459995in}{3.074593in}}%
\pgfpathlineto{\pgfqpoint{1.468137in}{3.068208in}}%
\pgfpathlineto{\pgfqpoint{1.476279in}{3.060991in}}%
\pgfpathlineto{\pgfqpoint{1.484421in}{3.052946in}}%
\pgfpathlineto{\pgfqpoint{1.492563in}{3.044079in}}%
\pgfpathlineto{\pgfqpoint{1.500704in}{3.034395in}}%
\pgfpathlineto{\pgfqpoint{1.508846in}{3.023901in}}%
\pgfpathlineto{\pgfqpoint{1.516988in}{3.012602in}}%
\pgfpathlineto{\pgfqpoint{1.525130in}{3.000507in}}%
\pgfpathlineto{\pgfqpoint{1.533272in}{2.987623in}}%
\pgfpathlineto{\pgfqpoint{1.541414in}{2.973957in}}%
\pgfpathlineto{\pgfqpoint{1.549556in}{2.959520in}}%
\pgfpathlineto{\pgfqpoint{1.557697in}{2.944319in}}%
\pgfpathlineto{\pgfqpoint{1.565839in}{2.928364in}}%
\pgfpathlineto{\pgfqpoint{1.573981in}{2.911666in}}%
\pgfpathlineto{\pgfqpoint{1.582123in}{2.894235in}}%
\pgfpathlineto{\pgfqpoint{1.590265in}{2.876082in}}%
\pgfpathlineto{\pgfqpoint{1.598407in}{2.857219in}}%
\pgfpathlineto{\pgfqpoint{1.606549in}{2.837657in}}%
\pgfpathlineto{\pgfqpoint{1.614691in}{2.817409in}}%
\pgfpathlineto{\pgfqpoint{1.626903in}{2.785779in}}%
\pgfpathlineto{\pgfqpoint{1.639116in}{2.752679in}}%
\pgfpathlineto{\pgfqpoint{1.651329in}{2.718157in}}%
\pgfpathlineto{\pgfqpoint{1.663542in}{2.682261in}}%
\pgfpathlineto{\pgfqpoint{1.675755in}{2.645042in}}%
\pgfpathlineto{\pgfqpoint{1.687967in}{2.606554in}}%
\pgfpathlineto{\pgfqpoint{1.700180in}{2.566852in}}%
\pgfpathlineto{\pgfqpoint{1.712393in}{2.525992in}}%
\pgfpathlineto{\pgfqpoint{1.724606in}{2.484032in}}%
\pgfpathlineto{\pgfqpoint{1.740890in}{2.426478in}}%
\pgfpathlineto{\pgfqpoint{1.757173in}{2.367220in}}%
\pgfpathlineto{\pgfqpoint{1.773457in}{2.306410in}}%
\pgfpathlineto{\pgfqpoint{1.789741in}{2.244200in}}%
\pgfpathlineto{\pgfqpoint{1.810095in}{2.164710in}}%
\pgfpathlineto{\pgfqpoint{1.830450in}{2.083594in}}%
\pgfpathlineto{\pgfqpoint{1.854876in}{1.984561in}}%
\pgfpathlineto{\pgfqpoint{1.883372in}{1.867401in}}%
\pgfpathlineto{\pgfqpoint{1.972933in}{1.497693in}}%
\pgfpathlineto{\pgfqpoint{1.997358in}{1.399077in}}%
\pgfpathlineto{\pgfqpoint{2.017713in}{1.318416in}}%
\pgfpathlineto{\pgfqpoint{2.038068in}{1.239475in}}%
\pgfpathlineto{\pgfqpoint{2.054352in}{1.177772in}}%
\pgfpathlineto{\pgfqpoint{2.070635in}{1.117526in}}%
\pgfpathlineto{\pgfqpoint{2.086919in}{1.058890in}}%
\pgfpathlineto{\pgfqpoint{2.099132in}{1.016059in}}%
\pgfpathlineto{\pgfqpoint{2.111345in}{0.974278in}}%
\pgfpathlineto{\pgfqpoint{2.123557in}{0.933606in}}%
\pgfpathlineto{\pgfqpoint{2.135770in}{0.894103in}}%
\pgfpathlineto{\pgfqpoint{2.147983in}{0.855823in}}%
\pgfpathlineto{\pgfqpoint{2.160196in}{0.818821in}}%
\pgfpathlineto{\pgfqpoint{2.172409in}{0.783151in}}%
\pgfpathlineto{\pgfqpoint{2.184621in}{0.748862in}}%
\pgfpathlineto{\pgfqpoint{2.196834in}{0.716004in}}%
\pgfpathlineto{\pgfqpoint{2.209047in}{0.684623in}}%
\pgfpathlineto{\pgfqpoint{2.217189in}{0.664545in}}%
\pgfpathlineto{\pgfqpoint{2.225331in}{0.645157in}}%
\pgfpathlineto{\pgfqpoint{2.233473in}{0.626470in}}%
\pgfpathlineto{\pgfqpoint{2.241615in}{0.608497in}}%
\pgfpathlineto{\pgfqpoint{2.249756in}{0.591248in}}%
\pgfpathlineto{\pgfqpoint{2.257898in}{0.574735in}}%
\pgfpathlineto{\pgfqpoint{2.266040in}{0.558968in}}%
\pgfpathlineto{\pgfqpoint{2.274182in}{0.543957in}}%
\pgfpathlineto{\pgfqpoint{2.282324in}{0.529711in}}%
\pgfpathlineto{\pgfqpoint{2.290466in}{0.516240in}}%
\pgfpathlineto{\pgfqpoint{2.298608in}{0.503552in}}%
\pgfpathlineto{\pgfqpoint{2.306749in}{0.491656in}}%
\pgfpathlineto{\pgfqpoint{2.314891in}{0.480558in}}%
\pgfpathlineto{\pgfqpoint{2.323033in}{0.470265in}}%
\pgfpathlineto{\pgfqpoint{2.331175in}{0.460785in}}%
\pgfpathlineto{\pgfqpoint{2.339317in}{0.452123in}}%
\pgfpathlineto{\pgfqpoint{2.347459in}{0.444285in}}%
\pgfpathlineto{\pgfqpoint{2.355601in}{0.437275in}}%
\pgfpathlineto{\pgfqpoint{2.363743in}{0.431099in}}%
\pgfpathlineto{\pgfqpoint{2.371884in}{0.425759in}}%
\pgfpathlineto{\pgfqpoint{2.380026in}{0.421260in}}%
\pgfpathlineto{\pgfqpoint{2.384097in}{0.419326in}}%
\pgfpathlineto{\pgfqpoint{2.388168in}{0.417604in}}%
\pgfpathlineto{\pgfqpoint{2.392239in}{0.416093in}}%
\pgfpathlineto{\pgfqpoint{2.396310in}{0.414793in}}%
\pgfpathlineto{\pgfqpoint{2.400381in}{0.413705in}}%
\pgfpathlineto{\pgfqpoint{2.404452in}{0.412830in}}%
\pgfpathlineto{\pgfqpoint{2.408523in}{0.412166in}}%
\pgfpathlineto{\pgfqpoint{2.412594in}{0.411715in}}%
\pgfpathlineto{\pgfqpoint{2.416665in}{0.411476in}}%
\pgfpathlineto{\pgfqpoint{2.420736in}{0.411450in}}%
\pgfpathlineto{\pgfqpoint{2.424807in}{0.411636in}}%
\pgfpathlineto{\pgfqpoint{2.428878in}{0.412034in}}%
\pgfpathlineto{\pgfqpoint{2.432948in}{0.412644in}}%
\pgfpathlineto{\pgfqpoint{2.437019in}{0.413467in}}%
\pgfpathlineto{\pgfqpoint{2.441090in}{0.414501in}}%
\pgfpathlineto{\pgfqpoint{2.445161in}{0.415748in}}%
\pgfpathlineto{\pgfqpoint{2.449232in}{0.417206in}}%
\pgfpathlineto{\pgfqpoint{2.453303in}{0.418876in}}%
\pgfpathlineto{\pgfqpoint{2.457374in}{0.420757in}}%
\pgfpathlineto{\pgfqpoint{2.461445in}{0.422848in}}%
\pgfpathlineto{\pgfqpoint{2.469587in}{0.427663in}}%
\pgfpathlineto{\pgfqpoint{2.477729in}{0.433317in}}%
\pgfpathlineto{\pgfqpoint{2.485871in}{0.439806in}}%
\pgfpathlineto{\pgfqpoint{2.494012in}{0.447127in}}%
\pgfpathlineto{\pgfqpoint{2.502154in}{0.455275in}}%
\pgfpathlineto{\pgfqpoint{2.510296in}{0.464245in}}%
\pgfpathlineto{\pgfqpoint{2.518438in}{0.474030in}}%
\pgfpathlineto{\pgfqpoint{2.526580in}{0.484625in}}%
\pgfpathlineto{\pgfqpoint{2.534722in}{0.496024in}}%
\pgfpathlineto{\pgfqpoint{2.542864in}{0.508218in}}%
\pgfpathlineto{\pgfqpoint{2.551006in}{0.521201in}}%
\pgfpathlineto{\pgfqpoint{2.559147in}{0.534963in}}%
\pgfpathlineto{\pgfqpoint{2.567289in}{0.549497in}}%
\pgfpathlineto{\pgfqpoint{2.575431in}{0.564792in}}%
\pgfpathlineto{\pgfqpoint{2.583573in}{0.580840in}}%
\pgfpathlineto{\pgfqpoint{2.591715in}{0.597631in}}%
\pgfpathlineto{\pgfqpoint{2.599857in}{0.615153in}}%
\pgfpathlineto{\pgfqpoint{2.607999in}{0.633395in}}%
\pgfpathlineto{\pgfqpoint{2.616140in}{0.652346in}}%
\pgfpathlineto{\pgfqpoint{2.624282in}{0.671994in}}%
\pgfpathlineto{\pgfqpoint{2.632424in}{0.692327in}}%
\pgfpathlineto{\pgfqpoint{2.644637in}{0.724082in}}%
\pgfpathlineto{\pgfqpoint{2.656850in}{0.757302in}}%
\pgfpathlineto{\pgfqpoint{2.669063in}{0.791941in}}%
\pgfpathlineto{\pgfqpoint{2.681275in}{0.827949in}}%
\pgfpathlineto{\pgfqpoint{2.693488in}{0.865275in}}%
\pgfpathlineto{\pgfqpoint{2.705701in}{0.903866in}}%
\pgfpathlineto{\pgfqpoint{2.717914in}{0.943667in}}%
\pgfpathlineto{\pgfqpoint{2.730127in}{0.984621in}}%
\pgfpathlineto{\pgfqpoint{2.742339in}{1.026670in}}%
\pgfpathlineto{\pgfqpoint{2.758623in}{1.084335in}}%
\pgfpathlineto{\pgfqpoint{2.774907in}{1.143694in}}%
\pgfpathlineto{\pgfqpoint{2.791191in}{1.204597in}}%
\pgfpathlineto{\pgfqpoint{2.807474in}{1.266889in}}%
\pgfpathlineto{\pgfqpoint{2.827829in}{1.346468in}}%
\pgfpathlineto{\pgfqpoint{2.848184in}{1.427656in}}%
\pgfpathlineto{\pgfqpoint{2.872609in}{1.526755in}}%
\pgfpathlineto{\pgfqpoint{2.901106in}{1.643963in}}%
\pgfpathlineto{\pgfqpoint{2.990666in}{2.013607in}}%
\pgfpathlineto{\pgfqpoint{3.015092in}{2.112150in}}%
\pgfpathlineto{\pgfqpoint{3.035447in}{2.192732in}}%
\pgfpathlineto{\pgfqpoint{3.055801in}{2.271579in}}%
\pgfpathlineto{\pgfqpoint{3.072085in}{2.333196in}}%
\pgfpathlineto{\pgfqpoint{3.088369in}{2.393345in}}%
\pgfpathlineto{\pgfqpoint{3.104653in}{2.451876in}}%
\pgfpathlineto{\pgfqpoint{3.116865in}{2.494622in}}%
\pgfpathlineto{\pgfqpoint{3.129078in}{2.536312in}}%
\pgfpathlineto{\pgfqpoint{3.141291in}{2.576889in}}%
\pgfpathlineto{\pgfqpoint{3.153504in}{2.616292in}}%
\pgfpathlineto{\pgfqpoint{3.165717in}{2.654468in}}%
\pgfpathlineto{\pgfqpoint{3.177929in}{2.691361in}}%
\pgfpathlineto{\pgfqpoint{3.190142in}{2.726918in}}%
\pgfpathlineto{\pgfqpoint{3.202355in}{2.761089in}}%
\pgfpathlineto{\pgfqpoint{3.214568in}{2.793826in}}%
\pgfpathlineto{\pgfqpoint{3.226781in}{2.825082in}}%
\pgfpathlineto{\pgfqpoint{3.234923in}{2.845074in}}%
\pgfpathlineto{\pgfqpoint{3.243064in}{2.864375in}}%
\pgfpathlineto{\pgfqpoint{3.251206in}{2.882973in}}%
\pgfpathlineto{\pgfqpoint{3.259348in}{2.900857in}}%
\pgfpathlineto{\pgfqpoint{3.267490in}{2.918014in}}%
\pgfpathlineto{\pgfqpoint{3.275632in}{2.934435in}}%
\pgfpathlineto{\pgfqpoint{3.283774in}{2.950108in}}%
\pgfpathlineto{\pgfqpoint{3.291916in}{2.965024in}}%
\pgfpathlineto{\pgfqpoint{3.300058in}{2.979173in}}%
\pgfpathlineto{\pgfqpoint{3.308199in}{2.992546in}}%
\pgfpathlineto{\pgfqpoint{3.316341in}{3.005136in}}%
\pgfpathlineto{\pgfqpoint{3.324483in}{3.016933in}}%
\pgfpathlineto{\pgfqpoint{3.332625in}{3.027931in}}%
\pgfpathlineto{\pgfqpoint{3.340767in}{3.038122in}}%
\pgfpathlineto{\pgfqpoint{3.348909in}{3.047500in}}%
\pgfpathlineto{\pgfqpoint{3.357051in}{3.056059in}}%
\pgfpathlineto{\pgfqpoint{3.365192in}{3.063794in}}%
\pgfpathlineto{\pgfqpoint{3.373334in}{3.070700in}}%
\pgfpathlineto{\pgfqpoint{3.381476in}{3.076772in}}%
\pgfpathlineto{\pgfqpoint{3.389618in}{3.082007in}}%
\pgfpathlineto{\pgfqpoint{3.397760in}{3.086401in}}%
\pgfpathlineto{\pgfqpoint{3.401831in}{3.088282in}}%
\pgfpathlineto{\pgfqpoint{3.405902in}{3.089952in}}%
\pgfpathlineto{\pgfqpoint{3.409973in}{3.091410in}}%
\pgfpathlineto{\pgfqpoint{3.414044in}{3.092656in}}%
\pgfpathlineto{\pgfqpoint{3.418115in}{3.093691in}}%
\pgfpathlineto{\pgfqpoint{3.422186in}{3.094514in}}%
\pgfpathlineto{\pgfqpoint{3.426256in}{3.095124in}}%
\pgfpathlineto{\pgfqpoint{3.430327in}{3.095522in}}%
\pgfpathlineto{\pgfqpoint{3.434398in}{3.095708in}}%
\pgfpathlineto{\pgfqpoint{3.438469in}{3.095681in}}%
\pgfpathlineto{\pgfqpoint{3.442540in}{3.095443in}}%
\pgfpathlineto{\pgfqpoint{3.446611in}{3.094991in}}%
\pgfpathlineto{\pgfqpoint{3.450682in}{3.094328in}}%
\pgfpathlineto{\pgfqpoint{3.454753in}{3.093452in}}%
\pgfpathlineto{\pgfqpoint{3.458824in}{3.092365in}}%
\pgfpathlineto{\pgfqpoint{3.462895in}{3.091065in}}%
\pgfpathlineto{\pgfqpoint{3.466966in}{3.089554in}}%
\pgfpathlineto{\pgfqpoint{3.471037in}{3.087832in}}%
\pgfpathlineto{\pgfqpoint{3.475108in}{3.085898in}}%
\pgfpathlineto{\pgfqpoint{3.479179in}{3.083754in}}%
\pgfpathlineto{\pgfqpoint{3.487321in}{3.078834in}}%
\pgfpathlineto{\pgfqpoint{3.495462in}{3.073075in}}%
\pgfpathlineto{\pgfqpoint{3.503604in}{3.066482in}}%
\pgfpathlineto{\pgfqpoint{3.511746in}{3.059057in}}%
\pgfpathlineto{\pgfqpoint{3.519888in}{3.050806in}}%
\pgfpathlineto{\pgfqpoint{3.528030in}{3.041734in}}%
\pgfpathlineto{\pgfqpoint{3.536172in}{3.031847in}}%
\pgfpathlineto{\pgfqpoint{3.544314in}{3.021151in}}%
\pgfpathlineto{\pgfqpoint{3.552455in}{3.009653in}}%
\pgfpathlineto{\pgfqpoint{3.560597in}{2.997360in}}%
\pgfpathlineto{\pgfqpoint{3.568739in}{2.984279in}}%
\pgfpathlineto{\pgfqpoint{3.576881in}{2.970420in}}%
\pgfpathlineto{\pgfqpoint{3.585023in}{2.955791in}}%
\pgfpathlineto{\pgfqpoint{3.593165in}{2.940400in}}%
\pgfpathlineto{\pgfqpoint{3.601307in}{2.924259in}}%
\pgfpathlineto{\pgfqpoint{3.609449in}{2.907377in}}%
\pgfpathlineto{\pgfqpoint{3.617590in}{2.889764in}}%
\pgfpathlineto{\pgfqpoint{3.625732in}{2.871432in}}%
\pgfpathlineto{\pgfqpoint{3.633874in}{2.852393in}}%
\pgfpathlineto{\pgfqpoint{3.642016in}{2.832659in}}%
\pgfpathlineto{\pgfqpoint{3.650158in}{2.812241in}}%
\pgfpathlineto{\pgfqpoint{3.662371in}{2.780363in}}%
\pgfpathlineto{\pgfqpoint{3.674584in}{2.747023in}}%
\pgfpathlineto{\pgfqpoint{3.686796in}{2.712268in}}%
\pgfpathlineto{\pgfqpoint{3.699009in}{2.676148in}}%
\pgfpathlineto{\pgfqpoint{3.711222in}{2.638714in}}%
\pgfpathlineto{\pgfqpoint{3.723435in}{2.600020in}}%
\pgfpathlineto{\pgfqpoint{3.735648in}{2.560121in}}%
\pgfpathlineto{\pgfqpoint{3.747860in}{2.519073in}}%
\pgfpathlineto{\pgfqpoint{3.760073in}{2.476936in}}%
\pgfpathlineto{\pgfqpoint{3.776357in}{2.419161in}}%
\pgfpathlineto{\pgfqpoint{3.792641in}{2.359701in}}%
\pgfpathlineto{\pgfqpoint{3.808924in}{2.298707in}}%
\pgfpathlineto{\pgfqpoint{3.825208in}{2.236333in}}%
\pgfpathlineto{\pgfqpoint{3.845563in}{2.156666in}}%
\pgfpathlineto{\pgfqpoint{3.865917in}{2.075405in}}%
\pgfpathlineto{\pgfqpoint{3.890343in}{1.976241in}}%
\pgfpathlineto{\pgfqpoint{3.922911in}{1.842149in}}%
\pgfpathlineto{\pgfqpoint{4.000258in}{1.522597in}}%
\pgfpathlineto{\pgfqpoint{4.024684in}{1.423564in}}%
\pgfpathlineto{\pgfqpoint{4.045039in}{1.342448in}}%
\pgfpathlineto{\pgfqpoint{4.065393in}{1.262958in}}%
\pgfpathlineto{\pgfqpoint{4.081677in}{1.200748in}}%
\pgfpathlineto{\pgfqpoint{4.097961in}{1.139937in}}%
\pgfpathlineto{\pgfqpoint{4.114244in}{1.080680in}}%
\pgfpathlineto{\pgfqpoint{4.130528in}{1.023126in}}%
\pgfpathlineto{\pgfqpoint{4.142741in}{0.981166in}}%
\pgfpathlineto{\pgfqpoint{4.154954in}{0.940305in}}%
\pgfpathlineto{\pgfqpoint{4.167167in}{0.900603in}}%
\pgfpathlineto{\pgfqpoint{4.179379in}{0.862116in}}%
\pgfpathlineto{\pgfqpoint{4.191592in}{0.824897in}}%
\pgfpathlineto{\pgfqpoint{4.203805in}{0.789001in}}%
\pgfpathlineto{\pgfqpoint{4.216018in}{0.754479in}}%
\pgfpathlineto{\pgfqpoint{4.228231in}{0.721379in}}%
\pgfpathlineto{\pgfqpoint{4.240443in}{0.689749in}}%
\pgfpathlineto{\pgfqpoint{4.248585in}{0.669501in}}%
\pgfpathlineto{\pgfqpoint{4.256727in}{0.649939in}}%
\pgfpathlineto{\pgfqpoint{4.264869in}{0.631076in}}%
\pgfpathlineto{\pgfqpoint{4.273011in}{0.612923in}}%
\pgfpathlineto{\pgfqpoint{4.281153in}{0.595492in}}%
\pgfpathlineto{\pgfqpoint{4.289295in}{0.578794in}}%
\pgfpathlineto{\pgfqpoint{4.297437in}{0.562839in}}%
\pgfpathlineto{\pgfqpoint{4.305578in}{0.547638in}}%
\pgfpathlineto{\pgfqpoint{4.313720in}{0.533200in}}%
\pgfpathlineto{\pgfqpoint{4.321862in}{0.519535in}}%
\pgfpathlineto{\pgfqpoint{4.330004in}{0.506651in}}%
\pgfpathlineto{\pgfqpoint{4.338146in}{0.494555in}}%
\pgfpathlineto{\pgfqpoint{4.346288in}{0.483257in}}%
\pgfpathlineto{\pgfqpoint{4.354430in}{0.472763in}}%
\pgfpathlineto{\pgfqpoint{4.362571in}{0.463079in}}%
\pgfpathlineto{\pgfqpoint{4.370713in}{0.454212in}}%
\pgfpathlineto{\pgfqpoint{4.378855in}{0.446167in}}%
\pgfpathlineto{\pgfqpoint{4.386997in}{0.438950in}}%
\pgfpathlineto{\pgfqpoint{4.395139in}{0.432564in}}%
\pgfpathlineto{\pgfqpoint{4.403281in}{0.427015in}}%
\pgfpathlineto{\pgfqpoint{4.411423in}{0.422306in}}%
\pgfpathlineto{\pgfqpoint{4.415494in}{0.420267in}}%
\pgfpathlineto{\pgfqpoint{4.419565in}{0.418438in}}%
\pgfpathlineto{\pgfqpoint{4.423635in}{0.416822in}}%
\pgfpathlineto{\pgfqpoint{4.427706in}{0.415416in}}%
\pgfpathlineto{\pgfqpoint{4.431777in}{0.414223in}}%
\pgfpathlineto{\pgfqpoint{4.435848in}{0.413241in}}%
\pgfpathlineto{\pgfqpoint{4.439919in}{0.412472in}}%
\pgfpathlineto{\pgfqpoint{4.443990in}{0.411914in}}%
\pgfpathlineto{\pgfqpoint{4.448061in}{0.411569in}}%
\pgfpathlineto{\pgfqpoint{4.452132in}{0.411436in}}%
\pgfpathlineto{\pgfqpoint{4.456203in}{0.411516in}}%
\pgfpathlineto{\pgfqpoint{4.460274in}{0.411808in}}%
\pgfpathlineto{\pgfqpoint{4.464345in}{0.412312in}}%
\pgfpathlineto{\pgfqpoint{4.468416in}{0.413029in}}%
\pgfpathlineto{\pgfqpoint{4.472487in}{0.413958in}}%
\pgfpathlineto{\pgfqpoint{4.476558in}{0.415098in}}%
\pgfpathlineto{\pgfqpoint{4.480629in}{0.416451in}}%
\pgfpathlineto{\pgfqpoint{4.484700in}{0.418014in}}%
\pgfpathlineto{\pgfqpoint{4.488770in}{0.419790in}}%
\pgfpathlineto{\pgfqpoint{4.492841in}{0.421776in}}%
\pgfpathlineto{\pgfqpoint{4.500983in}{0.426381in}}%
\pgfpathlineto{\pgfqpoint{4.509125in}{0.431825in}}%
\pgfpathlineto{\pgfqpoint{4.517267in}{0.438106in}}%
\pgfpathlineto{\pgfqpoint{4.525409in}{0.445219in}}%
\pgfpathlineto{\pgfqpoint{4.533551in}{0.453161in}}%
\pgfpathlineto{\pgfqpoint{4.541693in}{0.461926in}}%
\pgfpathlineto{\pgfqpoint{4.549834in}{0.471508in}}%
\pgfpathlineto{\pgfqpoint{4.557976in}{0.481901in}}%
\pgfpathlineto{\pgfqpoint{4.566118in}{0.493099in}}%
\pgfpathlineto{\pgfqpoint{4.574260in}{0.505095in}}%
\pgfpathlineto{\pgfqpoint{4.582402in}{0.517882in}}%
\pgfpathlineto{\pgfqpoint{4.590544in}{0.531450in}}%
\pgfpathlineto{\pgfqpoint{4.598686in}{0.545791in}}%
\pgfpathlineto{\pgfqpoint{4.606828in}{0.560897in}}%
\pgfpathlineto{\pgfqpoint{4.614969in}{0.576758in}}%
\pgfpathlineto{\pgfqpoint{4.623111in}{0.593364in}}%
\pgfpathlineto{\pgfqpoint{4.631253in}{0.610704in}}%
\pgfpathlineto{\pgfqpoint{4.639395in}{0.628767in}}%
\pgfpathlineto{\pgfqpoint{4.647537in}{0.647543in}}%
\pgfpathlineto{\pgfqpoint{4.655679in}{0.667018in}}%
\pgfpathlineto{\pgfqpoint{4.663821in}{0.687181in}}%
\pgfpathlineto{\pgfqpoint{4.676033in}{0.718686in}}%
\pgfpathlineto{\pgfqpoint{4.688246in}{0.751665in}}%
\pgfpathlineto{\pgfqpoint{4.700459in}{0.786071in}}%
\pgfpathlineto{\pgfqpoint{4.712672in}{0.821855in}}%
\pgfpathlineto{\pgfqpoint{4.724885in}{0.858965in}}%
\pgfpathlineto{\pgfqpoint{4.737097in}{0.897349in}}%
\pgfpathlineto{\pgfqpoint{4.749310in}{0.936952in}}%
\pgfpathlineto{\pgfqpoint{4.761523in}{0.977718in}}%
\pgfpathlineto{\pgfqpoint{4.773736in}{1.019589in}}%
\pgfpathlineto{\pgfqpoint{4.785949in}{1.062504in}}%
\pgfpathlineto{\pgfqpoint{4.802232in}{1.121246in}}%
\pgfpathlineto{\pgfqpoint{4.818516in}{1.181587in}}%
\pgfpathlineto{\pgfqpoint{4.834800in}{1.243376in}}%
\pgfpathlineto{\pgfqpoint{4.851084in}{1.306457in}}%
\pgfpathlineto{\pgfqpoint{4.871438in}{1.386881in}}%
\pgfpathlineto{\pgfqpoint{4.891793in}{1.468754in}}%
\pgfpathlineto{\pgfqpoint{4.916219in}{1.568461in}}%
\pgfpathlineto{\pgfqpoint{4.952857in}{1.719817in}}%
\pgfpathlineto{\pgfqpoint{4.960999in}{1.753579in}}%
\pgfpathlineto{\pgfqpoint{4.960999in}{1.753579in}}%
\pgfusepath{stroke}%
\end{pgfscope}%
\begin{pgfscope}%
\pgfpathrectangle{\pgfqpoint{0.690792in}{0.277222in}}{\pgfqpoint{4.473550in}{2.952713in}}%
\pgfusepath{clip}%
\pgfsetbuttcap%
\pgfsetroundjoin%
\pgfsetlinewidth{1.003750pt}%
\definecolor{currentstroke}{rgb}{0.501961,0.501961,0.501961}%
\pgfsetstrokecolor{currentstroke}%
\pgfsetdash{{3.000000pt}{2.000000pt}}{0.000000pt}%
\pgfpathmoveto{\pgfqpoint{0.894135in}{1.753579in}}%
\pgfpathlineto{\pgfqpoint{0.930774in}{1.980402in}}%
\pgfpathlineto{\pgfqpoint{0.951128in}{2.104008in}}%
\pgfpathlineto{\pgfqpoint{0.967412in}{2.200700in}}%
\pgfpathlineto{\pgfqpoint{0.983696in}{2.294847in}}%
\pgfpathlineto{\pgfqpoint{0.995908in}{2.363464in}}%
\pgfpathlineto{\pgfqpoint{1.008121in}{2.430126in}}%
\pgfpathlineto{\pgfqpoint{1.020334in}{2.494622in}}%
\pgfpathlineto{\pgfqpoint{1.032547in}{2.556743in}}%
\pgfpathlineto{\pgfqpoint{1.044760in}{2.616292in}}%
\pgfpathlineto{\pgfqpoint{1.052902in}{2.654468in}}%
\pgfpathlineto{\pgfqpoint{1.061043in}{2.691361in}}%
\pgfpathlineto{\pgfqpoint{1.069185in}{2.726918in}}%
\pgfpathlineto{\pgfqpoint{1.077327in}{2.761089in}}%
\pgfpathlineto{\pgfqpoint{1.085469in}{2.793826in}}%
\pgfpathlineto{\pgfqpoint{1.093611in}{2.825082in}}%
\pgfpathlineto{\pgfqpoint{1.101753in}{2.854812in}}%
\pgfpathlineto{\pgfqpoint{1.109895in}{2.882973in}}%
\pgfpathlineto{\pgfqpoint{1.118037in}{2.909527in}}%
\pgfpathlineto{\pgfqpoint{1.126178in}{2.934435in}}%
\pgfpathlineto{\pgfqpoint{1.134320in}{2.957661in}}%
\pgfpathlineto{\pgfqpoint{1.142462in}{2.979173in}}%
\pgfpathlineto{\pgfqpoint{1.150604in}{2.998940in}}%
\pgfpathlineto{\pgfqpoint{1.158746in}{3.016933in}}%
\pgfpathlineto{\pgfqpoint{1.162817in}{3.025257in}}%
\pgfpathlineto{\pgfqpoint{1.166888in}{3.033128in}}%
\pgfpathlineto{\pgfqpoint{1.170959in}{3.040543in}}%
\pgfpathlineto{\pgfqpoint{1.175030in}{3.047500in}}%
\pgfpathlineto{\pgfqpoint{1.179101in}{3.053997in}}%
\pgfpathlineto{\pgfqpoint{1.183171in}{3.060030in}}%
\pgfpathlineto{\pgfqpoint{1.187242in}{3.065599in}}%
\pgfpathlineto{\pgfqpoint{1.191313in}{3.070700in}}%
\pgfpathlineto{\pgfqpoint{1.195384in}{3.075333in}}%
\pgfpathlineto{\pgfqpoint{1.199455in}{3.079495in}}%
\pgfpathlineto{\pgfqpoint{1.203526in}{3.083185in}}%
\pgfpathlineto{\pgfqpoint{1.207597in}{3.086401in}}%
\pgfpathlineto{\pgfqpoint{1.211668in}{3.089143in}}%
\pgfpathlineto{\pgfqpoint{1.215739in}{3.091410in}}%
\pgfpathlineto{\pgfqpoint{1.219810in}{3.093200in}}%
\pgfpathlineto{\pgfqpoint{1.223881in}{3.094514in}}%
\pgfpathlineto{\pgfqpoint{1.227952in}{3.095350in}}%
\pgfpathlineto{\pgfqpoint{1.232023in}{3.095708in}}%
\pgfpathlineto{\pgfqpoint{1.236094in}{3.095589in}}%
\pgfpathlineto{\pgfqpoint{1.240165in}{3.094991in}}%
\pgfpathlineto{\pgfqpoint{1.244236in}{3.093917in}}%
\pgfpathlineto{\pgfqpoint{1.248306in}{3.092365in}}%
\pgfpathlineto{\pgfqpoint{1.252377in}{3.090336in}}%
\pgfpathlineto{\pgfqpoint{1.256448in}{3.087832in}}%
\pgfpathlineto{\pgfqpoint{1.260519in}{3.084852in}}%
\pgfpathlineto{\pgfqpoint{1.264590in}{3.081399in}}%
\pgfpathlineto{\pgfqpoint{1.268661in}{3.077473in}}%
\pgfpathlineto{\pgfqpoint{1.272732in}{3.073075in}}%
\pgfpathlineto{\pgfqpoint{1.276803in}{3.068208in}}%
\pgfpathlineto{\pgfqpoint{1.280874in}{3.062873in}}%
\pgfpathlineto{\pgfqpoint{1.284945in}{3.057072in}}%
\pgfpathlineto{\pgfqpoint{1.289016in}{3.050806in}}%
\pgfpathlineto{\pgfqpoint{1.293087in}{3.044079in}}%
\pgfpathlineto{\pgfqpoint{1.297158in}{3.036892in}}%
\pgfpathlineto{\pgfqpoint{1.301229in}{3.029249in}}%
\pgfpathlineto{\pgfqpoint{1.305300in}{3.021151in}}%
\pgfpathlineto{\pgfqpoint{1.309370in}{3.012602in}}%
\pgfpathlineto{\pgfqpoint{1.317512in}{2.994163in}}%
\pgfpathlineto{\pgfqpoint{1.325654in}{2.973957in}}%
\pgfpathlineto{\pgfqpoint{1.333796in}{2.952014in}}%
\pgfpathlineto{\pgfqpoint{1.341938in}{2.928364in}}%
\pgfpathlineto{\pgfqpoint{1.350080in}{2.903042in}}%
\pgfpathlineto{\pgfqpoint{1.358222in}{2.876082in}}%
\pgfpathlineto{\pgfqpoint{1.366364in}{2.847524in}}%
\pgfpathlineto{\pgfqpoint{1.374505in}{2.817409in}}%
\pgfpathlineto{\pgfqpoint{1.382647in}{2.785779in}}%
\pgfpathlineto{\pgfqpoint{1.390789in}{2.752679in}}%
\pgfpathlineto{\pgfqpoint{1.398931in}{2.718157in}}%
\pgfpathlineto{\pgfqpoint{1.407073in}{2.682261in}}%
\pgfpathlineto{\pgfqpoint{1.415215in}{2.645042in}}%
\pgfpathlineto{\pgfqpoint{1.423357in}{2.606554in}}%
\pgfpathlineto{\pgfqpoint{1.435569in}{2.546563in}}%
\pgfpathlineto{\pgfqpoint{1.447782in}{2.484032in}}%
\pgfpathlineto{\pgfqpoint{1.459995in}{2.419161in}}%
\pgfpathlineto{\pgfqpoint{1.472208in}{2.352157in}}%
\pgfpathlineto{\pgfqpoint{1.484421in}{2.283237in}}%
\pgfpathlineto{\pgfqpoint{1.500704in}{2.188742in}}%
\pgfpathlineto{\pgfqpoint{1.516988in}{2.091770in}}%
\pgfpathlineto{\pgfqpoint{1.537343in}{1.967912in}}%
\pgfpathlineto{\pgfqpoint{1.565839in}{1.791560in}}%
\pgfpathlineto{\pgfqpoint{1.606549in}{1.539245in}}%
\pgfpathlineto{\pgfqpoint{1.626903in}{1.415388in}}%
\pgfpathlineto{\pgfqpoint{1.643187in}{1.318416in}}%
\pgfpathlineto{\pgfqpoint{1.659471in}{1.223921in}}%
\pgfpathlineto{\pgfqpoint{1.671684in}{1.155000in}}%
\pgfpathlineto{\pgfqpoint{1.683896in}{1.087997in}}%
\pgfpathlineto{\pgfqpoint{1.696109in}{1.023126in}}%
\pgfpathlineto{\pgfqpoint{1.708322in}{0.960594in}}%
\pgfpathlineto{\pgfqpoint{1.720535in}{0.900603in}}%
\pgfpathlineto{\pgfqpoint{1.728677in}{0.862116in}}%
\pgfpathlineto{\pgfqpoint{1.736819in}{0.824897in}}%
\pgfpathlineto{\pgfqpoint{1.744960in}{0.789001in}}%
\pgfpathlineto{\pgfqpoint{1.753102in}{0.754479in}}%
\pgfpathlineto{\pgfqpoint{1.761244in}{0.721379in}}%
\pgfpathlineto{\pgfqpoint{1.769386in}{0.689749in}}%
\pgfpathlineto{\pgfqpoint{1.777528in}{0.659633in}}%
\pgfpathlineto{\pgfqpoint{1.785670in}{0.631076in}}%
\pgfpathlineto{\pgfqpoint{1.793812in}{0.604116in}}%
\pgfpathlineto{\pgfqpoint{1.801954in}{0.578794in}}%
\pgfpathlineto{\pgfqpoint{1.810095in}{0.555144in}}%
\pgfpathlineto{\pgfqpoint{1.818237in}{0.533200in}}%
\pgfpathlineto{\pgfqpoint{1.826379in}{0.512995in}}%
\pgfpathlineto{\pgfqpoint{1.834521in}{0.494555in}}%
\pgfpathlineto{\pgfqpoint{1.838592in}{0.486006in}}%
\pgfpathlineto{\pgfqpoint{1.842663in}{0.477909in}}%
\pgfpathlineto{\pgfqpoint{1.846734in}{0.470265in}}%
\pgfpathlineto{\pgfqpoint{1.850805in}{0.463079in}}%
\pgfpathlineto{\pgfqpoint{1.854876in}{0.456352in}}%
\pgfpathlineto{\pgfqpoint{1.858947in}{0.450086in}}%
\pgfpathlineto{\pgfqpoint{1.863018in}{0.444285in}}%
\pgfpathlineto{\pgfqpoint{1.867089in}{0.438950in}}%
\pgfpathlineto{\pgfqpoint{1.871159in}{0.434083in}}%
\pgfpathlineto{\pgfqpoint{1.875230in}{0.429685in}}%
\pgfpathlineto{\pgfqpoint{1.879301in}{0.425759in}}%
\pgfpathlineto{\pgfqpoint{1.883372in}{0.422306in}}%
\pgfpathlineto{\pgfqpoint{1.887443in}{0.419326in}}%
\pgfpathlineto{\pgfqpoint{1.891514in}{0.416822in}}%
\pgfpathlineto{\pgfqpoint{1.895585in}{0.414793in}}%
\pgfpathlineto{\pgfqpoint{1.899656in}{0.413241in}}%
\pgfpathlineto{\pgfqpoint{1.903727in}{0.412166in}}%
\pgfpathlineto{\pgfqpoint{1.907798in}{0.411569in}}%
\pgfpathlineto{\pgfqpoint{1.911869in}{0.411450in}}%
\pgfpathlineto{\pgfqpoint{1.915940in}{0.411808in}}%
\pgfpathlineto{\pgfqpoint{1.920011in}{0.412644in}}%
\pgfpathlineto{\pgfqpoint{1.924082in}{0.413958in}}%
\pgfpathlineto{\pgfqpoint{1.928153in}{0.415748in}}%
\pgfpathlineto{\pgfqpoint{1.932223in}{0.418014in}}%
\pgfpathlineto{\pgfqpoint{1.936294in}{0.420757in}}%
\pgfpathlineto{\pgfqpoint{1.940365in}{0.423973in}}%
\pgfpathlineto{\pgfqpoint{1.944436in}{0.427663in}}%
\pgfpathlineto{\pgfqpoint{1.948507in}{0.431825in}}%
\pgfpathlineto{\pgfqpoint{1.952578in}{0.436457in}}%
\pgfpathlineto{\pgfqpoint{1.956649in}{0.441559in}}%
\pgfpathlineto{\pgfqpoint{1.960720in}{0.447127in}}%
\pgfpathlineto{\pgfqpoint{1.964791in}{0.453161in}}%
\pgfpathlineto{\pgfqpoint{1.968862in}{0.459658in}}%
\pgfpathlineto{\pgfqpoint{1.972933in}{0.466615in}}%
\pgfpathlineto{\pgfqpoint{1.977004in}{0.474030in}}%
\pgfpathlineto{\pgfqpoint{1.981075in}{0.481901in}}%
\pgfpathlineto{\pgfqpoint{1.985146in}{0.490225in}}%
\pgfpathlineto{\pgfqpoint{1.989217in}{0.498998in}}%
\pgfpathlineto{\pgfqpoint{1.997358in}{0.517882in}}%
\pgfpathlineto{\pgfqpoint{2.005500in}{0.538524in}}%
\pgfpathlineto{\pgfqpoint{2.013642in}{0.560897in}}%
\pgfpathlineto{\pgfqpoint{2.021784in}{0.584969in}}%
\pgfpathlineto{\pgfqpoint{2.029926in}{0.610704in}}%
\pgfpathlineto{\pgfqpoint{2.038068in}{0.638067in}}%
\pgfpathlineto{\pgfqpoint{2.046210in}{0.667018in}}%
\pgfpathlineto{\pgfqpoint{2.054352in}{0.697516in}}%
\pgfpathlineto{\pgfqpoint{2.062493in}{0.729518in}}%
\pgfpathlineto{\pgfqpoint{2.070635in}{0.762978in}}%
\pgfpathlineto{\pgfqpoint{2.078777in}{0.797848in}}%
\pgfpathlineto{\pgfqpoint{2.086919in}{0.834080in}}%
\pgfpathlineto{\pgfqpoint{2.095061in}{0.871621in}}%
\pgfpathlineto{\pgfqpoint{2.103203in}{0.910417in}}%
\pgfpathlineto{\pgfqpoint{2.115416in}{0.970845in}}%
\pgfpathlineto{\pgfqpoint{2.127628in}{1.033781in}}%
\pgfpathlineto{\pgfqpoint{2.139841in}{1.099022in}}%
\pgfpathlineto{\pgfqpoint{2.152054in}{1.166360in}}%
\pgfpathlineto{\pgfqpoint{2.164267in}{1.235579in}}%
\pgfpathlineto{\pgfqpoint{2.180550in}{1.330413in}}%
\pgfpathlineto{\pgfqpoint{2.196834in}{1.427656in}}%
\pgfpathlineto{\pgfqpoint{2.217189in}{1.551754in}}%
\pgfpathlineto{\pgfqpoint{2.245685in}{1.728256in}}%
\pgfpathlineto{\pgfqpoint{2.286395in}{1.980402in}}%
\pgfpathlineto{\pgfqpoint{2.306749in}{2.104008in}}%
\pgfpathlineto{\pgfqpoint{2.323033in}{2.200700in}}%
\pgfpathlineto{\pgfqpoint{2.339317in}{2.294847in}}%
\pgfpathlineto{\pgfqpoint{2.351530in}{2.363464in}}%
\pgfpathlineto{\pgfqpoint{2.363743in}{2.430126in}}%
\pgfpathlineto{\pgfqpoint{2.375955in}{2.494622in}}%
\pgfpathlineto{\pgfqpoint{2.388168in}{2.556743in}}%
\pgfpathlineto{\pgfqpoint{2.400381in}{2.616292in}}%
\pgfpathlineto{\pgfqpoint{2.408523in}{2.654468in}}%
\pgfpathlineto{\pgfqpoint{2.416665in}{2.691361in}}%
\pgfpathlineto{\pgfqpoint{2.424807in}{2.726918in}}%
\pgfpathlineto{\pgfqpoint{2.432948in}{2.761089in}}%
\pgfpathlineto{\pgfqpoint{2.441090in}{2.793826in}}%
\pgfpathlineto{\pgfqpoint{2.449232in}{2.825082in}}%
\pgfpathlineto{\pgfqpoint{2.457374in}{2.854812in}}%
\pgfpathlineto{\pgfqpoint{2.465516in}{2.882973in}}%
\pgfpathlineto{\pgfqpoint{2.473658in}{2.909527in}}%
\pgfpathlineto{\pgfqpoint{2.481800in}{2.934435in}}%
\pgfpathlineto{\pgfqpoint{2.489942in}{2.957661in}}%
\pgfpathlineto{\pgfqpoint{2.498083in}{2.979173in}}%
\pgfpathlineto{\pgfqpoint{2.506225in}{2.998940in}}%
\pgfpathlineto{\pgfqpoint{2.514367in}{3.016933in}}%
\pgfpathlineto{\pgfqpoint{2.518438in}{3.025257in}}%
\pgfpathlineto{\pgfqpoint{2.522509in}{3.033128in}}%
\pgfpathlineto{\pgfqpoint{2.526580in}{3.040543in}}%
\pgfpathlineto{\pgfqpoint{2.530651in}{3.047500in}}%
\pgfpathlineto{\pgfqpoint{2.534722in}{3.053997in}}%
\pgfpathlineto{\pgfqpoint{2.538793in}{3.060030in}}%
\pgfpathlineto{\pgfqpoint{2.542864in}{3.065599in}}%
\pgfpathlineto{\pgfqpoint{2.546935in}{3.070700in}}%
\pgfpathlineto{\pgfqpoint{2.551006in}{3.075333in}}%
\pgfpathlineto{\pgfqpoint{2.555076in}{3.079495in}}%
\pgfpathlineto{\pgfqpoint{2.559147in}{3.083185in}}%
\pgfpathlineto{\pgfqpoint{2.563218in}{3.086401in}}%
\pgfpathlineto{\pgfqpoint{2.567289in}{3.089143in}}%
\pgfpathlineto{\pgfqpoint{2.571360in}{3.091410in}}%
\pgfpathlineto{\pgfqpoint{2.575431in}{3.093200in}}%
\pgfpathlineto{\pgfqpoint{2.579502in}{3.094514in}}%
\pgfpathlineto{\pgfqpoint{2.583573in}{3.095350in}}%
\pgfpathlineto{\pgfqpoint{2.587644in}{3.095708in}}%
\pgfpathlineto{\pgfqpoint{2.591715in}{3.095589in}}%
\pgfpathlineto{\pgfqpoint{2.595786in}{3.094991in}}%
\pgfpathlineto{\pgfqpoint{2.599857in}{3.093917in}}%
\pgfpathlineto{\pgfqpoint{2.603928in}{3.092365in}}%
\pgfpathlineto{\pgfqpoint{2.607999in}{3.090336in}}%
\pgfpathlineto{\pgfqpoint{2.612070in}{3.087832in}}%
\pgfpathlineto{\pgfqpoint{2.616140in}{3.084852in}}%
\pgfpathlineto{\pgfqpoint{2.620211in}{3.081399in}}%
\pgfpathlineto{\pgfqpoint{2.624282in}{3.077473in}}%
\pgfpathlineto{\pgfqpoint{2.628353in}{3.073075in}}%
\pgfpathlineto{\pgfqpoint{2.632424in}{3.068208in}}%
\pgfpathlineto{\pgfqpoint{2.636495in}{3.062873in}}%
\pgfpathlineto{\pgfqpoint{2.640566in}{3.057072in}}%
\pgfpathlineto{\pgfqpoint{2.644637in}{3.050806in}}%
\pgfpathlineto{\pgfqpoint{2.648708in}{3.044079in}}%
\pgfpathlineto{\pgfqpoint{2.652779in}{3.036892in}}%
\pgfpathlineto{\pgfqpoint{2.656850in}{3.029249in}}%
\pgfpathlineto{\pgfqpoint{2.660921in}{3.021151in}}%
\pgfpathlineto{\pgfqpoint{2.664992in}{3.012602in}}%
\pgfpathlineto{\pgfqpoint{2.673134in}{2.994163in}}%
\pgfpathlineto{\pgfqpoint{2.681275in}{2.973957in}}%
\pgfpathlineto{\pgfqpoint{2.689417in}{2.952014in}}%
\pgfpathlineto{\pgfqpoint{2.697559in}{2.928364in}}%
\pgfpathlineto{\pgfqpoint{2.705701in}{2.903042in}}%
\pgfpathlineto{\pgfqpoint{2.713843in}{2.876082in}}%
\pgfpathlineto{\pgfqpoint{2.721985in}{2.847524in}}%
\pgfpathlineto{\pgfqpoint{2.730127in}{2.817409in}}%
\pgfpathlineto{\pgfqpoint{2.738269in}{2.785779in}}%
\pgfpathlineto{\pgfqpoint{2.746410in}{2.752679in}}%
\pgfpathlineto{\pgfqpoint{2.754552in}{2.718157in}}%
\pgfpathlineto{\pgfqpoint{2.762694in}{2.682261in}}%
\pgfpathlineto{\pgfqpoint{2.770836in}{2.645042in}}%
\pgfpathlineto{\pgfqpoint{2.778978in}{2.606554in}}%
\pgfpathlineto{\pgfqpoint{2.791191in}{2.546563in}}%
\pgfpathlineto{\pgfqpoint{2.803403in}{2.484032in}}%
\pgfpathlineto{\pgfqpoint{2.815616in}{2.419161in}}%
\pgfpathlineto{\pgfqpoint{2.827829in}{2.352157in}}%
\pgfpathlineto{\pgfqpoint{2.840042in}{2.283237in}}%
\pgfpathlineto{\pgfqpoint{2.856326in}{2.188742in}}%
\pgfpathlineto{\pgfqpoint{2.872609in}{2.091770in}}%
\pgfpathlineto{\pgfqpoint{2.892964in}{1.967912in}}%
\pgfpathlineto{\pgfqpoint{2.921461in}{1.791560in}}%
\pgfpathlineto{\pgfqpoint{2.962170in}{1.539245in}}%
\pgfpathlineto{\pgfqpoint{2.982525in}{1.415388in}}%
\pgfpathlineto{\pgfqpoint{2.998808in}{1.318416in}}%
\pgfpathlineto{\pgfqpoint{3.015092in}{1.223921in}}%
\pgfpathlineto{\pgfqpoint{3.027305in}{1.155000in}}%
\pgfpathlineto{\pgfqpoint{3.039518in}{1.087997in}}%
\pgfpathlineto{\pgfqpoint{3.051731in}{1.023126in}}%
\pgfpathlineto{\pgfqpoint{3.063943in}{0.960594in}}%
\pgfpathlineto{\pgfqpoint{3.076156in}{0.900603in}}%
\pgfpathlineto{\pgfqpoint{3.084298in}{0.862116in}}%
\pgfpathlineto{\pgfqpoint{3.092440in}{0.824897in}}%
\pgfpathlineto{\pgfqpoint{3.100582in}{0.789001in}}%
\pgfpathlineto{\pgfqpoint{3.108724in}{0.754479in}}%
\pgfpathlineto{\pgfqpoint{3.116865in}{0.721379in}}%
\pgfpathlineto{\pgfqpoint{3.125007in}{0.689749in}}%
\pgfpathlineto{\pgfqpoint{3.133149in}{0.659633in}}%
\pgfpathlineto{\pgfqpoint{3.141291in}{0.631076in}}%
\pgfpathlineto{\pgfqpoint{3.149433in}{0.604116in}}%
\pgfpathlineto{\pgfqpoint{3.157575in}{0.578794in}}%
\pgfpathlineto{\pgfqpoint{3.165717in}{0.555144in}}%
\pgfpathlineto{\pgfqpoint{3.173859in}{0.533200in}}%
\pgfpathlineto{\pgfqpoint{3.182000in}{0.512995in}}%
\pgfpathlineto{\pgfqpoint{3.190142in}{0.494555in}}%
\pgfpathlineto{\pgfqpoint{3.194213in}{0.486006in}}%
\pgfpathlineto{\pgfqpoint{3.198284in}{0.477909in}}%
\pgfpathlineto{\pgfqpoint{3.202355in}{0.470265in}}%
\pgfpathlineto{\pgfqpoint{3.206426in}{0.463079in}}%
\pgfpathlineto{\pgfqpoint{3.210497in}{0.456352in}}%
\pgfpathlineto{\pgfqpoint{3.214568in}{0.450086in}}%
\pgfpathlineto{\pgfqpoint{3.218639in}{0.444285in}}%
\pgfpathlineto{\pgfqpoint{3.222710in}{0.438950in}}%
\pgfpathlineto{\pgfqpoint{3.226781in}{0.434083in}}%
\pgfpathlineto{\pgfqpoint{3.230852in}{0.429685in}}%
\pgfpathlineto{\pgfqpoint{3.234923in}{0.425759in}}%
\pgfpathlineto{\pgfqpoint{3.238993in}{0.422306in}}%
\pgfpathlineto{\pgfqpoint{3.243064in}{0.419326in}}%
\pgfpathlineto{\pgfqpoint{3.247135in}{0.416822in}}%
\pgfpathlineto{\pgfqpoint{3.251206in}{0.414793in}}%
\pgfpathlineto{\pgfqpoint{3.255277in}{0.413241in}}%
\pgfpathlineto{\pgfqpoint{3.259348in}{0.412166in}}%
\pgfpathlineto{\pgfqpoint{3.263419in}{0.411569in}}%
\pgfpathlineto{\pgfqpoint{3.267490in}{0.411450in}}%
\pgfpathlineto{\pgfqpoint{3.271561in}{0.411808in}}%
\pgfpathlineto{\pgfqpoint{3.275632in}{0.412644in}}%
\pgfpathlineto{\pgfqpoint{3.279703in}{0.413958in}}%
\pgfpathlineto{\pgfqpoint{3.283774in}{0.415748in}}%
\pgfpathlineto{\pgfqpoint{3.287845in}{0.418014in}}%
\pgfpathlineto{\pgfqpoint{3.291916in}{0.420757in}}%
\pgfpathlineto{\pgfqpoint{3.295987in}{0.423973in}}%
\pgfpathlineto{\pgfqpoint{3.300058in}{0.427663in}}%
\pgfpathlineto{\pgfqpoint{3.304128in}{0.431825in}}%
\pgfpathlineto{\pgfqpoint{3.308199in}{0.436457in}}%
\pgfpathlineto{\pgfqpoint{3.312270in}{0.441559in}}%
\pgfpathlineto{\pgfqpoint{3.316341in}{0.447127in}}%
\pgfpathlineto{\pgfqpoint{3.320412in}{0.453161in}}%
\pgfpathlineto{\pgfqpoint{3.324483in}{0.459658in}}%
\pgfpathlineto{\pgfqpoint{3.328554in}{0.466615in}}%
\pgfpathlineto{\pgfqpoint{3.332625in}{0.474030in}}%
\pgfpathlineto{\pgfqpoint{3.336696in}{0.481901in}}%
\pgfpathlineto{\pgfqpoint{3.340767in}{0.490225in}}%
\pgfpathlineto{\pgfqpoint{3.344838in}{0.498998in}}%
\pgfpathlineto{\pgfqpoint{3.352980in}{0.517882in}}%
\pgfpathlineto{\pgfqpoint{3.361122in}{0.538524in}}%
\pgfpathlineto{\pgfqpoint{3.369263in}{0.560897in}}%
\pgfpathlineto{\pgfqpoint{3.377405in}{0.584969in}}%
\pgfpathlineto{\pgfqpoint{3.385547in}{0.610704in}}%
\pgfpathlineto{\pgfqpoint{3.393689in}{0.638067in}}%
\pgfpathlineto{\pgfqpoint{3.401831in}{0.667018in}}%
\pgfpathlineto{\pgfqpoint{3.409973in}{0.697516in}}%
\pgfpathlineto{\pgfqpoint{3.418115in}{0.729518in}}%
\pgfpathlineto{\pgfqpoint{3.426256in}{0.762978in}}%
\pgfpathlineto{\pgfqpoint{3.434398in}{0.797848in}}%
\pgfpathlineto{\pgfqpoint{3.442540in}{0.834080in}}%
\pgfpathlineto{\pgfqpoint{3.450682in}{0.871621in}}%
\pgfpathlineto{\pgfqpoint{3.458824in}{0.910417in}}%
\pgfpathlineto{\pgfqpoint{3.471037in}{0.970845in}}%
\pgfpathlineto{\pgfqpoint{3.483250in}{1.033781in}}%
\pgfpathlineto{\pgfqpoint{3.495462in}{1.099022in}}%
\pgfpathlineto{\pgfqpoint{3.507675in}{1.166360in}}%
\pgfpathlineto{\pgfqpoint{3.519888in}{1.235579in}}%
\pgfpathlineto{\pgfqpoint{3.536172in}{1.330413in}}%
\pgfpathlineto{\pgfqpoint{3.552455in}{1.427656in}}%
\pgfpathlineto{\pgfqpoint{3.572810in}{1.551754in}}%
\pgfpathlineto{\pgfqpoint{3.601307in}{1.728256in}}%
\pgfpathlineto{\pgfqpoint{3.642016in}{1.980402in}}%
\pgfpathlineto{\pgfqpoint{3.662371in}{2.104008in}}%
\pgfpathlineto{\pgfqpoint{3.678654in}{2.200700in}}%
\pgfpathlineto{\pgfqpoint{3.694938in}{2.294847in}}%
\pgfpathlineto{\pgfqpoint{3.707151in}{2.363464in}}%
\pgfpathlineto{\pgfqpoint{3.719364in}{2.430126in}}%
\pgfpathlineto{\pgfqpoint{3.731577in}{2.494622in}}%
\pgfpathlineto{\pgfqpoint{3.743789in}{2.556743in}}%
\pgfpathlineto{\pgfqpoint{3.756002in}{2.616292in}}%
\pgfpathlineto{\pgfqpoint{3.764144in}{2.654468in}}%
\pgfpathlineto{\pgfqpoint{3.772286in}{2.691361in}}%
\pgfpathlineto{\pgfqpoint{3.780428in}{2.726918in}}%
\pgfpathlineto{\pgfqpoint{3.788570in}{2.761089in}}%
\pgfpathlineto{\pgfqpoint{3.796712in}{2.793826in}}%
\pgfpathlineto{\pgfqpoint{3.804853in}{2.825082in}}%
\pgfpathlineto{\pgfqpoint{3.812995in}{2.854812in}}%
\pgfpathlineto{\pgfqpoint{3.821137in}{2.882973in}}%
\pgfpathlineto{\pgfqpoint{3.829279in}{2.909527in}}%
\pgfpathlineto{\pgfqpoint{3.837421in}{2.934435in}}%
\pgfpathlineto{\pgfqpoint{3.845563in}{2.957661in}}%
\pgfpathlineto{\pgfqpoint{3.853705in}{2.979173in}}%
\pgfpathlineto{\pgfqpoint{3.861847in}{2.998940in}}%
\pgfpathlineto{\pgfqpoint{3.869988in}{3.016933in}}%
\pgfpathlineto{\pgfqpoint{3.874059in}{3.025257in}}%
\pgfpathlineto{\pgfqpoint{3.878130in}{3.033128in}}%
\pgfpathlineto{\pgfqpoint{3.882201in}{3.040543in}}%
\pgfpathlineto{\pgfqpoint{3.886272in}{3.047500in}}%
\pgfpathlineto{\pgfqpoint{3.890343in}{3.053997in}}%
\pgfpathlineto{\pgfqpoint{3.894414in}{3.060030in}}%
\pgfpathlineto{\pgfqpoint{3.898485in}{3.065599in}}%
\pgfpathlineto{\pgfqpoint{3.902556in}{3.070700in}}%
\pgfpathlineto{\pgfqpoint{3.906627in}{3.075333in}}%
\pgfpathlineto{\pgfqpoint{3.910698in}{3.079495in}}%
\pgfpathlineto{\pgfqpoint{3.914769in}{3.083185in}}%
\pgfpathlineto{\pgfqpoint{3.918840in}{3.086401in}}%
\pgfpathlineto{\pgfqpoint{3.922911in}{3.089143in}}%
\pgfpathlineto{\pgfqpoint{3.926981in}{3.091410in}}%
\pgfpathlineto{\pgfqpoint{3.931052in}{3.093200in}}%
\pgfpathlineto{\pgfqpoint{3.935123in}{3.094514in}}%
\pgfpathlineto{\pgfqpoint{3.939194in}{3.095350in}}%
\pgfpathlineto{\pgfqpoint{3.943265in}{3.095708in}}%
\pgfpathlineto{\pgfqpoint{3.947336in}{3.095589in}}%
\pgfpathlineto{\pgfqpoint{3.951407in}{3.094991in}}%
\pgfpathlineto{\pgfqpoint{3.955478in}{3.093917in}}%
\pgfpathlineto{\pgfqpoint{3.959549in}{3.092365in}}%
\pgfpathlineto{\pgfqpoint{3.963620in}{3.090336in}}%
\pgfpathlineto{\pgfqpoint{3.967691in}{3.087832in}}%
\pgfpathlineto{\pgfqpoint{3.971762in}{3.084852in}}%
\pgfpathlineto{\pgfqpoint{3.975833in}{3.081399in}}%
\pgfpathlineto{\pgfqpoint{3.979904in}{3.077473in}}%
\pgfpathlineto{\pgfqpoint{3.983975in}{3.073075in}}%
\pgfpathlineto{\pgfqpoint{3.988045in}{3.068208in}}%
\pgfpathlineto{\pgfqpoint{3.992116in}{3.062873in}}%
\pgfpathlineto{\pgfqpoint{3.996187in}{3.057072in}}%
\pgfpathlineto{\pgfqpoint{4.000258in}{3.050806in}}%
\pgfpathlineto{\pgfqpoint{4.004329in}{3.044079in}}%
\pgfpathlineto{\pgfqpoint{4.008400in}{3.036892in}}%
\pgfpathlineto{\pgfqpoint{4.012471in}{3.029249in}}%
\pgfpathlineto{\pgfqpoint{4.016542in}{3.021151in}}%
\pgfpathlineto{\pgfqpoint{4.020613in}{3.012602in}}%
\pgfpathlineto{\pgfqpoint{4.028755in}{2.994163in}}%
\pgfpathlineto{\pgfqpoint{4.036897in}{2.973957in}}%
\pgfpathlineto{\pgfqpoint{4.045039in}{2.952014in}}%
\pgfpathlineto{\pgfqpoint{4.053180in}{2.928364in}}%
\pgfpathlineto{\pgfqpoint{4.061322in}{2.903042in}}%
\pgfpathlineto{\pgfqpoint{4.069464in}{2.876082in}}%
\pgfpathlineto{\pgfqpoint{4.077606in}{2.847524in}}%
\pgfpathlineto{\pgfqpoint{4.085748in}{2.817409in}}%
\pgfpathlineto{\pgfqpoint{4.093890in}{2.785779in}}%
\pgfpathlineto{\pgfqpoint{4.102032in}{2.752679in}}%
\pgfpathlineto{\pgfqpoint{4.110174in}{2.718157in}}%
\pgfpathlineto{\pgfqpoint{4.118315in}{2.682261in}}%
\pgfpathlineto{\pgfqpoint{4.126457in}{2.645042in}}%
\pgfpathlineto{\pgfqpoint{4.134599in}{2.606554in}}%
\pgfpathlineto{\pgfqpoint{4.146812in}{2.546563in}}%
\pgfpathlineto{\pgfqpoint{4.159025in}{2.484032in}}%
\pgfpathlineto{\pgfqpoint{4.171238in}{2.419161in}}%
\pgfpathlineto{\pgfqpoint{4.183450in}{2.352157in}}%
\pgfpathlineto{\pgfqpoint{4.195663in}{2.283237in}}%
\pgfpathlineto{\pgfqpoint{4.211947in}{2.188742in}}%
\pgfpathlineto{\pgfqpoint{4.228231in}{2.091770in}}%
\pgfpathlineto{\pgfqpoint{4.248585in}{1.967912in}}%
\pgfpathlineto{\pgfqpoint{4.277082in}{1.791560in}}%
\pgfpathlineto{\pgfqpoint{4.317791in}{1.539245in}}%
\pgfpathlineto{\pgfqpoint{4.338146in}{1.415388in}}%
\pgfpathlineto{\pgfqpoint{4.354430in}{1.318416in}}%
\pgfpathlineto{\pgfqpoint{4.370713in}{1.223921in}}%
\pgfpathlineto{\pgfqpoint{4.382926in}{1.155000in}}%
\pgfpathlineto{\pgfqpoint{4.395139in}{1.087997in}}%
\pgfpathlineto{\pgfqpoint{4.407352in}{1.023126in}}%
\pgfpathlineto{\pgfqpoint{4.419565in}{0.960594in}}%
\pgfpathlineto{\pgfqpoint{4.431777in}{0.900603in}}%
\pgfpathlineto{\pgfqpoint{4.439919in}{0.862116in}}%
\pgfpathlineto{\pgfqpoint{4.448061in}{0.824897in}}%
\pgfpathlineto{\pgfqpoint{4.456203in}{0.789001in}}%
\pgfpathlineto{\pgfqpoint{4.464345in}{0.754479in}}%
\pgfpathlineto{\pgfqpoint{4.472487in}{0.721379in}}%
\pgfpathlineto{\pgfqpoint{4.480629in}{0.689749in}}%
\pgfpathlineto{\pgfqpoint{4.488770in}{0.659633in}}%
\pgfpathlineto{\pgfqpoint{4.496912in}{0.631076in}}%
\pgfpathlineto{\pgfqpoint{4.505054in}{0.604116in}}%
\pgfpathlineto{\pgfqpoint{4.513196in}{0.578794in}}%
\pgfpathlineto{\pgfqpoint{4.521338in}{0.555144in}}%
\pgfpathlineto{\pgfqpoint{4.529480in}{0.533200in}}%
\pgfpathlineto{\pgfqpoint{4.537622in}{0.512995in}}%
\pgfpathlineto{\pgfqpoint{4.545764in}{0.494555in}}%
\pgfpathlineto{\pgfqpoint{4.549834in}{0.486006in}}%
\pgfpathlineto{\pgfqpoint{4.553905in}{0.477909in}}%
\pgfpathlineto{\pgfqpoint{4.557976in}{0.470265in}}%
\pgfpathlineto{\pgfqpoint{4.562047in}{0.463079in}}%
\pgfpathlineto{\pgfqpoint{4.566118in}{0.456352in}}%
\pgfpathlineto{\pgfqpoint{4.570189in}{0.450086in}}%
\pgfpathlineto{\pgfqpoint{4.574260in}{0.444285in}}%
\pgfpathlineto{\pgfqpoint{4.578331in}{0.438950in}}%
\pgfpathlineto{\pgfqpoint{4.582402in}{0.434083in}}%
\pgfpathlineto{\pgfqpoint{4.586473in}{0.429685in}}%
\pgfpathlineto{\pgfqpoint{4.590544in}{0.425759in}}%
\pgfpathlineto{\pgfqpoint{4.594615in}{0.422306in}}%
\pgfpathlineto{\pgfqpoint{4.598686in}{0.419326in}}%
\pgfpathlineto{\pgfqpoint{4.602757in}{0.416822in}}%
\pgfpathlineto{\pgfqpoint{4.606828in}{0.414793in}}%
\pgfpathlineto{\pgfqpoint{4.610898in}{0.413241in}}%
\pgfpathlineto{\pgfqpoint{4.614969in}{0.412166in}}%
\pgfpathlineto{\pgfqpoint{4.619040in}{0.411569in}}%
\pgfpathlineto{\pgfqpoint{4.623111in}{0.411450in}}%
\pgfpathlineto{\pgfqpoint{4.627182in}{0.411808in}}%
\pgfpathlineto{\pgfqpoint{4.631253in}{0.412644in}}%
\pgfpathlineto{\pgfqpoint{4.635324in}{0.413958in}}%
\pgfpathlineto{\pgfqpoint{4.639395in}{0.415748in}}%
\pgfpathlineto{\pgfqpoint{4.643466in}{0.418014in}}%
\pgfpathlineto{\pgfqpoint{4.647537in}{0.420757in}}%
\pgfpathlineto{\pgfqpoint{4.651608in}{0.423973in}}%
\pgfpathlineto{\pgfqpoint{4.655679in}{0.427663in}}%
\pgfpathlineto{\pgfqpoint{4.659750in}{0.431825in}}%
\pgfpathlineto{\pgfqpoint{4.663821in}{0.436457in}}%
\pgfpathlineto{\pgfqpoint{4.667892in}{0.441559in}}%
\pgfpathlineto{\pgfqpoint{4.671963in}{0.447127in}}%
\pgfpathlineto{\pgfqpoint{4.676033in}{0.453161in}}%
\pgfpathlineto{\pgfqpoint{4.680104in}{0.459658in}}%
\pgfpathlineto{\pgfqpoint{4.684175in}{0.466615in}}%
\pgfpathlineto{\pgfqpoint{4.688246in}{0.474030in}}%
\pgfpathlineto{\pgfqpoint{4.692317in}{0.481901in}}%
\pgfpathlineto{\pgfqpoint{4.696388in}{0.490225in}}%
\pgfpathlineto{\pgfqpoint{4.700459in}{0.498998in}}%
\pgfpathlineto{\pgfqpoint{4.708601in}{0.517882in}}%
\pgfpathlineto{\pgfqpoint{4.716743in}{0.538524in}}%
\pgfpathlineto{\pgfqpoint{4.724885in}{0.560897in}}%
\pgfpathlineto{\pgfqpoint{4.733027in}{0.584969in}}%
\pgfpathlineto{\pgfqpoint{4.741168in}{0.610704in}}%
\pgfpathlineto{\pgfqpoint{4.749310in}{0.638067in}}%
\pgfpathlineto{\pgfqpoint{4.757452in}{0.667018in}}%
\pgfpathlineto{\pgfqpoint{4.765594in}{0.697516in}}%
\pgfpathlineto{\pgfqpoint{4.773736in}{0.729518in}}%
\pgfpathlineto{\pgfqpoint{4.781878in}{0.762978in}}%
\pgfpathlineto{\pgfqpoint{4.790020in}{0.797848in}}%
\pgfpathlineto{\pgfqpoint{4.798161in}{0.834080in}}%
\pgfpathlineto{\pgfqpoint{4.806303in}{0.871621in}}%
\pgfpathlineto{\pgfqpoint{4.814445in}{0.910417in}}%
\pgfpathlineto{\pgfqpoint{4.826658in}{0.970845in}}%
\pgfpathlineto{\pgfqpoint{4.838871in}{1.033781in}}%
\pgfpathlineto{\pgfqpoint{4.851084in}{1.099022in}}%
\pgfpathlineto{\pgfqpoint{4.863296in}{1.166360in}}%
\pgfpathlineto{\pgfqpoint{4.875509in}{1.235579in}}%
\pgfpathlineto{\pgfqpoint{4.891793in}{1.330413in}}%
\pgfpathlineto{\pgfqpoint{4.908077in}{1.427656in}}%
\pgfpathlineto{\pgfqpoint{4.928431in}{1.551754in}}%
\pgfpathlineto{\pgfqpoint{4.956928in}{1.728256in}}%
\pgfpathlineto{\pgfqpoint{4.960999in}{1.753579in}}%
\pgfpathlineto{\pgfqpoint{4.960999in}{1.753579in}}%
\pgfusepath{stroke}%
\end{pgfscope}%
\begin{pgfscope}%
\pgfpathrectangle{\pgfqpoint{0.690792in}{0.277222in}}{\pgfqpoint{4.473550in}{2.952713in}}%
\pgfusepath{clip}%
\pgfsetbuttcap%
\pgfsetroundjoin%
\pgfsetlinewidth{1.003750pt}%
\definecolor{currentstroke}{rgb}{0.752941,0.752941,0.752941}%
\pgfsetstrokecolor{currentstroke}%
\pgfsetdash{{4.000000pt}{3.000000pt}}{0.000000pt}%
\pgfpathmoveto{\pgfqpoint{0.894135in}{1.753579in}}%
\pgfpathlineto{\pgfqpoint{0.922632in}{1.988718in}}%
\pgfpathlineto{\pgfqpoint{0.938915in}{2.120277in}}%
\pgfpathlineto{\pgfqpoint{0.951128in}{2.216583in}}%
\pgfpathlineto{\pgfqpoint{0.963341in}{2.310253in}}%
\pgfpathlineto{\pgfqpoint{0.975554in}{2.400753in}}%
\pgfpathlineto{\pgfqpoint{0.987767in}{2.487569in}}%
\pgfpathlineto{\pgfqpoint{0.995908in}{2.543154in}}%
\pgfpathlineto{\pgfqpoint{1.004050in}{2.596741in}}%
\pgfpathlineto{\pgfqpoint{1.012192in}{2.648193in}}%
\pgfpathlineto{\pgfqpoint{1.020334in}{2.697381in}}%
\pgfpathlineto{\pgfqpoint{1.028476in}{2.744180in}}%
\pgfpathlineto{\pgfqpoint{1.036618in}{2.788472in}}%
\pgfpathlineto{\pgfqpoint{1.044760in}{2.830144in}}%
\pgfpathlineto{\pgfqpoint{1.052902in}{2.869091in}}%
\pgfpathlineto{\pgfqpoint{1.061043in}{2.905215in}}%
\pgfpathlineto{\pgfqpoint{1.069185in}{2.938424in}}%
\pgfpathlineto{\pgfqpoint{1.073256in}{2.953908in}}%
\pgfpathlineto{\pgfqpoint{1.077327in}{2.968633in}}%
\pgfpathlineto{\pgfqpoint{1.081398in}{2.982589in}}%
\pgfpathlineto{\pgfqpoint{1.085469in}{2.995768in}}%
\pgfpathlineto{\pgfqpoint{1.089540in}{3.008160in}}%
\pgfpathlineto{\pgfqpoint{1.093611in}{3.019758in}}%
\pgfpathlineto{\pgfqpoint{1.097682in}{3.030554in}}%
\pgfpathlineto{\pgfqpoint{1.101753in}{3.040543in}}%
\pgfpathlineto{\pgfqpoint{1.105824in}{3.049717in}}%
\pgfpathlineto{\pgfqpoint{1.109895in}{3.058071in}}%
\pgfpathlineto{\pgfqpoint{1.113966in}{3.065599in}}%
\pgfpathlineto{\pgfqpoint{1.118037in}{3.072297in}}%
\pgfpathlineto{\pgfqpoint{1.122107in}{3.078160in}}%
\pgfpathlineto{\pgfqpoint{1.126178in}{3.083185in}}%
\pgfpathlineto{\pgfqpoint{1.130249in}{3.087368in}}%
\pgfpathlineto{\pgfqpoint{1.134320in}{3.090707in}}%
\pgfpathlineto{\pgfqpoint{1.138391in}{3.093200in}}%
\pgfpathlineto{\pgfqpoint{1.142462in}{3.094845in}}%
\pgfpathlineto{\pgfqpoint{1.146533in}{3.095642in}}%
\pgfpathlineto{\pgfqpoint{1.150604in}{3.095589in}}%
\pgfpathlineto{\pgfqpoint{1.154675in}{3.094686in}}%
\pgfpathlineto{\pgfqpoint{1.158746in}{3.092935in}}%
\pgfpathlineto{\pgfqpoint{1.162817in}{3.090336in}}%
\pgfpathlineto{\pgfqpoint{1.166888in}{3.086891in}}%
\pgfpathlineto{\pgfqpoint{1.170959in}{3.082602in}}%
\pgfpathlineto{\pgfqpoint{1.175030in}{3.077473in}}%
\pgfpathlineto{\pgfqpoint{1.179101in}{3.071505in}}%
\pgfpathlineto{\pgfqpoint{1.183171in}{3.064703in}}%
\pgfpathlineto{\pgfqpoint{1.187242in}{3.057072in}}%
\pgfpathlineto{\pgfqpoint{1.191313in}{3.048615in}}%
\pgfpathlineto{\pgfqpoint{1.195384in}{3.039339in}}%
\pgfpathlineto{\pgfqpoint{1.199455in}{3.029249in}}%
\pgfpathlineto{\pgfqpoint{1.203526in}{3.018352in}}%
\pgfpathlineto{\pgfqpoint{1.207597in}{3.006654in}}%
\pgfpathlineto{\pgfqpoint{1.211668in}{2.994163in}}%
\pgfpathlineto{\pgfqpoint{1.215739in}{2.980887in}}%
\pgfpathlineto{\pgfqpoint{1.219810in}{2.966835in}}%
\pgfpathlineto{\pgfqpoint{1.223881in}{2.952014in}}%
\pgfpathlineto{\pgfqpoint{1.227952in}{2.936435in}}%
\pgfpathlineto{\pgfqpoint{1.236094in}{2.903042in}}%
\pgfpathlineto{\pgfqpoint{1.244236in}{2.866739in}}%
\pgfpathlineto{\pgfqpoint{1.252377in}{2.827618in}}%
\pgfpathlineto{\pgfqpoint{1.260519in}{2.785779in}}%
\pgfpathlineto{\pgfqpoint{1.268661in}{2.741327in}}%
\pgfpathlineto{\pgfqpoint{1.276803in}{2.694375in}}%
\pgfpathlineto{\pgfqpoint{1.284945in}{2.645042in}}%
\pgfpathlineto{\pgfqpoint{1.293087in}{2.593453in}}%
\pgfpathlineto{\pgfqpoint{1.301229in}{2.539737in}}%
\pgfpathlineto{\pgfqpoint{1.309370in}{2.484032in}}%
\pgfpathlineto{\pgfqpoint{1.317512in}{2.426478in}}%
\pgfpathlineto{\pgfqpoint{1.329725in}{2.337000in}}%
\pgfpathlineto{\pgfqpoint{1.341938in}{2.244200in}}%
\pgfpathlineto{\pgfqpoint{1.354151in}{2.148607in}}%
\pgfpathlineto{\pgfqpoint{1.370434in}{2.017746in}}%
\pgfpathlineto{\pgfqpoint{1.390789in}{1.850570in}}%
\pgfpathlineto{\pgfqpoint{1.435569in}{1.481140in}}%
\pgfpathlineto{\pgfqpoint{1.451853in}{1.350492in}}%
\pgfpathlineto{\pgfqpoint{1.464066in}{1.255111in}}%
\pgfpathlineto{\pgfqpoint{1.476279in}{1.162568in}}%
\pgfpathlineto{\pgfqpoint{1.488492in}{1.073390in}}%
\pgfpathlineto{\pgfqpoint{1.496633in}{1.016059in}}%
\pgfpathlineto{\pgfqpoint{1.504775in}{0.960594in}}%
\pgfpathlineto{\pgfqpoint{1.512917in}{0.907137in}}%
\pgfpathlineto{\pgfqpoint{1.521059in}{0.855823in}}%
\pgfpathlineto{\pgfqpoint{1.529201in}{0.806781in}}%
\pgfpathlineto{\pgfqpoint{1.537343in}{0.760135in}}%
\pgfpathlineto{\pgfqpoint{1.545485in}{0.716004in}}%
\pgfpathlineto{\pgfqpoint{1.553627in}{0.674499in}}%
\pgfpathlineto{\pgfqpoint{1.561768in}{0.635725in}}%
\pgfpathlineto{\pgfqpoint{1.569910in}{0.599781in}}%
\pgfpathlineto{\pgfqpoint{1.578052in}{0.566757in}}%
\pgfpathlineto{\pgfqpoint{1.582123in}{0.551367in}}%
\pgfpathlineto{\pgfqpoint{1.586194in}{0.536738in}}%
\pgfpathlineto{\pgfqpoint{1.590265in}{0.522878in}}%
\pgfpathlineto{\pgfqpoint{1.594336in}{0.509798in}}%
\pgfpathlineto{\pgfqpoint{1.598407in}{0.497505in}}%
\pgfpathlineto{\pgfqpoint{1.602478in}{0.486006in}}%
\pgfpathlineto{\pgfqpoint{1.606549in}{0.475310in}}%
\pgfpathlineto{\pgfqpoint{1.610620in}{0.465423in}}%
\pgfpathlineto{\pgfqpoint{1.614691in}{0.456352in}}%
\pgfpathlineto{\pgfqpoint{1.618762in}{0.448101in}}%
\pgfpathlineto{\pgfqpoint{1.622832in}{0.440676in}}%
\pgfpathlineto{\pgfqpoint{1.626903in}{0.434083in}}%
\pgfpathlineto{\pgfqpoint{1.630974in}{0.428324in}}%
\pgfpathlineto{\pgfqpoint{1.635045in}{0.423404in}}%
\pgfpathlineto{\pgfqpoint{1.639116in}{0.419326in}}%
\pgfpathlineto{\pgfqpoint{1.643187in}{0.416093in}}%
\pgfpathlineto{\pgfqpoint{1.647258in}{0.413705in}}%
\pgfpathlineto{\pgfqpoint{1.651329in}{0.412166in}}%
\pgfpathlineto{\pgfqpoint{1.655400in}{0.411476in}}%
\pgfpathlineto{\pgfqpoint{1.659471in}{0.411636in}}%
\pgfpathlineto{\pgfqpoint{1.663542in}{0.412644in}}%
\pgfpathlineto{\pgfqpoint{1.667613in}{0.414501in}}%
\pgfpathlineto{\pgfqpoint{1.671684in}{0.417206in}}%
\pgfpathlineto{\pgfqpoint{1.675755in}{0.420757in}}%
\pgfpathlineto{\pgfqpoint{1.679826in}{0.425151in}}%
\pgfpathlineto{\pgfqpoint{1.683896in}{0.430385in}}%
\pgfpathlineto{\pgfqpoint{1.687967in}{0.436457in}}%
\pgfpathlineto{\pgfqpoint{1.692038in}{0.443363in}}%
\pgfpathlineto{\pgfqpoint{1.696109in}{0.451098in}}%
\pgfpathlineto{\pgfqpoint{1.700180in}{0.459658in}}%
\pgfpathlineto{\pgfqpoint{1.704251in}{0.469036in}}%
\pgfpathlineto{\pgfqpoint{1.708322in}{0.479227in}}%
\pgfpathlineto{\pgfqpoint{1.712393in}{0.490225in}}%
\pgfpathlineto{\pgfqpoint{1.716464in}{0.502022in}}%
\pgfpathlineto{\pgfqpoint{1.720535in}{0.514611in}}%
\pgfpathlineto{\pgfqpoint{1.724606in}{0.527985in}}%
\pgfpathlineto{\pgfqpoint{1.728677in}{0.542134in}}%
\pgfpathlineto{\pgfqpoint{1.732748in}{0.557050in}}%
\pgfpathlineto{\pgfqpoint{1.736819in}{0.572723in}}%
\pgfpathlineto{\pgfqpoint{1.744960in}{0.606301in}}%
\pgfpathlineto{\pgfqpoint{1.753102in}{0.642783in}}%
\pgfpathlineto{\pgfqpoint{1.761244in}{0.682076in}}%
\pgfpathlineto{\pgfqpoint{1.769386in}{0.724082in}}%
\pgfpathlineto{\pgfqpoint{1.777528in}{0.768693in}}%
\pgfpathlineto{\pgfqpoint{1.785670in}{0.815797in}}%
\pgfpathlineto{\pgfqpoint{1.793812in}{0.865275in}}%
\pgfpathlineto{\pgfqpoint{1.801954in}{0.917001in}}%
\pgfpathlineto{\pgfqpoint{1.810095in}{0.970845in}}%
\pgfpathlineto{\pgfqpoint{1.818237in}{1.026670in}}%
\pgfpathlineto{\pgfqpoint{1.830450in}{1.113812in}}%
\pgfpathlineto{\pgfqpoint{1.842663in}{1.204597in}}%
\pgfpathlineto{\pgfqpoint{1.854876in}{1.298507in}}%
\pgfpathlineto{\pgfqpoint{1.867089in}{1.395008in}}%
\pgfpathlineto{\pgfqpoint{1.883372in}{1.526755in}}%
\pgfpathlineto{\pgfqpoint{1.907798in}{1.728256in}}%
\pgfpathlineto{\pgfqpoint{1.940365in}{1.997024in}}%
\pgfpathlineto{\pgfqpoint{1.956649in}{2.128390in}}%
\pgfpathlineto{\pgfqpoint{1.968862in}{2.224497in}}%
\pgfpathlineto{\pgfqpoint{1.981075in}{2.317923in}}%
\pgfpathlineto{\pgfqpoint{1.993287in}{2.408136in}}%
\pgfpathlineto{\pgfqpoint{2.005500in}{2.494622in}}%
\pgfpathlineto{\pgfqpoint{2.013642in}{2.549965in}}%
\pgfpathlineto{\pgfqpoint{2.021784in}{2.603292in}}%
\pgfpathlineto{\pgfqpoint{2.029926in}{2.654468in}}%
\pgfpathlineto{\pgfqpoint{2.038068in}{2.703364in}}%
\pgfpathlineto{\pgfqpoint{2.046210in}{2.749856in}}%
\pgfpathlineto{\pgfqpoint{2.054352in}{2.793826in}}%
\pgfpathlineto{\pgfqpoint{2.062493in}{2.835163in}}%
\pgfpathlineto{\pgfqpoint{2.070635in}{2.873763in}}%
\pgfpathlineto{\pgfqpoint{2.078777in}{2.909527in}}%
\pgfpathlineto{\pgfqpoint{2.086919in}{2.942365in}}%
\pgfpathlineto{\pgfqpoint{2.090990in}{2.957661in}}%
\pgfpathlineto{\pgfqpoint{2.095061in}{2.972195in}}%
\pgfpathlineto{\pgfqpoint{2.099132in}{2.985957in}}%
\pgfpathlineto{\pgfqpoint{2.103203in}{2.998940in}}%
\pgfpathlineto{\pgfqpoint{2.107274in}{3.011134in}}%
\pgfpathlineto{\pgfqpoint{2.111345in}{3.022532in}}%
\pgfpathlineto{\pgfqpoint{2.115416in}{3.033128in}}%
\pgfpathlineto{\pgfqpoint{2.119486in}{3.042913in}}%
\pgfpathlineto{\pgfqpoint{2.123557in}{3.051883in}}%
\pgfpathlineto{\pgfqpoint{2.127628in}{3.060030in}}%
\pgfpathlineto{\pgfqpoint{2.131699in}{3.067351in}}%
\pgfpathlineto{\pgfqpoint{2.135770in}{3.073841in}}%
\pgfpathlineto{\pgfqpoint{2.139841in}{3.079495in}}%
\pgfpathlineto{\pgfqpoint{2.143912in}{3.084309in}}%
\pgfpathlineto{\pgfqpoint{2.147983in}{3.088282in}}%
\pgfpathlineto{\pgfqpoint{2.152054in}{3.091410in}}%
\pgfpathlineto{\pgfqpoint{2.156125in}{3.093691in}}%
\pgfpathlineto{\pgfqpoint{2.160196in}{3.095124in}}%
\pgfpathlineto{\pgfqpoint{2.164267in}{3.095708in}}%
\pgfpathlineto{\pgfqpoint{2.168338in}{3.095443in}}%
\pgfpathlineto{\pgfqpoint{2.172409in}{3.094328in}}%
\pgfpathlineto{\pgfqpoint{2.176480in}{3.092365in}}%
\pgfpathlineto{\pgfqpoint{2.180550in}{3.089554in}}%
\pgfpathlineto{\pgfqpoint{2.184621in}{3.085898in}}%
\pgfpathlineto{\pgfqpoint{2.188692in}{3.081399in}}%
\pgfpathlineto{\pgfqpoint{2.192763in}{3.076059in}}%
\pgfpathlineto{\pgfqpoint{2.196834in}{3.069883in}}%
\pgfpathlineto{\pgfqpoint{2.200905in}{3.062873in}}%
\pgfpathlineto{\pgfqpoint{2.204976in}{3.055035in}}%
\pgfpathlineto{\pgfqpoint{2.209047in}{3.046373in}}%
\pgfpathlineto{\pgfqpoint{2.213118in}{3.036892in}}%
\pgfpathlineto{\pgfqpoint{2.217189in}{3.026600in}}%
\pgfpathlineto{\pgfqpoint{2.221260in}{3.015502in}}%
\pgfpathlineto{\pgfqpoint{2.225331in}{3.003605in}}%
\pgfpathlineto{\pgfqpoint{2.229402in}{2.990917in}}%
\pgfpathlineto{\pgfqpoint{2.233473in}{2.977447in}}%
\pgfpathlineto{\pgfqpoint{2.237544in}{2.963201in}}%
\pgfpathlineto{\pgfqpoint{2.241615in}{2.948190in}}%
\pgfpathlineto{\pgfqpoint{2.245685in}{2.932423in}}%
\pgfpathlineto{\pgfqpoint{2.253827in}{2.898661in}}%
\pgfpathlineto{\pgfqpoint{2.261969in}{2.862001in}}%
\pgfpathlineto{\pgfqpoint{2.270111in}{2.822535in}}%
\pgfpathlineto{\pgfqpoint{2.278253in}{2.780363in}}%
\pgfpathlineto{\pgfqpoint{2.286395in}{2.735593in}}%
\pgfpathlineto{\pgfqpoint{2.294537in}{2.688336in}}%
\pgfpathlineto{\pgfqpoint{2.302679in}{2.638714in}}%
\pgfpathlineto{\pgfqpoint{2.310820in}{2.586852in}}%
\pgfpathlineto{\pgfqpoint{2.318962in}{2.532880in}}%
\pgfpathlineto{\pgfqpoint{2.327104in}{2.476936in}}%
\pgfpathlineto{\pgfqpoint{2.339317in}{2.389632in}}%
\pgfpathlineto{\pgfqpoint{2.351530in}{2.298707in}}%
\pgfpathlineto{\pgfqpoint{2.363743in}{2.204678in}}%
\pgfpathlineto{\pgfqpoint{2.375955in}{2.108081in}}%
\pgfpathlineto{\pgfqpoint{2.392239in}{1.976241in}}%
\pgfpathlineto{\pgfqpoint{2.416665in}{1.774681in}}%
\pgfpathlineto{\pgfqpoint{2.449232in}{1.505984in}}%
\pgfpathlineto{\pgfqpoint{2.465516in}{1.374717in}}%
\pgfpathlineto{\pgfqpoint{2.477729in}{1.278711in}}%
\pgfpathlineto{\pgfqpoint{2.489942in}{1.185408in}}%
\pgfpathlineto{\pgfqpoint{2.502154in}{1.095341in}}%
\pgfpathlineto{\pgfqpoint{2.514367in}{1.009021in}}%
\pgfpathlineto{\pgfqpoint{2.522509in}{0.953800in}}%
\pgfpathlineto{\pgfqpoint{2.530651in}{0.900603in}}%
\pgfpathlineto{\pgfqpoint{2.538793in}{0.849566in}}%
\pgfpathlineto{\pgfqpoint{2.546935in}{0.800816in}}%
\pgfpathlineto{\pgfqpoint{2.555076in}{0.754479in}}%
\pgfpathlineto{\pgfqpoint{2.563218in}{0.710670in}}%
\pgfpathlineto{\pgfqpoint{2.571360in}{0.669501in}}%
\pgfpathlineto{\pgfqpoint{2.579502in}{0.631076in}}%
\pgfpathlineto{\pgfqpoint{2.587644in}{0.595492in}}%
\pgfpathlineto{\pgfqpoint{2.595786in}{0.562839in}}%
\pgfpathlineto{\pgfqpoint{2.599857in}{0.547638in}}%
\pgfpathlineto{\pgfqpoint{2.603928in}{0.533200in}}%
\pgfpathlineto{\pgfqpoint{2.607999in}{0.519535in}}%
\pgfpathlineto{\pgfqpoint{2.612070in}{0.506651in}}%
\pgfpathlineto{\pgfqpoint{2.616140in}{0.494555in}}%
\pgfpathlineto{\pgfqpoint{2.620211in}{0.483257in}}%
\pgfpathlineto{\pgfqpoint{2.624282in}{0.472763in}}%
\pgfpathlineto{\pgfqpoint{2.628353in}{0.463079in}}%
\pgfpathlineto{\pgfqpoint{2.632424in}{0.454212in}}%
\pgfpathlineto{\pgfqpoint{2.636495in}{0.446167in}}%
\pgfpathlineto{\pgfqpoint{2.640566in}{0.438950in}}%
\pgfpathlineto{\pgfqpoint{2.644637in}{0.432564in}}%
\pgfpathlineto{\pgfqpoint{2.648708in}{0.427015in}}%
\pgfpathlineto{\pgfqpoint{2.652779in}{0.422306in}}%
\pgfpathlineto{\pgfqpoint{2.656850in}{0.418438in}}%
\pgfpathlineto{\pgfqpoint{2.660921in}{0.415416in}}%
\pgfpathlineto{\pgfqpoint{2.664992in}{0.413241in}}%
\pgfpathlineto{\pgfqpoint{2.669063in}{0.411914in}}%
\pgfpathlineto{\pgfqpoint{2.673134in}{0.411436in}}%
\pgfpathlineto{\pgfqpoint{2.677205in}{0.411808in}}%
\pgfpathlineto{\pgfqpoint{2.681275in}{0.413029in}}%
\pgfpathlineto{\pgfqpoint{2.685346in}{0.415098in}}%
\pgfpathlineto{\pgfqpoint{2.689417in}{0.418014in}}%
\pgfpathlineto{\pgfqpoint{2.693488in}{0.421776in}}%
\pgfpathlineto{\pgfqpoint{2.697559in}{0.426381in}}%
\pgfpathlineto{\pgfqpoint{2.701630in}{0.431825in}}%
\pgfpathlineto{\pgfqpoint{2.705701in}{0.438106in}}%
\pgfpathlineto{\pgfqpoint{2.709772in}{0.445219in}}%
\pgfpathlineto{\pgfqpoint{2.713843in}{0.453161in}}%
\pgfpathlineto{\pgfqpoint{2.717914in}{0.461926in}}%
\pgfpathlineto{\pgfqpoint{2.721985in}{0.471508in}}%
\pgfpathlineto{\pgfqpoint{2.726056in}{0.481901in}}%
\pgfpathlineto{\pgfqpoint{2.730127in}{0.493099in}}%
\pgfpathlineto{\pgfqpoint{2.734198in}{0.505095in}}%
\pgfpathlineto{\pgfqpoint{2.738269in}{0.517882in}}%
\pgfpathlineto{\pgfqpoint{2.742339in}{0.531450in}}%
\pgfpathlineto{\pgfqpoint{2.746410in}{0.545791in}}%
\pgfpathlineto{\pgfqpoint{2.750481in}{0.560897in}}%
\pgfpathlineto{\pgfqpoint{2.758623in}{0.593364in}}%
\pgfpathlineto{\pgfqpoint{2.766765in}{0.628767in}}%
\pgfpathlineto{\pgfqpoint{2.774907in}{0.667018in}}%
\pgfpathlineto{\pgfqpoint{2.783049in}{0.708018in}}%
\pgfpathlineto{\pgfqpoint{2.791191in}{0.751665in}}%
\pgfpathlineto{\pgfqpoint{2.799333in}{0.797848in}}%
\pgfpathlineto{\pgfqpoint{2.807474in}{0.846451in}}%
\pgfpathlineto{\pgfqpoint{2.815616in}{0.897349in}}%
\pgfpathlineto{\pgfqpoint{2.823758in}{0.950414in}}%
\pgfpathlineto{\pgfqpoint{2.831900in}{1.005513in}}%
\pgfpathlineto{\pgfqpoint{2.840042in}{1.062504in}}%
\pgfpathlineto{\pgfqpoint{2.852255in}{1.151226in}}%
\pgfpathlineto{\pgfqpoint{2.864468in}{1.243376in}}%
\pgfpathlineto{\pgfqpoint{2.876680in}{1.338432in}}%
\pgfpathlineto{\pgfqpoint{2.892964in}{1.468754in}}%
\pgfpathlineto{\pgfqpoint{2.913319in}{1.635552in}}%
\pgfpathlineto{\pgfqpoint{2.962170in}{2.038403in}}%
\pgfpathlineto{\pgfqpoint{2.978454in}{2.168726in}}%
\pgfpathlineto{\pgfqpoint{2.990666in}{2.263781in}}%
\pgfpathlineto{\pgfqpoint{3.002879in}{2.355932in}}%
\pgfpathlineto{\pgfqpoint{3.015092in}{2.444653in}}%
\pgfpathlineto{\pgfqpoint{3.023234in}{2.501645in}}%
\pgfpathlineto{\pgfqpoint{3.031376in}{2.556743in}}%
\pgfpathlineto{\pgfqpoint{3.039518in}{2.609809in}}%
\pgfpathlineto{\pgfqpoint{3.047660in}{2.660707in}}%
\pgfpathlineto{\pgfqpoint{3.055801in}{2.709309in}}%
\pgfpathlineto{\pgfqpoint{3.063943in}{2.755492in}}%
\pgfpathlineto{\pgfqpoint{3.072085in}{2.799139in}}%
\pgfpathlineto{\pgfqpoint{3.080227in}{2.840140in}}%
\pgfpathlineto{\pgfqpoint{3.088369in}{2.878390in}}%
\pgfpathlineto{\pgfqpoint{3.096511in}{2.913794in}}%
\pgfpathlineto{\pgfqpoint{3.104653in}{2.946260in}}%
\pgfpathlineto{\pgfqpoint{3.108724in}{2.961366in}}%
\pgfpathlineto{\pgfqpoint{3.112795in}{2.975708in}}%
\pgfpathlineto{\pgfqpoint{3.116865in}{2.989276in}}%
\pgfpathlineto{\pgfqpoint{3.120936in}{3.002062in}}%
\pgfpathlineto{\pgfqpoint{3.125007in}{3.014058in}}%
\pgfpathlineto{\pgfqpoint{3.129078in}{3.025257in}}%
\pgfpathlineto{\pgfqpoint{3.133149in}{3.035650in}}%
\pgfpathlineto{\pgfqpoint{3.137220in}{3.045232in}}%
\pgfpathlineto{\pgfqpoint{3.141291in}{3.053997in}}%
\pgfpathlineto{\pgfqpoint{3.145362in}{3.061938in}}%
\pgfpathlineto{\pgfqpoint{3.149433in}{3.069052in}}%
\pgfpathlineto{\pgfqpoint{3.153504in}{3.075333in}}%
\pgfpathlineto{\pgfqpoint{3.157575in}{3.080777in}}%
\pgfpathlineto{\pgfqpoint{3.161646in}{3.085382in}}%
\pgfpathlineto{\pgfqpoint{3.165717in}{3.089143in}}%
\pgfpathlineto{\pgfqpoint{3.169788in}{3.092060in}}%
\pgfpathlineto{\pgfqpoint{3.173859in}{3.094129in}}%
\pgfpathlineto{\pgfqpoint{3.177929in}{3.095350in}}%
\pgfpathlineto{\pgfqpoint{3.182000in}{3.095721in}}%
\pgfpathlineto{\pgfqpoint{3.186071in}{3.095243in}}%
\pgfpathlineto{\pgfqpoint{3.190142in}{3.093917in}}%
\pgfpathlineto{\pgfqpoint{3.194213in}{3.091741in}}%
\pgfpathlineto{\pgfqpoint{3.198284in}{3.088719in}}%
\pgfpathlineto{\pgfqpoint{3.202355in}{3.084852in}}%
\pgfpathlineto{\pgfqpoint{3.206426in}{3.080143in}}%
\pgfpathlineto{\pgfqpoint{3.210497in}{3.074593in}}%
\pgfpathlineto{\pgfqpoint{3.214568in}{3.068208in}}%
\pgfpathlineto{\pgfqpoint{3.218639in}{3.060991in}}%
\pgfpathlineto{\pgfqpoint{3.222710in}{3.052946in}}%
\pgfpathlineto{\pgfqpoint{3.226781in}{3.044079in}}%
\pgfpathlineto{\pgfqpoint{3.230852in}{3.034395in}}%
\pgfpathlineto{\pgfqpoint{3.234923in}{3.023901in}}%
\pgfpathlineto{\pgfqpoint{3.238993in}{3.012602in}}%
\pgfpathlineto{\pgfqpoint{3.243064in}{3.000507in}}%
\pgfpathlineto{\pgfqpoint{3.247135in}{2.987623in}}%
\pgfpathlineto{\pgfqpoint{3.251206in}{2.973957in}}%
\pgfpathlineto{\pgfqpoint{3.255277in}{2.959520in}}%
\pgfpathlineto{\pgfqpoint{3.259348in}{2.944319in}}%
\pgfpathlineto{\pgfqpoint{3.267490in}{2.911666in}}%
\pgfpathlineto{\pgfqpoint{3.275632in}{2.876082in}}%
\pgfpathlineto{\pgfqpoint{3.283774in}{2.837657in}}%
\pgfpathlineto{\pgfqpoint{3.291916in}{2.796488in}}%
\pgfpathlineto{\pgfqpoint{3.300058in}{2.752679in}}%
\pgfpathlineto{\pgfqpoint{3.308199in}{2.706341in}}%
\pgfpathlineto{\pgfqpoint{3.316341in}{2.657592in}}%
\pgfpathlineto{\pgfqpoint{3.324483in}{2.606554in}}%
\pgfpathlineto{\pgfqpoint{3.332625in}{2.553358in}}%
\pgfpathlineto{\pgfqpoint{3.340767in}{2.498137in}}%
\pgfpathlineto{\pgfqpoint{3.348909in}{2.441032in}}%
\pgfpathlineto{\pgfqpoint{3.361122in}{2.352157in}}%
\pgfpathlineto{\pgfqpoint{3.373334in}{2.259875in}}%
\pgfpathlineto{\pgfqpoint{3.385547in}{2.164710in}}%
\pgfpathlineto{\pgfqpoint{3.401831in}{2.034277in}}%
\pgfpathlineto{\pgfqpoint{3.422186in}{1.867401in}}%
\pgfpathlineto{\pgfqpoint{3.471037in}{1.464631in}}%
\pgfpathlineto{\pgfqpoint{3.487321in}{1.334421in}}%
\pgfpathlineto{\pgfqpoint{3.499533in}{1.239475in}}%
\pgfpathlineto{\pgfqpoint{3.511746in}{1.147457in}}%
\pgfpathlineto{\pgfqpoint{3.523959in}{1.058890in}}%
\pgfpathlineto{\pgfqpoint{3.532101in}{1.002012in}}%
\pgfpathlineto{\pgfqpoint{3.540243in}{0.947037in}}%
\pgfpathlineto{\pgfqpoint{3.548385in}{0.894103in}}%
\pgfpathlineto{\pgfqpoint{3.556526in}{0.843344in}}%
\pgfpathlineto{\pgfqpoint{3.564668in}{0.794890in}}%
\pgfpathlineto{\pgfqpoint{3.572810in}{0.748862in}}%
\pgfpathlineto{\pgfqpoint{3.580952in}{0.705377in}}%
\pgfpathlineto{\pgfqpoint{3.589094in}{0.664545in}}%
\pgfpathlineto{\pgfqpoint{3.597236in}{0.626470in}}%
\pgfpathlineto{\pgfqpoint{3.605378in}{0.591248in}}%
\pgfpathlineto{\pgfqpoint{3.613519in}{0.558968in}}%
\pgfpathlineto{\pgfqpoint{3.617590in}{0.543957in}}%
\pgfpathlineto{\pgfqpoint{3.621661in}{0.529711in}}%
\pgfpathlineto{\pgfqpoint{3.625732in}{0.516240in}}%
\pgfpathlineto{\pgfqpoint{3.629803in}{0.503552in}}%
\pgfpathlineto{\pgfqpoint{3.633874in}{0.491656in}}%
\pgfpathlineto{\pgfqpoint{3.637945in}{0.480558in}}%
\pgfpathlineto{\pgfqpoint{3.642016in}{0.470265in}}%
\pgfpathlineto{\pgfqpoint{3.646087in}{0.460785in}}%
\pgfpathlineto{\pgfqpoint{3.650158in}{0.452123in}}%
\pgfpathlineto{\pgfqpoint{3.654229in}{0.444285in}}%
\pgfpathlineto{\pgfqpoint{3.658300in}{0.437275in}}%
\pgfpathlineto{\pgfqpoint{3.662371in}{0.431099in}}%
\pgfpathlineto{\pgfqpoint{3.666442in}{0.425759in}}%
\pgfpathlineto{\pgfqpoint{3.670513in}{0.421260in}}%
\pgfpathlineto{\pgfqpoint{3.674584in}{0.417604in}}%
\pgfpathlineto{\pgfqpoint{3.678654in}{0.414793in}}%
\pgfpathlineto{\pgfqpoint{3.682725in}{0.412830in}}%
\pgfpathlineto{\pgfqpoint{3.686796in}{0.411715in}}%
\pgfpathlineto{\pgfqpoint{3.690867in}{0.411450in}}%
\pgfpathlineto{\pgfqpoint{3.694938in}{0.412034in}}%
\pgfpathlineto{\pgfqpoint{3.699009in}{0.413467in}}%
\pgfpathlineto{\pgfqpoint{3.703080in}{0.415748in}}%
\pgfpathlineto{\pgfqpoint{3.707151in}{0.418876in}}%
\pgfpathlineto{\pgfqpoint{3.711222in}{0.422848in}}%
\pgfpathlineto{\pgfqpoint{3.715293in}{0.427663in}}%
\pgfpathlineto{\pgfqpoint{3.719364in}{0.433317in}}%
\pgfpathlineto{\pgfqpoint{3.723435in}{0.439806in}}%
\pgfpathlineto{\pgfqpoint{3.727506in}{0.447127in}}%
\pgfpathlineto{\pgfqpoint{3.731577in}{0.455275in}}%
\pgfpathlineto{\pgfqpoint{3.735648in}{0.464245in}}%
\pgfpathlineto{\pgfqpoint{3.739718in}{0.474030in}}%
\pgfpathlineto{\pgfqpoint{3.743789in}{0.484625in}}%
\pgfpathlineto{\pgfqpoint{3.747860in}{0.496024in}}%
\pgfpathlineto{\pgfqpoint{3.751931in}{0.508218in}}%
\pgfpathlineto{\pgfqpoint{3.756002in}{0.521201in}}%
\pgfpathlineto{\pgfqpoint{3.760073in}{0.534963in}}%
\pgfpathlineto{\pgfqpoint{3.764144in}{0.549497in}}%
\pgfpathlineto{\pgfqpoint{3.768215in}{0.564792in}}%
\pgfpathlineto{\pgfqpoint{3.776357in}{0.597631in}}%
\pgfpathlineto{\pgfqpoint{3.784499in}{0.633395in}}%
\pgfpathlineto{\pgfqpoint{3.792641in}{0.671994in}}%
\pgfpathlineto{\pgfqpoint{3.800782in}{0.713332in}}%
\pgfpathlineto{\pgfqpoint{3.808924in}{0.757302in}}%
\pgfpathlineto{\pgfqpoint{3.817066in}{0.803794in}}%
\pgfpathlineto{\pgfqpoint{3.825208in}{0.852690in}}%
\pgfpathlineto{\pgfqpoint{3.833350in}{0.903866in}}%
\pgfpathlineto{\pgfqpoint{3.841492in}{0.957193in}}%
\pgfpathlineto{\pgfqpoint{3.849634in}{1.012536in}}%
\pgfpathlineto{\pgfqpoint{3.857776in}{1.069754in}}%
\pgfpathlineto{\pgfqpoint{3.869988in}{1.158781in}}%
\pgfpathlineto{\pgfqpoint{3.882201in}{1.251194in}}%
\pgfpathlineto{\pgfqpoint{3.894414in}{1.346468in}}%
\pgfpathlineto{\pgfqpoint{3.910698in}{1.477009in}}%
\pgfpathlineto{\pgfqpoint{3.931052in}{1.643963in}}%
\pgfpathlineto{\pgfqpoint{3.979904in}{2.046647in}}%
\pgfpathlineto{\pgfqpoint{3.996187in}{2.176745in}}%
\pgfpathlineto{\pgfqpoint{4.008400in}{2.271579in}}%
\pgfpathlineto{\pgfqpoint{4.020613in}{2.363464in}}%
\pgfpathlineto{\pgfqpoint{4.032826in}{2.451876in}}%
\pgfpathlineto{\pgfqpoint{4.040968in}{2.508639in}}%
\pgfpathlineto{\pgfqpoint{4.049109in}{2.563491in}}%
\pgfpathlineto{\pgfqpoint{4.057251in}{2.616292in}}%
\pgfpathlineto{\pgfqpoint{4.065393in}{2.666911in}}%
\pgfpathlineto{\pgfqpoint{4.073535in}{2.715217in}}%
\pgfpathlineto{\pgfqpoint{4.081677in}{2.761089in}}%
\pgfpathlineto{\pgfqpoint{4.089819in}{2.804411in}}%
\pgfpathlineto{\pgfqpoint{4.097961in}{2.845074in}}%
\pgfpathlineto{\pgfqpoint{4.106103in}{2.882973in}}%
\pgfpathlineto{\pgfqpoint{4.114244in}{2.918014in}}%
\pgfpathlineto{\pgfqpoint{4.122386in}{2.950108in}}%
\pgfpathlineto{\pgfqpoint{4.126457in}{2.965024in}}%
\pgfpathlineto{\pgfqpoint{4.130528in}{2.979173in}}%
\pgfpathlineto{\pgfqpoint{4.134599in}{2.992546in}}%
\pgfpathlineto{\pgfqpoint{4.138670in}{3.005136in}}%
\pgfpathlineto{\pgfqpoint{4.142741in}{3.016933in}}%
\pgfpathlineto{\pgfqpoint{4.146812in}{3.027931in}}%
\pgfpathlineto{\pgfqpoint{4.150883in}{3.038122in}}%
\pgfpathlineto{\pgfqpoint{4.154954in}{3.047500in}}%
\pgfpathlineto{\pgfqpoint{4.159025in}{3.056059in}}%
\pgfpathlineto{\pgfqpoint{4.163096in}{3.063794in}}%
\pgfpathlineto{\pgfqpoint{4.167167in}{3.070700in}}%
\pgfpathlineto{\pgfqpoint{4.171238in}{3.076772in}}%
\pgfpathlineto{\pgfqpoint{4.175308in}{3.082007in}}%
\pgfpathlineto{\pgfqpoint{4.179379in}{3.086401in}}%
\pgfpathlineto{\pgfqpoint{4.183450in}{3.089952in}}%
\pgfpathlineto{\pgfqpoint{4.187521in}{3.092656in}}%
\pgfpathlineto{\pgfqpoint{4.191592in}{3.094514in}}%
\pgfpathlineto{\pgfqpoint{4.195663in}{3.095522in}}%
\pgfpathlineto{\pgfqpoint{4.199734in}{3.095681in}}%
\pgfpathlineto{\pgfqpoint{4.203805in}{3.094991in}}%
\pgfpathlineto{\pgfqpoint{4.207876in}{3.093452in}}%
\pgfpathlineto{\pgfqpoint{4.211947in}{3.091065in}}%
\pgfpathlineto{\pgfqpoint{4.216018in}{3.087832in}}%
\pgfpathlineto{\pgfqpoint{4.220089in}{3.083754in}}%
\pgfpathlineto{\pgfqpoint{4.224160in}{3.078834in}}%
\pgfpathlineto{\pgfqpoint{4.228231in}{3.073075in}}%
\pgfpathlineto{\pgfqpoint{4.232302in}{3.066482in}}%
\pgfpathlineto{\pgfqpoint{4.236372in}{3.059057in}}%
\pgfpathlineto{\pgfqpoint{4.240443in}{3.050806in}}%
\pgfpathlineto{\pgfqpoint{4.244514in}{3.041734in}}%
\pgfpathlineto{\pgfqpoint{4.248585in}{3.031847in}}%
\pgfpathlineto{\pgfqpoint{4.252656in}{3.021151in}}%
\pgfpathlineto{\pgfqpoint{4.256727in}{3.009653in}}%
\pgfpathlineto{\pgfqpoint{4.260798in}{2.997360in}}%
\pgfpathlineto{\pgfqpoint{4.264869in}{2.984279in}}%
\pgfpathlineto{\pgfqpoint{4.268940in}{2.970420in}}%
\pgfpathlineto{\pgfqpoint{4.273011in}{2.955791in}}%
\pgfpathlineto{\pgfqpoint{4.277082in}{2.940400in}}%
\pgfpathlineto{\pgfqpoint{4.285224in}{2.907377in}}%
\pgfpathlineto{\pgfqpoint{4.293366in}{2.871432in}}%
\pgfpathlineto{\pgfqpoint{4.301507in}{2.832659in}}%
\pgfpathlineto{\pgfqpoint{4.309649in}{2.791154in}}%
\pgfpathlineto{\pgfqpoint{4.317791in}{2.747023in}}%
\pgfpathlineto{\pgfqpoint{4.325933in}{2.700377in}}%
\pgfpathlineto{\pgfqpoint{4.334075in}{2.651335in}}%
\pgfpathlineto{\pgfqpoint{4.342217in}{2.600020in}}%
\pgfpathlineto{\pgfqpoint{4.350359in}{2.546563in}}%
\pgfpathlineto{\pgfqpoint{4.358501in}{2.491099in}}%
\pgfpathlineto{\pgfqpoint{4.366642in}{2.433768in}}%
\pgfpathlineto{\pgfqpoint{4.378855in}{2.344590in}}%
\pgfpathlineto{\pgfqpoint{4.391068in}{2.252047in}}%
\pgfpathlineto{\pgfqpoint{4.403281in}{2.156666in}}%
\pgfpathlineto{\pgfqpoint{4.419565in}{2.026017in}}%
\pgfpathlineto{\pgfqpoint{4.439919in}{1.858987in}}%
\pgfpathlineto{\pgfqpoint{4.484700in}{1.489411in}}%
\pgfpathlineto{\pgfqpoint{4.500983in}{1.358551in}}%
\pgfpathlineto{\pgfqpoint{4.513196in}{1.262958in}}%
\pgfpathlineto{\pgfqpoint{4.525409in}{1.170158in}}%
\pgfpathlineto{\pgfqpoint{4.537622in}{1.080680in}}%
\pgfpathlineto{\pgfqpoint{4.545764in}{1.023126in}}%
\pgfpathlineto{\pgfqpoint{4.553905in}{0.967421in}}%
\pgfpathlineto{\pgfqpoint{4.562047in}{0.913705in}}%
\pgfpathlineto{\pgfqpoint{4.570189in}{0.862116in}}%
\pgfpathlineto{\pgfqpoint{4.578331in}{0.812782in}}%
\pgfpathlineto{\pgfqpoint{4.586473in}{0.765831in}}%
\pgfpathlineto{\pgfqpoint{4.594615in}{0.721379in}}%
\pgfpathlineto{\pgfqpoint{4.602757in}{0.679540in}}%
\pgfpathlineto{\pgfqpoint{4.610898in}{0.640419in}}%
\pgfpathlineto{\pgfqpoint{4.619040in}{0.604116in}}%
\pgfpathlineto{\pgfqpoint{4.627182in}{0.570723in}}%
\pgfpathlineto{\pgfqpoint{4.631253in}{0.555144in}}%
\pgfpathlineto{\pgfqpoint{4.635324in}{0.540323in}}%
\pgfpathlineto{\pgfqpoint{4.639395in}{0.526271in}}%
\pgfpathlineto{\pgfqpoint{4.643466in}{0.512995in}}%
\pgfpathlineto{\pgfqpoint{4.647537in}{0.500504in}}%
\pgfpathlineto{\pgfqpoint{4.651608in}{0.488806in}}%
\pgfpathlineto{\pgfqpoint{4.655679in}{0.477909in}}%
\pgfpathlineto{\pgfqpoint{4.659750in}{0.467819in}}%
\pgfpathlineto{\pgfqpoint{4.663821in}{0.458543in}}%
\pgfpathlineto{\pgfqpoint{4.667892in}{0.450086in}}%
\pgfpathlineto{\pgfqpoint{4.671963in}{0.442455in}}%
\pgfpathlineto{\pgfqpoint{4.676033in}{0.435653in}}%
\pgfpathlineto{\pgfqpoint{4.680104in}{0.429685in}}%
\pgfpathlineto{\pgfqpoint{4.684175in}{0.424555in}}%
\pgfpathlineto{\pgfqpoint{4.688246in}{0.420267in}}%
\pgfpathlineto{\pgfqpoint{4.692317in}{0.416822in}}%
\pgfpathlineto{\pgfqpoint{4.696388in}{0.414223in}}%
\pgfpathlineto{\pgfqpoint{4.700459in}{0.412472in}}%
\pgfpathlineto{\pgfqpoint{4.704530in}{0.411569in}}%
\pgfpathlineto{\pgfqpoint{4.708601in}{0.411516in}}%
\pgfpathlineto{\pgfqpoint{4.712672in}{0.412312in}}%
\pgfpathlineto{\pgfqpoint{4.716743in}{0.413958in}}%
\pgfpathlineto{\pgfqpoint{4.720814in}{0.416451in}}%
\pgfpathlineto{\pgfqpoint{4.724885in}{0.419790in}}%
\pgfpathlineto{\pgfqpoint{4.728956in}{0.423973in}}%
\pgfpathlineto{\pgfqpoint{4.733027in}{0.428998in}}%
\pgfpathlineto{\pgfqpoint{4.737097in}{0.434861in}}%
\pgfpathlineto{\pgfqpoint{4.741168in}{0.441559in}}%
\pgfpathlineto{\pgfqpoint{4.745239in}{0.449087in}}%
\pgfpathlineto{\pgfqpoint{4.749310in}{0.457441in}}%
\pgfpathlineto{\pgfqpoint{4.753381in}{0.466615in}}%
\pgfpathlineto{\pgfqpoint{4.757452in}{0.476603in}}%
\pgfpathlineto{\pgfqpoint{4.761523in}{0.487400in}}%
\pgfpathlineto{\pgfqpoint{4.765594in}{0.498998in}}%
\pgfpathlineto{\pgfqpoint{4.769665in}{0.511390in}}%
\pgfpathlineto{\pgfqpoint{4.773736in}{0.524568in}}%
\pgfpathlineto{\pgfqpoint{4.777807in}{0.538524in}}%
\pgfpathlineto{\pgfqpoint{4.781878in}{0.553249in}}%
\pgfpathlineto{\pgfqpoint{4.785949in}{0.568734in}}%
\pgfpathlineto{\pgfqpoint{4.794091in}{0.601943in}}%
\pgfpathlineto{\pgfqpoint{4.802232in}{0.638067in}}%
\pgfpathlineto{\pgfqpoint{4.810374in}{0.677014in}}%
\pgfpathlineto{\pgfqpoint{4.818516in}{0.718686in}}%
\pgfpathlineto{\pgfqpoint{4.826658in}{0.762978in}}%
\pgfpathlineto{\pgfqpoint{4.834800in}{0.809777in}}%
\pgfpathlineto{\pgfqpoint{4.842942in}{0.858965in}}%
\pgfpathlineto{\pgfqpoint{4.851084in}{0.910417in}}%
\pgfpathlineto{\pgfqpoint{4.859225in}{0.964004in}}%
\pgfpathlineto{\pgfqpoint{4.867367in}{1.019589in}}%
\pgfpathlineto{\pgfqpoint{4.875509in}{1.077031in}}%
\pgfpathlineto{\pgfqpoint{4.887722in}{1.166360in}}%
\pgfpathlineto{\pgfqpoint{4.899935in}{1.259032in}}%
\pgfpathlineto{\pgfqpoint{4.912148in}{1.354519in}}%
\pgfpathlineto{\pgfqpoint{4.928431in}{1.485275in}}%
\pgfpathlineto{\pgfqpoint{4.948786in}{1.652378in}}%
\pgfpathlineto{\pgfqpoint{4.960999in}{1.753579in}}%
\pgfpathlineto{\pgfqpoint{4.960999in}{1.753579in}}%
\pgfusepath{stroke}%
\end{pgfscope}%
\begin{pgfscope}%
\pgfsetrectcap%
\pgfsetmiterjoin%
\pgfsetlinewidth{0.803000pt}%
\definecolor{currentstroke}{rgb}{0.000000,0.000000,0.000000}%
\pgfsetstrokecolor{currentstroke}%
\pgfsetdash{}{0pt}%
\pgfpathmoveto{\pgfqpoint{0.690792in}{0.277222in}}%
\pgfpathlineto{\pgfqpoint{0.690792in}{3.229936in}}%
\pgfusepath{stroke}%
\end{pgfscope}%
\begin{pgfscope}%
\pgfsetrectcap%
\pgfsetmiterjoin%
\pgfsetlinewidth{0.803000pt}%
\definecolor{currentstroke}{rgb}{0.000000,0.000000,0.000000}%
\pgfsetstrokecolor{currentstroke}%
\pgfsetdash{}{0pt}%
\pgfpathmoveto{\pgfqpoint{5.164342in}{0.277222in}}%
\pgfpathlineto{\pgfqpoint{5.164342in}{3.229936in}}%
\pgfusepath{stroke}%
\end{pgfscope}%
\begin{pgfscope}%
\pgfsetrectcap%
\pgfsetmiterjoin%
\pgfsetlinewidth{0.803000pt}%
\definecolor{currentstroke}{rgb}{0.000000,0.000000,0.000000}%
\pgfsetstrokecolor{currentstroke}%
\pgfsetdash{}{0pt}%
\pgfpathmoveto{\pgfqpoint{0.690792in}{0.277222in}}%
\pgfpathlineto{\pgfqpoint{5.164342in}{0.277222in}}%
\pgfusepath{stroke}%
\end{pgfscope}%
\begin{pgfscope}%
\pgfsetrectcap%
\pgfsetmiterjoin%
\pgfsetlinewidth{0.803000pt}%
\definecolor{currentstroke}{rgb}{0.000000,0.000000,0.000000}%
\pgfsetstrokecolor{currentstroke}%
\pgfsetdash{}{0pt}%
\pgfpathmoveto{\pgfqpoint{0.690792in}{3.229936in}}%
\pgfpathlineto{\pgfqpoint{5.164342in}{3.229936in}}%
\pgfusepath{stroke}%
\end{pgfscope}%
\begin{pgfscope}%
\pgfsetbuttcap%
\pgfsetmiterjoin%
\definecolor{currentfill}{rgb}{1.000000,1.000000,1.000000}%
\pgfsetfillcolor{currentfill}%
\pgfsetfillopacity{0.800000}%
\pgfsetlinewidth{1.003750pt}%
\definecolor{currentstroke}{rgb}{0.800000,0.800000,0.800000}%
\pgfsetstrokecolor{currentstroke}%
\pgfsetstrokeopacity{0.800000}%
\pgfsetdash{}{0pt}%
\pgfpathmoveto{\pgfqpoint{4.228689in}{2.326602in}}%
\pgfpathlineto{\pgfqpoint{5.094898in}{2.326602in}}%
\pgfpathlineto{\pgfqpoint{5.094898in}{3.160491in}}%
\pgfpathlineto{\pgfqpoint{4.228689in}{3.160491in}}%
\pgfpathlineto{\pgfqpoint{4.228689in}{2.326602in}}%
\pgfpathclose%
\pgfusepath{stroke,fill}%
\end{pgfscope}%
\begin{pgfscope}%
\pgfsetbuttcap%
\pgfsetroundjoin%
\pgfsetlinewidth{1.003750pt}%
\definecolor{currentstroke}{rgb}{0.000000,0.000000,0.000000}%
\pgfsetstrokecolor{currentstroke}%
\pgfsetdash{{1.000000pt}{0.000000pt}}{0.000000pt}%
\pgfpathmoveto{\pgfqpoint{4.284245in}{3.055491in}}%
\pgfpathlineto{\pgfqpoint{4.423133in}{3.055491in}}%
\pgfpathlineto{\pgfqpoint{4.562022in}{3.055491in}}%
\pgfusepath{stroke}%
\end{pgfscope}%
\begin{pgfscope}%
\definecolor{textcolor}{rgb}{0.000000,0.000000,0.000000}%
\pgfsetstrokecolor{textcolor}%
\pgfsetfillcolor{textcolor}%
\pgftext[x=4.673133in,y=3.006880in,left,base]{\color{textcolor}{\rmfamily\fontsize{10.000000}{12.000000}\selectfont\catcode`\^=\active\def^{\ifmmode\sp\else\^{}\fi}\catcode`\%=\active\def%{\%}\SI{1.0}{\hertz}}}%
\end{pgfscope}%
\begin{pgfscope}%
\pgfsetbuttcap%
\pgfsetroundjoin%
\pgfsetlinewidth{1.003750pt}%
\definecolor{currentstroke}{rgb}{0.250980,0.250980,0.250980}%
\pgfsetstrokecolor{currentstroke}%
\pgfsetdash{{2.000000pt}{1.000000pt}}{0.000000pt}%
\pgfpathmoveto{\pgfqpoint{4.284245in}{2.857436in}}%
\pgfpathlineto{\pgfqpoint{4.423133in}{2.857436in}}%
\pgfpathlineto{\pgfqpoint{4.562022in}{2.857436in}}%
\pgfusepath{stroke}%
\end{pgfscope}%
\begin{pgfscope}%
\definecolor{textcolor}{rgb}{0.000000,0.000000,0.000000}%
\pgfsetstrokecolor{textcolor}%
\pgfsetfillcolor{textcolor}%
\pgftext[x=4.673133in,y=2.808824in,left,base]{\color{textcolor}{\rmfamily\fontsize{10.000000}{12.000000}\selectfont\catcode`\^=\active\def^{\ifmmode\sp\else\^{}\fi}\catcode`\%=\active\def%{\%}\SI{2.0}{\hertz}}}%
\end{pgfscope}%
\begin{pgfscope}%
\pgfsetbuttcap%
\pgfsetroundjoin%
\pgfsetlinewidth{1.003750pt}%
\definecolor{currentstroke}{rgb}{0.501961,0.501961,0.501961}%
\pgfsetstrokecolor{currentstroke}%
\pgfsetdash{{3.000000pt}{2.000000pt}}{0.000000pt}%
\pgfpathmoveto{\pgfqpoint{4.284245in}{2.659380in}}%
\pgfpathlineto{\pgfqpoint{4.423133in}{2.659380in}}%
\pgfpathlineto{\pgfqpoint{4.562022in}{2.659380in}}%
\pgfusepath{stroke}%
\end{pgfscope}%
\begin{pgfscope}%
\definecolor{textcolor}{rgb}{0.000000,0.000000,0.000000}%
\pgfsetstrokecolor{textcolor}%
\pgfsetfillcolor{textcolor}%
\pgftext[x=4.673133in,y=2.610769in,left,base]{\color{textcolor}{\rmfamily\fontsize{10.000000}{12.000000}\selectfont\catcode`\^=\active\def^{\ifmmode\sp\else\^{}\fi}\catcode`\%=\active\def%{\%}\SI{3.0}{\hertz}}}%
\end{pgfscope}%
\begin{pgfscope}%
\pgfsetbuttcap%
\pgfsetroundjoin%
\pgfsetlinewidth{1.003750pt}%
\definecolor{currentstroke}{rgb}{0.752941,0.752941,0.752941}%
\pgfsetstrokecolor{currentstroke}%
\pgfsetdash{{4.000000pt}{3.000000pt}}{0.000000pt}%
\pgfpathmoveto{\pgfqpoint{4.284245in}{2.461324in}}%
\pgfpathlineto{\pgfqpoint{4.423133in}{2.461324in}}%
\pgfpathlineto{\pgfqpoint{4.562022in}{2.461324in}}%
\pgfusepath{stroke}%
\end{pgfscope}%
\begin{pgfscope}%
\definecolor{textcolor}{rgb}{0.000000,0.000000,0.000000}%
\pgfsetstrokecolor{textcolor}%
\pgfsetfillcolor{textcolor}%
\pgftext[x=4.673133in,y=2.412713in,left,base]{\color{textcolor}{\rmfamily\fontsize{10.000000}{12.000000}\selectfont\catcode`\^=\active\def^{\ifmmode\sp\else\^{}\fi}\catcode`\%=\active\def%{\%}\SI{4.0}{\hertz}}}%
\end{pgfscope}%
\end{pgfpicture}%
\makeatother%
\endgroup%

\caption{Example \thesection\ Plot}
\end{figure}

\end{document}
