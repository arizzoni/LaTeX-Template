\documentclass[12pt, titlepage]{article}
\newcommand{\theauthor}{Alessandro Rizzoni}
\newcommand{\thedoctitle}{Python\TeX\ Examples}

\usepackage{hyperref}

% Code Execution and Typesetting
\usepackage[%
    pygopt={
        style=bw
        },
    upquote=true
    ]{pythontex}
\setpythontexfv{%
    breaklines,
    breakanywhere,
    frame=leftline,
    framerule=0.8pt,
    firstnumber=last,
    numbers=left,
    numbersep=0pt
    }
\renewcommand{\FancyVerbFormatLine}[1]{\small #1}
\renewcommand{\theFancyVerbLine}{%
    \tiny\rmfamily{\arabic{FancyVerbLine}\,}
    }

% Page Layout
\usepackage[%
    letterpaper, % 8.5" x 11" paper size
    margin=1in
    ]{geometry}

% Mathematics
\usepackage{mathtools} % Mathematics typesetting
\usepackage[% Mathematics with modern fonts
    warnings-off={% Turn off default always-on warnings
        mathtools-colon,
        mathtools-overbracket
        }
    ]{unicode-math} 
\usepackage{lualatex-math} % Math fixes for LuaLaTeX
\usepackage{siunitx} % Handy typesetting of units

% Font
\usepackage{fontspec} % Modern .ttc, .ttf, and .otf fonts
\setmathfont{STIXTwoMath-Regular.otf}
\setmainfont{STIXTwoText-Regular.otf}[%
    Ligatures=TeX,
    ItalicFont=STIXTwoText-Italic.otf,
    BoldFont=STIXTwoText-Bold.otf,
    BoldItalicFont=STIXTwoText-BoldItalic.otf,
    ]
\setsansfont{Iosevka Aile}[%
    Ligatures=TeX,
    Scale=MatchLowercase
    ]
\setmonofont{Iosevka Fixed}[%
    Ligatures=TeX,
    Scale=MatchLowercase
    ]

% Localization
\usepackage[% American English localization
    USenglish
    ]{babel} 
\usepackage[% Localized context-sensitive quotes
    autostyle=true
    ]{csquotes}

% Headers
\usepackage{fancyhdr}
\renewcommand{\sectionmark}[1]{\markright{#1}}
\setlength{\headheight}{24pt}
\fancypagestyle{fancy}{
    \fancyhf{}
    \renewcommand{\headrulewidth}{0.8pt}
    \renewcommand{\footrulewidth}{0pt}
    \fancyhead[L]{\sffamily\thedoctitle}
    \fancyhead[R]{\sffamily\small\rightmark}
    \fancyfoot[C]{\sffamily\small\thepage}
}
\pagestyle{fancy}

% Section Title Formatting
\usepackage{titlesec}
\titleformat{\section}{\Large\sffamily}{\thesection}{1em}{\Large\sffamily}
\titleformat{\subsection}{\large\sffamily}{\thesubsection}{1em}{\large\sffamily}
\titleformat{\subsubsection}{\normalsize\sffamily}{\thesubsubsection}{1em}{\normalsize\sffamily}
% \AddToHook{cmd/section/before}{\clearpage}

% Table of Contents Formatting
\usepackage{titletoc}
\dottedcontents{section}[1em]{}{1em}{0.5em}
\dottedcontents{subsection}[1em]{}{1em}{0.5em}
\dottedcontents{subsubsection}[1em]{}{1em}{0.5em}

% Tables and Lists
\usepackage{tabularray} % Tables and Arrays
\usepackage{enumitem} % Reformat Enumerated Lists

% Graphics
\usepackage{pgf} % Base graphics package
\usepackage{blox} % System block diagrams
\usepackage{tikz-timing} % Timing diagram graphics
\usepackage[siunitx]{circuitikz} % Electrical circuit graphics
\ctikzloadstyle{romano} % Better styling for CircuiTikz
\tikzset{romano circuit style} % Apply the style

%% Pythontex 
% Get lengths (in LaTeX points!)
\newlength\inlength \inlength=1in
\newlength\cmlength \cmlength=1cm
\newlength\mmlength \mmlength=1mm
\newlength\emlength \emlength=1em
\newlength\exlength \exlength=1ex
\newlength\bplength \bplength=1bp
\newlength\ddlength \ddlength=1dd
\newlength\pclength \pclength=1pc

% Get font parameters
\makeatletter

% Font Sizes
\tiny\xdef\thetinysize{\f@size}
\scriptsize\xdef\thescriptsize{\f@size}
\footnotesize\xdef\thefootnotesize{\f@size}
\small\xdef\thesmallsize{\f@size}
\normalsize\xdef\thenormalsize{\f@size}
\large\xdef\thelargesize{\f@size}
\Large\xdef\theLargesize{\f@size}
\LARGE\xdef\theLARGEsize{\f@size}
\huge\xdef\thehugesize{\f@size}
\Huge\xdef\theHugesize{\f@size}

% Fonts 
\sffamily\mdseries\upshape\xdef\thesffont{\expandafter\string\the\font}
\ttfamily\mdseries\upshape\xdef\thettfont{\expandafter\string\the\font}
\rmfamily\mdseries\upshape\xdef\thermfont{\expandafter\string\the\font}
\sffamily\bfseries\upshape\xdef\theboldsffont{\expandafter\string\the\font}
\ttfamily\bfseries\upshape\xdef\theboldttfont{\expandafter\string\the\font}
\rmfamily\bfseries\upshape\xdef\theboldrmfont{\expandafter\string\the\font}
\sffamily\mdseries\itshape\xdef\theitalicsffont{\expandafter\string\the\font}
\ttfamily\mdseries\itshape\xdef\theitalicttfont{\expandafter\string\the\font}
\rmfamily\mdseries\itshape\xdef\theitalicrmfont{\expandafter\string\the\font}
\sffamily\bfseries\itshape\xdef\thebolditalicsffont{\expandafter\string\the\font}
\ttfamily\bfseries\itshape\xdef\thebolditalicttfont{\expandafter\string\the\font}
\rmfamily\bfseries\itshape\xdef\thebolditalicrmfont{\expandafter\string\the\font}
\sffamily\mdseries\scshape\xdef\thesmallcapssffont{\expandafter\string\the\font}
\ttfamily\mdseries\scshape\xdef\thesmallcapsttfont{\expandafter\string\the\font}
\rmfamily\mdseries\scshape\xdef\thesmallcapsrmfont{\expandafter\string\the\font}
\sffamily\bfseries\scshape\xdef\theboldsmallcapssffont{\expandafter\string\the\font}
\ttfamily\bfseries\scshape\xdef\theboldsmallcapsttfont{\expandafter\string\the\font}
\rmfamily\bfseries\scshape\xdef\theboldsmallcapsrmfont{\expandafter\string\the\font}
\sffamily\mdseries\slshape\xdef\theslantedsffont{\expandafter\string\the\font}
\ttfamily\mdseries\slshape\xdef\theslantedttfont{\expandafter\string\the\font}
\rmfamily\mdseries\slshape\xdef\theslantedrmfont{\expandafter\string\the\font}
\sffamily\bfseries\slshape\xdef\theboldslantedsffont{\expandafter\string\the\font}
\ttfamily\bfseries\slshape\xdef\theboldslantedttfont{\expandafter\string\the\font}
\rmfamily\bfseries\slshape\xdef\theboldslantedrmfont{\expandafter\string\the\font}

% Math Fonts
\xdef\themathfont{\expandafter\string\the\textfont0}
\xdef\themathfontone{\expandafter\string\the\textfont1}
\xdef\themathfonttwo{\expandafter\string\the\textfont2}
\xdef\themathfontthree{\expandafter\string\the\textfont3}

\makeatother

% Define Python Context
\setpythontexcontext{%
    romanfont=\thermfont,
    romanboldfont=\theboldrmfont,
    romanitalicfont=\theitalicrmfont,
    romanbolditalicfont=\thebolditalicrmfont,
    sansfont=\thesffont,
    sansboldfont=\theboldsffont,
    sansitalicfont=\theitalicsffont,
    sansbolditalicfont=\thebolditalicsffont,
    monofont=\thettfont,
    monoboldfont=\theboldttfont,
    monoitalicfont=\theitalicttfont,
    monobolditalicfont=\thebolditalicttfont,
    mathfont=\themathfont,
    mathfontone=\themathfontone,
    mathfonttwo=\themathfonttwo,
    mathfontthree=\themathfontthree,
    scsansfont = \thesmallcapssffont,
    scromanfont = \thesmallcapsrmfont,
    scmonofont = \thesmallcapsttfont,
    scsansboldfont = \theboldsmallcapssffont,
    scromanboldfont = \theboldsmallcapsrmfont,
    scmonoboldfont = \theboldsmallcapsttfont,
    slantedsansfont = \theslantedsffont,
    slantedromanfont = \theslantedrmfont,
    slantedmonofont = \theslantedttfont,
    slantedsansboldfont = \theboldslantedsffont,
    slantedromanboldfont = \theboldslantedrmfont,
    slantedmonoboldfont = \theboldslantedttfont,
    tiny=\thetinysize,
    scriptsize=\thescriptsize,
    footnotesize=\thefootnotesize,
    small=\thesmallsize,
    normalsize=\thenormalsize,
    large=\thelargesize,
    Large=\theLargesize,
    LARGE=\theLARGEsize,
    huge=\thehugesize,
    Huge=\theHugesize,
    baselineskip=\the\baselineskip,
    % baselinestretch=\the\baselinestretch,
    columnsep=\the\columnsep,
    columnwidth=\the\columnwidth,
    evensidemargin=\the\evensidemargin,
    linewidth=\the\linewidth,
    oddsidemargin=\the\oddsidemargin,
    paperwidth=\the\paperwidth,
    paperheight=\the\paperheight,
    parindent=\the\parindent,
    parskip=\the\parskip,
    textheight=\the\textheight,
    textwidth=\the\textwidth,
    topmargin=\the\topmargin,
    unitlength=\the\unitlength,
    in=\the\inlength,
    cm=\the\cmlength,
    mm=\the\mmlength,
    em=\the\emlength,
    ex=\the\exlength,
    bp=\the\bplength,
    dd=\the\ddlength,
    pc=\the\pclength
}

\AtBeginDocument{
    \renewcommand{\abstractname}{Introduction}
    \addtocontents{toc}{\protect\thispagestyle{empty}}
}

\begin{document}

\title{\sffamily\thedoctitle}
\author{\sffamily\theauthor}
\date{\sffamily\today}
\maketitle

\begin{abstract}
    These are some examples that show the capabilities of Python\TeX. These examples were written using the \Verb{\pyblock} environment - normally it is preferred to use a \Verb{\pythontexcustomcode} environment so that the settings can be applied to multiple Python evironments within the document.
\end{abstract}

\tableofcontents
\clearpage

\section{Python Environment}

The first example covers the python environment that we will use for this document. It's pretty simple! All we need for the first few examples is \Verb{numpy} and \Verb{matplotlib}. We can also use one of \Verb{matplotlib}'s dependencies, \Verb{cycler}, to save some time.

\begin{pyblock}
import os

import numpy as np # Python numerical computing library
import matplotlib as mpl

from cycler import cycler # Property cycler utilities
from matplotlib import pyplot as plt # Pyplot API
from matplotlib import rcParams as rc # Matplotlib plot styling
from matplotlib.ticker import EngFormatter # Plot tick formatting

class Environment():

    build_directory = os.path.abspath(os.getcwd())
    output_directory = os.sep.join(build_directory.split(os.sep)[:-1])
    figures_directory = os.path.join(output_directory, 'figures')
    pgf_directory = os.path.join(figures_directory, 'pgf')
    pdf_directory = os.path.join(figures_directory, 'pdf')

    def __init__(self):
        figure_directories = [ self.figures_directory, self.pgf_directory, self.pdf_directory ]
        for directory in figure_directories:
            if not os.path.exists(directory):
                os.mkdir(directory)

environment = Environment()

\end{pyblock}

\section{Utility Functions}

Example \thesection\ contains some functions that will be useful later on. The \Verb{metallic_ratio()} function returns the ``metallic ratio'' of a numeric argument, and the \Verb{eng_format()} function returns an \Verb{EngFormatter} object that is used to generate the engineering format plot ticks that are used later on. Information on the metallic ratio, or mean, can be found here: \url{https://en.wikipedia.org/wiki/Metallic_mean}.

\begin{pyblock}
def metallic_ratio(n):
    return 0.5 * ( n + np.sqrt(n**2 + 4) )
    
def eng_format(arg: str):
    return EngFormatter(unit=arg, sep=r'\,')

def save_pythontex_figure(figure, figure_name):
    if type(figure) == mpl.figure.Figure:
        figure.savefig(os.path.join(environment.pgf_directory, f'{figure_name}.pgf'))
        figure.savefig(os.path.join(environment.pdf_directory, f'{figure_name}.pdf'))
    return figure

\end{pyblock}

\section{Geometry Class}

Example \thesection\ starts to get to the good stuff - here we are starting to interact with the Python context we generated in the preamble to pull variables from \LaTeX\ into our Python environment. The \Verb{pytex.context} dictionary holds all of the values, etc. that we pass to Python, and we can access them as follows. Some additional sanitization and conditioning is required to give a plain string and convert from \LaTeX\ points to inches. the three values we will start off with are the column width, the text width, and the text height. They are generally described as they are named, but the text height and the column height are not necessarily the same, such as in a multiple column document.

\begin{pyblock}

class Geometry():

    # Convert from (LaTeX!) points to inches 
    # There is ~ 0.14 micron floating point error here
    in_length = float(pytex.context['in'][:-2]) / 72.27
    cm_length = float(pytex.context['cm'][:-2]) / 72.27
    mm_length = float(pytex.context['mm'][:-2]) / 72.27
    em_length = float(pytex.context['em'][:-2]) / 72.27
    ex_length = float(pytex.context['ex'][:-2]) / 72.27
    bp_length = float(pytex.context['bp'][:-2]) / 72.27
    dd_length = float(pytex.context['dd'][:-2]) / 72.27
    pc_length = float(pytex.context['pc'][:-2]) / 72.27

    column_width = float(pytex.context['columnwidth'][:-2]) / 72.27
    text_width = float(pytex.context['textwidth'][:-2]) / 72.27
    text_height = float(pytex.context['textheight'][:-2]) / 72.27
    figure_width = column_width - 2 * em_length
    figure_height = figure_width / metallic_ratio(1) # Define figure height as a function of the figure width and the golden ratio

    axis_dimensions = (0, 0, 1, 1) # 0 lr margin, 0 tb margin, 100% figure size

geometry = Geometry()

\end{pyblock}

\section{Font Class}

Here we pull the fonts from the \LaTeX\ document so we can use them in our \Verb{matplotlib} plots and have a consistently formatted document. The font names are taken from the context dictionary and sanitized as before. After the typefaces have been transferred, the font sizes can be taken. The font sizes are taken from the default \LaTeX\ font sizes - other sizes can be added or the base font commands can be patched if desired. The class is declared to allow for dot notation. Eventually, there may be a need for additional font configuration code, and having a class interface in place will simplify expansion later on.

\begin{pyblock}

# Determine font parameters
class Font():
    # Font faces
    roman = pytex.context['romanfont'].split('/')[1][:-3]
    bold_roman = pytex.context['romanboldfont'].split('/')[1][:-3]
    bold_italic_roman = pytex.context['romanbolditalicfont'].split('/')[1][:-3]
    italic_roman = pytex.context['romanitalicfont'].split('/')[1][:-3]
    sans = pytex.context['sansfont'].split('/')[1][:-3]
    mono = pytex.context['monofont'].split('/')[1][:-3]
    math = pytex.context['mathfont'].split('/')[1][:-3]
    
    # Font sizes
    tiny = pytex.context['tiny']
    script_size = pytex.context['scriptsize']
    footnote_size = pytex.context['footnotesize']
    small = pytex.context['small']
    normal_size = pytex.context['normalsize']
    large = pytex.context['large']
    llarge = pytex.context['Large']
    lllarge = pytex.context['LARGE']
    huge = pytex.context['huge']
    hhuge = pytex.context['Huge']

font = Font() # Instantiate the class for use

\end{pyblock}

\section{Plot Settings}

Example \thesection\ configures various options for the plots. The figure geometry is defined first, then the colormap is selected. The \Verb{grey} colorscheme was chosen to keep the document in black and white. Once the colormap and the number of styles are selected, a \Verb{cycler} is created to iterate over the different color and dash styles.

\begin{pyblock}

# figure settings
cmap = plt.get_cmap('grey') # Select colormap
num_plot_styles = 4 # number of colors for plotting

# Initialize empty lists for plot colors and styles
plot_colors = []
line_styles = []
for i in range(num_plot_styles): # Populate the color and style lists
    plot_colors.append(cmap(1.0 * i/num_plot_styles))
    line_styles.append((0, (i+1, i)))

# Define the main cycler with the two component lists 
style_cycler = cycler(color=plot_colors, linestyle=line_styles)

\end{pyblock}

\section{Matplotlib Configuration}

This example is interesting - the \Verb{matplotlib} rcparams can be modified for the active Python evironment. The more interesting feature is the PGF backend which allows for the output image to be produced as directly importable \LaTeX\ code with the \Verb{pgf} package. The backend allows for a \LaTeX\ preamble to be defined, and this feature is used here to create a unified plot style with the same font as the rest of the document. Other fonts are available in the Python environment - any font you define in the normal \LaTeX\ ways, such as by importing a package, using \Verb{fontspec} or \Verb{fontsetup}, etc.

\begin{pyblock}

# Document-wide Matplotlib Configuration
rc.update({
        'backend': 'pgf',
        'lines.linewidth': 1,
        'font.family': 'serif',
        'font.size': font.footnote_size,
        'text.usetex': True,
        'axes.prop_cycle': style_cycler,
        'axes.labelsize': font.footnote_size,
        'axes.linewidth': 0.8,
        'xtick.direction': 'in',
        'xtick.top': True,
        'xtick.bottom': True,
        'xtick.minor.visible': True,
        'ytick.direction': 'in',
        'ytick.left': True,
        'ytick.right': True,
        'ytick.minor.visible': True,
        'legend.fontsize': font.footnote_size,
        'legend.fancybox': False,
        'figure.figsize': (geometry.figure_width, geometry.figure_height),
        'figure.dpi': 600,
	'figure.constrained_layout.use': True,
        'figure.constrained_layout.hspace': 0,
        'figure.constrained_layout.wspace': 0,
        'figure.constrained_layout.w_pad': 0,
        'figure.constrained_layout.h_pad': 0,
        'savefig.format': 'pgf',
        'savefig.bbox': 'tight',
        'savefig.transparent': True,
        'pgf.rcfonts': False,
	'pgf.preamble': '\n'.join([
                r'\usepackage{mathtools}',
                r'\usepackage[warnings-off={mathtools-colon, mathtools-overbracket}]{unicode-math}',
                r'\usepackage{lualatex-math}',
                r'\usepackage{siunitx}',
                r'\usepackage{fontspec}',
                r'\setmainfont{%s}[Ligatures=TeX, ItalicFont=%s, BoldFont=%s, BoldItalicFont=%s]' %(font.roman, font.italic_roman, font.bold_roman, font.bold_italic_roman),
                r'\setmathfont{%s}' %(font.math),
                r'\setsansfont{%s}[Ligatures=TeX, Scale=MatchLowercase]' %(font.sans),
                r'\setmonofont{%s}[Ligatures=TeX, Scale=MatchLowercase]' %(font.mono),
		r'\usepackage[USenglish]{babel}',
                r'\usepackage[autostyle=true]{csquotes}'
	]),
        'pgf.texsystem': 'lualatex', # default is xetex, but lualatex is preferred
})

\end{pyblock}

\section{Example Plot}

This example shows a few different things that are possible. First, a plot can be generated with plot size and other parameters passed to Python from \LaTeX. The example also shows the utility provided by the use of the Matplotlib PGF backend's \LaTeX\ preamble, which was defined in the Matplotlib configuration code. This allows the package \texttt{siunitx} to be loaded inside Matplotlib so that mathematics and physical units can be properly typeset using \LaTeX.

\begin{pyblock}

fig, ax = plt.subplots()

x = np.linspace(0, 1, 1000)
for frequency in range(1, num_plot_styles + 1, 1):
    _ = ax.plot(
	x,
	np.sin(2*np.pi*frequency*x),
        label=r'\SI{%.1f}{\hertz}' % frequency
        )

ax.xaxis.set_major_formatter(
    eng_format(r'\unit{\second}')
    )

ax.yaxis.set_major_formatter(
    eng_format(r'\unit{\volt}')
    )

_ = ax.legend()

figure_name = 'example_figure'
save_pythontex_figure(fig, figure_name)

\end{pyblock}

\begin{figure}[h!]
%% Creator: Matplotlib, PGF backend
%%
%% To include the figure in your LaTeX document, write
%%   \input{<filename>.pgf}
%%
%% Make sure the required packages are loaded in your preamble
%%   \usepackage{pgf}
%%
%% Also ensure that all the required font packages are loaded; for instance,
%% the lmodern package is sometimes necessary when using math font.
%%   \usepackage{lmodern}
%%
%% Figures using additional raster images can only be included by \input if
%% they are in the same directory as the main LaTeX file. For loading figures
%% from other directories you can use the `import` package
%%   \usepackage{import}
%%
%% and then include the figures with
%%   \import{<path to file>}{<filename>.pgf}
%%
%% Matplotlib used the following preamble
%%   \def\mathdefault#1{#1}
%%   \everymath=\expandafter{\the\everymath\displaystyle}
%%   \usepackage{mathtools}
%%   \usepackage[warnings-off={mathtools-colon, mathtools-overbracket}]{unicode-math}
%%   \usepackage{lualatex-math}
%%   \usepackage{siunitx}
%%   \usepackage{fontspec}
%%   \setmainfont{STIXTwoText-Regular.otf}[Ligatures=TeX, ItalicFont=STIXTwoText-Regular.otf, BoldFont=STIXTwoText-Regular.otf, BoldItalicFont=STIXTwoText-Regular.otf]
%%   \setmathfont{STIXTwoMath-Regular.otf}
%%   \setsansfont{IosevkaAile}[Ligatures=TeX, Scale=MatchLowercase]
%%   \setmonofont{IosevkaFixed}[Ligatures=TeX, Scale=MatchLowercase]
%%   \usepackage[USenglish]{babel}
%%   \usepackage[autostyle=true]{csquotes}
%%   \usepackage{fontspec}
%%   \makeatletter\@ifpackageloaded{underscore}{}{\usepackage[strings]{underscore}}\makeatother
%%
\begingroup%
\makeatletter%
\begin{pgfpicture}%
\pgfpathrectangle{\pgfpointorigin}{\pgfqpoint{4.697025in}{2.979314in}}%
\pgfusepath{use as bounding box, clip}%
\begin{pgfscope}%
\pgfsetbuttcap%
\pgfsetmiterjoin%
\pgfsetlinewidth{0.000000pt}%
\definecolor{currentstroke}{rgb}{0.000000,0.000000,0.000000}%
\pgfsetstrokecolor{currentstroke}%
\pgfsetstrokeopacity{0.000000}%
\pgfsetdash{}{0pt}%
\pgfpathmoveto{\pgfqpoint{0.000000in}{0.000000in}}%
\pgfpathlineto{\pgfqpoint{4.697025in}{0.000000in}}%
\pgfpathlineto{\pgfqpoint{4.697025in}{2.979314in}}%
\pgfpathlineto{\pgfqpoint{0.000000in}{2.979314in}}%
\pgfpathlineto{\pgfqpoint{0.000000in}{0.000000in}}%
\pgfpathclose%
\pgfusepath{}%
\end{pgfscope}%
\begin{pgfscope}%
\pgfsetbuttcap%
\pgfsetmiterjoin%
\pgfsetlinewidth{0.000000pt}%
\definecolor{currentstroke}{rgb}{0.000000,0.000000,0.000000}%
\pgfsetstrokecolor{currentstroke}%
\pgfsetstrokeopacity{0.000000}%
\pgfsetdash{}{0pt}%
\pgfpathmoveto{\pgfqpoint{0.607800in}{0.251500in}}%
\pgfpathlineto{\pgfqpoint{4.597025in}{0.251500in}}%
\pgfpathlineto{\pgfqpoint{4.597025in}{2.879314in}}%
\pgfpathlineto{\pgfqpoint{0.607800in}{2.879314in}}%
\pgfpathlineto{\pgfqpoint{0.607800in}{0.251500in}}%
\pgfpathclose%
\pgfusepath{}%
\end{pgfscope}%
\begin{pgfscope}%
\pgfsetbuttcap%
\pgfsetroundjoin%
\definecolor{currentfill}{rgb}{0.000000,0.000000,0.000000}%
\pgfsetfillcolor{currentfill}%
\pgfsetlinewidth{0.803000pt}%
\definecolor{currentstroke}{rgb}{0.000000,0.000000,0.000000}%
\pgfsetstrokecolor{currentstroke}%
\pgfsetdash{}{0pt}%
\pgfsys@defobject{currentmarker}{\pgfqpoint{0.000000in}{0.000000in}}{\pgfqpoint{0.000000in}{0.048611in}}{%
\pgfpathmoveto{\pgfqpoint{0.000000in}{0.000000in}}%
\pgfpathlineto{\pgfqpoint{0.000000in}{0.048611in}}%
\pgfusepath{stroke,fill}%
}%
\begin{pgfscope}%
\pgfsys@transformshift{0.789129in}{0.251500in}%
\pgfsys@useobject{currentmarker}{}%
\end{pgfscope}%
\end{pgfscope}%
\begin{pgfscope}%
\pgfsetbuttcap%
\pgfsetroundjoin%
\definecolor{currentfill}{rgb}{0.000000,0.000000,0.000000}%
\pgfsetfillcolor{currentfill}%
\pgfsetlinewidth{0.803000pt}%
\definecolor{currentstroke}{rgb}{0.000000,0.000000,0.000000}%
\pgfsetstrokecolor{currentstroke}%
\pgfsetdash{}{0pt}%
\pgfsys@defobject{currentmarker}{\pgfqpoint{0.000000in}{-0.048611in}}{\pgfqpoint{0.000000in}{0.000000in}}{%
\pgfpathmoveto{\pgfqpoint{0.000000in}{0.000000in}}%
\pgfpathlineto{\pgfqpoint{0.000000in}{-0.048611in}}%
\pgfusepath{stroke,fill}%
}%
\begin{pgfscope}%
\pgfsys@transformshift{0.789129in}{2.879314in}%
\pgfsys@useobject{currentmarker}{}%
\end{pgfscope}%
\end{pgfscope}%
\begin{pgfscope}%
\definecolor{textcolor}{rgb}{0.000000,0.000000,0.000000}%
\pgfsetstrokecolor{textcolor}%
\pgfsetfillcolor{textcolor}%
\pgftext[x=0.789129in,y=0.202889in,,top]{\color{textcolor}{\rmfamily\fontsize{8.000000}{9.600000}\selectfont\catcode`\^=\active\def^{\ifmmode\sp\else\^{}\fi}\catcode`\%=\active\def%{\%}$0$\,\unit{\second}}}%
\end{pgfscope}%
\begin{pgfscope}%
\pgfsetbuttcap%
\pgfsetroundjoin%
\definecolor{currentfill}{rgb}{0.000000,0.000000,0.000000}%
\pgfsetfillcolor{currentfill}%
\pgfsetlinewidth{0.803000pt}%
\definecolor{currentstroke}{rgb}{0.000000,0.000000,0.000000}%
\pgfsetstrokecolor{currentstroke}%
\pgfsetdash{}{0pt}%
\pgfsys@defobject{currentmarker}{\pgfqpoint{0.000000in}{0.000000in}}{\pgfqpoint{0.000000in}{0.048611in}}{%
\pgfpathmoveto{\pgfqpoint{0.000000in}{0.000000in}}%
\pgfpathlineto{\pgfqpoint{0.000000in}{0.048611in}}%
\pgfusepath{stroke,fill}%
}%
\begin{pgfscope}%
\pgfsys@transformshift{1.514442in}{0.251500in}%
\pgfsys@useobject{currentmarker}{}%
\end{pgfscope}%
\end{pgfscope}%
\begin{pgfscope}%
\pgfsetbuttcap%
\pgfsetroundjoin%
\definecolor{currentfill}{rgb}{0.000000,0.000000,0.000000}%
\pgfsetfillcolor{currentfill}%
\pgfsetlinewidth{0.803000pt}%
\definecolor{currentstroke}{rgb}{0.000000,0.000000,0.000000}%
\pgfsetstrokecolor{currentstroke}%
\pgfsetdash{}{0pt}%
\pgfsys@defobject{currentmarker}{\pgfqpoint{0.000000in}{-0.048611in}}{\pgfqpoint{0.000000in}{0.000000in}}{%
\pgfpathmoveto{\pgfqpoint{0.000000in}{0.000000in}}%
\pgfpathlineto{\pgfqpoint{0.000000in}{-0.048611in}}%
\pgfusepath{stroke,fill}%
}%
\begin{pgfscope}%
\pgfsys@transformshift{1.514442in}{2.879314in}%
\pgfsys@useobject{currentmarker}{}%
\end{pgfscope}%
\end{pgfscope}%
\begin{pgfscope}%
\definecolor{textcolor}{rgb}{0.000000,0.000000,0.000000}%
\pgfsetstrokecolor{textcolor}%
\pgfsetfillcolor{textcolor}%
\pgftext[x=1.514442in,y=0.202889in,,top]{\color{textcolor}{\rmfamily\fontsize{8.000000}{9.600000}\selectfont\catcode`\^=\active\def^{\ifmmode\sp\else\^{}\fi}\catcode`\%=\active\def%{\%}$200$\,m\unit{\second}}}%
\end{pgfscope}%
\begin{pgfscope}%
\pgfsetbuttcap%
\pgfsetroundjoin%
\definecolor{currentfill}{rgb}{0.000000,0.000000,0.000000}%
\pgfsetfillcolor{currentfill}%
\pgfsetlinewidth{0.803000pt}%
\definecolor{currentstroke}{rgb}{0.000000,0.000000,0.000000}%
\pgfsetstrokecolor{currentstroke}%
\pgfsetdash{}{0pt}%
\pgfsys@defobject{currentmarker}{\pgfqpoint{0.000000in}{0.000000in}}{\pgfqpoint{0.000000in}{0.048611in}}{%
\pgfpathmoveto{\pgfqpoint{0.000000in}{0.000000in}}%
\pgfpathlineto{\pgfqpoint{0.000000in}{0.048611in}}%
\pgfusepath{stroke,fill}%
}%
\begin{pgfscope}%
\pgfsys@transformshift{2.239756in}{0.251500in}%
\pgfsys@useobject{currentmarker}{}%
\end{pgfscope}%
\end{pgfscope}%
\begin{pgfscope}%
\pgfsetbuttcap%
\pgfsetroundjoin%
\definecolor{currentfill}{rgb}{0.000000,0.000000,0.000000}%
\pgfsetfillcolor{currentfill}%
\pgfsetlinewidth{0.803000pt}%
\definecolor{currentstroke}{rgb}{0.000000,0.000000,0.000000}%
\pgfsetstrokecolor{currentstroke}%
\pgfsetdash{}{0pt}%
\pgfsys@defobject{currentmarker}{\pgfqpoint{0.000000in}{-0.048611in}}{\pgfqpoint{0.000000in}{0.000000in}}{%
\pgfpathmoveto{\pgfqpoint{0.000000in}{0.000000in}}%
\pgfpathlineto{\pgfqpoint{0.000000in}{-0.048611in}}%
\pgfusepath{stroke,fill}%
}%
\begin{pgfscope}%
\pgfsys@transformshift{2.239756in}{2.879314in}%
\pgfsys@useobject{currentmarker}{}%
\end{pgfscope}%
\end{pgfscope}%
\begin{pgfscope}%
\definecolor{textcolor}{rgb}{0.000000,0.000000,0.000000}%
\pgfsetstrokecolor{textcolor}%
\pgfsetfillcolor{textcolor}%
\pgftext[x=2.239756in,y=0.202889in,,top]{\color{textcolor}{\rmfamily\fontsize{8.000000}{9.600000}\selectfont\catcode`\^=\active\def^{\ifmmode\sp\else\^{}\fi}\catcode`\%=\active\def%{\%}$400$\,m\unit{\second}}}%
\end{pgfscope}%
\begin{pgfscope}%
\pgfsetbuttcap%
\pgfsetroundjoin%
\definecolor{currentfill}{rgb}{0.000000,0.000000,0.000000}%
\pgfsetfillcolor{currentfill}%
\pgfsetlinewidth{0.803000pt}%
\definecolor{currentstroke}{rgb}{0.000000,0.000000,0.000000}%
\pgfsetstrokecolor{currentstroke}%
\pgfsetdash{}{0pt}%
\pgfsys@defobject{currentmarker}{\pgfqpoint{0.000000in}{0.000000in}}{\pgfqpoint{0.000000in}{0.048611in}}{%
\pgfpathmoveto{\pgfqpoint{0.000000in}{0.000000in}}%
\pgfpathlineto{\pgfqpoint{0.000000in}{0.048611in}}%
\pgfusepath{stroke,fill}%
}%
\begin{pgfscope}%
\pgfsys@transformshift{2.965069in}{0.251500in}%
\pgfsys@useobject{currentmarker}{}%
\end{pgfscope}%
\end{pgfscope}%
\begin{pgfscope}%
\pgfsetbuttcap%
\pgfsetroundjoin%
\definecolor{currentfill}{rgb}{0.000000,0.000000,0.000000}%
\pgfsetfillcolor{currentfill}%
\pgfsetlinewidth{0.803000pt}%
\definecolor{currentstroke}{rgb}{0.000000,0.000000,0.000000}%
\pgfsetstrokecolor{currentstroke}%
\pgfsetdash{}{0pt}%
\pgfsys@defobject{currentmarker}{\pgfqpoint{0.000000in}{-0.048611in}}{\pgfqpoint{0.000000in}{0.000000in}}{%
\pgfpathmoveto{\pgfqpoint{0.000000in}{0.000000in}}%
\pgfpathlineto{\pgfqpoint{0.000000in}{-0.048611in}}%
\pgfusepath{stroke,fill}%
}%
\begin{pgfscope}%
\pgfsys@transformshift{2.965069in}{2.879314in}%
\pgfsys@useobject{currentmarker}{}%
\end{pgfscope}%
\end{pgfscope}%
\begin{pgfscope}%
\definecolor{textcolor}{rgb}{0.000000,0.000000,0.000000}%
\pgfsetstrokecolor{textcolor}%
\pgfsetfillcolor{textcolor}%
\pgftext[x=2.965069in,y=0.202889in,,top]{\color{textcolor}{\rmfamily\fontsize{8.000000}{9.600000}\selectfont\catcode`\^=\active\def^{\ifmmode\sp\else\^{}\fi}\catcode`\%=\active\def%{\%}$600$\,m\unit{\second}}}%
\end{pgfscope}%
\begin{pgfscope}%
\pgfsetbuttcap%
\pgfsetroundjoin%
\definecolor{currentfill}{rgb}{0.000000,0.000000,0.000000}%
\pgfsetfillcolor{currentfill}%
\pgfsetlinewidth{0.803000pt}%
\definecolor{currentstroke}{rgb}{0.000000,0.000000,0.000000}%
\pgfsetstrokecolor{currentstroke}%
\pgfsetdash{}{0pt}%
\pgfsys@defobject{currentmarker}{\pgfqpoint{0.000000in}{0.000000in}}{\pgfqpoint{0.000000in}{0.048611in}}{%
\pgfpathmoveto{\pgfqpoint{0.000000in}{0.000000in}}%
\pgfpathlineto{\pgfqpoint{0.000000in}{0.048611in}}%
\pgfusepath{stroke,fill}%
}%
\begin{pgfscope}%
\pgfsys@transformshift{3.690383in}{0.251500in}%
\pgfsys@useobject{currentmarker}{}%
\end{pgfscope}%
\end{pgfscope}%
\begin{pgfscope}%
\pgfsetbuttcap%
\pgfsetroundjoin%
\definecolor{currentfill}{rgb}{0.000000,0.000000,0.000000}%
\pgfsetfillcolor{currentfill}%
\pgfsetlinewidth{0.803000pt}%
\definecolor{currentstroke}{rgb}{0.000000,0.000000,0.000000}%
\pgfsetstrokecolor{currentstroke}%
\pgfsetdash{}{0pt}%
\pgfsys@defobject{currentmarker}{\pgfqpoint{0.000000in}{-0.048611in}}{\pgfqpoint{0.000000in}{0.000000in}}{%
\pgfpathmoveto{\pgfqpoint{0.000000in}{0.000000in}}%
\pgfpathlineto{\pgfqpoint{0.000000in}{-0.048611in}}%
\pgfusepath{stroke,fill}%
}%
\begin{pgfscope}%
\pgfsys@transformshift{3.690383in}{2.879314in}%
\pgfsys@useobject{currentmarker}{}%
\end{pgfscope}%
\end{pgfscope}%
\begin{pgfscope}%
\definecolor{textcolor}{rgb}{0.000000,0.000000,0.000000}%
\pgfsetstrokecolor{textcolor}%
\pgfsetfillcolor{textcolor}%
\pgftext[x=3.690383in,y=0.202889in,,top]{\color{textcolor}{\rmfamily\fontsize{8.000000}{9.600000}\selectfont\catcode`\^=\active\def^{\ifmmode\sp\else\^{}\fi}\catcode`\%=\active\def%{\%}$800$\,m\unit{\second}}}%
\end{pgfscope}%
\begin{pgfscope}%
\pgfsetbuttcap%
\pgfsetroundjoin%
\definecolor{currentfill}{rgb}{0.000000,0.000000,0.000000}%
\pgfsetfillcolor{currentfill}%
\pgfsetlinewidth{0.803000pt}%
\definecolor{currentstroke}{rgb}{0.000000,0.000000,0.000000}%
\pgfsetstrokecolor{currentstroke}%
\pgfsetdash{}{0pt}%
\pgfsys@defobject{currentmarker}{\pgfqpoint{0.000000in}{0.000000in}}{\pgfqpoint{0.000000in}{0.048611in}}{%
\pgfpathmoveto{\pgfqpoint{0.000000in}{0.000000in}}%
\pgfpathlineto{\pgfqpoint{0.000000in}{0.048611in}}%
\pgfusepath{stroke,fill}%
}%
\begin{pgfscope}%
\pgfsys@transformshift{4.415697in}{0.251500in}%
\pgfsys@useobject{currentmarker}{}%
\end{pgfscope}%
\end{pgfscope}%
\begin{pgfscope}%
\pgfsetbuttcap%
\pgfsetroundjoin%
\definecolor{currentfill}{rgb}{0.000000,0.000000,0.000000}%
\pgfsetfillcolor{currentfill}%
\pgfsetlinewidth{0.803000pt}%
\definecolor{currentstroke}{rgb}{0.000000,0.000000,0.000000}%
\pgfsetstrokecolor{currentstroke}%
\pgfsetdash{}{0pt}%
\pgfsys@defobject{currentmarker}{\pgfqpoint{0.000000in}{-0.048611in}}{\pgfqpoint{0.000000in}{0.000000in}}{%
\pgfpathmoveto{\pgfqpoint{0.000000in}{0.000000in}}%
\pgfpathlineto{\pgfqpoint{0.000000in}{-0.048611in}}%
\pgfusepath{stroke,fill}%
}%
\begin{pgfscope}%
\pgfsys@transformshift{4.415697in}{2.879314in}%
\pgfsys@useobject{currentmarker}{}%
\end{pgfscope}%
\end{pgfscope}%
\begin{pgfscope}%
\definecolor{textcolor}{rgb}{0.000000,0.000000,0.000000}%
\pgfsetstrokecolor{textcolor}%
\pgfsetfillcolor{textcolor}%
\pgftext[x=4.415697in,y=0.202889in,,top]{\color{textcolor}{\rmfamily\fontsize{8.000000}{9.600000}\selectfont\catcode`\^=\active\def^{\ifmmode\sp\else\^{}\fi}\catcode`\%=\active\def%{\%}$1$\,\unit{\second}}}%
\end{pgfscope}%
\begin{pgfscope}%
\pgfsetbuttcap%
\pgfsetroundjoin%
\definecolor{currentfill}{rgb}{0.000000,0.000000,0.000000}%
\pgfsetfillcolor{currentfill}%
\pgfsetlinewidth{0.602250pt}%
\definecolor{currentstroke}{rgb}{0.000000,0.000000,0.000000}%
\pgfsetstrokecolor{currentstroke}%
\pgfsetdash{}{0pt}%
\pgfsys@defobject{currentmarker}{\pgfqpoint{0.000000in}{0.000000in}}{\pgfqpoint{0.000000in}{0.027778in}}{%
\pgfpathmoveto{\pgfqpoint{0.000000in}{0.000000in}}%
\pgfpathlineto{\pgfqpoint{0.000000in}{0.027778in}}%
\pgfusepath{stroke,fill}%
}%
\begin{pgfscope}%
\pgfsys@transformshift{0.607800in}{0.251500in}%
\pgfsys@useobject{currentmarker}{}%
\end{pgfscope}%
\end{pgfscope}%
\begin{pgfscope}%
\pgfsetbuttcap%
\pgfsetroundjoin%
\definecolor{currentfill}{rgb}{0.000000,0.000000,0.000000}%
\pgfsetfillcolor{currentfill}%
\pgfsetlinewidth{0.602250pt}%
\definecolor{currentstroke}{rgb}{0.000000,0.000000,0.000000}%
\pgfsetstrokecolor{currentstroke}%
\pgfsetdash{}{0pt}%
\pgfsys@defobject{currentmarker}{\pgfqpoint{0.000000in}{-0.027778in}}{\pgfqpoint{0.000000in}{0.000000in}}{%
\pgfpathmoveto{\pgfqpoint{0.000000in}{0.000000in}}%
\pgfpathlineto{\pgfqpoint{0.000000in}{-0.027778in}}%
\pgfusepath{stroke,fill}%
}%
\begin{pgfscope}%
\pgfsys@transformshift{0.607800in}{2.879314in}%
\pgfsys@useobject{currentmarker}{}%
\end{pgfscope}%
\end{pgfscope}%
\begin{pgfscope}%
\pgfsetbuttcap%
\pgfsetroundjoin%
\definecolor{currentfill}{rgb}{0.000000,0.000000,0.000000}%
\pgfsetfillcolor{currentfill}%
\pgfsetlinewidth{0.602250pt}%
\definecolor{currentstroke}{rgb}{0.000000,0.000000,0.000000}%
\pgfsetstrokecolor{currentstroke}%
\pgfsetdash{}{0pt}%
\pgfsys@defobject{currentmarker}{\pgfqpoint{0.000000in}{0.000000in}}{\pgfqpoint{0.000000in}{0.027778in}}{%
\pgfpathmoveto{\pgfqpoint{0.000000in}{0.000000in}}%
\pgfpathlineto{\pgfqpoint{0.000000in}{0.027778in}}%
\pgfusepath{stroke,fill}%
}%
\begin{pgfscope}%
\pgfsys@transformshift{0.970457in}{0.251500in}%
\pgfsys@useobject{currentmarker}{}%
\end{pgfscope}%
\end{pgfscope}%
\begin{pgfscope}%
\pgfsetbuttcap%
\pgfsetroundjoin%
\definecolor{currentfill}{rgb}{0.000000,0.000000,0.000000}%
\pgfsetfillcolor{currentfill}%
\pgfsetlinewidth{0.602250pt}%
\definecolor{currentstroke}{rgb}{0.000000,0.000000,0.000000}%
\pgfsetstrokecolor{currentstroke}%
\pgfsetdash{}{0pt}%
\pgfsys@defobject{currentmarker}{\pgfqpoint{0.000000in}{-0.027778in}}{\pgfqpoint{0.000000in}{0.000000in}}{%
\pgfpathmoveto{\pgfqpoint{0.000000in}{0.000000in}}%
\pgfpathlineto{\pgfqpoint{0.000000in}{-0.027778in}}%
\pgfusepath{stroke,fill}%
}%
\begin{pgfscope}%
\pgfsys@transformshift{0.970457in}{2.879314in}%
\pgfsys@useobject{currentmarker}{}%
\end{pgfscope}%
\end{pgfscope}%
\begin{pgfscope}%
\pgfsetbuttcap%
\pgfsetroundjoin%
\definecolor{currentfill}{rgb}{0.000000,0.000000,0.000000}%
\pgfsetfillcolor{currentfill}%
\pgfsetlinewidth{0.602250pt}%
\definecolor{currentstroke}{rgb}{0.000000,0.000000,0.000000}%
\pgfsetstrokecolor{currentstroke}%
\pgfsetdash{}{0pt}%
\pgfsys@defobject{currentmarker}{\pgfqpoint{0.000000in}{0.000000in}}{\pgfqpoint{0.000000in}{0.027778in}}{%
\pgfpathmoveto{\pgfqpoint{0.000000in}{0.000000in}}%
\pgfpathlineto{\pgfqpoint{0.000000in}{0.027778in}}%
\pgfusepath{stroke,fill}%
}%
\begin{pgfscope}%
\pgfsys@transformshift{1.151785in}{0.251500in}%
\pgfsys@useobject{currentmarker}{}%
\end{pgfscope}%
\end{pgfscope}%
\begin{pgfscope}%
\pgfsetbuttcap%
\pgfsetroundjoin%
\definecolor{currentfill}{rgb}{0.000000,0.000000,0.000000}%
\pgfsetfillcolor{currentfill}%
\pgfsetlinewidth{0.602250pt}%
\definecolor{currentstroke}{rgb}{0.000000,0.000000,0.000000}%
\pgfsetstrokecolor{currentstroke}%
\pgfsetdash{}{0pt}%
\pgfsys@defobject{currentmarker}{\pgfqpoint{0.000000in}{-0.027778in}}{\pgfqpoint{0.000000in}{0.000000in}}{%
\pgfpathmoveto{\pgfqpoint{0.000000in}{0.000000in}}%
\pgfpathlineto{\pgfqpoint{0.000000in}{-0.027778in}}%
\pgfusepath{stroke,fill}%
}%
\begin{pgfscope}%
\pgfsys@transformshift{1.151785in}{2.879314in}%
\pgfsys@useobject{currentmarker}{}%
\end{pgfscope}%
\end{pgfscope}%
\begin{pgfscope}%
\pgfsetbuttcap%
\pgfsetroundjoin%
\definecolor{currentfill}{rgb}{0.000000,0.000000,0.000000}%
\pgfsetfillcolor{currentfill}%
\pgfsetlinewidth{0.602250pt}%
\definecolor{currentstroke}{rgb}{0.000000,0.000000,0.000000}%
\pgfsetstrokecolor{currentstroke}%
\pgfsetdash{}{0pt}%
\pgfsys@defobject{currentmarker}{\pgfqpoint{0.000000in}{0.000000in}}{\pgfqpoint{0.000000in}{0.027778in}}{%
\pgfpathmoveto{\pgfqpoint{0.000000in}{0.000000in}}%
\pgfpathlineto{\pgfqpoint{0.000000in}{0.027778in}}%
\pgfusepath{stroke,fill}%
}%
\begin{pgfscope}%
\pgfsys@transformshift{1.333114in}{0.251500in}%
\pgfsys@useobject{currentmarker}{}%
\end{pgfscope}%
\end{pgfscope}%
\begin{pgfscope}%
\pgfsetbuttcap%
\pgfsetroundjoin%
\definecolor{currentfill}{rgb}{0.000000,0.000000,0.000000}%
\pgfsetfillcolor{currentfill}%
\pgfsetlinewidth{0.602250pt}%
\definecolor{currentstroke}{rgb}{0.000000,0.000000,0.000000}%
\pgfsetstrokecolor{currentstroke}%
\pgfsetdash{}{0pt}%
\pgfsys@defobject{currentmarker}{\pgfqpoint{0.000000in}{-0.027778in}}{\pgfqpoint{0.000000in}{0.000000in}}{%
\pgfpathmoveto{\pgfqpoint{0.000000in}{0.000000in}}%
\pgfpathlineto{\pgfqpoint{0.000000in}{-0.027778in}}%
\pgfusepath{stroke,fill}%
}%
\begin{pgfscope}%
\pgfsys@transformshift{1.333114in}{2.879314in}%
\pgfsys@useobject{currentmarker}{}%
\end{pgfscope}%
\end{pgfscope}%
\begin{pgfscope}%
\pgfsetbuttcap%
\pgfsetroundjoin%
\definecolor{currentfill}{rgb}{0.000000,0.000000,0.000000}%
\pgfsetfillcolor{currentfill}%
\pgfsetlinewidth{0.602250pt}%
\definecolor{currentstroke}{rgb}{0.000000,0.000000,0.000000}%
\pgfsetstrokecolor{currentstroke}%
\pgfsetdash{}{0pt}%
\pgfsys@defobject{currentmarker}{\pgfqpoint{0.000000in}{0.000000in}}{\pgfqpoint{0.000000in}{0.027778in}}{%
\pgfpathmoveto{\pgfqpoint{0.000000in}{0.000000in}}%
\pgfpathlineto{\pgfqpoint{0.000000in}{0.027778in}}%
\pgfusepath{stroke,fill}%
}%
\begin{pgfscope}%
\pgfsys@transformshift{1.695771in}{0.251500in}%
\pgfsys@useobject{currentmarker}{}%
\end{pgfscope}%
\end{pgfscope}%
\begin{pgfscope}%
\pgfsetbuttcap%
\pgfsetroundjoin%
\definecolor{currentfill}{rgb}{0.000000,0.000000,0.000000}%
\pgfsetfillcolor{currentfill}%
\pgfsetlinewidth{0.602250pt}%
\definecolor{currentstroke}{rgb}{0.000000,0.000000,0.000000}%
\pgfsetstrokecolor{currentstroke}%
\pgfsetdash{}{0pt}%
\pgfsys@defobject{currentmarker}{\pgfqpoint{0.000000in}{-0.027778in}}{\pgfqpoint{0.000000in}{0.000000in}}{%
\pgfpathmoveto{\pgfqpoint{0.000000in}{0.000000in}}%
\pgfpathlineto{\pgfqpoint{0.000000in}{-0.027778in}}%
\pgfusepath{stroke,fill}%
}%
\begin{pgfscope}%
\pgfsys@transformshift{1.695771in}{2.879314in}%
\pgfsys@useobject{currentmarker}{}%
\end{pgfscope}%
\end{pgfscope}%
\begin{pgfscope}%
\pgfsetbuttcap%
\pgfsetroundjoin%
\definecolor{currentfill}{rgb}{0.000000,0.000000,0.000000}%
\pgfsetfillcolor{currentfill}%
\pgfsetlinewidth{0.602250pt}%
\definecolor{currentstroke}{rgb}{0.000000,0.000000,0.000000}%
\pgfsetstrokecolor{currentstroke}%
\pgfsetdash{}{0pt}%
\pgfsys@defobject{currentmarker}{\pgfqpoint{0.000000in}{0.000000in}}{\pgfqpoint{0.000000in}{0.027778in}}{%
\pgfpathmoveto{\pgfqpoint{0.000000in}{0.000000in}}%
\pgfpathlineto{\pgfqpoint{0.000000in}{0.027778in}}%
\pgfusepath{stroke,fill}%
}%
\begin{pgfscope}%
\pgfsys@transformshift{1.877099in}{0.251500in}%
\pgfsys@useobject{currentmarker}{}%
\end{pgfscope}%
\end{pgfscope}%
\begin{pgfscope}%
\pgfsetbuttcap%
\pgfsetroundjoin%
\definecolor{currentfill}{rgb}{0.000000,0.000000,0.000000}%
\pgfsetfillcolor{currentfill}%
\pgfsetlinewidth{0.602250pt}%
\definecolor{currentstroke}{rgb}{0.000000,0.000000,0.000000}%
\pgfsetstrokecolor{currentstroke}%
\pgfsetdash{}{0pt}%
\pgfsys@defobject{currentmarker}{\pgfqpoint{0.000000in}{-0.027778in}}{\pgfqpoint{0.000000in}{0.000000in}}{%
\pgfpathmoveto{\pgfqpoint{0.000000in}{0.000000in}}%
\pgfpathlineto{\pgfqpoint{0.000000in}{-0.027778in}}%
\pgfusepath{stroke,fill}%
}%
\begin{pgfscope}%
\pgfsys@transformshift{1.877099in}{2.879314in}%
\pgfsys@useobject{currentmarker}{}%
\end{pgfscope}%
\end{pgfscope}%
\begin{pgfscope}%
\pgfsetbuttcap%
\pgfsetroundjoin%
\definecolor{currentfill}{rgb}{0.000000,0.000000,0.000000}%
\pgfsetfillcolor{currentfill}%
\pgfsetlinewidth{0.602250pt}%
\definecolor{currentstroke}{rgb}{0.000000,0.000000,0.000000}%
\pgfsetstrokecolor{currentstroke}%
\pgfsetdash{}{0pt}%
\pgfsys@defobject{currentmarker}{\pgfqpoint{0.000000in}{0.000000in}}{\pgfqpoint{0.000000in}{0.027778in}}{%
\pgfpathmoveto{\pgfqpoint{0.000000in}{0.000000in}}%
\pgfpathlineto{\pgfqpoint{0.000000in}{0.027778in}}%
\pgfusepath{stroke,fill}%
}%
\begin{pgfscope}%
\pgfsys@transformshift{2.058427in}{0.251500in}%
\pgfsys@useobject{currentmarker}{}%
\end{pgfscope}%
\end{pgfscope}%
\begin{pgfscope}%
\pgfsetbuttcap%
\pgfsetroundjoin%
\definecolor{currentfill}{rgb}{0.000000,0.000000,0.000000}%
\pgfsetfillcolor{currentfill}%
\pgfsetlinewidth{0.602250pt}%
\definecolor{currentstroke}{rgb}{0.000000,0.000000,0.000000}%
\pgfsetstrokecolor{currentstroke}%
\pgfsetdash{}{0pt}%
\pgfsys@defobject{currentmarker}{\pgfqpoint{0.000000in}{-0.027778in}}{\pgfqpoint{0.000000in}{0.000000in}}{%
\pgfpathmoveto{\pgfqpoint{0.000000in}{0.000000in}}%
\pgfpathlineto{\pgfqpoint{0.000000in}{-0.027778in}}%
\pgfusepath{stroke,fill}%
}%
\begin{pgfscope}%
\pgfsys@transformshift{2.058427in}{2.879314in}%
\pgfsys@useobject{currentmarker}{}%
\end{pgfscope}%
\end{pgfscope}%
\begin{pgfscope}%
\pgfsetbuttcap%
\pgfsetroundjoin%
\definecolor{currentfill}{rgb}{0.000000,0.000000,0.000000}%
\pgfsetfillcolor{currentfill}%
\pgfsetlinewidth{0.602250pt}%
\definecolor{currentstroke}{rgb}{0.000000,0.000000,0.000000}%
\pgfsetstrokecolor{currentstroke}%
\pgfsetdash{}{0pt}%
\pgfsys@defobject{currentmarker}{\pgfqpoint{0.000000in}{0.000000in}}{\pgfqpoint{0.000000in}{0.027778in}}{%
\pgfpathmoveto{\pgfqpoint{0.000000in}{0.000000in}}%
\pgfpathlineto{\pgfqpoint{0.000000in}{0.027778in}}%
\pgfusepath{stroke,fill}%
}%
\begin{pgfscope}%
\pgfsys@transformshift{2.421084in}{0.251500in}%
\pgfsys@useobject{currentmarker}{}%
\end{pgfscope}%
\end{pgfscope}%
\begin{pgfscope}%
\pgfsetbuttcap%
\pgfsetroundjoin%
\definecolor{currentfill}{rgb}{0.000000,0.000000,0.000000}%
\pgfsetfillcolor{currentfill}%
\pgfsetlinewidth{0.602250pt}%
\definecolor{currentstroke}{rgb}{0.000000,0.000000,0.000000}%
\pgfsetstrokecolor{currentstroke}%
\pgfsetdash{}{0pt}%
\pgfsys@defobject{currentmarker}{\pgfqpoint{0.000000in}{-0.027778in}}{\pgfqpoint{0.000000in}{0.000000in}}{%
\pgfpathmoveto{\pgfqpoint{0.000000in}{0.000000in}}%
\pgfpathlineto{\pgfqpoint{0.000000in}{-0.027778in}}%
\pgfusepath{stroke,fill}%
}%
\begin{pgfscope}%
\pgfsys@transformshift{2.421084in}{2.879314in}%
\pgfsys@useobject{currentmarker}{}%
\end{pgfscope}%
\end{pgfscope}%
\begin{pgfscope}%
\pgfsetbuttcap%
\pgfsetroundjoin%
\definecolor{currentfill}{rgb}{0.000000,0.000000,0.000000}%
\pgfsetfillcolor{currentfill}%
\pgfsetlinewidth{0.602250pt}%
\definecolor{currentstroke}{rgb}{0.000000,0.000000,0.000000}%
\pgfsetstrokecolor{currentstroke}%
\pgfsetdash{}{0pt}%
\pgfsys@defobject{currentmarker}{\pgfqpoint{0.000000in}{0.000000in}}{\pgfqpoint{0.000000in}{0.027778in}}{%
\pgfpathmoveto{\pgfqpoint{0.000000in}{0.000000in}}%
\pgfpathlineto{\pgfqpoint{0.000000in}{0.027778in}}%
\pgfusepath{stroke,fill}%
}%
\begin{pgfscope}%
\pgfsys@transformshift{2.602413in}{0.251500in}%
\pgfsys@useobject{currentmarker}{}%
\end{pgfscope}%
\end{pgfscope}%
\begin{pgfscope}%
\pgfsetbuttcap%
\pgfsetroundjoin%
\definecolor{currentfill}{rgb}{0.000000,0.000000,0.000000}%
\pgfsetfillcolor{currentfill}%
\pgfsetlinewidth{0.602250pt}%
\definecolor{currentstroke}{rgb}{0.000000,0.000000,0.000000}%
\pgfsetstrokecolor{currentstroke}%
\pgfsetdash{}{0pt}%
\pgfsys@defobject{currentmarker}{\pgfqpoint{0.000000in}{-0.027778in}}{\pgfqpoint{0.000000in}{0.000000in}}{%
\pgfpathmoveto{\pgfqpoint{0.000000in}{0.000000in}}%
\pgfpathlineto{\pgfqpoint{0.000000in}{-0.027778in}}%
\pgfusepath{stroke,fill}%
}%
\begin{pgfscope}%
\pgfsys@transformshift{2.602413in}{2.879314in}%
\pgfsys@useobject{currentmarker}{}%
\end{pgfscope}%
\end{pgfscope}%
\begin{pgfscope}%
\pgfsetbuttcap%
\pgfsetroundjoin%
\definecolor{currentfill}{rgb}{0.000000,0.000000,0.000000}%
\pgfsetfillcolor{currentfill}%
\pgfsetlinewidth{0.602250pt}%
\definecolor{currentstroke}{rgb}{0.000000,0.000000,0.000000}%
\pgfsetstrokecolor{currentstroke}%
\pgfsetdash{}{0pt}%
\pgfsys@defobject{currentmarker}{\pgfqpoint{0.000000in}{0.000000in}}{\pgfqpoint{0.000000in}{0.027778in}}{%
\pgfpathmoveto{\pgfqpoint{0.000000in}{0.000000in}}%
\pgfpathlineto{\pgfqpoint{0.000000in}{0.027778in}}%
\pgfusepath{stroke,fill}%
}%
\begin{pgfscope}%
\pgfsys@transformshift{2.783741in}{0.251500in}%
\pgfsys@useobject{currentmarker}{}%
\end{pgfscope}%
\end{pgfscope}%
\begin{pgfscope}%
\pgfsetbuttcap%
\pgfsetroundjoin%
\definecolor{currentfill}{rgb}{0.000000,0.000000,0.000000}%
\pgfsetfillcolor{currentfill}%
\pgfsetlinewidth{0.602250pt}%
\definecolor{currentstroke}{rgb}{0.000000,0.000000,0.000000}%
\pgfsetstrokecolor{currentstroke}%
\pgfsetdash{}{0pt}%
\pgfsys@defobject{currentmarker}{\pgfqpoint{0.000000in}{-0.027778in}}{\pgfqpoint{0.000000in}{0.000000in}}{%
\pgfpathmoveto{\pgfqpoint{0.000000in}{0.000000in}}%
\pgfpathlineto{\pgfqpoint{0.000000in}{-0.027778in}}%
\pgfusepath{stroke,fill}%
}%
\begin{pgfscope}%
\pgfsys@transformshift{2.783741in}{2.879314in}%
\pgfsys@useobject{currentmarker}{}%
\end{pgfscope}%
\end{pgfscope}%
\begin{pgfscope}%
\pgfsetbuttcap%
\pgfsetroundjoin%
\definecolor{currentfill}{rgb}{0.000000,0.000000,0.000000}%
\pgfsetfillcolor{currentfill}%
\pgfsetlinewidth{0.602250pt}%
\definecolor{currentstroke}{rgb}{0.000000,0.000000,0.000000}%
\pgfsetstrokecolor{currentstroke}%
\pgfsetdash{}{0pt}%
\pgfsys@defobject{currentmarker}{\pgfqpoint{0.000000in}{0.000000in}}{\pgfqpoint{0.000000in}{0.027778in}}{%
\pgfpathmoveto{\pgfqpoint{0.000000in}{0.000000in}}%
\pgfpathlineto{\pgfqpoint{0.000000in}{0.027778in}}%
\pgfusepath{stroke,fill}%
}%
\begin{pgfscope}%
\pgfsys@transformshift{3.146398in}{0.251500in}%
\pgfsys@useobject{currentmarker}{}%
\end{pgfscope}%
\end{pgfscope}%
\begin{pgfscope}%
\pgfsetbuttcap%
\pgfsetroundjoin%
\definecolor{currentfill}{rgb}{0.000000,0.000000,0.000000}%
\pgfsetfillcolor{currentfill}%
\pgfsetlinewidth{0.602250pt}%
\definecolor{currentstroke}{rgb}{0.000000,0.000000,0.000000}%
\pgfsetstrokecolor{currentstroke}%
\pgfsetdash{}{0pt}%
\pgfsys@defobject{currentmarker}{\pgfqpoint{0.000000in}{-0.027778in}}{\pgfqpoint{0.000000in}{0.000000in}}{%
\pgfpathmoveto{\pgfqpoint{0.000000in}{0.000000in}}%
\pgfpathlineto{\pgfqpoint{0.000000in}{-0.027778in}}%
\pgfusepath{stroke,fill}%
}%
\begin{pgfscope}%
\pgfsys@transformshift{3.146398in}{2.879314in}%
\pgfsys@useobject{currentmarker}{}%
\end{pgfscope}%
\end{pgfscope}%
\begin{pgfscope}%
\pgfsetbuttcap%
\pgfsetroundjoin%
\definecolor{currentfill}{rgb}{0.000000,0.000000,0.000000}%
\pgfsetfillcolor{currentfill}%
\pgfsetlinewidth{0.602250pt}%
\definecolor{currentstroke}{rgb}{0.000000,0.000000,0.000000}%
\pgfsetstrokecolor{currentstroke}%
\pgfsetdash{}{0pt}%
\pgfsys@defobject{currentmarker}{\pgfqpoint{0.000000in}{0.000000in}}{\pgfqpoint{0.000000in}{0.027778in}}{%
\pgfpathmoveto{\pgfqpoint{0.000000in}{0.000000in}}%
\pgfpathlineto{\pgfqpoint{0.000000in}{0.027778in}}%
\pgfusepath{stroke,fill}%
}%
\begin{pgfscope}%
\pgfsys@transformshift{3.327726in}{0.251500in}%
\pgfsys@useobject{currentmarker}{}%
\end{pgfscope}%
\end{pgfscope}%
\begin{pgfscope}%
\pgfsetbuttcap%
\pgfsetroundjoin%
\definecolor{currentfill}{rgb}{0.000000,0.000000,0.000000}%
\pgfsetfillcolor{currentfill}%
\pgfsetlinewidth{0.602250pt}%
\definecolor{currentstroke}{rgb}{0.000000,0.000000,0.000000}%
\pgfsetstrokecolor{currentstroke}%
\pgfsetdash{}{0pt}%
\pgfsys@defobject{currentmarker}{\pgfqpoint{0.000000in}{-0.027778in}}{\pgfqpoint{0.000000in}{0.000000in}}{%
\pgfpathmoveto{\pgfqpoint{0.000000in}{0.000000in}}%
\pgfpathlineto{\pgfqpoint{0.000000in}{-0.027778in}}%
\pgfusepath{stroke,fill}%
}%
\begin{pgfscope}%
\pgfsys@transformshift{3.327726in}{2.879314in}%
\pgfsys@useobject{currentmarker}{}%
\end{pgfscope}%
\end{pgfscope}%
\begin{pgfscope}%
\pgfsetbuttcap%
\pgfsetroundjoin%
\definecolor{currentfill}{rgb}{0.000000,0.000000,0.000000}%
\pgfsetfillcolor{currentfill}%
\pgfsetlinewidth{0.602250pt}%
\definecolor{currentstroke}{rgb}{0.000000,0.000000,0.000000}%
\pgfsetstrokecolor{currentstroke}%
\pgfsetdash{}{0pt}%
\pgfsys@defobject{currentmarker}{\pgfqpoint{0.000000in}{0.000000in}}{\pgfqpoint{0.000000in}{0.027778in}}{%
\pgfpathmoveto{\pgfqpoint{0.000000in}{0.000000in}}%
\pgfpathlineto{\pgfqpoint{0.000000in}{0.027778in}}%
\pgfusepath{stroke,fill}%
}%
\begin{pgfscope}%
\pgfsys@transformshift{3.509055in}{0.251500in}%
\pgfsys@useobject{currentmarker}{}%
\end{pgfscope}%
\end{pgfscope}%
\begin{pgfscope}%
\pgfsetbuttcap%
\pgfsetroundjoin%
\definecolor{currentfill}{rgb}{0.000000,0.000000,0.000000}%
\pgfsetfillcolor{currentfill}%
\pgfsetlinewidth{0.602250pt}%
\definecolor{currentstroke}{rgb}{0.000000,0.000000,0.000000}%
\pgfsetstrokecolor{currentstroke}%
\pgfsetdash{}{0pt}%
\pgfsys@defobject{currentmarker}{\pgfqpoint{0.000000in}{-0.027778in}}{\pgfqpoint{0.000000in}{0.000000in}}{%
\pgfpathmoveto{\pgfqpoint{0.000000in}{0.000000in}}%
\pgfpathlineto{\pgfqpoint{0.000000in}{-0.027778in}}%
\pgfusepath{stroke,fill}%
}%
\begin{pgfscope}%
\pgfsys@transformshift{3.509055in}{2.879314in}%
\pgfsys@useobject{currentmarker}{}%
\end{pgfscope}%
\end{pgfscope}%
\begin{pgfscope}%
\pgfsetbuttcap%
\pgfsetroundjoin%
\definecolor{currentfill}{rgb}{0.000000,0.000000,0.000000}%
\pgfsetfillcolor{currentfill}%
\pgfsetlinewidth{0.602250pt}%
\definecolor{currentstroke}{rgb}{0.000000,0.000000,0.000000}%
\pgfsetstrokecolor{currentstroke}%
\pgfsetdash{}{0pt}%
\pgfsys@defobject{currentmarker}{\pgfqpoint{0.000000in}{0.000000in}}{\pgfqpoint{0.000000in}{0.027778in}}{%
\pgfpathmoveto{\pgfqpoint{0.000000in}{0.000000in}}%
\pgfpathlineto{\pgfqpoint{0.000000in}{0.027778in}}%
\pgfusepath{stroke,fill}%
}%
\begin{pgfscope}%
\pgfsys@transformshift{3.871711in}{0.251500in}%
\pgfsys@useobject{currentmarker}{}%
\end{pgfscope}%
\end{pgfscope}%
\begin{pgfscope}%
\pgfsetbuttcap%
\pgfsetroundjoin%
\definecolor{currentfill}{rgb}{0.000000,0.000000,0.000000}%
\pgfsetfillcolor{currentfill}%
\pgfsetlinewidth{0.602250pt}%
\definecolor{currentstroke}{rgb}{0.000000,0.000000,0.000000}%
\pgfsetstrokecolor{currentstroke}%
\pgfsetdash{}{0pt}%
\pgfsys@defobject{currentmarker}{\pgfqpoint{0.000000in}{-0.027778in}}{\pgfqpoint{0.000000in}{0.000000in}}{%
\pgfpathmoveto{\pgfqpoint{0.000000in}{0.000000in}}%
\pgfpathlineto{\pgfqpoint{0.000000in}{-0.027778in}}%
\pgfusepath{stroke,fill}%
}%
\begin{pgfscope}%
\pgfsys@transformshift{3.871711in}{2.879314in}%
\pgfsys@useobject{currentmarker}{}%
\end{pgfscope}%
\end{pgfscope}%
\begin{pgfscope}%
\pgfsetbuttcap%
\pgfsetroundjoin%
\definecolor{currentfill}{rgb}{0.000000,0.000000,0.000000}%
\pgfsetfillcolor{currentfill}%
\pgfsetlinewidth{0.602250pt}%
\definecolor{currentstroke}{rgb}{0.000000,0.000000,0.000000}%
\pgfsetstrokecolor{currentstroke}%
\pgfsetdash{}{0pt}%
\pgfsys@defobject{currentmarker}{\pgfqpoint{0.000000in}{0.000000in}}{\pgfqpoint{0.000000in}{0.027778in}}{%
\pgfpathmoveto{\pgfqpoint{0.000000in}{0.000000in}}%
\pgfpathlineto{\pgfqpoint{0.000000in}{0.027778in}}%
\pgfusepath{stroke,fill}%
}%
\begin{pgfscope}%
\pgfsys@transformshift{4.053040in}{0.251500in}%
\pgfsys@useobject{currentmarker}{}%
\end{pgfscope}%
\end{pgfscope}%
\begin{pgfscope}%
\pgfsetbuttcap%
\pgfsetroundjoin%
\definecolor{currentfill}{rgb}{0.000000,0.000000,0.000000}%
\pgfsetfillcolor{currentfill}%
\pgfsetlinewidth{0.602250pt}%
\definecolor{currentstroke}{rgb}{0.000000,0.000000,0.000000}%
\pgfsetstrokecolor{currentstroke}%
\pgfsetdash{}{0pt}%
\pgfsys@defobject{currentmarker}{\pgfqpoint{0.000000in}{-0.027778in}}{\pgfqpoint{0.000000in}{0.000000in}}{%
\pgfpathmoveto{\pgfqpoint{0.000000in}{0.000000in}}%
\pgfpathlineto{\pgfqpoint{0.000000in}{-0.027778in}}%
\pgfusepath{stroke,fill}%
}%
\begin{pgfscope}%
\pgfsys@transformshift{4.053040in}{2.879314in}%
\pgfsys@useobject{currentmarker}{}%
\end{pgfscope}%
\end{pgfscope}%
\begin{pgfscope}%
\pgfsetbuttcap%
\pgfsetroundjoin%
\definecolor{currentfill}{rgb}{0.000000,0.000000,0.000000}%
\pgfsetfillcolor{currentfill}%
\pgfsetlinewidth{0.602250pt}%
\definecolor{currentstroke}{rgb}{0.000000,0.000000,0.000000}%
\pgfsetstrokecolor{currentstroke}%
\pgfsetdash{}{0pt}%
\pgfsys@defobject{currentmarker}{\pgfqpoint{0.000000in}{0.000000in}}{\pgfqpoint{0.000000in}{0.027778in}}{%
\pgfpathmoveto{\pgfqpoint{0.000000in}{0.000000in}}%
\pgfpathlineto{\pgfqpoint{0.000000in}{0.027778in}}%
\pgfusepath{stroke,fill}%
}%
\begin{pgfscope}%
\pgfsys@transformshift{4.234368in}{0.251500in}%
\pgfsys@useobject{currentmarker}{}%
\end{pgfscope}%
\end{pgfscope}%
\begin{pgfscope}%
\pgfsetbuttcap%
\pgfsetroundjoin%
\definecolor{currentfill}{rgb}{0.000000,0.000000,0.000000}%
\pgfsetfillcolor{currentfill}%
\pgfsetlinewidth{0.602250pt}%
\definecolor{currentstroke}{rgb}{0.000000,0.000000,0.000000}%
\pgfsetstrokecolor{currentstroke}%
\pgfsetdash{}{0pt}%
\pgfsys@defobject{currentmarker}{\pgfqpoint{0.000000in}{-0.027778in}}{\pgfqpoint{0.000000in}{0.000000in}}{%
\pgfpathmoveto{\pgfqpoint{0.000000in}{0.000000in}}%
\pgfpathlineto{\pgfqpoint{0.000000in}{-0.027778in}}%
\pgfusepath{stroke,fill}%
}%
\begin{pgfscope}%
\pgfsys@transformshift{4.234368in}{2.879314in}%
\pgfsys@useobject{currentmarker}{}%
\end{pgfscope}%
\end{pgfscope}%
\begin{pgfscope}%
\pgfsetbuttcap%
\pgfsetroundjoin%
\definecolor{currentfill}{rgb}{0.000000,0.000000,0.000000}%
\pgfsetfillcolor{currentfill}%
\pgfsetlinewidth{0.602250pt}%
\definecolor{currentstroke}{rgb}{0.000000,0.000000,0.000000}%
\pgfsetstrokecolor{currentstroke}%
\pgfsetdash{}{0pt}%
\pgfsys@defobject{currentmarker}{\pgfqpoint{0.000000in}{0.000000in}}{\pgfqpoint{0.000000in}{0.027778in}}{%
\pgfpathmoveto{\pgfqpoint{0.000000in}{0.000000in}}%
\pgfpathlineto{\pgfqpoint{0.000000in}{0.027778in}}%
\pgfusepath{stroke,fill}%
}%
\begin{pgfscope}%
\pgfsys@transformshift{4.597025in}{0.251500in}%
\pgfsys@useobject{currentmarker}{}%
\end{pgfscope}%
\end{pgfscope}%
\begin{pgfscope}%
\pgfsetbuttcap%
\pgfsetroundjoin%
\definecolor{currentfill}{rgb}{0.000000,0.000000,0.000000}%
\pgfsetfillcolor{currentfill}%
\pgfsetlinewidth{0.602250pt}%
\definecolor{currentstroke}{rgb}{0.000000,0.000000,0.000000}%
\pgfsetstrokecolor{currentstroke}%
\pgfsetdash{}{0pt}%
\pgfsys@defobject{currentmarker}{\pgfqpoint{0.000000in}{-0.027778in}}{\pgfqpoint{0.000000in}{0.000000in}}{%
\pgfpathmoveto{\pgfqpoint{0.000000in}{0.000000in}}%
\pgfpathlineto{\pgfqpoint{0.000000in}{-0.027778in}}%
\pgfusepath{stroke,fill}%
}%
\begin{pgfscope}%
\pgfsys@transformshift{4.597025in}{2.879314in}%
\pgfsys@useobject{currentmarker}{}%
\end{pgfscope}%
\end{pgfscope}%
\begin{pgfscope}%
\pgfsetbuttcap%
\pgfsetroundjoin%
\definecolor{currentfill}{rgb}{0.000000,0.000000,0.000000}%
\pgfsetfillcolor{currentfill}%
\pgfsetlinewidth{0.803000pt}%
\definecolor{currentstroke}{rgb}{0.000000,0.000000,0.000000}%
\pgfsetstrokecolor{currentstroke}%
\pgfsetdash{}{0pt}%
\pgfsys@defobject{currentmarker}{\pgfqpoint{0.000000in}{0.000000in}}{\pgfqpoint{0.048611in}{0.000000in}}{%
\pgfpathmoveto{\pgfqpoint{0.000000in}{0.000000in}}%
\pgfpathlineto{\pgfqpoint{0.048611in}{0.000000in}}%
\pgfusepath{stroke,fill}%
}%
\begin{pgfscope}%
\pgfsys@transformshift{0.607800in}{0.370945in}%
\pgfsys@useobject{currentmarker}{}%
\end{pgfscope}%
\end{pgfscope}%
\begin{pgfscope}%
\pgfsetbuttcap%
\pgfsetroundjoin%
\definecolor{currentfill}{rgb}{0.000000,0.000000,0.000000}%
\pgfsetfillcolor{currentfill}%
\pgfsetlinewidth{0.803000pt}%
\definecolor{currentstroke}{rgb}{0.000000,0.000000,0.000000}%
\pgfsetstrokecolor{currentstroke}%
\pgfsetdash{}{0pt}%
\pgfsys@defobject{currentmarker}{\pgfqpoint{-0.048611in}{0.000000in}}{\pgfqpoint{-0.000000in}{0.000000in}}{%
\pgfpathmoveto{\pgfqpoint{-0.000000in}{0.000000in}}%
\pgfpathlineto{\pgfqpoint{-0.048611in}{0.000000in}}%
\pgfusepath{stroke,fill}%
}%
\begin{pgfscope}%
\pgfsys@transformshift{4.597025in}{0.370945in}%
\pgfsys@useobject{currentmarker}{}%
\end{pgfscope}%
\end{pgfscope}%
\begin{pgfscope}%
\definecolor{textcolor}{rgb}{0.000000,0.000000,0.000000}%
\pgfsetstrokecolor{textcolor}%
\pgfsetfillcolor{textcolor}%
\pgftext[x=0.322778in, y=0.331722in, left, base]{\color{textcolor}{\rmfamily\fontsize{8.000000}{9.600000}\selectfont\catcode`\^=\active\def^{\ifmmode\sp\else\^{}\fi}\catcode`\%=\active\def%{\%}$\ensuremath{-}1$\,\unit{\volt}}}%
\end{pgfscope}%
\begin{pgfscope}%
\pgfsetbuttcap%
\pgfsetroundjoin%
\definecolor{currentfill}{rgb}{0.000000,0.000000,0.000000}%
\pgfsetfillcolor{currentfill}%
\pgfsetlinewidth{0.803000pt}%
\definecolor{currentstroke}{rgb}{0.000000,0.000000,0.000000}%
\pgfsetstrokecolor{currentstroke}%
\pgfsetdash{}{0pt}%
\pgfsys@defobject{currentmarker}{\pgfqpoint{0.000000in}{0.000000in}}{\pgfqpoint{0.048611in}{0.000000in}}{%
\pgfpathmoveto{\pgfqpoint{0.000000in}{0.000000in}}%
\pgfpathlineto{\pgfqpoint{0.048611in}{0.000000in}}%
\pgfusepath{stroke,fill}%
}%
\begin{pgfscope}%
\pgfsys@transformshift{0.607800in}{0.669560in}%
\pgfsys@useobject{currentmarker}{}%
\end{pgfscope}%
\end{pgfscope}%
\begin{pgfscope}%
\pgfsetbuttcap%
\pgfsetroundjoin%
\definecolor{currentfill}{rgb}{0.000000,0.000000,0.000000}%
\pgfsetfillcolor{currentfill}%
\pgfsetlinewidth{0.803000pt}%
\definecolor{currentstroke}{rgb}{0.000000,0.000000,0.000000}%
\pgfsetstrokecolor{currentstroke}%
\pgfsetdash{}{0pt}%
\pgfsys@defobject{currentmarker}{\pgfqpoint{-0.048611in}{0.000000in}}{\pgfqpoint{-0.000000in}{0.000000in}}{%
\pgfpathmoveto{\pgfqpoint{-0.000000in}{0.000000in}}%
\pgfpathlineto{\pgfqpoint{-0.048611in}{0.000000in}}%
\pgfusepath{stroke,fill}%
}%
\begin{pgfscope}%
\pgfsys@transformshift{4.597025in}{0.669560in}%
\pgfsys@useobject{currentmarker}{}%
\end{pgfscope}%
\end{pgfscope}%
\begin{pgfscope}%
\definecolor{textcolor}{rgb}{0.000000,0.000000,0.000000}%
\pgfsetstrokecolor{textcolor}%
\pgfsetfillcolor{textcolor}%
\pgftext[x=0.107444in, y=0.630338in, left, base]{\color{textcolor}{\rmfamily\fontsize{8.000000}{9.600000}\selectfont\catcode`\^=\active\def^{\ifmmode\sp\else\^{}\fi}\catcode`\%=\active\def%{\%}$\ensuremath{-}750$\,m\unit{\volt}}}%
\end{pgfscope}%
\begin{pgfscope}%
\pgfsetbuttcap%
\pgfsetroundjoin%
\definecolor{currentfill}{rgb}{0.000000,0.000000,0.000000}%
\pgfsetfillcolor{currentfill}%
\pgfsetlinewidth{0.803000pt}%
\definecolor{currentstroke}{rgb}{0.000000,0.000000,0.000000}%
\pgfsetstrokecolor{currentstroke}%
\pgfsetdash{}{0pt}%
\pgfsys@defobject{currentmarker}{\pgfqpoint{0.000000in}{0.000000in}}{\pgfqpoint{0.048611in}{0.000000in}}{%
\pgfpathmoveto{\pgfqpoint{0.000000in}{0.000000in}}%
\pgfpathlineto{\pgfqpoint{0.048611in}{0.000000in}}%
\pgfusepath{stroke,fill}%
}%
\begin{pgfscope}%
\pgfsys@transformshift{0.607800in}{0.968176in}%
\pgfsys@useobject{currentmarker}{}%
\end{pgfscope}%
\end{pgfscope}%
\begin{pgfscope}%
\pgfsetbuttcap%
\pgfsetroundjoin%
\definecolor{currentfill}{rgb}{0.000000,0.000000,0.000000}%
\pgfsetfillcolor{currentfill}%
\pgfsetlinewidth{0.803000pt}%
\definecolor{currentstroke}{rgb}{0.000000,0.000000,0.000000}%
\pgfsetstrokecolor{currentstroke}%
\pgfsetdash{}{0pt}%
\pgfsys@defobject{currentmarker}{\pgfqpoint{-0.048611in}{0.000000in}}{\pgfqpoint{-0.000000in}{0.000000in}}{%
\pgfpathmoveto{\pgfqpoint{-0.000000in}{0.000000in}}%
\pgfpathlineto{\pgfqpoint{-0.048611in}{0.000000in}}%
\pgfusepath{stroke,fill}%
}%
\begin{pgfscope}%
\pgfsys@transformshift{4.597025in}{0.968176in}%
\pgfsys@useobject{currentmarker}{}%
\end{pgfscope}%
\end{pgfscope}%
\begin{pgfscope}%
\definecolor{textcolor}{rgb}{0.000000,0.000000,0.000000}%
\pgfsetstrokecolor{textcolor}%
\pgfsetfillcolor{textcolor}%
\pgftext[x=0.100000in, y=0.928954in, left, base]{\color{textcolor}{\rmfamily\fontsize{8.000000}{9.600000}\selectfont\catcode`\^=\active\def^{\ifmmode\sp\else\^{}\fi}\catcode`\%=\active\def%{\%}$\ensuremath{-}500$\,m\unit{\volt}}}%
\end{pgfscope}%
\begin{pgfscope}%
\pgfsetbuttcap%
\pgfsetroundjoin%
\definecolor{currentfill}{rgb}{0.000000,0.000000,0.000000}%
\pgfsetfillcolor{currentfill}%
\pgfsetlinewidth{0.803000pt}%
\definecolor{currentstroke}{rgb}{0.000000,0.000000,0.000000}%
\pgfsetstrokecolor{currentstroke}%
\pgfsetdash{}{0pt}%
\pgfsys@defobject{currentmarker}{\pgfqpoint{0.000000in}{0.000000in}}{\pgfqpoint{0.048611in}{0.000000in}}{%
\pgfpathmoveto{\pgfqpoint{0.000000in}{0.000000in}}%
\pgfpathlineto{\pgfqpoint{0.048611in}{0.000000in}}%
\pgfusepath{stroke,fill}%
}%
\begin{pgfscope}%
\pgfsys@transformshift{0.607800in}{1.266792in}%
\pgfsys@useobject{currentmarker}{}%
\end{pgfscope}%
\end{pgfscope}%
\begin{pgfscope}%
\pgfsetbuttcap%
\pgfsetroundjoin%
\definecolor{currentfill}{rgb}{0.000000,0.000000,0.000000}%
\pgfsetfillcolor{currentfill}%
\pgfsetlinewidth{0.803000pt}%
\definecolor{currentstroke}{rgb}{0.000000,0.000000,0.000000}%
\pgfsetstrokecolor{currentstroke}%
\pgfsetdash{}{0pt}%
\pgfsys@defobject{currentmarker}{\pgfqpoint{-0.048611in}{0.000000in}}{\pgfqpoint{-0.000000in}{0.000000in}}{%
\pgfpathmoveto{\pgfqpoint{-0.000000in}{0.000000in}}%
\pgfpathlineto{\pgfqpoint{-0.048611in}{0.000000in}}%
\pgfusepath{stroke,fill}%
}%
\begin{pgfscope}%
\pgfsys@transformshift{4.597025in}{1.266792in}%
\pgfsys@useobject{currentmarker}{}%
\end{pgfscope}%
\end{pgfscope}%
\begin{pgfscope}%
\definecolor{textcolor}{rgb}{0.000000,0.000000,0.000000}%
\pgfsetstrokecolor{textcolor}%
\pgfsetfillcolor{textcolor}%
\pgftext[x=0.106222in, y=1.227569in, left, base]{\color{textcolor}{\rmfamily\fontsize{8.000000}{9.600000}\selectfont\catcode`\^=\active\def^{\ifmmode\sp\else\^{}\fi}\catcode`\%=\active\def%{\%}$\ensuremath{-}250$\,m\unit{\volt}}}%
\end{pgfscope}%
\begin{pgfscope}%
\pgfsetbuttcap%
\pgfsetroundjoin%
\definecolor{currentfill}{rgb}{0.000000,0.000000,0.000000}%
\pgfsetfillcolor{currentfill}%
\pgfsetlinewidth{0.803000pt}%
\definecolor{currentstroke}{rgb}{0.000000,0.000000,0.000000}%
\pgfsetstrokecolor{currentstroke}%
\pgfsetdash{}{0pt}%
\pgfsys@defobject{currentmarker}{\pgfqpoint{0.000000in}{0.000000in}}{\pgfqpoint{0.048611in}{0.000000in}}{%
\pgfpathmoveto{\pgfqpoint{0.000000in}{0.000000in}}%
\pgfpathlineto{\pgfqpoint{0.048611in}{0.000000in}}%
\pgfusepath{stroke,fill}%
}%
\begin{pgfscope}%
\pgfsys@transformshift{0.607800in}{1.565407in}%
\pgfsys@useobject{currentmarker}{}%
\end{pgfscope}%
\end{pgfscope}%
\begin{pgfscope}%
\pgfsetbuttcap%
\pgfsetroundjoin%
\definecolor{currentfill}{rgb}{0.000000,0.000000,0.000000}%
\pgfsetfillcolor{currentfill}%
\pgfsetlinewidth{0.803000pt}%
\definecolor{currentstroke}{rgb}{0.000000,0.000000,0.000000}%
\pgfsetstrokecolor{currentstroke}%
\pgfsetdash{}{0pt}%
\pgfsys@defobject{currentmarker}{\pgfqpoint{-0.048611in}{0.000000in}}{\pgfqpoint{-0.000000in}{0.000000in}}{%
\pgfpathmoveto{\pgfqpoint{-0.000000in}{0.000000in}}%
\pgfpathlineto{\pgfqpoint{-0.048611in}{0.000000in}}%
\pgfusepath{stroke,fill}%
}%
\begin{pgfscope}%
\pgfsys@transformshift{4.597025in}{1.565407in}%
\pgfsys@useobject{currentmarker}{}%
\end{pgfscope}%
\end{pgfscope}%
\begin{pgfscope}%
\definecolor{textcolor}{rgb}{0.000000,0.000000,0.000000}%
\pgfsetstrokecolor{textcolor}%
\pgfsetfillcolor{textcolor}%
\pgftext[x=0.400667in, y=1.526185in, left, base]{\color{textcolor}{\rmfamily\fontsize{8.000000}{9.600000}\selectfont\catcode`\^=\active\def^{\ifmmode\sp\else\^{}\fi}\catcode`\%=\active\def%{\%}$0$\,\unit{\volt}}}%
\end{pgfscope}%
\begin{pgfscope}%
\pgfsetbuttcap%
\pgfsetroundjoin%
\definecolor{currentfill}{rgb}{0.000000,0.000000,0.000000}%
\pgfsetfillcolor{currentfill}%
\pgfsetlinewidth{0.803000pt}%
\definecolor{currentstroke}{rgb}{0.000000,0.000000,0.000000}%
\pgfsetstrokecolor{currentstroke}%
\pgfsetdash{}{0pt}%
\pgfsys@defobject{currentmarker}{\pgfqpoint{0.000000in}{0.000000in}}{\pgfqpoint{0.048611in}{0.000000in}}{%
\pgfpathmoveto{\pgfqpoint{0.000000in}{0.000000in}}%
\pgfpathlineto{\pgfqpoint{0.048611in}{0.000000in}}%
\pgfusepath{stroke,fill}%
}%
\begin{pgfscope}%
\pgfsys@transformshift{0.607800in}{1.864023in}%
\pgfsys@useobject{currentmarker}{}%
\end{pgfscope}%
\end{pgfscope}%
\begin{pgfscope}%
\pgfsetbuttcap%
\pgfsetroundjoin%
\definecolor{currentfill}{rgb}{0.000000,0.000000,0.000000}%
\pgfsetfillcolor{currentfill}%
\pgfsetlinewidth{0.803000pt}%
\definecolor{currentstroke}{rgb}{0.000000,0.000000,0.000000}%
\pgfsetstrokecolor{currentstroke}%
\pgfsetdash{}{0pt}%
\pgfsys@defobject{currentmarker}{\pgfqpoint{-0.048611in}{0.000000in}}{\pgfqpoint{-0.000000in}{0.000000in}}{%
\pgfpathmoveto{\pgfqpoint{-0.000000in}{0.000000in}}%
\pgfpathlineto{\pgfqpoint{-0.048611in}{0.000000in}}%
\pgfusepath{stroke,fill}%
}%
\begin{pgfscope}%
\pgfsys@transformshift{4.597025in}{1.864023in}%
\pgfsys@useobject{currentmarker}{}%
\end{pgfscope}%
\end{pgfscope}%
\begin{pgfscope}%
\definecolor{textcolor}{rgb}{0.000000,0.000000,0.000000}%
\pgfsetstrokecolor{textcolor}%
\pgfsetfillcolor{textcolor}%
\pgftext[x=0.192000in, y=1.824801in, left, base]{\color{textcolor}{\rmfamily\fontsize{8.000000}{9.600000}\selectfont\catcode`\^=\active\def^{\ifmmode\sp\else\^{}\fi}\catcode`\%=\active\def%{\%}$250$\,m\unit{\volt}}}%
\end{pgfscope}%
\begin{pgfscope}%
\pgfsetbuttcap%
\pgfsetroundjoin%
\definecolor{currentfill}{rgb}{0.000000,0.000000,0.000000}%
\pgfsetfillcolor{currentfill}%
\pgfsetlinewidth{0.803000pt}%
\definecolor{currentstroke}{rgb}{0.000000,0.000000,0.000000}%
\pgfsetstrokecolor{currentstroke}%
\pgfsetdash{}{0pt}%
\pgfsys@defobject{currentmarker}{\pgfqpoint{0.000000in}{0.000000in}}{\pgfqpoint{0.048611in}{0.000000in}}{%
\pgfpathmoveto{\pgfqpoint{0.000000in}{0.000000in}}%
\pgfpathlineto{\pgfqpoint{0.048611in}{0.000000in}}%
\pgfusepath{stroke,fill}%
}%
\begin{pgfscope}%
\pgfsys@transformshift{0.607800in}{2.162638in}%
\pgfsys@useobject{currentmarker}{}%
\end{pgfscope}%
\end{pgfscope}%
\begin{pgfscope}%
\pgfsetbuttcap%
\pgfsetroundjoin%
\definecolor{currentfill}{rgb}{0.000000,0.000000,0.000000}%
\pgfsetfillcolor{currentfill}%
\pgfsetlinewidth{0.803000pt}%
\definecolor{currentstroke}{rgb}{0.000000,0.000000,0.000000}%
\pgfsetstrokecolor{currentstroke}%
\pgfsetdash{}{0pt}%
\pgfsys@defobject{currentmarker}{\pgfqpoint{-0.048611in}{0.000000in}}{\pgfqpoint{-0.000000in}{0.000000in}}{%
\pgfpathmoveto{\pgfqpoint{-0.000000in}{0.000000in}}%
\pgfpathlineto{\pgfqpoint{-0.048611in}{0.000000in}}%
\pgfusepath{stroke,fill}%
}%
\begin{pgfscope}%
\pgfsys@transformshift{4.597025in}{2.162638in}%
\pgfsys@useobject{currentmarker}{}%
\end{pgfscope}%
\end{pgfscope}%
\begin{pgfscope}%
\definecolor{textcolor}{rgb}{0.000000,0.000000,0.000000}%
\pgfsetstrokecolor{textcolor}%
\pgfsetfillcolor{textcolor}%
\pgftext[x=0.185778in, y=2.123416in, left, base]{\color{textcolor}{\rmfamily\fontsize{8.000000}{9.600000}\selectfont\catcode`\^=\active\def^{\ifmmode\sp\else\^{}\fi}\catcode`\%=\active\def%{\%}$500$\,m\unit{\volt}}}%
\end{pgfscope}%
\begin{pgfscope}%
\pgfsetbuttcap%
\pgfsetroundjoin%
\definecolor{currentfill}{rgb}{0.000000,0.000000,0.000000}%
\pgfsetfillcolor{currentfill}%
\pgfsetlinewidth{0.803000pt}%
\definecolor{currentstroke}{rgb}{0.000000,0.000000,0.000000}%
\pgfsetstrokecolor{currentstroke}%
\pgfsetdash{}{0pt}%
\pgfsys@defobject{currentmarker}{\pgfqpoint{0.000000in}{0.000000in}}{\pgfqpoint{0.048611in}{0.000000in}}{%
\pgfpathmoveto{\pgfqpoint{0.000000in}{0.000000in}}%
\pgfpathlineto{\pgfqpoint{0.048611in}{0.000000in}}%
\pgfusepath{stroke,fill}%
}%
\begin{pgfscope}%
\pgfsys@transformshift{0.607800in}{2.461254in}%
\pgfsys@useobject{currentmarker}{}%
\end{pgfscope}%
\end{pgfscope}%
\begin{pgfscope}%
\pgfsetbuttcap%
\pgfsetroundjoin%
\definecolor{currentfill}{rgb}{0.000000,0.000000,0.000000}%
\pgfsetfillcolor{currentfill}%
\pgfsetlinewidth{0.803000pt}%
\definecolor{currentstroke}{rgb}{0.000000,0.000000,0.000000}%
\pgfsetstrokecolor{currentstroke}%
\pgfsetdash{}{0pt}%
\pgfsys@defobject{currentmarker}{\pgfqpoint{-0.048611in}{0.000000in}}{\pgfqpoint{-0.000000in}{0.000000in}}{%
\pgfpathmoveto{\pgfqpoint{-0.000000in}{0.000000in}}%
\pgfpathlineto{\pgfqpoint{-0.048611in}{0.000000in}}%
\pgfusepath{stroke,fill}%
}%
\begin{pgfscope}%
\pgfsys@transformshift{4.597025in}{2.461254in}%
\pgfsys@useobject{currentmarker}{}%
\end{pgfscope}%
\end{pgfscope}%
\begin{pgfscope}%
\definecolor{textcolor}{rgb}{0.000000,0.000000,0.000000}%
\pgfsetstrokecolor{textcolor}%
\pgfsetfillcolor{textcolor}%
\pgftext[x=0.193222in, y=2.422032in, left, base]{\color{textcolor}{\rmfamily\fontsize{8.000000}{9.600000}\selectfont\catcode`\^=\active\def^{\ifmmode\sp\else\^{}\fi}\catcode`\%=\active\def%{\%}$750$\,m\unit{\volt}}}%
\end{pgfscope}%
\begin{pgfscope}%
\pgfsetbuttcap%
\pgfsetroundjoin%
\definecolor{currentfill}{rgb}{0.000000,0.000000,0.000000}%
\pgfsetfillcolor{currentfill}%
\pgfsetlinewidth{0.803000pt}%
\definecolor{currentstroke}{rgb}{0.000000,0.000000,0.000000}%
\pgfsetstrokecolor{currentstroke}%
\pgfsetdash{}{0pt}%
\pgfsys@defobject{currentmarker}{\pgfqpoint{0.000000in}{0.000000in}}{\pgfqpoint{0.048611in}{0.000000in}}{%
\pgfpathmoveto{\pgfqpoint{0.000000in}{0.000000in}}%
\pgfpathlineto{\pgfqpoint{0.048611in}{0.000000in}}%
\pgfusepath{stroke,fill}%
}%
\begin{pgfscope}%
\pgfsys@transformshift{0.607800in}{2.759870in}%
\pgfsys@useobject{currentmarker}{}%
\end{pgfscope}%
\end{pgfscope}%
\begin{pgfscope}%
\pgfsetbuttcap%
\pgfsetroundjoin%
\definecolor{currentfill}{rgb}{0.000000,0.000000,0.000000}%
\pgfsetfillcolor{currentfill}%
\pgfsetlinewidth{0.803000pt}%
\definecolor{currentstroke}{rgb}{0.000000,0.000000,0.000000}%
\pgfsetstrokecolor{currentstroke}%
\pgfsetdash{}{0pt}%
\pgfsys@defobject{currentmarker}{\pgfqpoint{-0.048611in}{0.000000in}}{\pgfqpoint{-0.000000in}{0.000000in}}{%
\pgfpathmoveto{\pgfqpoint{-0.000000in}{0.000000in}}%
\pgfpathlineto{\pgfqpoint{-0.048611in}{0.000000in}}%
\pgfusepath{stroke,fill}%
}%
\begin{pgfscope}%
\pgfsys@transformshift{4.597025in}{2.759870in}%
\pgfsys@useobject{currentmarker}{}%
\end{pgfscope}%
\end{pgfscope}%
\begin{pgfscope}%
\definecolor{textcolor}{rgb}{0.000000,0.000000,0.000000}%
\pgfsetstrokecolor{textcolor}%
\pgfsetfillcolor{textcolor}%
\pgftext[x=0.408555in, y=2.720647in, left, base]{\color{textcolor}{\rmfamily\fontsize{8.000000}{9.600000}\selectfont\catcode`\^=\active\def^{\ifmmode\sp\else\^{}\fi}\catcode`\%=\active\def%{\%}$1$\,\unit{\volt}}}%
\end{pgfscope}%
\begin{pgfscope}%
\pgfsetbuttcap%
\pgfsetroundjoin%
\definecolor{currentfill}{rgb}{0.000000,0.000000,0.000000}%
\pgfsetfillcolor{currentfill}%
\pgfsetlinewidth{0.602250pt}%
\definecolor{currentstroke}{rgb}{0.000000,0.000000,0.000000}%
\pgfsetstrokecolor{currentstroke}%
\pgfsetdash{}{0pt}%
\pgfsys@defobject{currentmarker}{\pgfqpoint{0.000000in}{0.000000in}}{\pgfqpoint{0.027778in}{0.000000in}}{%
\pgfpathmoveto{\pgfqpoint{0.000000in}{0.000000in}}%
\pgfpathlineto{\pgfqpoint{0.027778in}{0.000000in}}%
\pgfusepath{stroke,fill}%
}%
\begin{pgfscope}%
\pgfsys@transformshift{0.607800in}{0.311222in}%
\pgfsys@useobject{currentmarker}{}%
\end{pgfscope}%
\end{pgfscope}%
\begin{pgfscope}%
\pgfsetbuttcap%
\pgfsetroundjoin%
\definecolor{currentfill}{rgb}{0.000000,0.000000,0.000000}%
\pgfsetfillcolor{currentfill}%
\pgfsetlinewidth{0.602250pt}%
\definecolor{currentstroke}{rgb}{0.000000,0.000000,0.000000}%
\pgfsetstrokecolor{currentstroke}%
\pgfsetdash{}{0pt}%
\pgfsys@defobject{currentmarker}{\pgfqpoint{-0.027778in}{0.000000in}}{\pgfqpoint{-0.000000in}{0.000000in}}{%
\pgfpathmoveto{\pgfqpoint{-0.000000in}{0.000000in}}%
\pgfpathlineto{\pgfqpoint{-0.027778in}{0.000000in}}%
\pgfusepath{stroke,fill}%
}%
\begin{pgfscope}%
\pgfsys@transformshift{4.597025in}{0.311222in}%
\pgfsys@useobject{currentmarker}{}%
\end{pgfscope}%
\end{pgfscope}%
\begin{pgfscope}%
\pgfsetbuttcap%
\pgfsetroundjoin%
\definecolor{currentfill}{rgb}{0.000000,0.000000,0.000000}%
\pgfsetfillcolor{currentfill}%
\pgfsetlinewidth{0.602250pt}%
\definecolor{currentstroke}{rgb}{0.000000,0.000000,0.000000}%
\pgfsetstrokecolor{currentstroke}%
\pgfsetdash{}{0pt}%
\pgfsys@defobject{currentmarker}{\pgfqpoint{0.000000in}{0.000000in}}{\pgfqpoint{0.027778in}{0.000000in}}{%
\pgfpathmoveto{\pgfqpoint{0.000000in}{0.000000in}}%
\pgfpathlineto{\pgfqpoint{0.027778in}{0.000000in}}%
\pgfusepath{stroke,fill}%
}%
\begin{pgfscope}%
\pgfsys@transformshift{0.607800in}{0.430668in}%
\pgfsys@useobject{currentmarker}{}%
\end{pgfscope}%
\end{pgfscope}%
\begin{pgfscope}%
\pgfsetbuttcap%
\pgfsetroundjoin%
\definecolor{currentfill}{rgb}{0.000000,0.000000,0.000000}%
\pgfsetfillcolor{currentfill}%
\pgfsetlinewidth{0.602250pt}%
\definecolor{currentstroke}{rgb}{0.000000,0.000000,0.000000}%
\pgfsetstrokecolor{currentstroke}%
\pgfsetdash{}{0pt}%
\pgfsys@defobject{currentmarker}{\pgfqpoint{-0.027778in}{0.000000in}}{\pgfqpoint{-0.000000in}{0.000000in}}{%
\pgfpathmoveto{\pgfqpoint{-0.000000in}{0.000000in}}%
\pgfpathlineto{\pgfqpoint{-0.027778in}{0.000000in}}%
\pgfusepath{stroke,fill}%
}%
\begin{pgfscope}%
\pgfsys@transformshift{4.597025in}{0.430668in}%
\pgfsys@useobject{currentmarker}{}%
\end{pgfscope}%
\end{pgfscope}%
\begin{pgfscope}%
\pgfsetbuttcap%
\pgfsetroundjoin%
\definecolor{currentfill}{rgb}{0.000000,0.000000,0.000000}%
\pgfsetfillcolor{currentfill}%
\pgfsetlinewidth{0.602250pt}%
\definecolor{currentstroke}{rgb}{0.000000,0.000000,0.000000}%
\pgfsetstrokecolor{currentstroke}%
\pgfsetdash{}{0pt}%
\pgfsys@defobject{currentmarker}{\pgfqpoint{0.000000in}{0.000000in}}{\pgfqpoint{0.027778in}{0.000000in}}{%
\pgfpathmoveto{\pgfqpoint{0.000000in}{0.000000in}}%
\pgfpathlineto{\pgfqpoint{0.027778in}{0.000000in}}%
\pgfusepath{stroke,fill}%
}%
\begin{pgfscope}%
\pgfsys@transformshift{0.607800in}{0.490391in}%
\pgfsys@useobject{currentmarker}{}%
\end{pgfscope}%
\end{pgfscope}%
\begin{pgfscope}%
\pgfsetbuttcap%
\pgfsetroundjoin%
\definecolor{currentfill}{rgb}{0.000000,0.000000,0.000000}%
\pgfsetfillcolor{currentfill}%
\pgfsetlinewidth{0.602250pt}%
\definecolor{currentstroke}{rgb}{0.000000,0.000000,0.000000}%
\pgfsetstrokecolor{currentstroke}%
\pgfsetdash{}{0pt}%
\pgfsys@defobject{currentmarker}{\pgfqpoint{-0.027778in}{0.000000in}}{\pgfqpoint{-0.000000in}{0.000000in}}{%
\pgfpathmoveto{\pgfqpoint{-0.000000in}{0.000000in}}%
\pgfpathlineto{\pgfqpoint{-0.027778in}{0.000000in}}%
\pgfusepath{stroke,fill}%
}%
\begin{pgfscope}%
\pgfsys@transformshift{4.597025in}{0.490391in}%
\pgfsys@useobject{currentmarker}{}%
\end{pgfscope}%
\end{pgfscope}%
\begin{pgfscope}%
\pgfsetbuttcap%
\pgfsetroundjoin%
\definecolor{currentfill}{rgb}{0.000000,0.000000,0.000000}%
\pgfsetfillcolor{currentfill}%
\pgfsetlinewidth{0.602250pt}%
\definecolor{currentstroke}{rgb}{0.000000,0.000000,0.000000}%
\pgfsetstrokecolor{currentstroke}%
\pgfsetdash{}{0pt}%
\pgfsys@defobject{currentmarker}{\pgfqpoint{0.000000in}{0.000000in}}{\pgfqpoint{0.027778in}{0.000000in}}{%
\pgfpathmoveto{\pgfqpoint{0.000000in}{0.000000in}}%
\pgfpathlineto{\pgfqpoint{0.027778in}{0.000000in}}%
\pgfusepath{stroke,fill}%
}%
\begin{pgfscope}%
\pgfsys@transformshift{0.607800in}{0.550114in}%
\pgfsys@useobject{currentmarker}{}%
\end{pgfscope}%
\end{pgfscope}%
\begin{pgfscope}%
\pgfsetbuttcap%
\pgfsetroundjoin%
\definecolor{currentfill}{rgb}{0.000000,0.000000,0.000000}%
\pgfsetfillcolor{currentfill}%
\pgfsetlinewidth{0.602250pt}%
\definecolor{currentstroke}{rgb}{0.000000,0.000000,0.000000}%
\pgfsetstrokecolor{currentstroke}%
\pgfsetdash{}{0pt}%
\pgfsys@defobject{currentmarker}{\pgfqpoint{-0.027778in}{0.000000in}}{\pgfqpoint{-0.000000in}{0.000000in}}{%
\pgfpathmoveto{\pgfqpoint{-0.000000in}{0.000000in}}%
\pgfpathlineto{\pgfqpoint{-0.027778in}{0.000000in}}%
\pgfusepath{stroke,fill}%
}%
\begin{pgfscope}%
\pgfsys@transformshift{4.597025in}{0.550114in}%
\pgfsys@useobject{currentmarker}{}%
\end{pgfscope}%
\end{pgfscope}%
\begin{pgfscope}%
\pgfsetbuttcap%
\pgfsetroundjoin%
\definecolor{currentfill}{rgb}{0.000000,0.000000,0.000000}%
\pgfsetfillcolor{currentfill}%
\pgfsetlinewidth{0.602250pt}%
\definecolor{currentstroke}{rgb}{0.000000,0.000000,0.000000}%
\pgfsetstrokecolor{currentstroke}%
\pgfsetdash{}{0pt}%
\pgfsys@defobject{currentmarker}{\pgfqpoint{0.000000in}{0.000000in}}{\pgfqpoint{0.027778in}{0.000000in}}{%
\pgfpathmoveto{\pgfqpoint{0.000000in}{0.000000in}}%
\pgfpathlineto{\pgfqpoint{0.027778in}{0.000000in}}%
\pgfusepath{stroke,fill}%
}%
\begin{pgfscope}%
\pgfsys@transformshift{0.607800in}{0.609837in}%
\pgfsys@useobject{currentmarker}{}%
\end{pgfscope}%
\end{pgfscope}%
\begin{pgfscope}%
\pgfsetbuttcap%
\pgfsetroundjoin%
\definecolor{currentfill}{rgb}{0.000000,0.000000,0.000000}%
\pgfsetfillcolor{currentfill}%
\pgfsetlinewidth{0.602250pt}%
\definecolor{currentstroke}{rgb}{0.000000,0.000000,0.000000}%
\pgfsetstrokecolor{currentstroke}%
\pgfsetdash{}{0pt}%
\pgfsys@defobject{currentmarker}{\pgfqpoint{-0.027778in}{0.000000in}}{\pgfqpoint{-0.000000in}{0.000000in}}{%
\pgfpathmoveto{\pgfqpoint{-0.000000in}{0.000000in}}%
\pgfpathlineto{\pgfqpoint{-0.027778in}{0.000000in}}%
\pgfusepath{stroke,fill}%
}%
\begin{pgfscope}%
\pgfsys@transformshift{4.597025in}{0.609837in}%
\pgfsys@useobject{currentmarker}{}%
\end{pgfscope}%
\end{pgfscope}%
\begin{pgfscope}%
\pgfsetbuttcap%
\pgfsetroundjoin%
\definecolor{currentfill}{rgb}{0.000000,0.000000,0.000000}%
\pgfsetfillcolor{currentfill}%
\pgfsetlinewidth{0.602250pt}%
\definecolor{currentstroke}{rgb}{0.000000,0.000000,0.000000}%
\pgfsetstrokecolor{currentstroke}%
\pgfsetdash{}{0pt}%
\pgfsys@defobject{currentmarker}{\pgfqpoint{0.000000in}{0.000000in}}{\pgfqpoint{0.027778in}{0.000000in}}{%
\pgfpathmoveto{\pgfqpoint{0.000000in}{0.000000in}}%
\pgfpathlineto{\pgfqpoint{0.027778in}{0.000000in}}%
\pgfusepath{stroke,fill}%
}%
\begin{pgfscope}%
\pgfsys@transformshift{0.607800in}{0.729283in}%
\pgfsys@useobject{currentmarker}{}%
\end{pgfscope}%
\end{pgfscope}%
\begin{pgfscope}%
\pgfsetbuttcap%
\pgfsetroundjoin%
\definecolor{currentfill}{rgb}{0.000000,0.000000,0.000000}%
\pgfsetfillcolor{currentfill}%
\pgfsetlinewidth{0.602250pt}%
\definecolor{currentstroke}{rgb}{0.000000,0.000000,0.000000}%
\pgfsetstrokecolor{currentstroke}%
\pgfsetdash{}{0pt}%
\pgfsys@defobject{currentmarker}{\pgfqpoint{-0.027778in}{0.000000in}}{\pgfqpoint{-0.000000in}{0.000000in}}{%
\pgfpathmoveto{\pgfqpoint{-0.000000in}{0.000000in}}%
\pgfpathlineto{\pgfqpoint{-0.027778in}{0.000000in}}%
\pgfusepath{stroke,fill}%
}%
\begin{pgfscope}%
\pgfsys@transformshift{4.597025in}{0.729283in}%
\pgfsys@useobject{currentmarker}{}%
\end{pgfscope}%
\end{pgfscope}%
\begin{pgfscope}%
\pgfsetbuttcap%
\pgfsetroundjoin%
\definecolor{currentfill}{rgb}{0.000000,0.000000,0.000000}%
\pgfsetfillcolor{currentfill}%
\pgfsetlinewidth{0.602250pt}%
\definecolor{currentstroke}{rgb}{0.000000,0.000000,0.000000}%
\pgfsetstrokecolor{currentstroke}%
\pgfsetdash{}{0pt}%
\pgfsys@defobject{currentmarker}{\pgfqpoint{0.000000in}{0.000000in}}{\pgfqpoint{0.027778in}{0.000000in}}{%
\pgfpathmoveto{\pgfqpoint{0.000000in}{0.000000in}}%
\pgfpathlineto{\pgfqpoint{0.027778in}{0.000000in}}%
\pgfusepath{stroke,fill}%
}%
\begin{pgfscope}%
\pgfsys@transformshift{0.607800in}{0.789007in}%
\pgfsys@useobject{currentmarker}{}%
\end{pgfscope}%
\end{pgfscope}%
\begin{pgfscope}%
\pgfsetbuttcap%
\pgfsetroundjoin%
\definecolor{currentfill}{rgb}{0.000000,0.000000,0.000000}%
\pgfsetfillcolor{currentfill}%
\pgfsetlinewidth{0.602250pt}%
\definecolor{currentstroke}{rgb}{0.000000,0.000000,0.000000}%
\pgfsetstrokecolor{currentstroke}%
\pgfsetdash{}{0pt}%
\pgfsys@defobject{currentmarker}{\pgfqpoint{-0.027778in}{0.000000in}}{\pgfqpoint{-0.000000in}{0.000000in}}{%
\pgfpathmoveto{\pgfqpoint{-0.000000in}{0.000000in}}%
\pgfpathlineto{\pgfqpoint{-0.027778in}{0.000000in}}%
\pgfusepath{stroke,fill}%
}%
\begin{pgfscope}%
\pgfsys@transformshift{4.597025in}{0.789007in}%
\pgfsys@useobject{currentmarker}{}%
\end{pgfscope}%
\end{pgfscope}%
\begin{pgfscope}%
\pgfsetbuttcap%
\pgfsetroundjoin%
\definecolor{currentfill}{rgb}{0.000000,0.000000,0.000000}%
\pgfsetfillcolor{currentfill}%
\pgfsetlinewidth{0.602250pt}%
\definecolor{currentstroke}{rgb}{0.000000,0.000000,0.000000}%
\pgfsetstrokecolor{currentstroke}%
\pgfsetdash{}{0pt}%
\pgfsys@defobject{currentmarker}{\pgfqpoint{0.000000in}{0.000000in}}{\pgfqpoint{0.027778in}{0.000000in}}{%
\pgfpathmoveto{\pgfqpoint{0.000000in}{0.000000in}}%
\pgfpathlineto{\pgfqpoint{0.027778in}{0.000000in}}%
\pgfusepath{stroke,fill}%
}%
\begin{pgfscope}%
\pgfsys@transformshift{0.607800in}{0.848730in}%
\pgfsys@useobject{currentmarker}{}%
\end{pgfscope}%
\end{pgfscope}%
\begin{pgfscope}%
\pgfsetbuttcap%
\pgfsetroundjoin%
\definecolor{currentfill}{rgb}{0.000000,0.000000,0.000000}%
\pgfsetfillcolor{currentfill}%
\pgfsetlinewidth{0.602250pt}%
\definecolor{currentstroke}{rgb}{0.000000,0.000000,0.000000}%
\pgfsetstrokecolor{currentstroke}%
\pgfsetdash{}{0pt}%
\pgfsys@defobject{currentmarker}{\pgfqpoint{-0.027778in}{0.000000in}}{\pgfqpoint{-0.000000in}{0.000000in}}{%
\pgfpathmoveto{\pgfqpoint{-0.000000in}{0.000000in}}%
\pgfpathlineto{\pgfqpoint{-0.027778in}{0.000000in}}%
\pgfusepath{stroke,fill}%
}%
\begin{pgfscope}%
\pgfsys@transformshift{4.597025in}{0.848730in}%
\pgfsys@useobject{currentmarker}{}%
\end{pgfscope}%
\end{pgfscope}%
\begin{pgfscope}%
\pgfsetbuttcap%
\pgfsetroundjoin%
\definecolor{currentfill}{rgb}{0.000000,0.000000,0.000000}%
\pgfsetfillcolor{currentfill}%
\pgfsetlinewidth{0.602250pt}%
\definecolor{currentstroke}{rgb}{0.000000,0.000000,0.000000}%
\pgfsetstrokecolor{currentstroke}%
\pgfsetdash{}{0pt}%
\pgfsys@defobject{currentmarker}{\pgfqpoint{0.000000in}{0.000000in}}{\pgfqpoint{0.027778in}{0.000000in}}{%
\pgfpathmoveto{\pgfqpoint{0.000000in}{0.000000in}}%
\pgfpathlineto{\pgfqpoint{0.027778in}{0.000000in}}%
\pgfusepath{stroke,fill}%
}%
\begin{pgfscope}%
\pgfsys@transformshift{0.607800in}{0.908453in}%
\pgfsys@useobject{currentmarker}{}%
\end{pgfscope}%
\end{pgfscope}%
\begin{pgfscope}%
\pgfsetbuttcap%
\pgfsetroundjoin%
\definecolor{currentfill}{rgb}{0.000000,0.000000,0.000000}%
\pgfsetfillcolor{currentfill}%
\pgfsetlinewidth{0.602250pt}%
\definecolor{currentstroke}{rgb}{0.000000,0.000000,0.000000}%
\pgfsetstrokecolor{currentstroke}%
\pgfsetdash{}{0pt}%
\pgfsys@defobject{currentmarker}{\pgfqpoint{-0.027778in}{0.000000in}}{\pgfqpoint{-0.000000in}{0.000000in}}{%
\pgfpathmoveto{\pgfqpoint{-0.000000in}{0.000000in}}%
\pgfpathlineto{\pgfqpoint{-0.027778in}{0.000000in}}%
\pgfusepath{stroke,fill}%
}%
\begin{pgfscope}%
\pgfsys@transformshift{4.597025in}{0.908453in}%
\pgfsys@useobject{currentmarker}{}%
\end{pgfscope}%
\end{pgfscope}%
\begin{pgfscope}%
\pgfsetbuttcap%
\pgfsetroundjoin%
\definecolor{currentfill}{rgb}{0.000000,0.000000,0.000000}%
\pgfsetfillcolor{currentfill}%
\pgfsetlinewidth{0.602250pt}%
\definecolor{currentstroke}{rgb}{0.000000,0.000000,0.000000}%
\pgfsetstrokecolor{currentstroke}%
\pgfsetdash{}{0pt}%
\pgfsys@defobject{currentmarker}{\pgfqpoint{0.000000in}{0.000000in}}{\pgfqpoint{0.027778in}{0.000000in}}{%
\pgfpathmoveto{\pgfqpoint{0.000000in}{0.000000in}}%
\pgfpathlineto{\pgfqpoint{0.027778in}{0.000000in}}%
\pgfusepath{stroke,fill}%
}%
\begin{pgfscope}%
\pgfsys@transformshift{0.607800in}{1.027899in}%
\pgfsys@useobject{currentmarker}{}%
\end{pgfscope}%
\end{pgfscope}%
\begin{pgfscope}%
\pgfsetbuttcap%
\pgfsetroundjoin%
\definecolor{currentfill}{rgb}{0.000000,0.000000,0.000000}%
\pgfsetfillcolor{currentfill}%
\pgfsetlinewidth{0.602250pt}%
\definecolor{currentstroke}{rgb}{0.000000,0.000000,0.000000}%
\pgfsetstrokecolor{currentstroke}%
\pgfsetdash{}{0pt}%
\pgfsys@defobject{currentmarker}{\pgfqpoint{-0.027778in}{0.000000in}}{\pgfqpoint{-0.000000in}{0.000000in}}{%
\pgfpathmoveto{\pgfqpoint{-0.000000in}{0.000000in}}%
\pgfpathlineto{\pgfqpoint{-0.027778in}{0.000000in}}%
\pgfusepath{stroke,fill}%
}%
\begin{pgfscope}%
\pgfsys@transformshift{4.597025in}{1.027899in}%
\pgfsys@useobject{currentmarker}{}%
\end{pgfscope}%
\end{pgfscope}%
\begin{pgfscope}%
\pgfsetbuttcap%
\pgfsetroundjoin%
\definecolor{currentfill}{rgb}{0.000000,0.000000,0.000000}%
\pgfsetfillcolor{currentfill}%
\pgfsetlinewidth{0.602250pt}%
\definecolor{currentstroke}{rgb}{0.000000,0.000000,0.000000}%
\pgfsetstrokecolor{currentstroke}%
\pgfsetdash{}{0pt}%
\pgfsys@defobject{currentmarker}{\pgfqpoint{0.000000in}{0.000000in}}{\pgfqpoint{0.027778in}{0.000000in}}{%
\pgfpathmoveto{\pgfqpoint{0.000000in}{0.000000in}}%
\pgfpathlineto{\pgfqpoint{0.027778in}{0.000000in}}%
\pgfusepath{stroke,fill}%
}%
\begin{pgfscope}%
\pgfsys@transformshift{0.607800in}{1.087622in}%
\pgfsys@useobject{currentmarker}{}%
\end{pgfscope}%
\end{pgfscope}%
\begin{pgfscope}%
\pgfsetbuttcap%
\pgfsetroundjoin%
\definecolor{currentfill}{rgb}{0.000000,0.000000,0.000000}%
\pgfsetfillcolor{currentfill}%
\pgfsetlinewidth{0.602250pt}%
\definecolor{currentstroke}{rgb}{0.000000,0.000000,0.000000}%
\pgfsetstrokecolor{currentstroke}%
\pgfsetdash{}{0pt}%
\pgfsys@defobject{currentmarker}{\pgfqpoint{-0.027778in}{0.000000in}}{\pgfqpoint{-0.000000in}{0.000000in}}{%
\pgfpathmoveto{\pgfqpoint{-0.000000in}{0.000000in}}%
\pgfpathlineto{\pgfqpoint{-0.027778in}{0.000000in}}%
\pgfusepath{stroke,fill}%
}%
\begin{pgfscope}%
\pgfsys@transformshift{4.597025in}{1.087622in}%
\pgfsys@useobject{currentmarker}{}%
\end{pgfscope}%
\end{pgfscope}%
\begin{pgfscope}%
\pgfsetbuttcap%
\pgfsetroundjoin%
\definecolor{currentfill}{rgb}{0.000000,0.000000,0.000000}%
\pgfsetfillcolor{currentfill}%
\pgfsetlinewidth{0.602250pt}%
\definecolor{currentstroke}{rgb}{0.000000,0.000000,0.000000}%
\pgfsetstrokecolor{currentstroke}%
\pgfsetdash{}{0pt}%
\pgfsys@defobject{currentmarker}{\pgfqpoint{0.000000in}{0.000000in}}{\pgfqpoint{0.027778in}{0.000000in}}{%
\pgfpathmoveto{\pgfqpoint{0.000000in}{0.000000in}}%
\pgfpathlineto{\pgfqpoint{0.027778in}{0.000000in}}%
\pgfusepath{stroke,fill}%
}%
\begin{pgfscope}%
\pgfsys@transformshift{0.607800in}{1.147345in}%
\pgfsys@useobject{currentmarker}{}%
\end{pgfscope}%
\end{pgfscope}%
\begin{pgfscope}%
\pgfsetbuttcap%
\pgfsetroundjoin%
\definecolor{currentfill}{rgb}{0.000000,0.000000,0.000000}%
\pgfsetfillcolor{currentfill}%
\pgfsetlinewidth{0.602250pt}%
\definecolor{currentstroke}{rgb}{0.000000,0.000000,0.000000}%
\pgfsetstrokecolor{currentstroke}%
\pgfsetdash{}{0pt}%
\pgfsys@defobject{currentmarker}{\pgfqpoint{-0.027778in}{0.000000in}}{\pgfqpoint{-0.000000in}{0.000000in}}{%
\pgfpathmoveto{\pgfqpoint{-0.000000in}{0.000000in}}%
\pgfpathlineto{\pgfqpoint{-0.027778in}{0.000000in}}%
\pgfusepath{stroke,fill}%
}%
\begin{pgfscope}%
\pgfsys@transformshift{4.597025in}{1.147345in}%
\pgfsys@useobject{currentmarker}{}%
\end{pgfscope}%
\end{pgfscope}%
\begin{pgfscope}%
\pgfsetbuttcap%
\pgfsetroundjoin%
\definecolor{currentfill}{rgb}{0.000000,0.000000,0.000000}%
\pgfsetfillcolor{currentfill}%
\pgfsetlinewidth{0.602250pt}%
\definecolor{currentstroke}{rgb}{0.000000,0.000000,0.000000}%
\pgfsetstrokecolor{currentstroke}%
\pgfsetdash{}{0pt}%
\pgfsys@defobject{currentmarker}{\pgfqpoint{0.000000in}{0.000000in}}{\pgfqpoint{0.027778in}{0.000000in}}{%
\pgfpathmoveto{\pgfqpoint{0.000000in}{0.000000in}}%
\pgfpathlineto{\pgfqpoint{0.027778in}{0.000000in}}%
\pgfusepath{stroke,fill}%
}%
\begin{pgfscope}%
\pgfsys@transformshift{0.607800in}{1.207068in}%
\pgfsys@useobject{currentmarker}{}%
\end{pgfscope}%
\end{pgfscope}%
\begin{pgfscope}%
\pgfsetbuttcap%
\pgfsetroundjoin%
\definecolor{currentfill}{rgb}{0.000000,0.000000,0.000000}%
\pgfsetfillcolor{currentfill}%
\pgfsetlinewidth{0.602250pt}%
\definecolor{currentstroke}{rgb}{0.000000,0.000000,0.000000}%
\pgfsetstrokecolor{currentstroke}%
\pgfsetdash{}{0pt}%
\pgfsys@defobject{currentmarker}{\pgfqpoint{-0.027778in}{0.000000in}}{\pgfqpoint{-0.000000in}{0.000000in}}{%
\pgfpathmoveto{\pgfqpoint{-0.000000in}{0.000000in}}%
\pgfpathlineto{\pgfqpoint{-0.027778in}{0.000000in}}%
\pgfusepath{stroke,fill}%
}%
\begin{pgfscope}%
\pgfsys@transformshift{4.597025in}{1.207068in}%
\pgfsys@useobject{currentmarker}{}%
\end{pgfscope}%
\end{pgfscope}%
\begin{pgfscope}%
\pgfsetbuttcap%
\pgfsetroundjoin%
\definecolor{currentfill}{rgb}{0.000000,0.000000,0.000000}%
\pgfsetfillcolor{currentfill}%
\pgfsetlinewidth{0.602250pt}%
\definecolor{currentstroke}{rgb}{0.000000,0.000000,0.000000}%
\pgfsetstrokecolor{currentstroke}%
\pgfsetdash{}{0pt}%
\pgfsys@defobject{currentmarker}{\pgfqpoint{0.000000in}{0.000000in}}{\pgfqpoint{0.027778in}{0.000000in}}{%
\pgfpathmoveto{\pgfqpoint{0.000000in}{0.000000in}}%
\pgfpathlineto{\pgfqpoint{0.027778in}{0.000000in}}%
\pgfusepath{stroke,fill}%
}%
\begin{pgfscope}%
\pgfsys@transformshift{0.607800in}{1.326515in}%
\pgfsys@useobject{currentmarker}{}%
\end{pgfscope}%
\end{pgfscope}%
\begin{pgfscope}%
\pgfsetbuttcap%
\pgfsetroundjoin%
\definecolor{currentfill}{rgb}{0.000000,0.000000,0.000000}%
\pgfsetfillcolor{currentfill}%
\pgfsetlinewidth{0.602250pt}%
\definecolor{currentstroke}{rgb}{0.000000,0.000000,0.000000}%
\pgfsetstrokecolor{currentstroke}%
\pgfsetdash{}{0pt}%
\pgfsys@defobject{currentmarker}{\pgfqpoint{-0.027778in}{0.000000in}}{\pgfqpoint{-0.000000in}{0.000000in}}{%
\pgfpathmoveto{\pgfqpoint{-0.000000in}{0.000000in}}%
\pgfpathlineto{\pgfqpoint{-0.027778in}{0.000000in}}%
\pgfusepath{stroke,fill}%
}%
\begin{pgfscope}%
\pgfsys@transformshift{4.597025in}{1.326515in}%
\pgfsys@useobject{currentmarker}{}%
\end{pgfscope}%
\end{pgfscope}%
\begin{pgfscope}%
\pgfsetbuttcap%
\pgfsetroundjoin%
\definecolor{currentfill}{rgb}{0.000000,0.000000,0.000000}%
\pgfsetfillcolor{currentfill}%
\pgfsetlinewidth{0.602250pt}%
\definecolor{currentstroke}{rgb}{0.000000,0.000000,0.000000}%
\pgfsetstrokecolor{currentstroke}%
\pgfsetdash{}{0pt}%
\pgfsys@defobject{currentmarker}{\pgfqpoint{0.000000in}{0.000000in}}{\pgfqpoint{0.027778in}{0.000000in}}{%
\pgfpathmoveto{\pgfqpoint{0.000000in}{0.000000in}}%
\pgfpathlineto{\pgfqpoint{0.027778in}{0.000000in}}%
\pgfusepath{stroke,fill}%
}%
\begin{pgfscope}%
\pgfsys@transformshift{0.607800in}{1.386238in}%
\pgfsys@useobject{currentmarker}{}%
\end{pgfscope}%
\end{pgfscope}%
\begin{pgfscope}%
\pgfsetbuttcap%
\pgfsetroundjoin%
\definecolor{currentfill}{rgb}{0.000000,0.000000,0.000000}%
\pgfsetfillcolor{currentfill}%
\pgfsetlinewidth{0.602250pt}%
\definecolor{currentstroke}{rgb}{0.000000,0.000000,0.000000}%
\pgfsetstrokecolor{currentstroke}%
\pgfsetdash{}{0pt}%
\pgfsys@defobject{currentmarker}{\pgfqpoint{-0.027778in}{0.000000in}}{\pgfqpoint{-0.000000in}{0.000000in}}{%
\pgfpathmoveto{\pgfqpoint{-0.000000in}{0.000000in}}%
\pgfpathlineto{\pgfqpoint{-0.027778in}{0.000000in}}%
\pgfusepath{stroke,fill}%
}%
\begin{pgfscope}%
\pgfsys@transformshift{4.597025in}{1.386238in}%
\pgfsys@useobject{currentmarker}{}%
\end{pgfscope}%
\end{pgfscope}%
\begin{pgfscope}%
\pgfsetbuttcap%
\pgfsetroundjoin%
\definecolor{currentfill}{rgb}{0.000000,0.000000,0.000000}%
\pgfsetfillcolor{currentfill}%
\pgfsetlinewidth{0.602250pt}%
\definecolor{currentstroke}{rgb}{0.000000,0.000000,0.000000}%
\pgfsetstrokecolor{currentstroke}%
\pgfsetdash{}{0pt}%
\pgfsys@defobject{currentmarker}{\pgfqpoint{0.000000in}{0.000000in}}{\pgfqpoint{0.027778in}{0.000000in}}{%
\pgfpathmoveto{\pgfqpoint{0.000000in}{0.000000in}}%
\pgfpathlineto{\pgfqpoint{0.027778in}{0.000000in}}%
\pgfusepath{stroke,fill}%
}%
\begin{pgfscope}%
\pgfsys@transformshift{0.607800in}{1.445961in}%
\pgfsys@useobject{currentmarker}{}%
\end{pgfscope}%
\end{pgfscope}%
\begin{pgfscope}%
\pgfsetbuttcap%
\pgfsetroundjoin%
\definecolor{currentfill}{rgb}{0.000000,0.000000,0.000000}%
\pgfsetfillcolor{currentfill}%
\pgfsetlinewidth{0.602250pt}%
\definecolor{currentstroke}{rgb}{0.000000,0.000000,0.000000}%
\pgfsetstrokecolor{currentstroke}%
\pgfsetdash{}{0pt}%
\pgfsys@defobject{currentmarker}{\pgfqpoint{-0.027778in}{0.000000in}}{\pgfqpoint{-0.000000in}{0.000000in}}{%
\pgfpathmoveto{\pgfqpoint{-0.000000in}{0.000000in}}%
\pgfpathlineto{\pgfqpoint{-0.027778in}{0.000000in}}%
\pgfusepath{stroke,fill}%
}%
\begin{pgfscope}%
\pgfsys@transformshift{4.597025in}{1.445961in}%
\pgfsys@useobject{currentmarker}{}%
\end{pgfscope}%
\end{pgfscope}%
\begin{pgfscope}%
\pgfsetbuttcap%
\pgfsetroundjoin%
\definecolor{currentfill}{rgb}{0.000000,0.000000,0.000000}%
\pgfsetfillcolor{currentfill}%
\pgfsetlinewidth{0.602250pt}%
\definecolor{currentstroke}{rgb}{0.000000,0.000000,0.000000}%
\pgfsetstrokecolor{currentstroke}%
\pgfsetdash{}{0pt}%
\pgfsys@defobject{currentmarker}{\pgfqpoint{0.000000in}{0.000000in}}{\pgfqpoint{0.027778in}{0.000000in}}{%
\pgfpathmoveto{\pgfqpoint{0.000000in}{0.000000in}}%
\pgfpathlineto{\pgfqpoint{0.027778in}{0.000000in}}%
\pgfusepath{stroke,fill}%
}%
\begin{pgfscope}%
\pgfsys@transformshift{0.607800in}{1.505684in}%
\pgfsys@useobject{currentmarker}{}%
\end{pgfscope}%
\end{pgfscope}%
\begin{pgfscope}%
\pgfsetbuttcap%
\pgfsetroundjoin%
\definecolor{currentfill}{rgb}{0.000000,0.000000,0.000000}%
\pgfsetfillcolor{currentfill}%
\pgfsetlinewidth{0.602250pt}%
\definecolor{currentstroke}{rgb}{0.000000,0.000000,0.000000}%
\pgfsetstrokecolor{currentstroke}%
\pgfsetdash{}{0pt}%
\pgfsys@defobject{currentmarker}{\pgfqpoint{-0.027778in}{0.000000in}}{\pgfqpoint{-0.000000in}{0.000000in}}{%
\pgfpathmoveto{\pgfqpoint{-0.000000in}{0.000000in}}%
\pgfpathlineto{\pgfqpoint{-0.027778in}{0.000000in}}%
\pgfusepath{stroke,fill}%
}%
\begin{pgfscope}%
\pgfsys@transformshift{4.597025in}{1.505684in}%
\pgfsys@useobject{currentmarker}{}%
\end{pgfscope}%
\end{pgfscope}%
\begin{pgfscope}%
\pgfsetbuttcap%
\pgfsetroundjoin%
\definecolor{currentfill}{rgb}{0.000000,0.000000,0.000000}%
\pgfsetfillcolor{currentfill}%
\pgfsetlinewidth{0.602250pt}%
\definecolor{currentstroke}{rgb}{0.000000,0.000000,0.000000}%
\pgfsetstrokecolor{currentstroke}%
\pgfsetdash{}{0pt}%
\pgfsys@defobject{currentmarker}{\pgfqpoint{0.000000in}{0.000000in}}{\pgfqpoint{0.027778in}{0.000000in}}{%
\pgfpathmoveto{\pgfqpoint{0.000000in}{0.000000in}}%
\pgfpathlineto{\pgfqpoint{0.027778in}{0.000000in}}%
\pgfusepath{stroke,fill}%
}%
\begin{pgfscope}%
\pgfsys@transformshift{0.607800in}{1.625130in}%
\pgfsys@useobject{currentmarker}{}%
\end{pgfscope}%
\end{pgfscope}%
\begin{pgfscope}%
\pgfsetbuttcap%
\pgfsetroundjoin%
\definecolor{currentfill}{rgb}{0.000000,0.000000,0.000000}%
\pgfsetfillcolor{currentfill}%
\pgfsetlinewidth{0.602250pt}%
\definecolor{currentstroke}{rgb}{0.000000,0.000000,0.000000}%
\pgfsetstrokecolor{currentstroke}%
\pgfsetdash{}{0pt}%
\pgfsys@defobject{currentmarker}{\pgfqpoint{-0.027778in}{0.000000in}}{\pgfqpoint{-0.000000in}{0.000000in}}{%
\pgfpathmoveto{\pgfqpoint{-0.000000in}{0.000000in}}%
\pgfpathlineto{\pgfqpoint{-0.027778in}{0.000000in}}%
\pgfusepath{stroke,fill}%
}%
\begin{pgfscope}%
\pgfsys@transformshift{4.597025in}{1.625130in}%
\pgfsys@useobject{currentmarker}{}%
\end{pgfscope}%
\end{pgfscope}%
\begin{pgfscope}%
\pgfsetbuttcap%
\pgfsetroundjoin%
\definecolor{currentfill}{rgb}{0.000000,0.000000,0.000000}%
\pgfsetfillcolor{currentfill}%
\pgfsetlinewidth{0.602250pt}%
\definecolor{currentstroke}{rgb}{0.000000,0.000000,0.000000}%
\pgfsetstrokecolor{currentstroke}%
\pgfsetdash{}{0pt}%
\pgfsys@defobject{currentmarker}{\pgfqpoint{0.000000in}{0.000000in}}{\pgfqpoint{0.027778in}{0.000000in}}{%
\pgfpathmoveto{\pgfqpoint{0.000000in}{0.000000in}}%
\pgfpathlineto{\pgfqpoint{0.027778in}{0.000000in}}%
\pgfusepath{stroke,fill}%
}%
\begin{pgfscope}%
\pgfsys@transformshift{0.607800in}{1.684853in}%
\pgfsys@useobject{currentmarker}{}%
\end{pgfscope}%
\end{pgfscope}%
\begin{pgfscope}%
\pgfsetbuttcap%
\pgfsetroundjoin%
\definecolor{currentfill}{rgb}{0.000000,0.000000,0.000000}%
\pgfsetfillcolor{currentfill}%
\pgfsetlinewidth{0.602250pt}%
\definecolor{currentstroke}{rgb}{0.000000,0.000000,0.000000}%
\pgfsetstrokecolor{currentstroke}%
\pgfsetdash{}{0pt}%
\pgfsys@defobject{currentmarker}{\pgfqpoint{-0.027778in}{0.000000in}}{\pgfqpoint{-0.000000in}{0.000000in}}{%
\pgfpathmoveto{\pgfqpoint{-0.000000in}{0.000000in}}%
\pgfpathlineto{\pgfqpoint{-0.027778in}{0.000000in}}%
\pgfusepath{stroke,fill}%
}%
\begin{pgfscope}%
\pgfsys@transformshift{4.597025in}{1.684853in}%
\pgfsys@useobject{currentmarker}{}%
\end{pgfscope}%
\end{pgfscope}%
\begin{pgfscope}%
\pgfsetbuttcap%
\pgfsetroundjoin%
\definecolor{currentfill}{rgb}{0.000000,0.000000,0.000000}%
\pgfsetfillcolor{currentfill}%
\pgfsetlinewidth{0.602250pt}%
\definecolor{currentstroke}{rgb}{0.000000,0.000000,0.000000}%
\pgfsetstrokecolor{currentstroke}%
\pgfsetdash{}{0pt}%
\pgfsys@defobject{currentmarker}{\pgfqpoint{0.000000in}{0.000000in}}{\pgfqpoint{0.027778in}{0.000000in}}{%
\pgfpathmoveto{\pgfqpoint{0.000000in}{0.000000in}}%
\pgfpathlineto{\pgfqpoint{0.027778in}{0.000000in}}%
\pgfusepath{stroke,fill}%
}%
\begin{pgfscope}%
\pgfsys@transformshift{0.607800in}{1.744577in}%
\pgfsys@useobject{currentmarker}{}%
\end{pgfscope}%
\end{pgfscope}%
\begin{pgfscope}%
\pgfsetbuttcap%
\pgfsetroundjoin%
\definecolor{currentfill}{rgb}{0.000000,0.000000,0.000000}%
\pgfsetfillcolor{currentfill}%
\pgfsetlinewidth{0.602250pt}%
\definecolor{currentstroke}{rgb}{0.000000,0.000000,0.000000}%
\pgfsetstrokecolor{currentstroke}%
\pgfsetdash{}{0pt}%
\pgfsys@defobject{currentmarker}{\pgfqpoint{-0.027778in}{0.000000in}}{\pgfqpoint{-0.000000in}{0.000000in}}{%
\pgfpathmoveto{\pgfqpoint{-0.000000in}{0.000000in}}%
\pgfpathlineto{\pgfqpoint{-0.027778in}{0.000000in}}%
\pgfusepath{stroke,fill}%
}%
\begin{pgfscope}%
\pgfsys@transformshift{4.597025in}{1.744577in}%
\pgfsys@useobject{currentmarker}{}%
\end{pgfscope}%
\end{pgfscope}%
\begin{pgfscope}%
\pgfsetbuttcap%
\pgfsetroundjoin%
\definecolor{currentfill}{rgb}{0.000000,0.000000,0.000000}%
\pgfsetfillcolor{currentfill}%
\pgfsetlinewidth{0.602250pt}%
\definecolor{currentstroke}{rgb}{0.000000,0.000000,0.000000}%
\pgfsetstrokecolor{currentstroke}%
\pgfsetdash{}{0pt}%
\pgfsys@defobject{currentmarker}{\pgfqpoint{0.000000in}{0.000000in}}{\pgfqpoint{0.027778in}{0.000000in}}{%
\pgfpathmoveto{\pgfqpoint{0.000000in}{0.000000in}}%
\pgfpathlineto{\pgfqpoint{0.027778in}{0.000000in}}%
\pgfusepath{stroke,fill}%
}%
\begin{pgfscope}%
\pgfsys@transformshift{0.607800in}{1.804300in}%
\pgfsys@useobject{currentmarker}{}%
\end{pgfscope}%
\end{pgfscope}%
\begin{pgfscope}%
\pgfsetbuttcap%
\pgfsetroundjoin%
\definecolor{currentfill}{rgb}{0.000000,0.000000,0.000000}%
\pgfsetfillcolor{currentfill}%
\pgfsetlinewidth{0.602250pt}%
\definecolor{currentstroke}{rgb}{0.000000,0.000000,0.000000}%
\pgfsetstrokecolor{currentstroke}%
\pgfsetdash{}{0pt}%
\pgfsys@defobject{currentmarker}{\pgfqpoint{-0.027778in}{0.000000in}}{\pgfqpoint{-0.000000in}{0.000000in}}{%
\pgfpathmoveto{\pgfqpoint{-0.000000in}{0.000000in}}%
\pgfpathlineto{\pgfqpoint{-0.027778in}{0.000000in}}%
\pgfusepath{stroke,fill}%
}%
\begin{pgfscope}%
\pgfsys@transformshift{4.597025in}{1.804300in}%
\pgfsys@useobject{currentmarker}{}%
\end{pgfscope}%
\end{pgfscope}%
\begin{pgfscope}%
\pgfsetbuttcap%
\pgfsetroundjoin%
\definecolor{currentfill}{rgb}{0.000000,0.000000,0.000000}%
\pgfsetfillcolor{currentfill}%
\pgfsetlinewidth{0.602250pt}%
\definecolor{currentstroke}{rgb}{0.000000,0.000000,0.000000}%
\pgfsetstrokecolor{currentstroke}%
\pgfsetdash{}{0pt}%
\pgfsys@defobject{currentmarker}{\pgfqpoint{0.000000in}{0.000000in}}{\pgfqpoint{0.027778in}{0.000000in}}{%
\pgfpathmoveto{\pgfqpoint{0.000000in}{0.000000in}}%
\pgfpathlineto{\pgfqpoint{0.027778in}{0.000000in}}%
\pgfusepath{stroke,fill}%
}%
\begin{pgfscope}%
\pgfsys@transformshift{0.607800in}{1.923746in}%
\pgfsys@useobject{currentmarker}{}%
\end{pgfscope}%
\end{pgfscope}%
\begin{pgfscope}%
\pgfsetbuttcap%
\pgfsetroundjoin%
\definecolor{currentfill}{rgb}{0.000000,0.000000,0.000000}%
\pgfsetfillcolor{currentfill}%
\pgfsetlinewidth{0.602250pt}%
\definecolor{currentstroke}{rgb}{0.000000,0.000000,0.000000}%
\pgfsetstrokecolor{currentstroke}%
\pgfsetdash{}{0pt}%
\pgfsys@defobject{currentmarker}{\pgfqpoint{-0.027778in}{0.000000in}}{\pgfqpoint{-0.000000in}{0.000000in}}{%
\pgfpathmoveto{\pgfqpoint{-0.000000in}{0.000000in}}%
\pgfpathlineto{\pgfqpoint{-0.027778in}{0.000000in}}%
\pgfusepath{stroke,fill}%
}%
\begin{pgfscope}%
\pgfsys@transformshift{4.597025in}{1.923746in}%
\pgfsys@useobject{currentmarker}{}%
\end{pgfscope}%
\end{pgfscope}%
\begin{pgfscope}%
\pgfsetbuttcap%
\pgfsetroundjoin%
\definecolor{currentfill}{rgb}{0.000000,0.000000,0.000000}%
\pgfsetfillcolor{currentfill}%
\pgfsetlinewidth{0.602250pt}%
\definecolor{currentstroke}{rgb}{0.000000,0.000000,0.000000}%
\pgfsetstrokecolor{currentstroke}%
\pgfsetdash{}{0pt}%
\pgfsys@defobject{currentmarker}{\pgfqpoint{0.000000in}{0.000000in}}{\pgfqpoint{0.027778in}{0.000000in}}{%
\pgfpathmoveto{\pgfqpoint{0.000000in}{0.000000in}}%
\pgfpathlineto{\pgfqpoint{0.027778in}{0.000000in}}%
\pgfusepath{stroke,fill}%
}%
\begin{pgfscope}%
\pgfsys@transformshift{0.607800in}{1.983469in}%
\pgfsys@useobject{currentmarker}{}%
\end{pgfscope}%
\end{pgfscope}%
\begin{pgfscope}%
\pgfsetbuttcap%
\pgfsetroundjoin%
\definecolor{currentfill}{rgb}{0.000000,0.000000,0.000000}%
\pgfsetfillcolor{currentfill}%
\pgfsetlinewidth{0.602250pt}%
\definecolor{currentstroke}{rgb}{0.000000,0.000000,0.000000}%
\pgfsetstrokecolor{currentstroke}%
\pgfsetdash{}{0pt}%
\pgfsys@defobject{currentmarker}{\pgfqpoint{-0.027778in}{0.000000in}}{\pgfqpoint{-0.000000in}{0.000000in}}{%
\pgfpathmoveto{\pgfqpoint{-0.000000in}{0.000000in}}%
\pgfpathlineto{\pgfqpoint{-0.027778in}{0.000000in}}%
\pgfusepath{stroke,fill}%
}%
\begin{pgfscope}%
\pgfsys@transformshift{4.597025in}{1.983469in}%
\pgfsys@useobject{currentmarker}{}%
\end{pgfscope}%
\end{pgfscope}%
\begin{pgfscope}%
\pgfsetbuttcap%
\pgfsetroundjoin%
\definecolor{currentfill}{rgb}{0.000000,0.000000,0.000000}%
\pgfsetfillcolor{currentfill}%
\pgfsetlinewidth{0.602250pt}%
\definecolor{currentstroke}{rgb}{0.000000,0.000000,0.000000}%
\pgfsetstrokecolor{currentstroke}%
\pgfsetdash{}{0pt}%
\pgfsys@defobject{currentmarker}{\pgfqpoint{0.000000in}{0.000000in}}{\pgfqpoint{0.027778in}{0.000000in}}{%
\pgfpathmoveto{\pgfqpoint{0.000000in}{0.000000in}}%
\pgfpathlineto{\pgfqpoint{0.027778in}{0.000000in}}%
\pgfusepath{stroke,fill}%
}%
\begin{pgfscope}%
\pgfsys@transformshift{0.607800in}{2.043192in}%
\pgfsys@useobject{currentmarker}{}%
\end{pgfscope}%
\end{pgfscope}%
\begin{pgfscope}%
\pgfsetbuttcap%
\pgfsetroundjoin%
\definecolor{currentfill}{rgb}{0.000000,0.000000,0.000000}%
\pgfsetfillcolor{currentfill}%
\pgfsetlinewidth{0.602250pt}%
\definecolor{currentstroke}{rgb}{0.000000,0.000000,0.000000}%
\pgfsetstrokecolor{currentstroke}%
\pgfsetdash{}{0pt}%
\pgfsys@defobject{currentmarker}{\pgfqpoint{-0.027778in}{0.000000in}}{\pgfqpoint{-0.000000in}{0.000000in}}{%
\pgfpathmoveto{\pgfqpoint{-0.000000in}{0.000000in}}%
\pgfpathlineto{\pgfqpoint{-0.027778in}{0.000000in}}%
\pgfusepath{stroke,fill}%
}%
\begin{pgfscope}%
\pgfsys@transformshift{4.597025in}{2.043192in}%
\pgfsys@useobject{currentmarker}{}%
\end{pgfscope}%
\end{pgfscope}%
\begin{pgfscope}%
\pgfsetbuttcap%
\pgfsetroundjoin%
\definecolor{currentfill}{rgb}{0.000000,0.000000,0.000000}%
\pgfsetfillcolor{currentfill}%
\pgfsetlinewidth{0.602250pt}%
\definecolor{currentstroke}{rgb}{0.000000,0.000000,0.000000}%
\pgfsetstrokecolor{currentstroke}%
\pgfsetdash{}{0pt}%
\pgfsys@defobject{currentmarker}{\pgfqpoint{0.000000in}{0.000000in}}{\pgfqpoint{0.027778in}{0.000000in}}{%
\pgfpathmoveto{\pgfqpoint{0.000000in}{0.000000in}}%
\pgfpathlineto{\pgfqpoint{0.027778in}{0.000000in}}%
\pgfusepath{stroke,fill}%
}%
\begin{pgfscope}%
\pgfsys@transformshift{0.607800in}{2.102915in}%
\pgfsys@useobject{currentmarker}{}%
\end{pgfscope}%
\end{pgfscope}%
\begin{pgfscope}%
\pgfsetbuttcap%
\pgfsetroundjoin%
\definecolor{currentfill}{rgb}{0.000000,0.000000,0.000000}%
\pgfsetfillcolor{currentfill}%
\pgfsetlinewidth{0.602250pt}%
\definecolor{currentstroke}{rgb}{0.000000,0.000000,0.000000}%
\pgfsetstrokecolor{currentstroke}%
\pgfsetdash{}{0pt}%
\pgfsys@defobject{currentmarker}{\pgfqpoint{-0.027778in}{0.000000in}}{\pgfqpoint{-0.000000in}{0.000000in}}{%
\pgfpathmoveto{\pgfqpoint{-0.000000in}{0.000000in}}%
\pgfpathlineto{\pgfqpoint{-0.027778in}{0.000000in}}%
\pgfusepath{stroke,fill}%
}%
\begin{pgfscope}%
\pgfsys@transformshift{4.597025in}{2.102915in}%
\pgfsys@useobject{currentmarker}{}%
\end{pgfscope}%
\end{pgfscope}%
\begin{pgfscope}%
\pgfsetbuttcap%
\pgfsetroundjoin%
\definecolor{currentfill}{rgb}{0.000000,0.000000,0.000000}%
\pgfsetfillcolor{currentfill}%
\pgfsetlinewidth{0.602250pt}%
\definecolor{currentstroke}{rgb}{0.000000,0.000000,0.000000}%
\pgfsetstrokecolor{currentstroke}%
\pgfsetdash{}{0pt}%
\pgfsys@defobject{currentmarker}{\pgfqpoint{0.000000in}{0.000000in}}{\pgfqpoint{0.027778in}{0.000000in}}{%
\pgfpathmoveto{\pgfqpoint{0.000000in}{0.000000in}}%
\pgfpathlineto{\pgfqpoint{0.027778in}{0.000000in}}%
\pgfusepath{stroke,fill}%
}%
\begin{pgfscope}%
\pgfsys@transformshift{0.607800in}{2.222362in}%
\pgfsys@useobject{currentmarker}{}%
\end{pgfscope}%
\end{pgfscope}%
\begin{pgfscope}%
\pgfsetbuttcap%
\pgfsetroundjoin%
\definecolor{currentfill}{rgb}{0.000000,0.000000,0.000000}%
\pgfsetfillcolor{currentfill}%
\pgfsetlinewidth{0.602250pt}%
\definecolor{currentstroke}{rgb}{0.000000,0.000000,0.000000}%
\pgfsetstrokecolor{currentstroke}%
\pgfsetdash{}{0pt}%
\pgfsys@defobject{currentmarker}{\pgfqpoint{-0.027778in}{0.000000in}}{\pgfqpoint{-0.000000in}{0.000000in}}{%
\pgfpathmoveto{\pgfqpoint{-0.000000in}{0.000000in}}%
\pgfpathlineto{\pgfqpoint{-0.027778in}{0.000000in}}%
\pgfusepath{stroke,fill}%
}%
\begin{pgfscope}%
\pgfsys@transformshift{4.597025in}{2.222362in}%
\pgfsys@useobject{currentmarker}{}%
\end{pgfscope}%
\end{pgfscope}%
\begin{pgfscope}%
\pgfsetbuttcap%
\pgfsetroundjoin%
\definecolor{currentfill}{rgb}{0.000000,0.000000,0.000000}%
\pgfsetfillcolor{currentfill}%
\pgfsetlinewidth{0.602250pt}%
\definecolor{currentstroke}{rgb}{0.000000,0.000000,0.000000}%
\pgfsetstrokecolor{currentstroke}%
\pgfsetdash{}{0pt}%
\pgfsys@defobject{currentmarker}{\pgfqpoint{0.000000in}{0.000000in}}{\pgfqpoint{0.027778in}{0.000000in}}{%
\pgfpathmoveto{\pgfqpoint{0.000000in}{0.000000in}}%
\pgfpathlineto{\pgfqpoint{0.027778in}{0.000000in}}%
\pgfusepath{stroke,fill}%
}%
\begin{pgfscope}%
\pgfsys@transformshift{0.607800in}{2.282085in}%
\pgfsys@useobject{currentmarker}{}%
\end{pgfscope}%
\end{pgfscope}%
\begin{pgfscope}%
\pgfsetbuttcap%
\pgfsetroundjoin%
\definecolor{currentfill}{rgb}{0.000000,0.000000,0.000000}%
\pgfsetfillcolor{currentfill}%
\pgfsetlinewidth{0.602250pt}%
\definecolor{currentstroke}{rgb}{0.000000,0.000000,0.000000}%
\pgfsetstrokecolor{currentstroke}%
\pgfsetdash{}{0pt}%
\pgfsys@defobject{currentmarker}{\pgfqpoint{-0.027778in}{0.000000in}}{\pgfqpoint{-0.000000in}{0.000000in}}{%
\pgfpathmoveto{\pgfqpoint{-0.000000in}{0.000000in}}%
\pgfpathlineto{\pgfqpoint{-0.027778in}{0.000000in}}%
\pgfusepath{stroke,fill}%
}%
\begin{pgfscope}%
\pgfsys@transformshift{4.597025in}{2.282085in}%
\pgfsys@useobject{currentmarker}{}%
\end{pgfscope}%
\end{pgfscope}%
\begin{pgfscope}%
\pgfsetbuttcap%
\pgfsetroundjoin%
\definecolor{currentfill}{rgb}{0.000000,0.000000,0.000000}%
\pgfsetfillcolor{currentfill}%
\pgfsetlinewidth{0.602250pt}%
\definecolor{currentstroke}{rgb}{0.000000,0.000000,0.000000}%
\pgfsetstrokecolor{currentstroke}%
\pgfsetdash{}{0pt}%
\pgfsys@defobject{currentmarker}{\pgfqpoint{0.000000in}{0.000000in}}{\pgfqpoint{0.027778in}{0.000000in}}{%
\pgfpathmoveto{\pgfqpoint{0.000000in}{0.000000in}}%
\pgfpathlineto{\pgfqpoint{0.027778in}{0.000000in}}%
\pgfusepath{stroke,fill}%
}%
\begin{pgfscope}%
\pgfsys@transformshift{0.607800in}{2.341808in}%
\pgfsys@useobject{currentmarker}{}%
\end{pgfscope}%
\end{pgfscope}%
\begin{pgfscope}%
\pgfsetbuttcap%
\pgfsetroundjoin%
\definecolor{currentfill}{rgb}{0.000000,0.000000,0.000000}%
\pgfsetfillcolor{currentfill}%
\pgfsetlinewidth{0.602250pt}%
\definecolor{currentstroke}{rgb}{0.000000,0.000000,0.000000}%
\pgfsetstrokecolor{currentstroke}%
\pgfsetdash{}{0pt}%
\pgfsys@defobject{currentmarker}{\pgfqpoint{-0.027778in}{0.000000in}}{\pgfqpoint{-0.000000in}{0.000000in}}{%
\pgfpathmoveto{\pgfqpoint{-0.000000in}{0.000000in}}%
\pgfpathlineto{\pgfqpoint{-0.027778in}{0.000000in}}%
\pgfusepath{stroke,fill}%
}%
\begin{pgfscope}%
\pgfsys@transformshift{4.597025in}{2.341808in}%
\pgfsys@useobject{currentmarker}{}%
\end{pgfscope}%
\end{pgfscope}%
\begin{pgfscope}%
\pgfsetbuttcap%
\pgfsetroundjoin%
\definecolor{currentfill}{rgb}{0.000000,0.000000,0.000000}%
\pgfsetfillcolor{currentfill}%
\pgfsetlinewidth{0.602250pt}%
\definecolor{currentstroke}{rgb}{0.000000,0.000000,0.000000}%
\pgfsetstrokecolor{currentstroke}%
\pgfsetdash{}{0pt}%
\pgfsys@defobject{currentmarker}{\pgfqpoint{0.000000in}{0.000000in}}{\pgfqpoint{0.027778in}{0.000000in}}{%
\pgfpathmoveto{\pgfqpoint{0.000000in}{0.000000in}}%
\pgfpathlineto{\pgfqpoint{0.027778in}{0.000000in}}%
\pgfusepath{stroke,fill}%
}%
\begin{pgfscope}%
\pgfsys@transformshift{0.607800in}{2.401531in}%
\pgfsys@useobject{currentmarker}{}%
\end{pgfscope}%
\end{pgfscope}%
\begin{pgfscope}%
\pgfsetbuttcap%
\pgfsetroundjoin%
\definecolor{currentfill}{rgb}{0.000000,0.000000,0.000000}%
\pgfsetfillcolor{currentfill}%
\pgfsetlinewidth{0.602250pt}%
\definecolor{currentstroke}{rgb}{0.000000,0.000000,0.000000}%
\pgfsetstrokecolor{currentstroke}%
\pgfsetdash{}{0pt}%
\pgfsys@defobject{currentmarker}{\pgfqpoint{-0.027778in}{0.000000in}}{\pgfqpoint{-0.000000in}{0.000000in}}{%
\pgfpathmoveto{\pgfqpoint{-0.000000in}{0.000000in}}%
\pgfpathlineto{\pgfqpoint{-0.027778in}{0.000000in}}%
\pgfusepath{stroke,fill}%
}%
\begin{pgfscope}%
\pgfsys@transformshift{4.597025in}{2.401531in}%
\pgfsys@useobject{currentmarker}{}%
\end{pgfscope}%
\end{pgfscope}%
\begin{pgfscope}%
\pgfsetbuttcap%
\pgfsetroundjoin%
\definecolor{currentfill}{rgb}{0.000000,0.000000,0.000000}%
\pgfsetfillcolor{currentfill}%
\pgfsetlinewidth{0.602250pt}%
\definecolor{currentstroke}{rgb}{0.000000,0.000000,0.000000}%
\pgfsetstrokecolor{currentstroke}%
\pgfsetdash{}{0pt}%
\pgfsys@defobject{currentmarker}{\pgfqpoint{0.000000in}{0.000000in}}{\pgfqpoint{0.027778in}{0.000000in}}{%
\pgfpathmoveto{\pgfqpoint{0.000000in}{0.000000in}}%
\pgfpathlineto{\pgfqpoint{0.027778in}{0.000000in}}%
\pgfusepath{stroke,fill}%
}%
\begin{pgfscope}%
\pgfsys@transformshift{0.607800in}{2.520977in}%
\pgfsys@useobject{currentmarker}{}%
\end{pgfscope}%
\end{pgfscope}%
\begin{pgfscope}%
\pgfsetbuttcap%
\pgfsetroundjoin%
\definecolor{currentfill}{rgb}{0.000000,0.000000,0.000000}%
\pgfsetfillcolor{currentfill}%
\pgfsetlinewidth{0.602250pt}%
\definecolor{currentstroke}{rgb}{0.000000,0.000000,0.000000}%
\pgfsetstrokecolor{currentstroke}%
\pgfsetdash{}{0pt}%
\pgfsys@defobject{currentmarker}{\pgfqpoint{-0.027778in}{0.000000in}}{\pgfqpoint{-0.000000in}{0.000000in}}{%
\pgfpathmoveto{\pgfqpoint{-0.000000in}{0.000000in}}%
\pgfpathlineto{\pgfqpoint{-0.027778in}{0.000000in}}%
\pgfusepath{stroke,fill}%
}%
\begin{pgfscope}%
\pgfsys@transformshift{4.597025in}{2.520977in}%
\pgfsys@useobject{currentmarker}{}%
\end{pgfscope}%
\end{pgfscope}%
\begin{pgfscope}%
\pgfsetbuttcap%
\pgfsetroundjoin%
\definecolor{currentfill}{rgb}{0.000000,0.000000,0.000000}%
\pgfsetfillcolor{currentfill}%
\pgfsetlinewidth{0.602250pt}%
\definecolor{currentstroke}{rgb}{0.000000,0.000000,0.000000}%
\pgfsetstrokecolor{currentstroke}%
\pgfsetdash{}{0pt}%
\pgfsys@defobject{currentmarker}{\pgfqpoint{0.000000in}{0.000000in}}{\pgfqpoint{0.027778in}{0.000000in}}{%
\pgfpathmoveto{\pgfqpoint{0.000000in}{0.000000in}}%
\pgfpathlineto{\pgfqpoint{0.027778in}{0.000000in}}%
\pgfusepath{stroke,fill}%
}%
\begin{pgfscope}%
\pgfsys@transformshift{0.607800in}{2.580700in}%
\pgfsys@useobject{currentmarker}{}%
\end{pgfscope}%
\end{pgfscope}%
\begin{pgfscope}%
\pgfsetbuttcap%
\pgfsetroundjoin%
\definecolor{currentfill}{rgb}{0.000000,0.000000,0.000000}%
\pgfsetfillcolor{currentfill}%
\pgfsetlinewidth{0.602250pt}%
\definecolor{currentstroke}{rgb}{0.000000,0.000000,0.000000}%
\pgfsetstrokecolor{currentstroke}%
\pgfsetdash{}{0pt}%
\pgfsys@defobject{currentmarker}{\pgfqpoint{-0.027778in}{0.000000in}}{\pgfqpoint{-0.000000in}{0.000000in}}{%
\pgfpathmoveto{\pgfqpoint{-0.000000in}{0.000000in}}%
\pgfpathlineto{\pgfqpoint{-0.027778in}{0.000000in}}%
\pgfusepath{stroke,fill}%
}%
\begin{pgfscope}%
\pgfsys@transformshift{4.597025in}{2.580700in}%
\pgfsys@useobject{currentmarker}{}%
\end{pgfscope}%
\end{pgfscope}%
\begin{pgfscope}%
\pgfsetbuttcap%
\pgfsetroundjoin%
\definecolor{currentfill}{rgb}{0.000000,0.000000,0.000000}%
\pgfsetfillcolor{currentfill}%
\pgfsetlinewidth{0.602250pt}%
\definecolor{currentstroke}{rgb}{0.000000,0.000000,0.000000}%
\pgfsetstrokecolor{currentstroke}%
\pgfsetdash{}{0pt}%
\pgfsys@defobject{currentmarker}{\pgfqpoint{0.000000in}{0.000000in}}{\pgfqpoint{0.027778in}{0.000000in}}{%
\pgfpathmoveto{\pgfqpoint{0.000000in}{0.000000in}}%
\pgfpathlineto{\pgfqpoint{0.027778in}{0.000000in}}%
\pgfusepath{stroke,fill}%
}%
\begin{pgfscope}%
\pgfsys@transformshift{0.607800in}{2.640423in}%
\pgfsys@useobject{currentmarker}{}%
\end{pgfscope}%
\end{pgfscope}%
\begin{pgfscope}%
\pgfsetbuttcap%
\pgfsetroundjoin%
\definecolor{currentfill}{rgb}{0.000000,0.000000,0.000000}%
\pgfsetfillcolor{currentfill}%
\pgfsetlinewidth{0.602250pt}%
\definecolor{currentstroke}{rgb}{0.000000,0.000000,0.000000}%
\pgfsetstrokecolor{currentstroke}%
\pgfsetdash{}{0pt}%
\pgfsys@defobject{currentmarker}{\pgfqpoint{-0.027778in}{0.000000in}}{\pgfqpoint{-0.000000in}{0.000000in}}{%
\pgfpathmoveto{\pgfqpoint{-0.000000in}{0.000000in}}%
\pgfpathlineto{\pgfqpoint{-0.027778in}{0.000000in}}%
\pgfusepath{stroke,fill}%
}%
\begin{pgfscope}%
\pgfsys@transformshift{4.597025in}{2.640423in}%
\pgfsys@useobject{currentmarker}{}%
\end{pgfscope}%
\end{pgfscope}%
\begin{pgfscope}%
\pgfsetbuttcap%
\pgfsetroundjoin%
\definecolor{currentfill}{rgb}{0.000000,0.000000,0.000000}%
\pgfsetfillcolor{currentfill}%
\pgfsetlinewidth{0.602250pt}%
\definecolor{currentstroke}{rgb}{0.000000,0.000000,0.000000}%
\pgfsetstrokecolor{currentstroke}%
\pgfsetdash{}{0pt}%
\pgfsys@defobject{currentmarker}{\pgfqpoint{0.000000in}{0.000000in}}{\pgfqpoint{0.027778in}{0.000000in}}{%
\pgfpathmoveto{\pgfqpoint{0.000000in}{0.000000in}}%
\pgfpathlineto{\pgfqpoint{0.027778in}{0.000000in}}%
\pgfusepath{stroke,fill}%
}%
\begin{pgfscope}%
\pgfsys@transformshift{0.607800in}{2.700147in}%
\pgfsys@useobject{currentmarker}{}%
\end{pgfscope}%
\end{pgfscope}%
\begin{pgfscope}%
\pgfsetbuttcap%
\pgfsetroundjoin%
\definecolor{currentfill}{rgb}{0.000000,0.000000,0.000000}%
\pgfsetfillcolor{currentfill}%
\pgfsetlinewidth{0.602250pt}%
\definecolor{currentstroke}{rgb}{0.000000,0.000000,0.000000}%
\pgfsetstrokecolor{currentstroke}%
\pgfsetdash{}{0pt}%
\pgfsys@defobject{currentmarker}{\pgfqpoint{-0.027778in}{0.000000in}}{\pgfqpoint{-0.000000in}{0.000000in}}{%
\pgfpathmoveto{\pgfqpoint{-0.000000in}{0.000000in}}%
\pgfpathlineto{\pgfqpoint{-0.027778in}{0.000000in}}%
\pgfusepath{stroke,fill}%
}%
\begin{pgfscope}%
\pgfsys@transformshift{4.597025in}{2.700147in}%
\pgfsys@useobject{currentmarker}{}%
\end{pgfscope}%
\end{pgfscope}%
\begin{pgfscope}%
\pgfsetbuttcap%
\pgfsetroundjoin%
\definecolor{currentfill}{rgb}{0.000000,0.000000,0.000000}%
\pgfsetfillcolor{currentfill}%
\pgfsetlinewidth{0.602250pt}%
\definecolor{currentstroke}{rgb}{0.000000,0.000000,0.000000}%
\pgfsetstrokecolor{currentstroke}%
\pgfsetdash{}{0pt}%
\pgfsys@defobject{currentmarker}{\pgfqpoint{0.000000in}{0.000000in}}{\pgfqpoint{0.027778in}{0.000000in}}{%
\pgfpathmoveto{\pgfqpoint{0.000000in}{0.000000in}}%
\pgfpathlineto{\pgfqpoint{0.027778in}{0.000000in}}%
\pgfusepath{stroke,fill}%
}%
\begin{pgfscope}%
\pgfsys@transformshift{0.607800in}{2.819593in}%
\pgfsys@useobject{currentmarker}{}%
\end{pgfscope}%
\end{pgfscope}%
\begin{pgfscope}%
\pgfsetbuttcap%
\pgfsetroundjoin%
\definecolor{currentfill}{rgb}{0.000000,0.000000,0.000000}%
\pgfsetfillcolor{currentfill}%
\pgfsetlinewidth{0.602250pt}%
\definecolor{currentstroke}{rgb}{0.000000,0.000000,0.000000}%
\pgfsetstrokecolor{currentstroke}%
\pgfsetdash{}{0pt}%
\pgfsys@defobject{currentmarker}{\pgfqpoint{-0.027778in}{0.000000in}}{\pgfqpoint{-0.000000in}{0.000000in}}{%
\pgfpathmoveto{\pgfqpoint{-0.000000in}{0.000000in}}%
\pgfpathlineto{\pgfqpoint{-0.027778in}{0.000000in}}%
\pgfusepath{stroke,fill}%
}%
\begin{pgfscope}%
\pgfsys@transformshift{4.597025in}{2.819593in}%
\pgfsys@useobject{currentmarker}{}%
\end{pgfscope}%
\end{pgfscope}%
\begin{pgfscope}%
\pgfpathrectangle{\pgfqpoint{0.607800in}{0.251500in}}{\pgfqpoint{3.989225in}{2.627814in}}%
\pgfusepath{clip}%
\pgfsetbuttcap%
\pgfsetroundjoin%
\pgfsetlinewidth{1.003750pt}%
\definecolor{currentstroke}{rgb}{0.000000,0.000000,0.000000}%
\pgfsetstrokecolor{currentstroke}%
\pgfsetdash{{1.000000pt}{0.000000pt}}{0.000000pt}%
\pgfpathmoveto{\pgfqpoint{0.789129in}{1.565407in}}%
\pgfpathlineto{\pgfqpoint{0.861733in}{1.715262in}}%
\pgfpathlineto{\pgfqpoint{0.901665in}{1.796823in}}%
\pgfpathlineto{\pgfqpoint{0.934336in}{1.862749in}}%
\pgfpathlineto{\pgfqpoint{0.963378in}{1.920557in}}%
\pgfpathlineto{\pgfqpoint{0.988789in}{1.970406in}}%
\pgfpathlineto{\pgfqpoint{1.014201in}{2.019470in}}%
\pgfpathlineto{\pgfqpoint{1.039612in}{2.067654in}}%
\pgfpathlineto{\pgfqpoint{1.061393in}{2.108184in}}%
\pgfpathlineto{\pgfqpoint{1.083175in}{2.147941in}}%
\pgfpathlineto{\pgfqpoint{1.104956in}{2.186868in}}%
\pgfpathlineto{\pgfqpoint{1.123107in}{2.218633in}}%
\pgfpathlineto{\pgfqpoint{1.141258in}{2.249753in}}%
\pgfpathlineto{\pgfqpoint{1.159409in}{2.280196in}}%
\pgfpathlineto{\pgfqpoint{1.177560in}{2.309933in}}%
\pgfpathlineto{\pgfqpoint{1.195711in}{2.338933in}}%
\pgfpathlineto{\pgfqpoint{1.213862in}{2.367168in}}%
\pgfpathlineto{\pgfqpoint{1.232013in}{2.394610in}}%
\pgfpathlineto{\pgfqpoint{1.246534in}{2.415975in}}%
\pgfpathlineto{\pgfqpoint{1.261054in}{2.436801in}}%
\pgfpathlineto{\pgfqpoint{1.275575in}{2.457076in}}%
\pgfpathlineto{\pgfqpoint{1.290096in}{2.476787in}}%
\pgfpathlineto{\pgfqpoint{1.304617in}{2.495920in}}%
\pgfpathlineto{\pgfqpoint{1.319137in}{2.514465in}}%
\pgfpathlineto{\pgfqpoint{1.333658in}{2.532410in}}%
\pgfpathlineto{\pgfqpoint{1.348179in}{2.549742in}}%
\pgfpathlineto{\pgfqpoint{1.362700in}{2.566451in}}%
\pgfpathlineto{\pgfqpoint{1.377221in}{2.582527in}}%
\pgfpathlineto{\pgfqpoint{1.391741in}{2.597959in}}%
\pgfpathlineto{\pgfqpoint{1.406262in}{2.612737in}}%
\pgfpathlineto{\pgfqpoint{1.420783in}{2.626853in}}%
\pgfpathlineto{\pgfqpoint{1.435304in}{2.640297in}}%
\pgfpathlineto{\pgfqpoint{1.446194in}{2.649934in}}%
\pgfpathlineto{\pgfqpoint{1.457085in}{2.659185in}}%
\pgfpathlineto{\pgfqpoint{1.467976in}{2.668046in}}%
\pgfpathlineto{\pgfqpoint{1.478866in}{2.676515in}}%
\pgfpathlineto{\pgfqpoint{1.489757in}{2.684588in}}%
\pgfpathlineto{\pgfqpoint{1.500647in}{2.692263in}}%
\pgfpathlineto{\pgfqpoint{1.511538in}{2.699537in}}%
\pgfpathlineto{\pgfqpoint{1.522429in}{2.706407in}}%
\pgfpathlineto{\pgfqpoint{1.533319in}{2.712871in}}%
\pgfpathlineto{\pgfqpoint{1.544210in}{2.718926in}}%
\pgfpathlineto{\pgfqpoint{1.555100in}{2.724571in}}%
\pgfpathlineto{\pgfqpoint{1.565991in}{2.729803in}}%
\pgfpathlineto{\pgfqpoint{1.576882in}{2.734620in}}%
\pgfpathlineto{\pgfqpoint{1.587772in}{2.739021in}}%
\pgfpathlineto{\pgfqpoint{1.598663in}{2.743004in}}%
\pgfpathlineto{\pgfqpoint{1.609553in}{2.746568in}}%
\pgfpathlineto{\pgfqpoint{1.620444in}{2.749712in}}%
\pgfpathlineto{\pgfqpoint{1.631335in}{2.752434in}}%
\pgfpathlineto{\pgfqpoint{1.642225in}{2.754733in}}%
\pgfpathlineto{\pgfqpoint{1.653116in}{2.756609in}}%
\pgfpathlineto{\pgfqpoint{1.664006in}{2.758061in}}%
\pgfpathlineto{\pgfqpoint{1.674897in}{2.759089in}}%
\pgfpathlineto{\pgfqpoint{1.685788in}{2.759691in}}%
\pgfpathlineto{\pgfqpoint{1.696678in}{2.759868in}}%
\pgfpathlineto{\pgfqpoint{1.707569in}{2.759620in}}%
\pgfpathlineto{\pgfqpoint{1.718459in}{2.758947in}}%
\pgfpathlineto{\pgfqpoint{1.729350in}{2.757849in}}%
\pgfpathlineto{\pgfqpoint{1.740240in}{2.756326in}}%
\pgfpathlineto{\pgfqpoint{1.751131in}{2.754380in}}%
\pgfpathlineto{\pgfqpoint{1.762022in}{2.752010in}}%
\pgfpathlineto{\pgfqpoint{1.772912in}{2.749217in}}%
\pgfpathlineto{\pgfqpoint{1.783803in}{2.746004in}}%
\pgfpathlineto{\pgfqpoint{1.794693in}{2.742370in}}%
\pgfpathlineto{\pgfqpoint{1.805584in}{2.738317in}}%
\pgfpathlineto{\pgfqpoint{1.816475in}{2.733846in}}%
\pgfpathlineto{\pgfqpoint{1.827365in}{2.728959in}}%
\pgfpathlineto{\pgfqpoint{1.838256in}{2.723658in}}%
\pgfpathlineto{\pgfqpoint{1.849146in}{2.717945in}}%
\pgfpathlineto{\pgfqpoint{1.860037in}{2.711822in}}%
\pgfpathlineto{\pgfqpoint{1.870928in}{2.705290in}}%
\pgfpathlineto{\pgfqpoint{1.881818in}{2.698353in}}%
\pgfpathlineto{\pgfqpoint{1.892709in}{2.691012in}}%
\pgfpathlineto{\pgfqpoint{1.903599in}{2.683270in}}%
\pgfpathlineto{\pgfqpoint{1.914490in}{2.675131in}}%
\pgfpathlineto{\pgfqpoint{1.925381in}{2.666596in}}%
\pgfpathlineto{\pgfqpoint{1.936271in}{2.657670in}}%
\pgfpathlineto{\pgfqpoint{1.947162in}{2.648354in}}%
\pgfpathlineto{\pgfqpoint{1.958052in}{2.638653in}}%
\pgfpathlineto{\pgfqpoint{1.968943in}{2.628570in}}%
\pgfpathlineto{\pgfqpoint{1.983464in}{2.614538in}}%
\pgfpathlineto{\pgfqpoint{1.997985in}{2.599842in}}%
\pgfpathlineto{\pgfqpoint{2.012505in}{2.584491in}}%
\pgfpathlineto{\pgfqpoint{2.027026in}{2.568495in}}%
\pgfpathlineto{\pgfqpoint{2.041547in}{2.551865in}}%
\pgfpathlineto{\pgfqpoint{2.056068in}{2.534610in}}%
\pgfpathlineto{\pgfqpoint{2.070589in}{2.516741in}}%
\pgfpathlineto{\pgfqpoint{2.085109in}{2.498271in}}%
\pgfpathlineto{\pgfqpoint{2.099630in}{2.479210in}}%
\pgfpathlineto{\pgfqpoint{2.114151in}{2.459571in}}%
\pgfpathlineto{\pgfqpoint{2.128672in}{2.439366in}}%
\pgfpathlineto{\pgfqpoint{2.143193in}{2.418608in}}%
\pgfpathlineto{\pgfqpoint{2.157713in}{2.397310in}}%
\pgfpathlineto{\pgfqpoint{2.175864in}{2.369948in}}%
\pgfpathlineto{\pgfqpoint{2.194015in}{2.341791in}}%
\pgfpathlineto{\pgfqpoint{2.212166in}{2.312866in}}%
\pgfpathlineto{\pgfqpoint{2.230317in}{2.283202in}}%
\pgfpathlineto{\pgfqpoint{2.248468in}{2.252829in}}%
\pgfpathlineto{\pgfqpoint{2.266619in}{2.221775in}}%
\pgfpathlineto{\pgfqpoint{2.284770in}{2.190073in}}%
\pgfpathlineto{\pgfqpoint{2.306551in}{2.151217in}}%
\pgfpathlineto{\pgfqpoint{2.328333in}{2.111527in}}%
\pgfpathlineto{\pgfqpoint{2.350114in}{2.071060in}}%
\pgfpathlineto{\pgfqpoint{2.371895in}{2.029872in}}%
\pgfpathlineto{\pgfqpoint{2.397306in}{1.980989in}}%
\pgfpathlineto{\pgfqpoint{2.422718in}{1.931300in}}%
\pgfpathlineto{\pgfqpoint{2.451759in}{1.873650in}}%
\pgfpathlineto{\pgfqpoint{2.484431in}{1.807868in}}%
\pgfpathlineto{\pgfqpoint{2.520733in}{1.733876in}}%
\pgfpathlineto{\pgfqpoint{2.567926in}{1.636734in}}%
\pgfpathlineto{\pgfqpoint{2.691352in}{1.382077in}}%
\pgfpathlineto{\pgfqpoint{2.727654in}{1.308254in}}%
\pgfpathlineto{\pgfqpoint{2.760326in}{1.242673in}}%
\pgfpathlineto{\pgfqpoint{2.789368in}{1.185241in}}%
\pgfpathlineto{\pgfqpoint{2.814779in}{1.135772in}}%
\pgfpathlineto{\pgfqpoint{2.840191in}{1.087136in}}%
\pgfpathlineto{\pgfqpoint{2.861972in}{1.046183in}}%
\pgfpathlineto{\pgfqpoint{2.883753in}{1.005968in}}%
\pgfpathlineto{\pgfqpoint{2.905534in}{0.966550in}}%
\pgfpathlineto{\pgfqpoint{2.927315in}{0.927985in}}%
\pgfpathlineto{\pgfqpoint{2.945466in}{0.896538in}}%
\pgfpathlineto{\pgfqpoint{2.963617in}{0.865753in}}%
\pgfpathlineto{\pgfqpoint{2.981768in}{0.835660in}}%
\pgfpathlineto{\pgfqpoint{2.999919in}{0.806288in}}%
\pgfpathlineto{\pgfqpoint{3.018070in}{0.777667in}}%
\pgfpathlineto{\pgfqpoint{3.036221in}{0.749825in}}%
\pgfpathlineto{\pgfqpoint{3.054372in}{0.722789in}}%
\pgfpathlineto{\pgfqpoint{3.068893in}{0.701759in}}%
\pgfpathlineto{\pgfqpoint{3.083414in}{0.681276in}}%
\pgfpathlineto{\pgfqpoint{3.097935in}{0.661352in}}%
\pgfpathlineto{\pgfqpoint{3.112455in}{0.642001in}}%
\pgfpathlineto{\pgfqpoint{3.126976in}{0.623234in}}%
\pgfpathlineto{\pgfqpoint{3.141497in}{0.605063in}}%
\pgfpathlineto{\pgfqpoint{3.156018in}{0.587500in}}%
\pgfpathlineto{\pgfqpoint{3.170539in}{0.570556in}}%
\pgfpathlineto{\pgfqpoint{3.185059in}{0.554241in}}%
\pgfpathlineto{\pgfqpoint{3.199580in}{0.538566in}}%
\pgfpathlineto{\pgfqpoint{3.214101in}{0.523542in}}%
\pgfpathlineto{\pgfqpoint{3.228622in}{0.509176in}}%
\pgfpathlineto{\pgfqpoint{3.243143in}{0.495480in}}%
\pgfpathlineto{\pgfqpoint{3.254033in}{0.485651in}}%
\pgfpathlineto{\pgfqpoint{3.264924in}{0.476207in}}%
\pgfpathlineto{\pgfqpoint{3.275814in}{0.467150in}}%
\pgfpathlineto{\pgfqpoint{3.286705in}{0.458485in}}%
\pgfpathlineto{\pgfqpoint{3.297596in}{0.450213in}}%
\pgfpathlineto{\pgfqpoint{3.308486in}{0.442339in}}%
\pgfpathlineto{\pgfqpoint{3.319377in}{0.434864in}}%
\pgfpathlineto{\pgfqpoint{3.330267in}{0.427792in}}%
\pgfpathlineto{\pgfqpoint{3.341158in}{0.421125in}}%
\pgfpathlineto{\pgfqpoint{3.352049in}{0.414865in}}%
\pgfpathlineto{\pgfqpoint{3.362939in}{0.409015in}}%
\pgfpathlineto{\pgfqpoint{3.373830in}{0.403576in}}%
\pgfpathlineto{\pgfqpoint{3.384720in}{0.398551in}}%
\pgfpathlineto{\pgfqpoint{3.395611in}{0.393942in}}%
\pgfpathlineto{\pgfqpoint{3.406502in}{0.389749in}}%
\pgfpathlineto{\pgfqpoint{3.417392in}{0.385975in}}%
\pgfpathlineto{\pgfqpoint{3.428283in}{0.382621in}}%
\pgfpathlineto{\pgfqpoint{3.439173in}{0.379688in}}%
\pgfpathlineto{\pgfqpoint{3.450064in}{0.377178in}}%
\pgfpathlineto{\pgfqpoint{3.460954in}{0.375090in}}%
\pgfpathlineto{\pgfqpoint{3.471845in}{0.373426in}}%
\pgfpathlineto{\pgfqpoint{3.482736in}{0.372186in}}%
\pgfpathlineto{\pgfqpoint{3.493626in}{0.371371in}}%
\pgfpathlineto{\pgfqpoint{3.504517in}{0.370982in}}%
\pgfpathlineto{\pgfqpoint{3.515407in}{0.371017in}}%
\pgfpathlineto{\pgfqpoint{3.526298in}{0.371478in}}%
\pgfpathlineto{\pgfqpoint{3.537189in}{0.372363in}}%
\pgfpathlineto{\pgfqpoint{3.548079in}{0.373674in}}%
\pgfpathlineto{\pgfqpoint{3.558970in}{0.375408in}}%
\pgfpathlineto{\pgfqpoint{3.569860in}{0.377567in}}%
\pgfpathlineto{\pgfqpoint{3.580751in}{0.380148in}}%
\pgfpathlineto{\pgfqpoint{3.591642in}{0.383151in}}%
\pgfpathlineto{\pgfqpoint{3.602532in}{0.386575in}}%
\pgfpathlineto{\pgfqpoint{3.613423in}{0.390419in}}%
\pgfpathlineto{\pgfqpoint{3.624313in}{0.394681in}}%
\pgfpathlineto{\pgfqpoint{3.635204in}{0.399360in}}%
\pgfpathlineto{\pgfqpoint{3.646095in}{0.404454in}}%
\pgfpathlineto{\pgfqpoint{3.656985in}{0.409961in}}%
\pgfpathlineto{\pgfqpoint{3.667876in}{0.415880in}}%
\pgfpathlineto{\pgfqpoint{3.678766in}{0.422208in}}%
\pgfpathlineto{\pgfqpoint{3.689657in}{0.428942in}}%
\pgfpathlineto{\pgfqpoint{3.700548in}{0.436082in}}%
\pgfpathlineto{\pgfqpoint{3.711438in}{0.443623in}}%
\pgfpathlineto{\pgfqpoint{3.722329in}{0.451564in}}%
\pgfpathlineto{\pgfqpoint{3.733219in}{0.459901in}}%
\pgfpathlineto{\pgfqpoint{3.744110in}{0.468632in}}%
\pgfpathlineto{\pgfqpoint{3.755001in}{0.477754in}}%
\pgfpathlineto{\pgfqpoint{3.765891in}{0.487262in}}%
\pgfpathlineto{\pgfqpoint{3.776782in}{0.497155in}}%
\pgfpathlineto{\pgfqpoint{3.791303in}{0.510936in}}%
\pgfpathlineto{\pgfqpoint{3.805823in}{0.525384in}}%
\pgfpathlineto{\pgfqpoint{3.820344in}{0.540490in}}%
\pgfpathlineto{\pgfqpoint{3.834865in}{0.556246in}}%
\pgfpathlineto{\pgfqpoint{3.849386in}{0.572639in}}%
\pgfpathlineto{\pgfqpoint{3.863906in}{0.589661in}}%
\pgfpathlineto{\pgfqpoint{3.878427in}{0.607301in}}%
\pgfpathlineto{\pgfqpoint{3.892948in}{0.625547in}}%
\pgfpathlineto{\pgfqpoint{3.907469in}{0.644388in}}%
\pgfpathlineto{\pgfqpoint{3.921990in}{0.663812in}}%
\pgfpathlineto{\pgfqpoint{3.936510in}{0.683806in}}%
\pgfpathlineto{\pgfqpoint{3.951031in}{0.704358in}}%
\pgfpathlineto{\pgfqpoint{3.965552in}{0.725456in}}%
\pgfpathlineto{\pgfqpoint{3.983703in}{0.752573in}}%
\pgfpathlineto{\pgfqpoint{4.001854in}{0.780494in}}%
\pgfpathlineto{\pgfqpoint{4.020005in}{0.809192in}}%
\pgfpathlineto{\pgfqpoint{4.038156in}{0.838637in}}%
\pgfpathlineto{\pgfqpoint{4.056307in}{0.868801in}}%
\pgfpathlineto{\pgfqpoint{4.074458in}{0.899654in}}%
\pgfpathlineto{\pgfqpoint{4.092609in}{0.931165in}}%
\pgfpathlineto{\pgfqpoint{4.114390in}{0.969803in}}%
\pgfpathlineto{\pgfqpoint{4.136171in}{1.009290in}}%
\pgfpathlineto{\pgfqpoint{4.157953in}{1.049568in}}%
\pgfpathlineto{\pgfqpoint{4.179734in}{1.090581in}}%
\pgfpathlineto{\pgfqpoint{4.205145in}{1.139279in}}%
\pgfpathlineto{\pgfqpoint{4.230557in}{1.188804in}}%
\pgfpathlineto{\pgfqpoint{4.259598in}{1.246291in}}%
\pgfpathlineto{\pgfqpoint{4.288640in}{1.304587in}}%
\pgfpathlineto{\pgfqpoint{4.324942in}{1.378367in}}%
\pgfpathlineto{\pgfqpoint{4.368504in}{1.467853in}}%
\pgfpathlineto{\pgfqpoint{4.415697in}{1.565407in}}%
\pgfpathlineto{\pgfqpoint{4.415697in}{1.565407in}}%
\pgfusepath{stroke}%
\end{pgfscope}%
\begin{pgfscope}%
\pgfpathrectangle{\pgfqpoint{0.607800in}{0.251500in}}{\pgfqpoint{3.989225in}{2.627814in}}%
\pgfusepath{clip}%
\pgfsetbuttcap%
\pgfsetroundjoin%
\pgfsetlinewidth{1.003750pt}%
\definecolor{currentstroke}{rgb}{0.250980,0.250980,0.250980}%
\pgfsetstrokecolor{currentstroke}%
\pgfsetdash{{2.000000pt}{1.000000pt}}{0.000000pt}%
\pgfpathmoveto{\pgfqpoint{0.789129in}{1.565407in}}%
\pgfpathlineto{\pgfqpoint{0.832691in}{1.745024in}}%
\pgfpathlineto{\pgfqpoint{0.858102in}{1.848174in}}%
\pgfpathlineto{\pgfqpoint{0.879884in}{1.934874in}}%
\pgfpathlineto{\pgfqpoint{0.898034in}{2.005537in}}%
\pgfpathlineto{\pgfqpoint{0.912555in}{2.060828in}}%
\pgfpathlineto{\pgfqpoint{0.927076in}{2.114865in}}%
\pgfpathlineto{\pgfqpoint{0.941597in}{2.167511in}}%
\pgfpathlineto{\pgfqpoint{0.956118in}{2.218633in}}%
\pgfpathlineto{\pgfqpoint{0.970638in}{2.268102in}}%
\pgfpathlineto{\pgfqpoint{0.981529in}{2.304043in}}%
\pgfpathlineto{\pgfqpoint{0.992420in}{2.338933in}}%
\pgfpathlineto{\pgfqpoint{1.003310in}{2.372720in}}%
\pgfpathlineto{\pgfqpoint{1.014201in}{2.405359in}}%
\pgfpathlineto{\pgfqpoint{1.025091in}{2.436801in}}%
\pgfpathlineto{\pgfqpoint{1.035982in}{2.467003in}}%
\pgfpathlineto{\pgfqpoint{1.046873in}{2.495920in}}%
\pgfpathlineto{\pgfqpoint{1.057763in}{2.523513in}}%
\pgfpathlineto{\pgfqpoint{1.068654in}{2.549742in}}%
\pgfpathlineto{\pgfqpoint{1.075914in}{2.566451in}}%
\pgfpathlineto{\pgfqpoint{1.083175in}{2.582527in}}%
\pgfpathlineto{\pgfqpoint{1.090435in}{2.597959in}}%
\pgfpathlineto{\pgfqpoint{1.097695in}{2.612737in}}%
\pgfpathlineto{\pgfqpoint{1.104956in}{2.626853in}}%
\pgfpathlineto{\pgfqpoint{1.112216in}{2.640297in}}%
\pgfpathlineto{\pgfqpoint{1.119477in}{2.653061in}}%
\pgfpathlineto{\pgfqpoint{1.126737in}{2.665136in}}%
\pgfpathlineto{\pgfqpoint{1.133997in}{2.676515in}}%
\pgfpathlineto{\pgfqpoint{1.141258in}{2.687191in}}%
\pgfpathlineto{\pgfqpoint{1.148518in}{2.697157in}}%
\pgfpathlineto{\pgfqpoint{1.155779in}{2.706407in}}%
\pgfpathlineto{\pgfqpoint{1.163039in}{2.714935in}}%
\pgfpathlineto{\pgfqpoint{1.170299in}{2.722735in}}%
\pgfpathlineto{\pgfqpoint{1.177560in}{2.729803in}}%
\pgfpathlineto{\pgfqpoint{1.184820in}{2.736133in}}%
\pgfpathlineto{\pgfqpoint{1.192081in}{2.741723in}}%
\pgfpathlineto{\pgfqpoint{1.199341in}{2.746568in}}%
\pgfpathlineto{\pgfqpoint{1.206601in}{2.750666in}}%
\pgfpathlineto{\pgfqpoint{1.213862in}{2.754014in}}%
\pgfpathlineto{\pgfqpoint{1.221122in}{2.756609in}}%
\pgfpathlineto{\pgfqpoint{1.228383in}{2.758451in}}%
\pgfpathlineto{\pgfqpoint{1.232013in}{2.759089in}}%
\pgfpathlineto{\pgfqpoint{1.235643in}{2.759537in}}%
\pgfpathlineto{\pgfqpoint{1.239273in}{2.759797in}}%
\pgfpathlineto{\pgfqpoint{1.242903in}{2.759868in}}%
\pgfpathlineto{\pgfqpoint{1.246534in}{2.759750in}}%
\pgfpathlineto{\pgfqpoint{1.250164in}{2.759443in}}%
\pgfpathlineto{\pgfqpoint{1.253794in}{2.758947in}}%
\pgfpathlineto{\pgfqpoint{1.257424in}{2.758262in}}%
\pgfpathlineto{\pgfqpoint{1.264685in}{2.756326in}}%
\pgfpathlineto{\pgfqpoint{1.271945in}{2.753637in}}%
\pgfpathlineto{\pgfqpoint{1.279205in}{2.750195in}}%
\pgfpathlineto{\pgfqpoint{1.286466in}{2.746004in}}%
\pgfpathlineto{\pgfqpoint{1.293726in}{2.741065in}}%
\pgfpathlineto{\pgfqpoint{1.300987in}{2.735382in}}%
\pgfpathlineto{\pgfqpoint{1.308247in}{2.728959in}}%
\pgfpathlineto{\pgfqpoint{1.315507in}{2.721800in}}%
\pgfpathlineto{\pgfqpoint{1.322768in}{2.713908in}}%
\pgfpathlineto{\pgfqpoint{1.330028in}{2.705290in}}%
\pgfpathlineto{\pgfqpoint{1.337288in}{2.695950in}}%
\pgfpathlineto{\pgfqpoint{1.344549in}{2.685895in}}%
\pgfpathlineto{\pgfqpoint{1.351809in}{2.675131in}}%
\pgfpathlineto{\pgfqpoint{1.359070in}{2.663664in}}%
\pgfpathlineto{\pgfqpoint{1.366330in}{2.651503in}}%
\pgfpathlineto{\pgfqpoint{1.373590in}{2.638653in}}%
\pgfpathlineto{\pgfqpoint{1.380851in}{2.625125in}}%
\pgfpathlineto{\pgfqpoint{1.388111in}{2.610926in}}%
\pgfpathlineto{\pgfqpoint{1.395372in}{2.596065in}}%
\pgfpathlineto{\pgfqpoint{1.402632in}{2.580552in}}%
\pgfpathlineto{\pgfqpoint{1.409892in}{2.564397in}}%
\pgfpathlineto{\pgfqpoint{1.417153in}{2.547609in}}%
\pgfpathlineto{\pgfqpoint{1.428043in}{2.521265in}}%
\pgfpathlineto{\pgfqpoint{1.438934in}{2.493561in}}%
\pgfpathlineto{\pgfqpoint{1.449825in}{2.464534in}}%
\pgfpathlineto{\pgfqpoint{1.460715in}{2.434228in}}%
\pgfpathlineto{\pgfqpoint{1.471606in}{2.402684in}}%
\pgfpathlineto{\pgfqpoint{1.482496in}{2.369948in}}%
\pgfpathlineto{\pgfqpoint{1.493387in}{2.336067in}}%
\pgfpathlineto{\pgfqpoint{1.504278in}{2.301088in}}%
\pgfpathlineto{\pgfqpoint{1.515168in}{2.265061in}}%
\pgfpathlineto{\pgfqpoint{1.529689in}{2.215485in}}%
\pgfpathlineto{\pgfqpoint{1.544210in}{2.164264in}}%
\pgfpathlineto{\pgfqpoint{1.558731in}{2.111527in}}%
\pgfpathlineto{\pgfqpoint{1.573251in}{2.057408in}}%
\pgfpathlineto{\pgfqpoint{1.587772in}{2.002043in}}%
\pgfpathlineto{\pgfqpoint{1.605923in}{1.931300in}}%
\pgfpathlineto{\pgfqpoint{1.624074in}{1.859109in}}%
\pgfpathlineto{\pgfqpoint{1.645855in}{1.770974in}}%
\pgfpathlineto{\pgfqpoint{1.674897in}{1.651726in}}%
\pgfpathlineto{\pgfqpoint{1.747501in}{1.352444in}}%
\pgfpathlineto{\pgfqpoint{1.769282in}{1.264429in}}%
\pgfpathlineto{\pgfqpoint{1.787433in}{1.192370in}}%
\pgfpathlineto{\pgfqpoint{1.805584in}{1.121787in}}%
\pgfpathlineto{\pgfqpoint{1.820105in}{1.066571in}}%
\pgfpathlineto{\pgfqpoint{1.834626in}{1.012617in}}%
\pgfpathlineto{\pgfqpoint{1.849146in}{0.960062in}}%
\pgfpathlineto{\pgfqpoint{1.863667in}{0.909039in}}%
\pgfpathlineto{\pgfqpoint{1.878188in}{0.859678in}}%
\pgfpathlineto{\pgfqpoint{1.889079in}{0.823823in}}%
\pgfpathlineto{\pgfqpoint{1.899969in}{0.789023in}}%
\pgfpathlineto{\pgfqpoint{1.910860in}{0.755329in}}%
\pgfpathlineto{\pgfqpoint{1.921750in}{0.722789in}}%
\pgfpathlineto{\pgfqpoint{1.932641in}{0.691448in}}%
\pgfpathlineto{\pgfqpoint{1.943532in}{0.661352in}}%
\pgfpathlineto{\pgfqpoint{1.954422in}{0.632543in}}%
\pgfpathlineto{\pgfqpoint{1.965313in}{0.605063in}}%
\pgfpathlineto{\pgfqpoint{1.976203in}{0.578950in}}%
\pgfpathlineto{\pgfqpoint{1.983464in}{0.562319in}}%
\pgfpathlineto{\pgfqpoint{1.990724in}{0.546323in}}%
\pgfpathlineto{\pgfqpoint{1.997985in}{0.530972in}}%
\pgfpathlineto{\pgfqpoint{2.005245in}{0.516276in}}%
\pgfpathlineto{\pgfqpoint{2.012505in}{0.502244in}}%
\pgfpathlineto{\pgfqpoint{2.019766in}{0.488885in}}%
\pgfpathlineto{\pgfqpoint{2.027026in}{0.476207in}}%
\pgfpathlineto{\pgfqpoint{2.034287in}{0.464218in}}%
\pgfpathlineto{\pgfqpoint{2.041547in}{0.452926in}}%
\pgfpathlineto{\pgfqpoint{2.048807in}{0.442339in}}%
\pgfpathlineto{\pgfqpoint{2.056068in}{0.432462in}}%
\pgfpathlineto{\pgfqpoint{2.063328in}{0.423302in}}%
\pgfpathlineto{\pgfqpoint{2.070589in}{0.414865in}}%
\pgfpathlineto{\pgfqpoint{2.077849in}{0.407156in}}%
\pgfpathlineto{\pgfqpoint{2.085109in}{0.400180in}}%
\pgfpathlineto{\pgfqpoint{2.092370in}{0.393942in}}%
\pgfpathlineto{\pgfqpoint{2.099630in}{0.388445in}}%
\pgfpathlineto{\pgfqpoint{2.106891in}{0.383693in}}%
\pgfpathlineto{\pgfqpoint{2.114151in}{0.379688in}}%
\pgfpathlineto{\pgfqpoint{2.121411in}{0.376435in}}%
\pgfpathlineto{\pgfqpoint{2.128672in}{0.373933in}}%
\pgfpathlineto{\pgfqpoint{2.135932in}{0.372186in}}%
\pgfpathlineto{\pgfqpoint{2.139562in}{0.371596in}}%
\pgfpathlineto{\pgfqpoint{2.143193in}{0.371194in}}%
\pgfpathlineto{\pgfqpoint{2.146823in}{0.370982in}}%
\pgfpathlineto{\pgfqpoint{2.150453in}{0.370958in}}%
\pgfpathlineto{\pgfqpoint{2.154083in}{0.371123in}}%
\pgfpathlineto{\pgfqpoint{2.157713in}{0.371478in}}%
\pgfpathlineto{\pgfqpoint{2.161343in}{0.372021in}}%
\pgfpathlineto{\pgfqpoint{2.168604in}{0.373674in}}%
\pgfpathlineto{\pgfqpoint{2.175864in}{0.376081in}}%
\pgfpathlineto{\pgfqpoint{2.183125in}{0.379241in}}%
\pgfpathlineto{\pgfqpoint{2.190385in}{0.383151in}}%
\pgfpathlineto{\pgfqpoint{2.197645in}{0.387810in}}%
\pgfpathlineto{\pgfqpoint{2.204906in}{0.393214in}}%
\pgfpathlineto{\pgfqpoint{2.212166in}{0.399360in}}%
\pgfpathlineto{\pgfqpoint{2.219427in}{0.406244in}}%
\pgfpathlineto{\pgfqpoint{2.226687in}{0.413861in}}%
\pgfpathlineto{\pgfqpoint{2.233947in}{0.422208in}}%
\pgfpathlineto{\pgfqpoint{2.241208in}{0.431277in}}%
\pgfpathlineto{\pgfqpoint{2.248468in}{0.441065in}}%
\pgfpathlineto{\pgfqpoint{2.255729in}{0.451564in}}%
\pgfpathlineto{\pgfqpoint{2.262989in}{0.462768in}}%
\pgfpathlineto{\pgfqpoint{2.270249in}{0.474670in}}%
\pgfpathlineto{\pgfqpoint{2.277510in}{0.487262in}}%
\pgfpathlineto{\pgfqpoint{2.284770in}{0.500537in}}%
\pgfpathlineto{\pgfqpoint{2.292031in}{0.514486in}}%
\pgfpathlineto{\pgfqpoint{2.299291in}{0.529099in}}%
\pgfpathlineto{\pgfqpoint{2.306551in}{0.544369in}}%
\pgfpathlineto{\pgfqpoint{2.313812in}{0.560285in}}%
\pgfpathlineto{\pgfqpoint{2.321072in}{0.576836in}}%
\pgfpathlineto{\pgfqpoint{2.331963in}{0.602834in}}%
\pgfpathlineto{\pgfqpoint{2.342853in}{0.630202in}}%
\pgfpathlineto{\pgfqpoint{2.353744in}{0.658902in}}%
\pgfpathlineto{\pgfqpoint{2.364635in}{0.688892in}}%
\pgfpathlineto{\pgfqpoint{2.375525in}{0.720131in}}%
\pgfpathlineto{\pgfqpoint{2.386416in}{0.752573in}}%
\pgfpathlineto{\pgfqpoint{2.397306in}{0.786173in}}%
\pgfpathlineto{\pgfqpoint{2.408197in}{0.820882in}}%
\pgfpathlineto{\pgfqpoint{2.419088in}{0.856651in}}%
\pgfpathlineto{\pgfqpoint{2.429978in}{0.893429in}}%
\pgfpathlineto{\pgfqpoint{2.444499in}{0.943947in}}%
\pgfpathlineto{\pgfqpoint{2.459020in}{0.996037in}}%
\pgfpathlineto{\pgfqpoint{2.473541in}{1.049568in}}%
\pgfpathlineto{\pgfqpoint{2.488061in}{1.104405in}}%
\pgfpathlineto{\pgfqpoint{2.506212in}{1.174575in}}%
\pgfpathlineto{\pgfqpoint{2.524363in}{1.246291in}}%
\pgfpathlineto{\pgfqpoint{2.546145in}{1.333991in}}%
\pgfpathlineto{\pgfqpoint{2.571556in}{1.437937in}}%
\pgfpathlineto{\pgfqpoint{2.622379in}{1.647979in}}%
\pgfpathlineto{\pgfqpoint{2.655050in}{1.782065in}}%
\pgfpathlineto{\pgfqpoint{2.676832in}{1.870019in}}%
\pgfpathlineto{\pgfqpoint{2.694983in}{1.942010in}}%
\pgfpathlineto{\pgfqpoint{2.713134in}{2.012512in}}%
\pgfpathlineto{\pgfqpoint{2.727654in}{2.067654in}}%
\pgfpathlineto{\pgfqpoint{2.742175in}{2.121525in}}%
\pgfpathlineto{\pgfqpoint{2.756696in}{2.173988in}}%
\pgfpathlineto{\pgfqpoint{2.771217in}{2.224910in}}%
\pgfpathlineto{\pgfqpoint{2.785738in}{2.274163in}}%
\pgfpathlineto{\pgfqpoint{2.796628in}{2.309933in}}%
\pgfpathlineto{\pgfqpoint{2.807519in}{2.344642in}}%
\pgfpathlineto{\pgfqpoint{2.818409in}{2.378241in}}%
\pgfpathlineto{\pgfqpoint{2.829300in}{2.410683in}}%
\pgfpathlineto{\pgfqpoint{2.840191in}{2.441922in}}%
\pgfpathlineto{\pgfqpoint{2.851081in}{2.471913in}}%
\pgfpathlineto{\pgfqpoint{2.861972in}{2.500612in}}%
\pgfpathlineto{\pgfqpoint{2.872862in}{2.527980in}}%
\pgfpathlineto{\pgfqpoint{2.883753in}{2.553978in}}%
\pgfpathlineto{\pgfqpoint{2.891013in}{2.570530in}}%
\pgfpathlineto{\pgfqpoint{2.898274in}{2.586446in}}%
\pgfpathlineto{\pgfqpoint{2.905534in}{2.601715in}}%
\pgfpathlineto{\pgfqpoint{2.912795in}{2.616329in}}%
\pgfpathlineto{\pgfqpoint{2.920055in}{2.630277in}}%
\pgfpathlineto{\pgfqpoint{2.927315in}{2.643552in}}%
\pgfpathlineto{\pgfqpoint{2.934576in}{2.656144in}}%
\pgfpathlineto{\pgfqpoint{2.941836in}{2.668046in}}%
\pgfpathlineto{\pgfqpoint{2.949097in}{2.679250in}}%
\pgfpathlineto{\pgfqpoint{2.956357in}{2.689749in}}%
\pgfpathlineto{\pgfqpoint{2.963617in}{2.699537in}}%
\pgfpathlineto{\pgfqpoint{2.970878in}{2.708607in}}%
\pgfpathlineto{\pgfqpoint{2.978138in}{2.716953in}}%
\pgfpathlineto{\pgfqpoint{2.985399in}{2.724571in}}%
\pgfpathlineto{\pgfqpoint{2.992659in}{2.731454in}}%
\pgfpathlineto{\pgfqpoint{2.999919in}{2.737600in}}%
\pgfpathlineto{\pgfqpoint{3.007180in}{2.743004in}}%
\pgfpathlineto{\pgfqpoint{3.014440in}{2.747663in}}%
\pgfpathlineto{\pgfqpoint{3.021700in}{2.751574in}}%
\pgfpathlineto{\pgfqpoint{3.028961in}{2.754733in}}%
\pgfpathlineto{\pgfqpoint{3.036221in}{2.757141in}}%
\pgfpathlineto{\pgfqpoint{3.043482in}{2.758793in}}%
\pgfpathlineto{\pgfqpoint{3.047112in}{2.759337in}}%
\pgfpathlineto{\pgfqpoint{3.050742in}{2.759691in}}%
\pgfpathlineto{\pgfqpoint{3.054372in}{2.759856in}}%
\pgfpathlineto{\pgfqpoint{3.058002in}{2.759833in}}%
\pgfpathlineto{\pgfqpoint{3.061633in}{2.759620in}}%
\pgfpathlineto{\pgfqpoint{3.065263in}{2.759219in}}%
\pgfpathlineto{\pgfqpoint{3.068893in}{2.758628in}}%
\pgfpathlineto{\pgfqpoint{3.076153in}{2.756881in}}%
\pgfpathlineto{\pgfqpoint{3.083414in}{2.754380in}}%
\pgfpathlineto{\pgfqpoint{3.090674in}{2.751126in}}%
\pgfpathlineto{\pgfqpoint{3.097935in}{2.747122in}}%
\pgfpathlineto{\pgfqpoint{3.105195in}{2.742370in}}%
\pgfpathlineto{\pgfqpoint{3.112455in}{2.736873in}}%
\pgfpathlineto{\pgfqpoint{3.119716in}{2.730634in}}%
\pgfpathlineto{\pgfqpoint{3.126976in}{2.723658in}}%
\pgfpathlineto{\pgfqpoint{3.134237in}{2.715950in}}%
\pgfpathlineto{\pgfqpoint{3.141497in}{2.707513in}}%
\pgfpathlineto{\pgfqpoint{3.148757in}{2.698353in}}%
\pgfpathlineto{\pgfqpoint{3.156018in}{2.688476in}}%
\pgfpathlineto{\pgfqpoint{3.163278in}{2.677888in}}%
\pgfpathlineto{\pgfqpoint{3.170539in}{2.666596in}}%
\pgfpathlineto{\pgfqpoint{3.177799in}{2.654608in}}%
\pgfpathlineto{\pgfqpoint{3.185059in}{2.641930in}}%
\pgfpathlineto{\pgfqpoint{3.192320in}{2.628570in}}%
\pgfpathlineto{\pgfqpoint{3.199580in}{2.614538in}}%
\pgfpathlineto{\pgfqpoint{3.206841in}{2.599842in}}%
\pgfpathlineto{\pgfqpoint{3.214101in}{2.584491in}}%
\pgfpathlineto{\pgfqpoint{3.221361in}{2.568495in}}%
\pgfpathlineto{\pgfqpoint{3.228622in}{2.551865in}}%
\pgfpathlineto{\pgfqpoint{3.239512in}{2.525752in}}%
\pgfpathlineto{\pgfqpoint{3.250403in}{2.498271in}}%
\pgfpathlineto{\pgfqpoint{3.261294in}{2.469462in}}%
\pgfpathlineto{\pgfqpoint{3.272184in}{2.439366in}}%
\pgfpathlineto{\pgfqpoint{3.283075in}{2.408025in}}%
\pgfpathlineto{\pgfqpoint{3.293965in}{2.375485in}}%
\pgfpathlineto{\pgfqpoint{3.304856in}{2.341791in}}%
\pgfpathlineto{\pgfqpoint{3.315747in}{2.306992in}}%
\pgfpathlineto{\pgfqpoint{3.326637in}{2.271136in}}%
\pgfpathlineto{\pgfqpoint{3.337528in}{2.234276in}}%
\pgfpathlineto{\pgfqpoint{3.352049in}{2.183657in}}%
\pgfpathlineto{\pgfqpoint{3.366569in}{2.131473in}}%
\pgfpathlineto{\pgfqpoint{3.381090in}{2.077856in}}%
\pgfpathlineto{\pgfqpoint{3.395611in}{2.022942in}}%
\pgfpathlineto{\pgfqpoint{3.413762in}{1.952688in}}%
\pgfpathlineto{\pgfqpoint{3.431913in}{1.880902in}}%
\pgfpathlineto{\pgfqpoint{3.453694in}{1.793137in}}%
\pgfpathlineto{\pgfqpoint{3.479105in}{1.689142in}}%
\pgfpathlineto{\pgfqpoint{3.573491in}{1.300922in}}%
\pgfpathlineto{\pgfqpoint{3.595272in}{1.213846in}}%
\pgfpathlineto{\pgfqpoint{3.613423in}{1.142791in}}%
\pgfpathlineto{\pgfqpoint{3.631574in}{1.073407in}}%
\pgfpathlineto{\pgfqpoint{3.646095in}{1.019287in}}%
\pgfpathlineto{\pgfqpoint{3.660615in}{0.966550in}}%
\pgfpathlineto{\pgfqpoint{3.675136in}{0.915329in}}%
\pgfpathlineto{\pgfqpoint{3.689657in}{0.865753in}}%
\pgfpathlineto{\pgfqpoint{3.700548in}{0.829727in}}%
\pgfpathlineto{\pgfqpoint{3.711438in}{0.794748in}}%
\pgfpathlineto{\pgfqpoint{3.722329in}{0.760866in}}%
\pgfpathlineto{\pgfqpoint{3.733219in}{0.728130in}}%
\pgfpathlineto{\pgfqpoint{3.744110in}{0.696587in}}%
\pgfpathlineto{\pgfqpoint{3.755001in}{0.666280in}}%
\pgfpathlineto{\pgfqpoint{3.765891in}{0.637254in}}%
\pgfpathlineto{\pgfqpoint{3.776782in}{0.609549in}}%
\pgfpathlineto{\pgfqpoint{3.787672in}{0.583205in}}%
\pgfpathlineto{\pgfqpoint{3.794933in}{0.566417in}}%
\pgfpathlineto{\pgfqpoint{3.802193in}{0.550262in}}%
\pgfpathlineto{\pgfqpoint{3.809454in}{0.534749in}}%
\pgfpathlineto{\pgfqpoint{3.816714in}{0.519888in}}%
\pgfpathlineto{\pgfqpoint{3.823974in}{0.505689in}}%
\pgfpathlineto{\pgfqpoint{3.831235in}{0.492161in}}%
\pgfpathlineto{\pgfqpoint{3.838495in}{0.479312in}}%
\pgfpathlineto{\pgfqpoint{3.845756in}{0.467150in}}%
\pgfpathlineto{\pgfqpoint{3.853016in}{0.455683in}}%
\pgfpathlineto{\pgfqpoint{3.860276in}{0.444919in}}%
\pgfpathlineto{\pgfqpoint{3.867537in}{0.434864in}}%
\pgfpathlineto{\pgfqpoint{3.874797in}{0.425524in}}%
\pgfpathlineto{\pgfqpoint{3.882057in}{0.416906in}}%
\pgfpathlineto{\pgfqpoint{3.889318in}{0.409015in}}%
\pgfpathlineto{\pgfqpoint{3.896578in}{0.401855in}}%
\pgfpathlineto{\pgfqpoint{3.903839in}{0.395432in}}%
\pgfpathlineto{\pgfqpoint{3.911099in}{0.389749in}}%
\pgfpathlineto{\pgfqpoint{3.918359in}{0.384811in}}%
\pgfpathlineto{\pgfqpoint{3.925620in}{0.380619in}}%
\pgfpathlineto{\pgfqpoint{3.932880in}{0.377178in}}%
\pgfpathlineto{\pgfqpoint{3.940141in}{0.374488in}}%
\pgfpathlineto{\pgfqpoint{3.947401in}{0.372552in}}%
\pgfpathlineto{\pgfqpoint{3.951031in}{0.371867in}}%
\pgfpathlineto{\pgfqpoint{3.954661in}{0.371371in}}%
\pgfpathlineto{\pgfqpoint{3.958292in}{0.371064in}}%
\pgfpathlineto{\pgfqpoint{3.961922in}{0.370946in}}%
\pgfpathlineto{\pgfqpoint{3.965552in}{0.371017in}}%
\pgfpathlineto{\pgfqpoint{3.969182in}{0.371277in}}%
\pgfpathlineto{\pgfqpoint{3.972812in}{0.371726in}}%
\pgfpathlineto{\pgfqpoint{3.976443in}{0.372363in}}%
\pgfpathlineto{\pgfqpoint{3.983703in}{0.374205in}}%
\pgfpathlineto{\pgfqpoint{3.990963in}{0.376800in}}%
\pgfpathlineto{\pgfqpoint{3.998224in}{0.380148in}}%
\pgfpathlineto{\pgfqpoint{4.005484in}{0.384246in}}%
\pgfpathlineto{\pgfqpoint{4.012745in}{0.389091in}}%
\pgfpathlineto{\pgfqpoint{4.020005in}{0.394681in}}%
\pgfpathlineto{\pgfqpoint{4.027265in}{0.401012in}}%
\pgfpathlineto{\pgfqpoint{4.034526in}{0.408080in}}%
\pgfpathlineto{\pgfqpoint{4.041786in}{0.415880in}}%
\pgfpathlineto{\pgfqpoint{4.049047in}{0.424407in}}%
\pgfpathlineto{\pgfqpoint{4.056307in}{0.433657in}}%
\pgfpathlineto{\pgfqpoint{4.063567in}{0.443623in}}%
\pgfpathlineto{\pgfqpoint{4.070828in}{0.454299in}}%
\pgfpathlineto{\pgfqpoint{4.078088in}{0.465679in}}%
\pgfpathlineto{\pgfqpoint{4.085349in}{0.477754in}}%
\pgfpathlineto{\pgfqpoint{4.092609in}{0.490517in}}%
\pgfpathlineto{\pgfqpoint{4.099869in}{0.503961in}}%
\pgfpathlineto{\pgfqpoint{4.107130in}{0.518077in}}%
\pgfpathlineto{\pgfqpoint{4.114390in}{0.532856in}}%
\pgfpathlineto{\pgfqpoint{4.121651in}{0.548288in}}%
\pgfpathlineto{\pgfqpoint{4.128911in}{0.564363in}}%
\pgfpathlineto{\pgfqpoint{4.136171in}{0.581072in}}%
\pgfpathlineto{\pgfqpoint{4.147062in}{0.607301in}}%
\pgfpathlineto{\pgfqpoint{4.157953in}{0.634894in}}%
\pgfpathlineto{\pgfqpoint{4.168843in}{0.663812in}}%
\pgfpathlineto{\pgfqpoint{4.179734in}{0.694013in}}%
\pgfpathlineto{\pgfqpoint{4.190624in}{0.725456in}}%
\pgfpathlineto{\pgfqpoint{4.201515in}{0.758094in}}%
\pgfpathlineto{\pgfqpoint{4.212406in}{0.791882in}}%
\pgfpathlineto{\pgfqpoint{4.223296in}{0.826771in}}%
\pgfpathlineto{\pgfqpoint{4.234187in}{0.862712in}}%
\pgfpathlineto{\pgfqpoint{4.248708in}{0.912181in}}%
\pgfpathlineto{\pgfqpoint{4.263228in}{0.963303in}}%
\pgfpathlineto{\pgfqpoint{4.277749in}{1.015949in}}%
\pgfpathlineto{\pgfqpoint{4.292270in}{1.069986in}}%
\pgfpathlineto{\pgfqpoint{4.306791in}{1.125277in}}%
\pgfpathlineto{\pgfqpoint{4.324942in}{1.195941in}}%
\pgfpathlineto{\pgfqpoint{4.343093in}{1.268065in}}%
\pgfpathlineto{\pgfqpoint{4.364874in}{1.356142in}}%
\pgfpathlineto{\pgfqpoint{4.393915in}{1.475342in}}%
\pgfpathlineto{\pgfqpoint{4.415697in}{1.565407in}}%
\pgfpathlineto{\pgfqpoint{4.415697in}{1.565407in}}%
\pgfusepath{stroke}%
\end{pgfscope}%
\begin{pgfscope}%
\pgfpathrectangle{\pgfqpoint{0.607800in}{0.251500in}}{\pgfqpoint{3.989225in}{2.627814in}}%
\pgfusepath{clip}%
\pgfsetbuttcap%
\pgfsetroundjoin%
\pgfsetlinewidth{1.003750pt}%
\definecolor{currentstroke}{rgb}{0.501961,0.501961,0.501961}%
\pgfsetstrokecolor{currentstroke}%
\pgfsetdash{{3.000000pt}{2.000000pt}}{0.000000pt}%
\pgfpathmoveto{\pgfqpoint{0.789129in}{1.565407in}}%
\pgfpathlineto{\pgfqpoint{0.821800in}{1.767272in}}%
\pgfpathlineto{\pgfqpoint{0.839951in}{1.877277in}}%
\pgfpathlineto{\pgfqpoint{0.854472in}{1.963330in}}%
\pgfpathlineto{\pgfqpoint{0.868993in}{2.047118in}}%
\pgfpathlineto{\pgfqpoint{0.879884in}{2.108184in}}%
\pgfpathlineto{\pgfqpoint{0.890774in}{2.167511in}}%
\pgfpathlineto{\pgfqpoint{0.901665in}{2.224910in}}%
\pgfpathlineto{\pgfqpoint{0.912555in}{2.280196in}}%
\pgfpathlineto{\pgfqpoint{0.923446in}{2.333193in}}%
\pgfpathlineto{\pgfqpoint{0.934336in}{2.383730in}}%
\pgfpathlineto{\pgfqpoint{0.941597in}{2.415975in}}%
\pgfpathlineto{\pgfqpoint{0.948857in}{2.447008in}}%
\pgfpathlineto{\pgfqpoint{0.956118in}{2.476787in}}%
\pgfpathlineto{\pgfqpoint{0.963378in}{2.505267in}}%
\pgfpathlineto{\pgfqpoint{0.970638in}{2.532410in}}%
\pgfpathlineto{\pgfqpoint{0.977899in}{2.558175in}}%
\pgfpathlineto{\pgfqpoint{0.985159in}{2.582527in}}%
\pgfpathlineto{\pgfqpoint{0.992420in}{2.605430in}}%
\pgfpathlineto{\pgfqpoint{0.999680in}{2.626853in}}%
\pgfpathlineto{\pgfqpoint{1.006940in}{2.646764in}}%
\pgfpathlineto{\pgfqpoint{1.014201in}{2.665136in}}%
\pgfpathlineto{\pgfqpoint{1.021461in}{2.681941in}}%
\pgfpathlineto{\pgfqpoint{1.028722in}{2.697157in}}%
\pgfpathlineto{\pgfqpoint{1.032352in}{2.704162in}}%
\pgfpathlineto{\pgfqpoint{1.035982in}{2.710761in}}%
\pgfpathlineto{\pgfqpoint{1.039612in}{2.716953in}}%
\pgfpathlineto{\pgfqpoint{1.043242in}{2.722735in}}%
\pgfpathlineto{\pgfqpoint{1.046873in}{2.728104in}}%
\pgfpathlineto{\pgfqpoint{1.050503in}{2.733060in}}%
\pgfpathlineto{\pgfqpoint{1.054133in}{2.737600in}}%
\pgfpathlineto{\pgfqpoint{1.057763in}{2.741723in}}%
\pgfpathlineto{\pgfqpoint{1.061393in}{2.745427in}}%
\pgfpathlineto{\pgfqpoint{1.065024in}{2.748711in}}%
\pgfpathlineto{\pgfqpoint{1.068654in}{2.751574in}}%
\pgfpathlineto{\pgfqpoint{1.072284in}{2.754014in}}%
\pgfpathlineto{\pgfqpoint{1.075914in}{2.756031in}}%
\pgfpathlineto{\pgfqpoint{1.079544in}{2.757625in}}%
\pgfpathlineto{\pgfqpoint{1.083175in}{2.758793in}}%
\pgfpathlineto{\pgfqpoint{1.086805in}{2.759537in}}%
\pgfpathlineto{\pgfqpoint{1.090435in}{2.759856in}}%
\pgfpathlineto{\pgfqpoint{1.094065in}{2.759750in}}%
\pgfpathlineto{\pgfqpoint{1.097695in}{2.759219in}}%
\pgfpathlineto{\pgfqpoint{1.101326in}{2.758262in}}%
\pgfpathlineto{\pgfqpoint{1.104956in}{2.756881in}}%
\pgfpathlineto{\pgfqpoint{1.108586in}{2.755076in}}%
\pgfpathlineto{\pgfqpoint{1.112216in}{2.752847in}}%
\pgfpathlineto{\pgfqpoint{1.115846in}{2.750195in}}%
\pgfpathlineto{\pgfqpoint{1.119477in}{2.747122in}}%
\pgfpathlineto{\pgfqpoint{1.123107in}{2.743628in}}%
\pgfpathlineto{\pgfqpoint{1.126737in}{2.739714in}}%
\pgfpathlineto{\pgfqpoint{1.130367in}{2.735382in}}%
\pgfpathlineto{\pgfqpoint{1.133997in}{2.730634in}}%
\pgfpathlineto{\pgfqpoint{1.137628in}{2.725471in}}%
\pgfpathlineto{\pgfqpoint{1.141258in}{2.719895in}}%
\pgfpathlineto{\pgfqpoint{1.144888in}{2.713908in}}%
\pgfpathlineto{\pgfqpoint{1.148518in}{2.707513in}}%
\pgfpathlineto{\pgfqpoint{1.152148in}{2.700710in}}%
\pgfpathlineto{\pgfqpoint{1.159409in}{2.685895in}}%
\pgfpathlineto{\pgfqpoint{1.166669in}{2.669485in}}%
\pgfpathlineto{\pgfqpoint{1.173930in}{2.651503in}}%
\pgfpathlineto{\pgfqpoint{1.181190in}{2.631974in}}%
\pgfpathlineto{\pgfqpoint{1.188450in}{2.610926in}}%
\pgfpathlineto{\pgfqpoint{1.195711in}{2.588390in}}%
\pgfpathlineto{\pgfqpoint{1.202971in}{2.564397in}}%
\pgfpathlineto{\pgfqpoint{1.210232in}{2.538981in}}%
\pgfpathlineto{\pgfqpoint{1.217492in}{2.512180in}}%
\pgfpathlineto{\pgfqpoint{1.224752in}{2.484030in}}%
\pgfpathlineto{\pgfqpoint{1.232013in}{2.454572in}}%
\pgfpathlineto{\pgfqpoint{1.239273in}{2.423848in}}%
\pgfpathlineto{\pgfqpoint{1.246534in}{2.391902in}}%
\pgfpathlineto{\pgfqpoint{1.253794in}{2.358779in}}%
\pgfpathlineto{\pgfqpoint{1.264685in}{2.306992in}}%
\pgfpathlineto{\pgfqpoint{1.275575in}{2.252829in}}%
\pgfpathlineto{\pgfqpoint{1.286466in}{2.196463in}}%
\pgfpathlineto{\pgfqpoint{1.297356in}{2.138077in}}%
\pgfpathlineto{\pgfqpoint{1.308247in}{2.077856in}}%
\pgfpathlineto{\pgfqpoint{1.322768in}{1.995042in}}%
\pgfpathlineto{\pgfqpoint{1.337288in}{1.909782in}}%
\pgfpathlineto{\pgfqpoint{1.355439in}{1.800507in}}%
\pgfpathlineto{\pgfqpoint{1.377221in}{1.666705in}}%
\pgfpathlineto{\pgfqpoint{1.435304in}{1.308254in}}%
\pgfpathlineto{\pgfqpoint{1.453455in}{1.199515in}}%
\pgfpathlineto{\pgfqpoint{1.467976in}{1.114821in}}%
\pgfpathlineto{\pgfqpoint{1.482496in}{1.032693in}}%
\pgfpathlineto{\pgfqpoint{1.493387in}{0.973062in}}%
\pgfpathlineto{\pgfqpoint{1.504278in}{0.915329in}}%
\pgfpathlineto{\pgfqpoint{1.515168in}{0.859678in}}%
\pgfpathlineto{\pgfqpoint{1.526059in}{0.806288in}}%
\pgfpathlineto{\pgfqpoint{1.536949in}{0.755329in}}%
\pgfpathlineto{\pgfqpoint{1.544210in}{0.722789in}}%
\pgfpathlineto{\pgfqpoint{1.551470in}{0.691448in}}%
\pgfpathlineto{\pgfqpoint{1.558731in}{0.661352in}}%
\pgfpathlineto{\pgfqpoint{1.565991in}{0.632543in}}%
\pgfpathlineto{\pgfqpoint{1.573251in}{0.605063in}}%
\pgfpathlineto{\pgfqpoint{1.580512in}{0.578950in}}%
\pgfpathlineto{\pgfqpoint{1.587772in}{0.554241in}}%
\pgfpathlineto{\pgfqpoint{1.595033in}{0.530972in}}%
\pgfpathlineto{\pgfqpoint{1.602293in}{0.509176in}}%
\pgfpathlineto{\pgfqpoint{1.609553in}{0.488885in}}%
\pgfpathlineto{\pgfqpoint{1.616814in}{0.470126in}}%
\pgfpathlineto{\pgfqpoint{1.624074in}{0.452926in}}%
\pgfpathlineto{\pgfqpoint{1.631335in}{0.437311in}}%
\pgfpathlineto{\pgfqpoint{1.638595in}{0.423302in}}%
\pgfpathlineto{\pgfqpoint{1.642225in}{0.416906in}}%
\pgfpathlineto{\pgfqpoint{1.645855in}{0.410919in}}%
\pgfpathlineto{\pgfqpoint{1.649486in}{0.405343in}}%
\pgfpathlineto{\pgfqpoint{1.653116in}{0.400180in}}%
\pgfpathlineto{\pgfqpoint{1.656746in}{0.395432in}}%
\pgfpathlineto{\pgfqpoint{1.660376in}{0.391100in}}%
\pgfpathlineto{\pgfqpoint{1.664006in}{0.387187in}}%
\pgfpathlineto{\pgfqpoint{1.667637in}{0.383693in}}%
\pgfpathlineto{\pgfqpoint{1.671267in}{0.380619in}}%
\pgfpathlineto{\pgfqpoint{1.674897in}{0.377968in}}%
\pgfpathlineto{\pgfqpoint{1.678527in}{0.375739in}}%
\pgfpathlineto{\pgfqpoint{1.682157in}{0.373933in}}%
\pgfpathlineto{\pgfqpoint{1.685788in}{0.372552in}}%
\pgfpathlineto{\pgfqpoint{1.689418in}{0.371596in}}%
\pgfpathlineto{\pgfqpoint{1.693048in}{0.371064in}}%
\pgfpathlineto{\pgfqpoint{1.696678in}{0.370958in}}%
\pgfpathlineto{\pgfqpoint{1.700308in}{0.371277in}}%
\pgfpathlineto{\pgfqpoint{1.703939in}{0.372021in}}%
\pgfpathlineto{\pgfqpoint{1.707569in}{0.373190in}}%
\pgfpathlineto{\pgfqpoint{1.711199in}{0.374783in}}%
\pgfpathlineto{\pgfqpoint{1.714829in}{0.376800in}}%
\pgfpathlineto{\pgfqpoint{1.718459in}{0.379241in}}%
\pgfpathlineto{\pgfqpoint{1.722090in}{0.382103in}}%
\pgfpathlineto{\pgfqpoint{1.725720in}{0.385387in}}%
\pgfpathlineto{\pgfqpoint{1.729350in}{0.389091in}}%
\pgfpathlineto{\pgfqpoint{1.732980in}{0.393214in}}%
\pgfpathlineto{\pgfqpoint{1.736610in}{0.397754in}}%
\pgfpathlineto{\pgfqpoint{1.740240in}{0.402710in}}%
\pgfpathlineto{\pgfqpoint{1.743871in}{0.408080in}}%
\pgfpathlineto{\pgfqpoint{1.747501in}{0.413861in}}%
\pgfpathlineto{\pgfqpoint{1.751131in}{0.420053in}}%
\pgfpathlineto{\pgfqpoint{1.754761in}{0.426652in}}%
\pgfpathlineto{\pgfqpoint{1.758391in}{0.433657in}}%
\pgfpathlineto{\pgfqpoint{1.765652in}{0.448873in}}%
\pgfpathlineto{\pgfqpoint{1.772912in}{0.465679in}}%
\pgfpathlineto{\pgfqpoint{1.780173in}{0.484050in}}%
\pgfpathlineto{\pgfqpoint{1.787433in}{0.503961in}}%
\pgfpathlineto{\pgfqpoint{1.794693in}{0.525384in}}%
\pgfpathlineto{\pgfqpoint{1.801954in}{0.548288in}}%
\pgfpathlineto{\pgfqpoint{1.809214in}{0.572639in}}%
\pgfpathlineto{\pgfqpoint{1.816475in}{0.598405in}}%
\pgfpathlineto{\pgfqpoint{1.823735in}{0.625547in}}%
\pgfpathlineto{\pgfqpoint{1.830995in}{0.654028in}}%
\pgfpathlineto{\pgfqpoint{1.838256in}{0.683806in}}%
\pgfpathlineto{\pgfqpoint{1.845516in}{0.714840in}}%
\pgfpathlineto{\pgfqpoint{1.852777in}{0.747084in}}%
\pgfpathlineto{\pgfqpoint{1.863667in}{0.797621in}}%
\pgfpathlineto{\pgfqpoint{1.874558in}{0.850618in}}%
\pgfpathlineto{\pgfqpoint{1.885448in}{0.905904in}}%
\pgfpathlineto{\pgfqpoint{1.896339in}{0.963303in}}%
\pgfpathlineto{\pgfqpoint{1.907230in}{1.022631in}}%
\pgfpathlineto{\pgfqpoint{1.918120in}{1.083697in}}%
\pgfpathlineto{\pgfqpoint{1.932641in}{1.167484in}}%
\pgfpathlineto{\pgfqpoint{1.947162in}{1.253537in}}%
\pgfpathlineto{\pgfqpoint{1.965313in}{1.363542in}}%
\pgfpathlineto{\pgfqpoint{1.990724in}{1.520343in}}%
\pgfpathlineto{\pgfqpoint{2.030656in}{1.767272in}}%
\pgfpathlineto{\pgfqpoint{2.048807in}{1.877277in}}%
\pgfpathlineto{\pgfqpoint{2.063328in}{1.963330in}}%
\pgfpathlineto{\pgfqpoint{2.077849in}{2.047118in}}%
\pgfpathlineto{\pgfqpoint{2.088740in}{2.108184in}}%
\pgfpathlineto{\pgfqpoint{2.099630in}{2.167511in}}%
\pgfpathlineto{\pgfqpoint{2.110521in}{2.224910in}}%
\pgfpathlineto{\pgfqpoint{2.121411in}{2.280196in}}%
\pgfpathlineto{\pgfqpoint{2.132302in}{2.333193in}}%
\pgfpathlineto{\pgfqpoint{2.143193in}{2.383730in}}%
\pgfpathlineto{\pgfqpoint{2.150453in}{2.415975in}}%
\pgfpathlineto{\pgfqpoint{2.157713in}{2.447008in}}%
\pgfpathlineto{\pgfqpoint{2.164974in}{2.476787in}}%
\pgfpathlineto{\pgfqpoint{2.172234in}{2.505267in}}%
\pgfpathlineto{\pgfqpoint{2.179494in}{2.532410in}}%
\pgfpathlineto{\pgfqpoint{2.186755in}{2.558175in}}%
\pgfpathlineto{\pgfqpoint{2.194015in}{2.582527in}}%
\pgfpathlineto{\pgfqpoint{2.201276in}{2.605430in}}%
\pgfpathlineto{\pgfqpoint{2.208536in}{2.626853in}}%
\pgfpathlineto{\pgfqpoint{2.215796in}{2.646764in}}%
\pgfpathlineto{\pgfqpoint{2.223057in}{2.665136in}}%
\pgfpathlineto{\pgfqpoint{2.230317in}{2.681941in}}%
\pgfpathlineto{\pgfqpoint{2.237578in}{2.697157in}}%
\pgfpathlineto{\pgfqpoint{2.241208in}{2.704162in}}%
\pgfpathlineto{\pgfqpoint{2.244838in}{2.710761in}}%
\pgfpathlineto{\pgfqpoint{2.248468in}{2.716953in}}%
\pgfpathlineto{\pgfqpoint{2.252098in}{2.722735in}}%
\pgfpathlineto{\pgfqpoint{2.255729in}{2.728104in}}%
\pgfpathlineto{\pgfqpoint{2.259359in}{2.733060in}}%
\pgfpathlineto{\pgfqpoint{2.262989in}{2.737600in}}%
\pgfpathlineto{\pgfqpoint{2.266619in}{2.741723in}}%
\pgfpathlineto{\pgfqpoint{2.270249in}{2.745427in}}%
\pgfpathlineto{\pgfqpoint{2.273880in}{2.748711in}}%
\pgfpathlineto{\pgfqpoint{2.277510in}{2.751574in}}%
\pgfpathlineto{\pgfqpoint{2.281140in}{2.754014in}}%
\pgfpathlineto{\pgfqpoint{2.284770in}{2.756031in}}%
\pgfpathlineto{\pgfqpoint{2.288400in}{2.757625in}}%
\pgfpathlineto{\pgfqpoint{2.292031in}{2.758793in}}%
\pgfpathlineto{\pgfqpoint{2.295661in}{2.759537in}}%
\pgfpathlineto{\pgfqpoint{2.299291in}{2.759856in}}%
\pgfpathlineto{\pgfqpoint{2.302921in}{2.759750in}}%
\pgfpathlineto{\pgfqpoint{2.306551in}{2.759219in}}%
\pgfpathlineto{\pgfqpoint{2.310182in}{2.758262in}}%
\pgfpathlineto{\pgfqpoint{2.313812in}{2.756881in}}%
\pgfpathlineto{\pgfqpoint{2.317442in}{2.755076in}}%
\pgfpathlineto{\pgfqpoint{2.321072in}{2.752847in}}%
\pgfpathlineto{\pgfqpoint{2.324702in}{2.750195in}}%
\pgfpathlineto{\pgfqpoint{2.328333in}{2.747122in}}%
\pgfpathlineto{\pgfqpoint{2.331963in}{2.743628in}}%
\pgfpathlineto{\pgfqpoint{2.335593in}{2.739714in}}%
\pgfpathlineto{\pgfqpoint{2.339223in}{2.735382in}}%
\pgfpathlineto{\pgfqpoint{2.342853in}{2.730634in}}%
\pgfpathlineto{\pgfqpoint{2.346484in}{2.725471in}}%
\pgfpathlineto{\pgfqpoint{2.350114in}{2.719895in}}%
\pgfpathlineto{\pgfqpoint{2.353744in}{2.713908in}}%
\pgfpathlineto{\pgfqpoint{2.357374in}{2.707513in}}%
\pgfpathlineto{\pgfqpoint{2.361004in}{2.700710in}}%
\pgfpathlineto{\pgfqpoint{2.368265in}{2.685895in}}%
\pgfpathlineto{\pgfqpoint{2.375525in}{2.669485in}}%
\pgfpathlineto{\pgfqpoint{2.382786in}{2.651503in}}%
\pgfpathlineto{\pgfqpoint{2.390046in}{2.631974in}}%
\pgfpathlineto{\pgfqpoint{2.397306in}{2.610926in}}%
\pgfpathlineto{\pgfqpoint{2.404567in}{2.588390in}}%
\pgfpathlineto{\pgfqpoint{2.411827in}{2.564397in}}%
\pgfpathlineto{\pgfqpoint{2.419088in}{2.538981in}}%
\pgfpathlineto{\pgfqpoint{2.426348in}{2.512180in}}%
\pgfpathlineto{\pgfqpoint{2.433608in}{2.484030in}}%
\pgfpathlineto{\pgfqpoint{2.440869in}{2.454572in}}%
\pgfpathlineto{\pgfqpoint{2.448129in}{2.423848in}}%
\pgfpathlineto{\pgfqpoint{2.455390in}{2.391902in}}%
\pgfpathlineto{\pgfqpoint{2.462650in}{2.358779in}}%
\pgfpathlineto{\pgfqpoint{2.473541in}{2.306992in}}%
\pgfpathlineto{\pgfqpoint{2.484431in}{2.252829in}}%
\pgfpathlineto{\pgfqpoint{2.495322in}{2.196463in}}%
\pgfpathlineto{\pgfqpoint{2.506212in}{2.138077in}}%
\pgfpathlineto{\pgfqpoint{2.517103in}{2.077856in}}%
\pgfpathlineto{\pgfqpoint{2.531624in}{1.995042in}}%
\pgfpathlineto{\pgfqpoint{2.546145in}{1.909782in}}%
\pgfpathlineto{\pgfqpoint{2.564296in}{1.800507in}}%
\pgfpathlineto{\pgfqpoint{2.586077in}{1.666705in}}%
\pgfpathlineto{\pgfqpoint{2.644160in}{1.308254in}}%
\pgfpathlineto{\pgfqpoint{2.662311in}{1.199515in}}%
\pgfpathlineto{\pgfqpoint{2.676832in}{1.114821in}}%
\pgfpathlineto{\pgfqpoint{2.691352in}{1.032693in}}%
\pgfpathlineto{\pgfqpoint{2.702243in}{0.973062in}}%
\pgfpathlineto{\pgfqpoint{2.713134in}{0.915329in}}%
\pgfpathlineto{\pgfqpoint{2.724024in}{0.859678in}}%
\pgfpathlineto{\pgfqpoint{2.734915in}{0.806288in}}%
\pgfpathlineto{\pgfqpoint{2.745805in}{0.755329in}}%
\pgfpathlineto{\pgfqpoint{2.753066in}{0.722789in}}%
\pgfpathlineto{\pgfqpoint{2.760326in}{0.691448in}}%
\pgfpathlineto{\pgfqpoint{2.767587in}{0.661352in}}%
\pgfpathlineto{\pgfqpoint{2.774847in}{0.632543in}}%
\pgfpathlineto{\pgfqpoint{2.782107in}{0.605063in}}%
\pgfpathlineto{\pgfqpoint{2.789368in}{0.578950in}}%
\pgfpathlineto{\pgfqpoint{2.796628in}{0.554241in}}%
\pgfpathlineto{\pgfqpoint{2.803889in}{0.530972in}}%
\pgfpathlineto{\pgfqpoint{2.811149in}{0.509176in}}%
\pgfpathlineto{\pgfqpoint{2.818409in}{0.488885in}}%
\pgfpathlineto{\pgfqpoint{2.825670in}{0.470126in}}%
\pgfpathlineto{\pgfqpoint{2.832930in}{0.452926in}}%
\pgfpathlineto{\pgfqpoint{2.840191in}{0.437311in}}%
\pgfpathlineto{\pgfqpoint{2.847451in}{0.423302in}}%
\pgfpathlineto{\pgfqpoint{2.851081in}{0.416906in}}%
\pgfpathlineto{\pgfqpoint{2.854711in}{0.410919in}}%
\pgfpathlineto{\pgfqpoint{2.858342in}{0.405343in}}%
\pgfpathlineto{\pgfqpoint{2.861972in}{0.400180in}}%
\pgfpathlineto{\pgfqpoint{2.865602in}{0.395432in}}%
\pgfpathlineto{\pgfqpoint{2.869232in}{0.391100in}}%
\pgfpathlineto{\pgfqpoint{2.872862in}{0.387187in}}%
\pgfpathlineto{\pgfqpoint{2.876493in}{0.383693in}}%
\pgfpathlineto{\pgfqpoint{2.880123in}{0.380619in}}%
\pgfpathlineto{\pgfqpoint{2.883753in}{0.377968in}}%
\pgfpathlineto{\pgfqpoint{2.887383in}{0.375739in}}%
\pgfpathlineto{\pgfqpoint{2.891013in}{0.373933in}}%
\pgfpathlineto{\pgfqpoint{2.894644in}{0.372552in}}%
\pgfpathlineto{\pgfqpoint{2.898274in}{0.371596in}}%
\pgfpathlineto{\pgfqpoint{2.901904in}{0.371064in}}%
\pgfpathlineto{\pgfqpoint{2.905534in}{0.370958in}}%
\pgfpathlineto{\pgfqpoint{2.909164in}{0.371277in}}%
\pgfpathlineto{\pgfqpoint{2.912795in}{0.372021in}}%
\pgfpathlineto{\pgfqpoint{2.916425in}{0.373190in}}%
\pgfpathlineto{\pgfqpoint{2.920055in}{0.374783in}}%
\pgfpathlineto{\pgfqpoint{2.923685in}{0.376800in}}%
\pgfpathlineto{\pgfqpoint{2.927315in}{0.379241in}}%
\pgfpathlineto{\pgfqpoint{2.930946in}{0.382103in}}%
\pgfpathlineto{\pgfqpoint{2.934576in}{0.385387in}}%
\pgfpathlineto{\pgfqpoint{2.938206in}{0.389091in}}%
\pgfpathlineto{\pgfqpoint{2.941836in}{0.393214in}}%
\pgfpathlineto{\pgfqpoint{2.945466in}{0.397754in}}%
\pgfpathlineto{\pgfqpoint{2.949097in}{0.402710in}}%
\pgfpathlineto{\pgfqpoint{2.952727in}{0.408080in}}%
\pgfpathlineto{\pgfqpoint{2.956357in}{0.413861in}}%
\pgfpathlineto{\pgfqpoint{2.959987in}{0.420053in}}%
\pgfpathlineto{\pgfqpoint{2.963617in}{0.426652in}}%
\pgfpathlineto{\pgfqpoint{2.967248in}{0.433657in}}%
\pgfpathlineto{\pgfqpoint{2.974508in}{0.448873in}}%
\pgfpathlineto{\pgfqpoint{2.981768in}{0.465679in}}%
\pgfpathlineto{\pgfqpoint{2.989029in}{0.484050in}}%
\pgfpathlineto{\pgfqpoint{2.996289in}{0.503961in}}%
\pgfpathlineto{\pgfqpoint{3.003550in}{0.525384in}}%
\pgfpathlineto{\pgfqpoint{3.010810in}{0.548288in}}%
\pgfpathlineto{\pgfqpoint{3.018070in}{0.572639in}}%
\pgfpathlineto{\pgfqpoint{3.025331in}{0.598405in}}%
\pgfpathlineto{\pgfqpoint{3.032591in}{0.625547in}}%
\pgfpathlineto{\pgfqpoint{3.039851in}{0.654028in}}%
\pgfpathlineto{\pgfqpoint{3.047112in}{0.683806in}}%
\pgfpathlineto{\pgfqpoint{3.054372in}{0.714840in}}%
\pgfpathlineto{\pgfqpoint{3.061633in}{0.747084in}}%
\pgfpathlineto{\pgfqpoint{3.072523in}{0.797621in}}%
\pgfpathlineto{\pgfqpoint{3.083414in}{0.850618in}}%
\pgfpathlineto{\pgfqpoint{3.094304in}{0.905904in}}%
\pgfpathlineto{\pgfqpoint{3.105195in}{0.963303in}}%
\pgfpathlineto{\pgfqpoint{3.116086in}{1.022631in}}%
\pgfpathlineto{\pgfqpoint{3.126976in}{1.083697in}}%
\pgfpathlineto{\pgfqpoint{3.141497in}{1.167484in}}%
\pgfpathlineto{\pgfqpoint{3.156018in}{1.253537in}}%
\pgfpathlineto{\pgfqpoint{3.174169in}{1.363542in}}%
\pgfpathlineto{\pgfqpoint{3.199580in}{1.520343in}}%
\pgfpathlineto{\pgfqpoint{3.239512in}{1.767272in}}%
\pgfpathlineto{\pgfqpoint{3.257663in}{1.877277in}}%
\pgfpathlineto{\pgfqpoint{3.272184in}{1.963330in}}%
\pgfpathlineto{\pgfqpoint{3.286705in}{2.047118in}}%
\pgfpathlineto{\pgfqpoint{3.297596in}{2.108184in}}%
\pgfpathlineto{\pgfqpoint{3.308486in}{2.167511in}}%
\pgfpathlineto{\pgfqpoint{3.319377in}{2.224910in}}%
\pgfpathlineto{\pgfqpoint{3.330267in}{2.280196in}}%
\pgfpathlineto{\pgfqpoint{3.341158in}{2.333193in}}%
\pgfpathlineto{\pgfqpoint{3.352049in}{2.383730in}}%
\pgfpathlineto{\pgfqpoint{3.359309in}{2.415975in}}%
\pgfpathlineto{\pgfqpoint{3.366569in}{2.447008in}}%
\pgfpathlineto{\pgfqpoint{3.373830in}{2.476787in}}%
\pgfpathlineto{\pgfqpoint{3.381090in}{2.505267in}}%
\pgfpathlineto{\pgfqpoint{3.388351in}{2.532410in}}%
\pgfpathlineto{\pgfqpoint{3.395611in}{2.558175in}}%
\pgfpathlineto{\pgfqpoint{3.402871in}{2.582527in}}%
\pgfpathlineto{\pgfqpoint{3.410132in}{2.605430in}}%
\pgfpathlineto{\pgfqpoint{3.417392in}{2.626853in}}%
\pgfpathlineto{\pgfqpoint{3.424653in}{2.646764in}}%
\pgfpathlineto{\pgfqpoint{3.431913in}{2.665136in}}%
\pgfpathlineto{\pgfqpoint{3.439173in}{2.681941in}}%
\pgfpathlineto{\pgfqpoint{3.446434in}{2.697157in}}%
\pgfpathlineto{\pgfqpoint{3.450064in}{2.704162in}}%
\pgfpathlineto{\pgfqpoint{3.453694in}{2.710761in}}%
\pgfpathlineto{\pgfqpoint{3.457324in}{2.716953in}}%
\pgfpathlineto{\pgfqpoint{3.460954in}{2.722735in}}%
\pgfpathlineto{\pgfqpoint{3.464585in}{2.728104in}}%
\pgfpathlineto{\pgfqpoint{3.468215in}{2.733060in}}%
\pgfpathlineto{\pgfqpoint{3.471845in}{2.737600in}}%
\pgfpathlineto{\pgfqpoint{3.475475in}{2.741723in}}%
\pgfpathlineto{\pgfqpoint{3.479105in}{2.745427in}}%
\pgfpathlineto{\pgfqpoint{3.482736in}{2.748711in}}%
\pgfpathlineto{\pgfqpoint{3.486366in}{2.751574in}}%
\pgfpathlineto{\pgfqpoint{3.489996in}{2.754014in}}%
\pgfpathlineto{\pgfqpoint{3.493626in}{2.756031in}}%
\pgfpathlineto{\pgfqpoint{3.497256in}{2.757625in}}%
\pgfpathlineto{\pgfqpoint{3.500887in}{2.758793in}}%
\pgfpathlineto{\pgfqpoint{3.504517in}{2.759537in}}%
\pgfpathlineto{\pgfqpoint{3.508147in}{2.759856in}}%
\pgfpathlineto{\pgfqpoint{3.511777in}{2.759750in}}%
\pgfpathlineto{\pgfqpoint{3.515407in}{2.759219in}}%
\pgfpathlineto{\pgfqpoint{3.519038in}{2.758262in}}%
\pgfpathlineto{\pgfqpoint{3.522668in}{2.756881in}}%
\pgfpathlineto{\pgfqpoint{3.526298in}{2.755076in}}%
\pgfpathlineto{\pgfqpoint{3.529928in}{2.752847in}}%
\pgfpathlineto{\pgfqpoint{3.533558in}{2.750195in}}%
\pgfpathlineto{\pgfqpoint{3.537189in}{2.747122in}}%
\pgfpathlineto{\pgfqpoint{3.540819in}{2.743628in}}%
\pgfpathlineto{\pgfqpoint{3.544449in}{2.739714in}}%
\pgfpathlineto{\pgfqpoint{3.548079in}{2.735382in}}%
\pgfpathlineto{\pgfqpoint{3.551709in}{2.730634in}}%
\pgfpathlineto{\pgfqpoint{3.555340in}{2.725471in}}%
\pgfpathlineto{\pgfqpoint{3.558970in}{2.719895in}}%
\pgfpathlineto{\pgfqpoint{3.562600in}{2.713908in}}%
\pgfpathlineto{\pgfqpoint{3.566230in}{2.707513in}}%
\pgfpathlineto{\pgfqpoint{3.569860in}{2.700710in}}%
\pgfpathlineto{\pgfqpoint{3.577121in}{2.685895in}}%
\pgfpathlineto{\pgfqpoint{3.584381in}{2.669485in}}%
\pgfpathlineto{\pgfqpoint{3.591642in}{2.651503in}}%
\pgfpathlineto{\pgfqpoint{3.598902in}{2.631974in}}%
\pgfpathlineto{\pgfqpoint{3.606162in}{2.610926in}}%
\pgfpathlineto{\pgfqpoint{3.613423in}{2.588390in}}%
\pgfpathlineto{\pgfqpoint{3.620683in}{2.564397in}}%
\pgfpathlineto{\pgfqpoint{3.627944in}{2.538981in}}%
\pgfpathlineto{\pgfqpoint{3.635204in}{2.512180in}}%
\pgfpathlineto{\pgfqpoint{3.642464in}{2.484030in}}%
\pgfpathlineto{\pgfqpoint{3.649725in}{2.454572in}}%
\pgfpathlineto{\pgfqpoint{3.656985in}{2.423848in}}%
\pgfpathlineto{\pgfqpoint{3.664246in}{2.391902in}}%
\pgfpathlineto{\pgfqpoint{3.671506in}{2.358779in}}%
\pgfpathlineto{\pgfqpoint{3.682397in}{2.306992in}}%
\pgfpathlineto{\pgfqpoint{3.693287in}{2.252829in}}%
\pgfpathlineto{\pgfqpoint{3.704178in}{2.196463in}}%
\pgfpathlineto{\pgfqpoint{3.715068in}{2.138077in}}%
\pgfpathlineto{\pgfqpoint{3.725959in}{2.077856in}}%
\pgfpathlineto{\pgfqpoint{3.740480in}{1.995042in}}%
\pgfpathlineto{\pgfqpoint{3.755001in}{1.909782in}}%
\pgfpathlineto{\pgfqpoint{3.773152in}{1.800507in}}%
\pgfpathlineto{\pgfqpoint{3.794933in}{1.666705in}}%
\pgfpathlineto{\pgfqpoint{3.853016in}{1.308254in}}%
\pgfpathlineto{\pgfqpoint{3.871167in}{1.199515in}}%
\pgfpathlineto{\pgfqpoint{3.885688in}{1.114821in}}%
\pgfpathlineto{\pgfqpoint{3.900208in}{1.032693in}}%
\pgfpathlineto{\pgfqpoint{3.911099in}{0.973062in}}%
\pgfpathlineto{\pgfqpoint{3.921990in}{0.915329in}}%
\pgfpathlineto{\pgfqpoint{3.932880in}{0.859678in}}%
\pgfpathlineto{\pgfqpoint{3.943771in}{0.806288in}}%
\pgfpathlineto{\pgfqpoint{3.954661in}{0.755329in}}%
\pgfpathlineto{\pgfqpoint{3.961922in}{0.722789in}}%
\pgfpathlineto{\pgfqpoint{3.969182in}{0.691448in}}%
\pgfpathlineto{\pgfqpoint{3.976443in}{0.661352in}}%
\pgfpathlineto{\pgfqpoint{3.983703in}{0.632543in}}%
\pgfpathlineto{\pgfqpoint{3.990963in}{0.605063in}}%
\pgfpathlineto{\pgfqpoint{3.998224in}{0.578950in}}%
\pgfpathlineto{\pgfqpoint{4.005484in}{0.554241in}}%
\pgfpathlineto{\pgfqpoint{4.012745in}{0.530972in}}%
\pgfpathlineto{\pgfqpoint{4.020005in}{0.509176in}}%
\pgfpathlineto{\pgfqpoint{4.027265in}{0.488885in}}%
\pgfpathlineto{\pgfqpoint{4.034526in}{0.470126in}}%
\pgfpathlineto{\pgfqpoint{4.041786in}{0.452926in}}%
\pgfpathlineto{\pgfqpoint{4.049047in}{0.437311in}}%
\pgfpathlineto{\pgfqpoint{4.056307in}{0.423302in}}%
\pgfpathlineto{\pgfqpoint{4.059937in}{0.416906in}}%
\pgfpathlineto{\pgfqpoint{4.063567in}{0.410919in}}%
\pgfpathlineto{\pgfqpoint{4.067198in}{0.405343in}}%
\pgfpathlineto{\pgfqpoint{4.070828in}{0.400180in}}%
\pgfpathlineto{\pgfqpoint{4.074458in}{0.395432in}}%
\pgfpathlineto{\pgfqpoint{4.078088in}{0.391100in}}%
\pgfpathlineto{\pgfqpoint{4.081718in}{0.387187in}}%
\pgfpathlineto{\pgfqpoint{4.085349in}{0.383693in}}%
\pgfpathlineto{\pgfqpoint{4.088979in}{0.380619in}}%
\pgfpathlineto{\pgfqpoint{4.092609in}{0.377968in}}%
\pgfpathlineto{\pgfqpoint{4.096239in}{0.375739in}}%
\pgfpathlineto{\pgfqpoint{4.099869in}{0.373933in}}%
\pgfpathlineto{\pgfqpoint{4.103500in}{0.372552in}}%
\pgfpathlineto{\pgfqpoint{4.107130in}{0.371596in}}%
\pgfpathlineto{\pgfqpoint{4.110760in}{0.371064in}}%
\pgfpathlineto{\pgfqpoint{4.114390in}{0.370958in}}%
\pgfpathlineto{\pgfqpoint{4.118020in}{0.371277in}}%
\pgfpathlineto{\pgfqpoint{4.121651in}{0.372021in}}%
\pgfpathlineto{\pgfqpoint{4.125281in}{0.373190in}}%
\pgfpathlineto{\pgfqpoint{4.128911in}{0.374783in}}%
\pgfpathlineto{\pgfqpoint{4.132541in}{0.376800in}}%
\pgfpathlineto{\pgfqpoint{4.136171in}{0.379241in}}%
\pgfpathlineto{\pgfqpoint{4.139802in}{0.382103in}}%
\pgfpathlineto{\pgfqpoint{4.143432in}{0.385387in}}%
\pgfpathlineto{\pgfqpoint{4.147062in}{0.389091in}}%
\pgfpathlineto{\pgfqpoint{4.150692in}{0.393214in}}%
\pgfpathlineto{\pgfqpoint{4.154322in}{0.397754in}}%
\pgfpathlineto{\pgfqpoint{4.157953in}{0.402710in}}%
\pgfpathlineto{\pgfqpoint{4.161583in}{0.408080in}}%
\pgfpathlineto{\pgfqpoint{4.165213in}{0.413861in}}%
\pgfpathlineto{\pgfqpoint{4.168843in}{0.420053in}}%
\pgfpathlineto{\pgfqpoint{4.172473in}{0.426652in}}%
\pgfpathlineto{\pgfqpoint{4.176104in}{0.433657in}}%
\pgfpathlineto{\pgfqpoint{4.183364in}{0.448873in}}%
\pgfpathlineto{\pgfqpoint{4.190624in}{0.465679in}}%
\pgfpathlineto{\pgfqpoint{4.197885in}{0.484050in}}%
\pgfpathlineto{\pgfqpoint{4.205145in}{0.503961in}}%
\pgfpathlineto{\pgfqpoint{4.212406in}{0.525384in}}%
\pgfpathlineto{\pgfqpoint{4.219666in}{0.548288in}}%
\pgfpathlineto{\pgfqpoint{4.226926in}{0.572639in}}%
\pgfpathlineto{\pgfqpoint{4.234187in}{0.598405in}}%
\pgfpathlineto{\pgfqpoint{4.241447in}{0.625547in}}%
\pgfpathlineto{\pgfqpoint{4.248708in}{0.654028in}}%
\pgfpathlineto{\pgfqpoint{4.255968in}{0.683806in}}%
\pgfpathlineto{\pgfqpoint{4.263228in}{0.714840in}}%
\pgfpathlineto{\pgfqpoint{4.270489in}{0.747084in}}%
\pgfpathlineto{\pgfqpoint{4.281379in}{0.797621in}}%
\pgfpathlineto{\pgfqpoint{4.292270in}{0.850618in}}%
\pgfpathlineto{\pgfqpoint{4.303160in}{0.905904in}}%
\pgfpathlineto{\pgfqpoint{4.314051in}{0.963303in}}%
\pgfpathlineto{\pgfqpoint{4.324942in}{1.022631in}}%
\pgfpathlineto{\pgfqpoint{4.335832in}{1.083697in}}%
\pgfpathlineto{\pgfqpoint{4.350353in}{1.167484in}}%
\pgfpathlineto{\pgfqpoint{4.364874in}{1.253537in}}%
\pgfpathlineto{\pgfqpoint{4.383025in}{1.363542in}}%
\pgfpathlineto{\pgfqpoint{4.408436in}{1.520343in}}%
\pgfpathlineto{\pgfqpoint{4.415697in}{1.565407in}}%
\pgfpathlineto{\pgfqpoint{4.415697in}{1.565407in}}%
\pgfusepath{stroke}%
\end{pgfscope}%
\begin{pgfscope}%
\pgfpathrectangle{\pgfqpoint{0.607800in}{0.251500in}}{\pgfqpoint{3.989225in}{2.627814in}}%
\pgfusepath{clip}%
\pgfsetbuttcap%
\pgfsetroundjoin%
\pgfsetlinewidth{1.003750pt}%
\definecolor{currentstroke}{rgb}{0.752941,0.752941,0.752941}%
\pgfsetstrokecolor{currentstroke}%
\pgfsetdash{{4.000000pt}{3.000000pt}}{0.000000pt}%
\pgfpathmoveto{\pgfqpoint{0.789129in}{1.565407in}}%
\pgfpathlineto{\pgfqpoint{0.814540in}{1.774673in}}%
\pgfpathlineto{\pgfqpoint{0.829061in}{1.891756in}}%
\pgfpathlineto{\pgfqpoint{0.843582in}{2.005537in}}%
\pgfpathlineto{\pgfqpoint{0.854472in}{2.088012in}}%
\pgfpathlineto{\pgfqpoint{0.865363in}{2.167511in}}%
\pgfpathlineto{\pgfqpoint{0.876253in}{2.243582in}}%
\pgfpathlineto{\pgfqpoint{0.883514in}{2.292177in}}%
\pgfpathlineto{\pgfqpoint{0.890774in}{2.338933in}}%
\pgfpathlineto{\pgfqpoint{0.898034in}{2.383730in}}%
\pgfpathlineto{\pgfqpoint{0.905295in}{2.426456in}}%
\pgfpathlineto{\pgfqpoint{0.912555in}{2.467003in}}%
\pgfpathlineto{\pgfqpoint{0.919816in}{2.505267in}}%
\pgfpathlineto{\pgfqpoint{0.927076in}{2.541153in}}%
\pgfpathlineto{\pgfqpoint{0.934336in}{2.574569in}}%
\pgfpathlineto{\pgfqpoint{0.941597in}{2.605430in}}%
\pgfpathlineto{\pgfqpoint{0.948857in}{2.633659in}}%
\pgfpathlineto{\pgfqpoint{0.956118in}{2.659185in}}%
\pgfpathlineto{\pgfqpoint{0.959748in}{2.670913in}}%
\pgfpathlineto{\pgfqpoint{0.963378in}{2.681941in}}%
\pgfpathlineto{\pgfqpoint{0.967008in}{2.692263in}}%
\pgfpathlineto{\pgfqpoint{0.970638in}{2.701872in}}%
\pgfpathlineto{\pgfqpoint{0.974269in}{2.710761in}}%
\pgfpathlineto{\pgfqpoint{0.977899in}{2.718926in}}%
\pgfpathlineto{\pgfqpoint{0.981529in}{2.726360in}}%
\pgfpathlineto{\pgfqpoint{0.985159in}{2.733060in}}%
\pgfpathlineto{\pgfqpoint{0.988789in}{2.739021in}}%
\pgfpathlineto{\pgfqpoint{0.992420in}{2.744239in}}%
\pgfpathlineto{\pgfqpoint{0.996050in}{2.748711in}}%
\pgfpathlineto{\pgfqpoint{0.999680in}{2.752434in}}%
\pgfpathlineto{\pgfqpoint{1.003310in}{2.755406in}}%
\pgfpathlineto{\pgfqpoint{1.006940in}{2.757625in}}%
\pgfpathlineto{\pgfqpoint{1.010571in}{2.759089in}}%
\pgfpathlineto{\pgfqpoint{1.014201in}{2.759797in}}%
\pgfpathlineto{\pgfqpoint{1.017831in}{2.759750in}}%
\pgfpathlineto{\pgfqpoint{1.021461in}{2.758947in}}%
\pgfpathlineto{\pgfqpoint{1.025091in}{2.757388in}}%
\pgfpathlineto{\pgfqpoint{1.028722in}{2.755076in}}%
\pgfpathlineto{\pgfqpoint{1.032352in}{2.752010in}}%
\pgfpathlineto{\pgfqpoint{1.035982in}{2.748193in}}%
\pgfpathlineto{\pgfqpoint{1.039612in}{2.743628in}}%
\pgfpathlineto{\pgfqpoint{1.043242in}{2.738317in}}%
\pgfpathlineto{\pgfqpoint{1.046873in}{2.732263in}}%
\pgfpathlineto{\pgfqpoint{1.050503in}{2.725471in}}%
\pgfpathlineto{\pgfqpoint{1.054133in}{2.717945in}}%
\pgfpathlineto{\pgfqpoint{1.057763in}{2.709690in}}%
\pgfpathlineto{\pgfqpoint{1.061393in}{2.700710in}}%
\pgfpathlineto{\pgfqpoint{1.065024in}{2.691012in}}%
\pgfpathlineto{\pgfqpoint{1.068654in}{2.680601in}}%
\pgfpathlineto{\pgfqpoint{1.072284in}{2.669485in}}%
\pgfpathlineto{\pgfqpoint{1.075914in}{2.657670in}}%
\pgfpathlineto{\pgfqpoint{1.079544in}{2.645163in}}%
\pgfpathlineto{\pgfqpoint{1.086805in}{2.618109in}}%
\pgfpathlineto{\pgfqpoint{1.094065in}{2.588390in}}%
\pgfpathlineto{\pgfqpoint{1.101326in}{2.556081in}}%
\pgfpathlineto{\pgfqpoint{1.108586in}{2.521265in}}%
\pgfpathlineto{\pgfqpoint{1.115846in}{2.484030in}}%
\pgfpathlineto{\pgfqpoint{1.123107in}{2.444470in}}%
\pgfpathlineto{\pgfqpoint{1.130367in}{2.402684in}}%
\pgfpathlineto{\pgfqpoint{1.137628in}{2.358779in}}%
\pgfpathlineto{\pgfqpoint{1.144888in}{2.312866in}}%
\pgfpathlineto{\pgfqpoint{1.152148in}{2.265061in}}%
\pgfpathlineto{\pgfqpoint{1.163039in}{2.190073in}}%
\pgfpathlineto{\pgfqpoint{1.173930in}{2.111527in}}%
\pgfpathlineto{\pgfqpoint{1.184820in}{2.029872in}}%
\pgfpathlineto{\pgfqpoint{1.195711in}{1.945573in}}%
\pgfpathlineto{\pgfqpoint{1.210232in}{1.829892in}}%
\pgfpathlineto{\pgfqpoint{1.228383in}{1.681667in}}%
\pgfpathlineto{\pgfqpoint{1.275575in}{1.293602in}}%
\pgfpathlineto{\pgfqpoint{1.290096in}{1.178127in}}%
\pgfpathlineto{\pgfqpoint{1.300987in}{1.094030in}}%
\pgfpathlineto{\pgfqpoint{1.311877in}{1.012617in}}%
\pgfpathlineto{\pgfqpoint{1.322768in}{0.934351in}}%
\pgfpathlineto{\pgfqpoint{1.333658in}{0.859678in}}%
\pgfpathlineto{\pgfqpoint{1.340919in}{0.812103in}}%
\pgfpathlineto{\pgfqpoint{1.348179in}{0.766435in}}%
\pgfpathlineto{\pgfqpoint{1.355439in}{0.722789in}}%
\pgfpathlineto{\pgfqpoint{1.362700in}{0.681276in}}%
\pgfpathlineto{\pgfqpoint{1.369960in}{0.642001in}}%
\pgfpathlineto{\pgfqpoint{1.377221in}{0.605063in}}%
\pgfpathlineto{\pgfqpoint{1.384481in}{0.570556in}}%
\pgfpathlineto{\pgfqpoint{1.391741in}{0.538566in}}%
\pgfpathlineto{\pgfqpoint{1.399002in}{0.509176in}}%
\pgfpathlineto{\pgfqpoint{1.406262in}{0.482460in}}%
\pgfpathlineto{\pgfqpoint{1.409892in}{0.470126in}}%
\pgfpathlineto{\pgfqpoint{1.413523in}{0.458485in}}%
\pgfpathlineto{\pgfqpoint{1.417153in}{0.447544in}}%
\pgfpathlineto{\pgfqpoint{1.420783in}{0.437311in}}%
\pgfpathlineto{\pgfqpoint{1.424413in}{0.427792in}}%
\pgfpathlineto{\pgfqpoint{1.428043in}{0.418993in}}%
\pgfpathlineto{\pgfqpoint{1.431674in}{0.410919in}}%
\pgfpathlineto{\pgfqpoint{1.435304in}{0.403576in}}%
\pgfpathlineto{\pgfqpoint{1.438934in}{0.396968in}}%
\pgfpathlineto{\pgfqpoint{1.442564in}{0.391100in}}%
\pgfpathlineto{\pgfqpoint{1.446194in}{0.385975in}}%
\pgfpathlineto{\pgfqpoint{1.449825in}{0.381597in}}%
\pgfpathlineto{\pgfqpoint{1.453455in}{0.377968in}}%
\pgfpathlineto{\pgfqpoint{1.457085in}{0.375090in}}%
\pgfpathlineto{\pgfqpoint{1.460715in}{0.372965in}}%
\pgfpathlineto{\pgfqpoint{1.464345in}{0.371596in}}%
\pgfpathlineto{\pgfqpoint{1.467976in}{0.370982in}}%
\pgfpathlineto{\pgfqpoint{1.471606in}{0.371123in}}%
\pgfpathlineto{\pgfqpoint{1.475236in}{0.372021in}}%
\pgfpathlineto{\pgfqpoint{1.478866in}{0.373674in}}%
\pgfpathlineto{\pgfqpoint{1.482496in}{0.376081in}}%
\pgfpathlineto{\pgfqpoint{1.486127in}{0.379241in}}%
\pgfpathlineto{\pgfqpoint{1.489757in}{0.383151in}}%
\pgfpathlineto{\pgfqpoint{1.493387in}{0.387810in}}%
\pgfpathlineto{\pgfqpoint{1.497017in}{0.393214in}}%
\pgfpathlineto{\pgfqpoint{1.500647in}{0.399360in}}%
\pgfpathlineto{\pgfqpoint{1.504278in}{0.406244in}}%
\pgfpathlineto{\pgfqpoint{1.507908in}{0.413861in}}%
\pgfpathlineto{\pgfqpoint{1.511538in}{0.422208in}}%
\pgfpathlineto{\pgfqpoint{1.515168in}{0.431277in}}%
\pgfpathlineto{\pgfqpoint{1.518798in}{0.441065in}}%
\pgfpathlineto{\pgfqpoint{1.522429in}{0.451564in}}%
\pgfpathlineto{\pgfqpoint{1.526059in}{0.462768in}}%
\pgfpathlineto{\pgfqpoint{1.529689in}{0.474670in}}%
\pgfpathlineto{\pgfqpoint{1.533319in}{0.487262in}}%
\pgfpathlineto{\pgfqpoint{1.540580in}{0.514486in}}%
\pgfpathlineto{\pgfqpoint{1.547840in}{0.544369in}}%
\pgfpathlineto{\pgfqpoint{1.555100in}{0.576836in}}%
\pgfpathlineto{\pgfqpoint{1.562361in}{0.611806in}}%
\pgfpathlineto{\pgfqpoint{1.569621in}{0.649190in}}%
\pgfpathlineto{\pgfqpoint{1.576882in}{0.688892in}}%
\pgfpathlineto{\pgfqpoint{1.584142in}{0.730813in}}%
\pgfpathlineto{\pgfqpoint{1.591402in}{0.774847in}}%
\pgfpathlineto{\pgfqpoint{1.598663in}{0.820882in}}%
\pgfpathlineto{\pgfqpoint{1.605923in}{0.868801in}}%
\pgfpathlineto{\pgfqpoint{1.616814in}{0.943947in}}%
\pgfpathlineto{\pgfqpoint{1.627704in}{1.022631in}}%
\pgfpathlineto{\pgfqpoint{1.638595in}{1.104405in}}%
\pgfpathlineto{\pgfqpoint{1.649486in}{1.188804in}}%
\pgfpathlineto{\pgfqpoint{1.664006in}{1.304587in}}%
\pgfpathlineto{\pgfqpoint{1.682157in}{1.452886in}}%
\pgfpathlineto{\pgfqpoint{1.729350in}{1.840869in}}%
\pgfpathlineto{\pgfqpoint{1.743871in}{1.956239in}}%
\pgfpathlineto{\pgfqpoint{1.754761in}{2.040234in}}%
\pgfpathlineto{\pgfqpoint{1.765652in}{2.121525in}}%
\pgfpathlineto{\pgfqpoint{1.776542in}{2.199650in}}%
\pgfpathlineto{\pgfqpoint{1.787433in}{2.274163in}}%
\pgfpathlineto{\pgfqpoint{1.794693in}{2.321623in}}%
\pgfpathlineto{\pgfqpoint{1.801954in}{2.367168in}}%
\pgfpathlineto{\pgfqpoint{1.809214in}{2.410683in}}%
\pgfpathlineto{\pgfqpoint{1.816475in}{2.452060in}}%
\pgfpathlineto{\pgfqpoint{1.823735in}{2.491192in}}%
\pgfpathlineto{\pgfqpoint{1.830995in}{2.527980in}}%
\pgfpathlineto{\pgfqpoint{1.838256in}{2.562333in}}%
\pgfpathlineto{\pgfqpoint{1.845516in}{2.594162in}}%
\pgfpathlineto{\pgfqpoint{1.852777in}{2.623387in}}%
\pgfpathlineto{\pgfqpoint{1.860037in}{2.649934in}}%
\pgfpathlineto{\pgfqpoint{1.863667in}{2.662182in}}%
\pgfpathlineto{\pgfqpoint{1.867297in}{2.673736in}}%
\pgfpathlineto{\pgfqpoint{1.870928in}{2.684588in}}%
\pgfpathlineto{\pgfqpoint{1.874558in}{2.694733in}}%
\pgfpathlineto{\pgfqpoint{1.878188in}{2.704162in}}%
\pgfpathlineto{\pgfqpoint{1.881818in}{2.712871in}}%
\pgfpathlineto{\pgfqpoint{1.885448in}{2.720853in}}%
\pgfpathlineto{\pgfqpoint{1.889079in}{2.728104in}}%
\pgfpathlineto{\pgfqpoint{1.892709in}{2.734620in}}%
\pgfpathlineto{\pgfqpoint{1.896339in}{2.740395in}}%
\pgfpathlineto{\pgfqpoint{1.899969in}{2.745427in}}%
\pgfpathlineto{\pgfqpoint{1.903599in}{2.749712in}}%
\pgfpathlineto{\pgfqpoint{1.907230in}{2.753248in}}%
\pgfpathlineto{\pgfqpoint{1.910860in}{2.756031in}}%
\pgfpathlineto{\pgfqpoint{1.914490in}{2.758061in}}%
\pgfpathlineto{\pgfqpoint{1.918120in}{2.759337in}}%
\pgfpathlineto{\pgfqpoint{1.921750in}{2.759856in}}%
\pgfpathlineto{\pgfqpoint{1.925381in}{2.759620in}}%
\pgfpathlineto{\pgfqpoint{1.929011in}{2.758628in}}%
\pgfpathlineto{\pgfqpoint{1.932641in}{2.756881in}}%
\pgfpathlineto{\pgfqpoint{1.936271in}{2.754380in}}%
\pgfpathlineto{\pgfqpoint{1.939901in}{2.751126in}}%
\pgfpathlineto{\pgfqpoint{1.943532in}{2.747122in}}%
\pgfpathlineto{\pgfqpoint{1.947162in}{2.742370in}}%
\pgfpathlineto{\pgfqpoint{1.950792in}{2.736873in}}%
\pgfpathlineto{\pgfqpoint{1.954422in}{2.730634in}}%
\pgfpathlineto{\pgfqpoint{1.958052in}{2.723658in}}%
\pgfpathlineto{\pgfqpoint{1.961683in}{2.715950in}}%
\pgfpathlineto{\pgfqpoint{1.965313in}{2.707513in}}%
\pgfpathlineto{\pgfqpoint{1.968943in}{2.698353in}}%
\pgfpathlineto{\pgfqpoint{1.972573in}{2.688476in}}%
\pgfpathlineto{\pgfqpoint{1.976203in}{2.677888in}}%
\pgfpathlineto{\pgfqpoint{1.979834in}{2.666596in}}%
\pgfpathlineto{\pgfqpoint{1.983464in}{2.654608in}}%
\pgfpathlineto{\pgfqpoint{1.987094in}{2.641930in}}%
\pgfpathlineto{\pgfqpoint{1.994354in}{2.614538in}}%
\pgfpathlineto{\pgfqpoint{2.001615in}{2.584491in}}%
\pgfpathlineto{\pgfqpoint{2.008875in}{2.551865in}}%
\pgfpathlineto{\pgfqpoint{2.016136in}{2.516741in}}%
\pgfpathlineto{\pgfqpoint{2.023396in}{2.479210in}}%
\pgfpathlineto{\pgfqpoint{2.030656in}{2.439366in}}%
\pgfpathlineto{\pgfqpoint{2.037917in}{2.397310in}}%
\pgfpathlineto{\pgfqpoint{2.045177in}{2.353148in}}%
\pgfpathlineto{\pgfqpoint{2.052438in}{2.306992in}}%
\pgfpathlineto{\pgfqpoint{2.059698in}{2.258959in}}%
\pgfpathlineto{\pgfqpoint{2.070589in}{2.183657in}}%
\pgfpathlineto{\pgfqpoint{2.081479in}{2.104835in}}%
\pgfpathlineto{\pgfqpoint{2.092370in}{2.022942in}}%
\pgfpathlineto{\pgfqpoint{2.103260in}{1.938444in}}%
\pgfpathlineto{\pgfqpoint{2.117781in}{1.822561in}}%
\pgfpathlineto{\pgfqpoint{2.135932in}{1.674188in}}%
\pgfpathlineto{\pgfqpoint{2.183125in}{1.286291in}}%
\pgfpathlineto{\pgfqpoint{2.197645in}{1.171028in}}%
\pgfpathlineto{\pgfqpoint{2.208536in}{1.087136in}}%
\pgfpathlineto{\pgfqpoint{2.219427in}{1.005968in}}%
\pgfpathlineto{\pgfqpoint{2.230317in}{0.927985in}}%
\pgfpathlineto{\pgfqpoint{2.241208in}{0.853631in}}%
\pgfpathlineto{\pgfqpoint{2.248468in}{0.806288in}}%
\pgfpathlineto{\pgfqpoint{2.255729in}{0.760866in}}%
\pgfpathlineto{\pgfqpoint{2.262989in}{0.717481in}}%
\pgfpathlineto{\pgfqpoint{2.270249in}{0.676242in}}%
\pgfpathlineto{\pgfqpoint{2.277510in}{0.637254in}}%
\pgfpathlineto{\pgfqpoint{2.284770in}{0.600615in}}%
\pgfpathlineto{\pgfqpoint{2.292031in}{0.566417in}}%
\pgfpathlineto{\pgfqpoint{2.299291in}{0.534749in}}%
\pgfpathlineto{\pgfqpoint{2.306551in}{0.505689in}}%
\pgfpathlineto{\pgfqpoint{2.313812in}{0.479312in}}%
\pgfpathlineto{\pgfqpoint{2.317442in}{0.467150in}}%
\pgfpathlineto{\pgfqpoint{2.321072in}{0.455683in}}%
\pgfpathlineto{\pgfqpoint{2.324702in}{0.444919in}}%
\pgfpathlineto{\pgfqpoint{2.328333in}{0.434864in}}%
\pgfpathlineto{\pgfqpoint{2.331963in}{0.425524in}}%
\pgfpathlineto{\pgfqpoint{2.335593in}{0.416906in}}%
\pgfpathlineto{\pgfqpoint{2.339223in}{0.409015in}}%
\pgfpathlineto{\pgfqpoint{2.342853in}{0.401855in}}%
\pgfpathlineto{\pgfqpoint{2.346484in}{0.395432in}}%
\pgfpathlineto{\pgfqpoint{2.350114in}{0.389749in}}%
\pgfpathlineto{\pgfqpoint{2.353744in}{0.384811in}}%
\pgfpathlineto{\pgfqpoint{2.357374in}{0.380619in}}%
\pgfpathlineto{\pgfqpoint{2.361004in}{0.377178in}}%
\pgfpathlineto{\pgfqpoint{2.364635in}{0.374488in}}%
\pgfpathlineto{\pgfqpoint{2.368265in}{0.372552in}}%
\pgfpathlineto{\pgfqpoint{2.371895in}{0.371371in}}%
\pgfpathlineto{\pgfqpoint{2.375525in}{0.370946in}}%
\pgfpathlineto{\pgfqpoint{2.379155in}{0.371277in}}%
\pgfpathlineto{\pgfqpoint{2.382786in}{0.372363in}}%
\pgfpathlineto{\pgfqpoint{2.386416in}{0.374205in}}%
\pgfpathlineto{\pgfqpoint{2.390046in}{0.376800in}}%
\pgfpathlineto{\pgfqpoint{2.393676in}{0.380148in}}%
\pgfpathlineto{\pgfqpoint{2.397306in}{0.384246in}}%
\pgfpathlineto{\pgfqpoint{2.400937in}{0.389091in}}%
\pgfpathlineto{\pgfqpoint{2.404567in}{0.394681in}}%
\pgfpathlineto{\pgfqpoint{2.408197in}{0.401012in}}%
\pgfpathlineto{\pgfqpoint{2.411827in}{0.408080in}}%
\pgfpathlineto{\pgfqpoint{2.415457in}{0.415880in}}%
\pgfpathlineto{\pgfqpoint{2.419088in}{0.424407in}}%
\pgfpathlineto{\pgfqpoint{2.422718in}{0.433657in}}%
\pgfpathlineto{\pgfqpoint{2.426348in}{0.443623in}}%
\pgfpathlineto{\pgfqpoint{2.429978in}{0.454299in}}%
\pgfpathlineto{\pgfqpoint{2.433608in}{0.465679in}}%
\pgfpathlineto{\pgfqpoint{2.437239in}{0.477754in}}%
\pgfpathlineto{\pgfqpoint{2.440869in}{0.490517in}}%
\pgfpathlineto{\pgfqpoint{2.448129in}{0.518077in}}%
\pgfpathlineto{\pgfqpoint{2.455390in}{0.548288in}}%
\pgfpathlineto{\pgfqpoint{2.462650in}{0.581072in}}%
\pgfpathlineto{\pgfqpoint{2.469910in}{0.616349in}}%
\pgfpathlineto{\pgfqpoint{2.477171in}{0.654028in}}%
\pgfpathlineto{\pgfqpoint{2.484431in}{0.694013in}}%
\pgfpathlineto{\pgfqpoint{2.491692in}{0.736204in}}%
\pgfpathlineto{\pgfqpoint{2.498952in}{0.780494in}}%
\pgfpathlineto{\pgfqpoint{2.506212in}{0.826771in}}%
\pgfpathlineto{\pgfqpoint{2.513473in}{0.874917in}}%
\pgfpathlineto{\pgfqpoint{2.524363in}{0.950374in}}%
\pgfpathlineto{\pgfqpoint{2.535254in}{1.029333in}}%
\pgfpathlineto{\pgfqpoint{2.546145in}{1.111344in}}%
\pgfpathlineto{\pgfqpoint{2.557035in}{1.195941in}}%
\pgfpathlineto{\pgfqpoint{2.571556in}{1.311923in}}%
\pgfpathlineto{\pgfqpoint{2.589707in}{1.460367in}}%
\pgfpathlineto{\pgfqpoint{2.633269in}{1.818891in}}%
\pgfpathlineto{\pgfqpoint{2.647790in}{1.934874in}}%
\pgfpathlineto{\pgfqpoint{2.658681in}{2.019470in}}%
\pgfpathlineto{\pgfqpoint{2.669571in}{2.101481in}}%
\pgfpathlineto{\pgfqpoint{2.680462in}{2.180440in}}%
\pgfpathlineto{\pgfqpoint{2.691352in}{2.255897in}}%
\pgfpathlineto{\pgfqpoint{2.698613in}{2.304043in}}%
\pgfpathlineto{\pgfqpoint{2.705873in}{2.350320in}}%
\pgfpathlineto{\pgfqpoint{2.713134in}{2.394610in}}%
\pgfpathlineto{\pgfqpoint{2.720394in}{2.436801in}}%
\pgfpathlineto{\pgfqpoint{2.727654in}{2.476787in}}%
\pgfpathlineto{\pgfqpoint{2.734915in}{2.514465in}}%
\pgfpathlineto{\pgfqpoint{2.742175in}{2.549742in}}%
\pgfpathlineto{\pgfqpoint{2.749436in}{2.582527in}}%
\pgfpathlineto{\pgfqpoint{2.756696in}{2.612737in}}%
\pgfpathlineto{\pgfqpoint{2.763956in}{2.640297in}}%
\pgfpathlineto{\pgfqpoint{2.767587in}{2.653061in}}%
\pgfpathlineto{\pgfqpoint{2.771217in}{2.665136in}}%
\pgfpathlineto{\pgfqpoint{2.774847in}{2.676515in}}%
\pgfpathlineto{\pgfqpoint{2.778477in}{2.687191in}}%
\pgfpathlineto{\pgfqpoint{2.782107in}{2.697157in}}%
\pgfpathlineto{\pgfqpoint{2.785738in}{2.706407in}}%
\pgfpathlineto{\pgfqpoint{2.789368in}{2.714935in}}%
\pgfpathlineto{\pgfqpoint{2.792998in}{2.722735in}}%
\pgfpathlineto{\pgfqpoint{2.796628in}{2.729803in}}%
\pgfpathlineto{\pgfqpoint{2.800258in}{2.736133in}}%
\pgfpathlineto{\pgfqpoint{2.803889in}{2.741723in}}%
\pgfpathlineto{\pgfqpoint{2.807519in}{2.746568in}}%
\pgfpathlineto{\pgfqpoint{2.811149in}{2.750666in}}%
\pgfpathlineto{\pgfqpoint{2.814779in}{2.754014in}}%
\pgfpathlineto{\pgfqpoint{2.818409in}{2.756609in}}%
\pgfpathlineto{\pgfqpoint{2.822040in}{2.758451in}}%
\pgfpathlineto{\pgfqpoint{2.825670in}{2.759537in}}%
\pgfpathlineto{\pgfqpoint{2.829300in}{2.759868in}}%
\pgfpathlineto{\pgfqpoint{2.832930in}{2.759443in}}%
\pgfpathlineto{\pgfqpoint{2.836560in}{2.758262in}}%
\pgfpathlineto{\pgfqpoint{2.840191in}{2.756326in}}%
\pgfpathlineto{\pgfqpoint{2.843821in}{2.753637in}}%
\pgfpathlineto{\pgfqpoint{2.847451in}{2.750195in}}%
\pgfpathlineto{\pgfqpoint{2.851081in}{2.746004in}}%
\pgfpathlineto{\pgfqpoint{2.854711in}{2.741065in}}%
\pgfpathlineto{\pgfqpoint{2.858342in}{2.735382in}}%
\pgfpathlineto{\pgfqpoint{2.861972in}{2.728959in}}%
\pgfpathlineto{\pgfqpoint{2.865602in}{2.721800in}}%
\pgfpathlineto{\pgfqpoint{2.869232in}{2.713908in}}%
\pgfpathlineto{\pgfqpoint{2.872862in}{2.705290in}}%
\pgfpathlineto{\pgfqpoint{2.876493in}{2.695950in}}%
\pgfpathlineto{\pgfqpoint{2.880123in}{2.685895in}}%
\pgfpathlineto{\pgfqpoint{2.883753in}{2.675131in}}%
\pgfpathlineto{\pgfqpoint{2.887383in}{2.663664in}}%
\pgfpathlineto{\pgfqpoint{2.891013in}{2.651503in}}%
\pgfpathlineto{\pgfqpoint{2.898274in}{2.625125in}}%
\pgfpathlineto{\pgfqpoint{2.905534in}{2.596065in}}%
\pgfpathlineto{\pgfqpoint{2.912795in}{2.564397in}}%
\pgfpathlineto{\pgfqpoint{2.920055in}{2.530200in}}%
\pgfpathlineto{\pgfqpoint{2.927315in}{2.493561in}}%
\pgfpathlineto{\pgfqpoint{2.934576in}{2.454572in}}%
\pgfpathlineto{\pgfqpoint{2.941836in}{2.413333in}}%
\pgfpathlineto{\pgfqpoint{2.949097in}{2.369948in}}%
\pgfpathlineto{\pgfqpoint{2.956357in}{2.324526in}}%
\pgfpathlineto{\pgfqpoint{2.963617in}{2.277183in}}%
\pgfpathlineto{\pgfqpoint{2.970878in}{2.228039in}}%
\pgfpathlineto{\pgfqpoint{2.981768in}{2.151217in}}%
\pgfpathlineto{\pgfqpoint{2.992659in}{2.071060in}}%
\pgfpathlineto{\pgfqpoint{3.003550in}{1.988024in}}%
\pgfpathlineto{\pgfqpoint{3.018070in}{1.873650in}}%
\pgfpathlineto{\pgfqpoint{3.032591in}{1.756157in}}%
\pgfpathlineto{\pgfqpoint{3.058002in}{1.546627in}}%
\pgfpathlineto{\pgfqpoint{3.083414in}{1.337677in}}%
\pgfpathlineto{\pgfqpoint{3.097935in}{1.221033in}}%
\pgfpathlineto{\pgfqpoint{3.108825in}{1.135772in}}%
\pgfpathlineto{\pgfqpoint{3.119716in}{1.052958in}}%
\pgfpathlineto{\pgfqpoint{3.130606in}{0.973062in}}%
\pgfpathlineto{\pgfqpoint{3.141497in}{0.896538in}}%
\pgfpathlineto{\pgfqpoint{3.148757in}{0.847612in}}%
\pgfpathlineto{\pgfqpoint{3.156018in}{0.800503in}}%
\pgfpathlineto{\pgfqpoint{3.163278in}{0.755329in}}%
\pgfpathlineto{\pgfqpoint{3.170539in}{0.712207in}}%
\pgfpathlineto{\pgfqpoint{3.177799in}{0.671243in}}%
\pgfpathlineto{\pgfqpoint{3.185059in}{0.632543in}}%
\pgfpathlineto{\pgfqpoint{3.192320in}{0.596205in}}%
\pgfpathlineto{\pgfqpoint{3.199580in}{0.562319in}}%
\pgfpathlineto{\pgfqpoint{3.206841in}{0.530972in}}%
\pgfpathlineto{\pgfqpoint{3.214101in}{0.502244in}}%
\pgfpathlineto{\pgfqpoint{3.221361in}{0.476207in}}%
\pgfpathlineto{\pgfqpoint{3.224992in}{0.464218in}}%
\pgfpathlineto{\pgfqpoint{3.228622in}{0.452926in}}%
\pgfpathlineto{\pgfqpoint{3.232252in}{0.442339in}}%
\pgfpathlineto{\pgfqpoint{3.235882in}{0.432462in}}%
\pgfpathlineto{\pgfqpoint{3.239512in}{0.423302in}}%
\pgfpathlineto{\pgfqpoint{3.243143in}{0.414865in}}%
\pgfpathlineto{\pgfqpoint{3.246773in}{0.407156in}}%
\pgfpathlineto{\pgfqpoint{3.250403in}{0.400180in}}%
\pgfpathlineto{\pgfqpoint{3.254033in}{0.393942in}}%
\pgfpathlineto{\pgfqpoint{3.257663in}{0.388445in}}%
\pgfpathlineto{\pgfqpoint{3.261294in}{0.383693in}}%
\pgfpathlineto{\pgfqpoint{3.264924in}{0.379688in}}%
\pgfpathlineto{\pgfqpoint{3.268554in}{0.376435in}}%
\pgfpathlineto{\pgfqpoint{3.272184in}{0.373933in}}%
\pgfpathlineto{\pgfqpoint{3.275814in}{0.372186in}}%
\pgfpathlineto{\pgfqpoint{3.279445in}{0.371194in}}%
\pgfpathlineto{\pgfqpoint{3.283075in}{0.370958in}}%
\pgfpathlineto{\pgfqpoint{3.286705in}{0.371478in}}%
\pgfpathlineto{\pgfqpoint{3.290335in}{0.372753in}}%
\pgfpathlineto{\pgfqpoint{3.293965in}{0.374783in}}%
\pgfpathlineto{\pgfqpoint{3.297596in}{0.377567in}}%
\pgfpathlineto{\pgfqpoint{3.301226in}{0.381102in}}%
\pgfpathlineto{\pgfqpoint{3.304856in}{0.385387in}}%
\pgfpathlineto{\pgfqpoint{3.308486in}{0.390419in}}%
\pgfpathlineto{\pgfqpoint{3.312116in}{0.396194in}}%
\pgfpathlineto{\pgfqpoint{3.315747in}{0.402710in}}%
\pgfpathlineto{\pgfqpoint{3.319377in}{0.409961in}}%
\pgfpathlineto{\pgfqpoint{3.323007in}{0.417944in}}%
\pgfpathlineto{\pgfqpoint{3.326637in}{0.426652in}}%
\pgfpathlineto{\pgfqpoint{3.330267in}{0.436082in}}%
\pgfpathlineto{\pgfqpoint{3.333898in}{0.446226in}}%
\pgfpathlineto{\pgfqpoint{3.337528in}{0.457078in}}%
\pgfpathlineto{\pgfqpoint{3.341158in}{0.468632in}}%
\pgfpathlineto{\pgfqpoint{3.344788in}{0.480881in}}%
\pgfpathlineto{\pgfqpoint{3.352049in}{0.507428in}}%
\pgfpathlineto{\pgfqpoint{3.359309in}{0.536653in}}%
\pgfpathlineto{\pgfqpoint{3.366569in}{0.568482in}}%
\pgfpathlineto{\pgfqpoint{3.373830in}{0.602834in}}%
\pgfpathlineto{\pgfqpoint{3.381090in}{0.639623in}}%
\pgfpathlineto{\pgfqpoint{3.388351in}{0.678755in}}%
\pgfpathlineto{\pgfqpoint{3.395611in}{0.720131in}}%
\pgfpathlineto{\pgfqpoint{3.402871in}{0.763647in}}%
\pgfpathlineto{\pgfqpoint{3.410132in}{0.809192in}}%
\pgfpathlineto{\pgfqpoint{3.417392in}{0.856651in}}%
\pgfpathlineto{\pgfqpoint{3.424653in}{0.905904in}}%
\pgfpathlineto{\pgfqpoint{3.435543in}{0.982874in}}%
\pgfpathlineto{\pgfqpoint{3.446434in}{1.063160in}}%
\pgfpathlineto{\pgfqpoint{3.457324in}{1.146306in}}%
\pgfpathlineto{\pgfqpoint{3.471845in}{1.260795in}}%
\pgfpathlineto{\pgfqpoint{3.486366in}{1.378367in}}%
\pgfpathlineto{\pgfqpoint{3.511777in}{1.587943in}}%
\pgfpathlineto{\pgfqpoint{3.537189in}{1.796823in}}%
\pgfpathlineto{\pgfqpoint{3.551709in}{1.913377in}}%
\pgfpathlineto{\pgfqpoint{3.562600in}{1.998545in}}%
\pgfpathlineto{\pgfqpoint{3.573491in}{2.081246in}}%
\pgfpathlineto{\pgfqpoint{3.584381in}{2.161011in}}%
\pgfpathlineto{\pgfqpoint{3.595272in}{2.237385in}}%
\pgfpathlineto{\pgfqpoint{3.602532in}{2.286201in}}%
\pgfpathlineto{\pgfqpoint{3.609793in}{2.333193in}}%
\pgfpathlineto{\pgfqpoint{3.617053in}{2.378241in}}%
\pgfpathlineto{\pgfqpoint{3.624313in}{2.421232in}}%
\pgfpathlineto{\pgfqpoint{3.631574in}{2.462057in}}%
\pgfpathlineto{\pgfqpoint{3.638834in}{2.500612in}}%
\pgfpathlineto{\pgfqpoint{3.646095in}{2.536800in}}%
\pgfpathlineto{\pgfqpoint{3.653355in}{2.570530in}}%
\pgfpathlineto{\pgfqpoint{3.660615in}{2.601715in}}%
\pgfpathlineto{\pgfqpoint{3.667876in}{2.630277in}}%
\pgfpathlineto{\pgfqpoint{3.675136in}{2.656144in}}%
\pgfpathlineto{\pgfqpoint{3.678766in}{2.668046in}}%
\pgfpathlineto{\pgfqpoint{3.682397in}{2.679250in}}%
\pgfpathlineto{\pgfqpoint{3.686027in}{2.689749in}}%
\pgfpathlineto{\pgfqpoint{3.689657in}{2.699537in}}%
\pgfpathlineto{\pgfqpoint{3.693287in}{2.708607in}}%
\pgfpathlineto{\pgfqpoint{3.696917in}{2.716953in}}%
\pgfpathlineto{\pgfqpoint{3.700548in}{2.724571in}}%
\pgfpathlineto{\pgfqpoint{3.704178in}{2.731454in}}%
\pgfpathlineto{\pgfqpoint{3.707808in}{2.737600in}}%
\pgfpathlineto{\pgfqpoint{3.711438in}{2.743004in}}%
\pgfpathlineto{\pgfqpoint{3.715068in}{2.747663in}}%
\pgfpathlineto{\pgfqpoint{3.718699in}{2.751574in}}%
\pgfpathlineto{\pgfqpoint{3.722329in}{2.754733in}}%
\pgfpathlineto{\pgfqpoint{3.725959in}{2.757141in}}%
\pgfpathlineto{\pgfqpoint{3.729589in}{2.758793in}}%
\pgfpathlineto{\pgfqpoint{3.733219in}{2.759691in}}%
\pgfpathlineto{\pgfqpoint{3.736850in}{2.759833in}}%
\pgfpathlineto{\pgfqpoint{3.740480in}{2.759219in}}%
\pgfpathlineto{\pgfqpoint{3.744110in}{2.757849in}}%
\pgfpathlineto{\pgfqpoint{3.747740in}{2.755724in}}%
\pgfpathlineto{\pgfqpoint{3.751370in}{2.752847in}}%
\pgfpathlineto{\pgfqpoint{3.755001in}{2.749217in}}%
\pgfpathlineto{\pgfqpoint{3.758631in}{2.744839in}}%
\pgfpathlineto{\pgfqpoint{3.762261in}{2.739714in}}%
\pgfpathlineto{\pgfqpoint{3.765891in}{2.733846in}}%
\pgfpathlineto{\pgfqpoint{3.769521in}{2.727238in}}%
\pgfpathlineto{\pgfqpoint{3.773152in}{2.719895in}}%
\pgfpathlineto{\pgfqpoint{3.776782in}{2.711822in}}%
\pgfpathlineto{\pgfqpoint{3.780412in}{2.703023in}}%
\pgfpathlineto{\pgfqpoint{3.784042in}{2.693503in}}%
\pgfpathlineto{\pgfqpoint{3.787672in}{2.683270in}}%
\pgfpathlineto{\pgfqpoint{3.791303in}{2.672330in}}%
\pgfpathlineto{\pgfqpoint{3.794933in}{2.660689in}}%
\pgfpathlineto{\pgfqpoint{3.798563in}{2.648354in}}%
\pgfpathlineto{\pgfqpoint{3.805823in}{2.621638in}}%
\pgfpathlineto{\pgfqpoint{3.813084in}{2.592248in}}%
\pgfpathlineto{\pgfqpoint{3.820344in}{2.560259in}}%
\pgfpathlineto{\pgfqpoint{3.827605in}{2.525752in}}%
\pgfpathlineto{\pgfqpoint{3.834865in}{2.488814in}}%
\pgfpathlineto{\pgfqpoint{3.842125in}{2.449538in}}%
\pgfpathlineto{\pgfqpoint{3.849386in}{2.408025in}}%
\pgfpathlineto{\pgfqpoint{3.856646in}{2.364379in}}%
\pgfpathlineto{\pgfqpoint{3.863906in}{2.318711in}}%
\pgfpathlineto{\pgfqpoint{3.871167in}{2.271136in}}%
\pgfpathlineto{\pgfqpoint{3.878427in}{2.221775in}}%
\pgfpathlineto{\pgfqpoint{3.889318in}{2.144658in}}%
\pgfpathlineto{\pgfqpoint{3.900208in}{2.064244in}}%
\pgfpathlineto{\pgfqpoint{3.911099in}{1.980989in}}%
\pgfpathlineto{\pgfqpoint{3.925620in}{1.866386in}}%
\pgfpathlineto{\pgfqpoint{3.940141in}{1.748737in}}%
\pgfpathlineto{\pgfqpoint{3.965552in}{1.539115in}}%
\pgfpathlineto{\pgfqpoint{3.990963in}{1.330307in}}%
\pgfpathlineto{\pgfqpoint{4.005484in}{1.213846in}}%
\pgfpathlineto{\pgfqpoint{4.016375in}{1.128771in}}%
\pgfpathlineto{\pgfqpoint{4.027265in}{1.046183in}}%
\pgfpathlineto{\pgfqpoint{4.038156in}{0.966550in}}%
\pgfpathlineto{\pgfqpoint{4.049047in}{0.890327in}}%
\pgfpathlineto{\pgfqpoint{4.056307in}{0.841622in}}%
\pgfpathlineto{\pgfqpoint{4.063567in}{0.794748in}}%
\pgfpathlineto{\pgfqpoint{4.070828in}{0.749825in}}%
\pgfpathlineto{\pgfqpoint{4.078088in}{0.706966in}}%
\pgfpathlineto{\pgfqpoint{4.085349in}{0.666280in}}%
\pgfpathlineto{\pgfqpoint{4.092609in}{0.627870in}}%
\pgfpathlineto{\pgfqpoint{4.099869in}{0.591833in}}%
\pgfpathlineto{\pgfqpoint{4.107130in}{0.558260in}}%
\pgfpathlineto{\pgfqpoint{4.114390in}{0.527236in}}%
\pgfpathlineto{\pgfqpoint{4.121651in}{0.498841in}}%
\pgfpathlineto{\pgfqpoint{4.128911in}{0.473145in}}%
\pgfpathlineto{\pgfqpoint{4.132541in}{0.461329in}}%
\pgfpathlineto{\pgfqpoint{4.136171in}{0.450213in}}%
\pgfpathlineto{\pgfqpoint{4.139802in}{0.439802in}}%
\pgfpathlineto{\pgfqpoint{4.143432in}{0.430104in}}%
\pgfpathlineto{\pgfqpoint{4.147062in}{0.421125in}}%
\pgfpathlineto{\pgfqpoint{4.150692in}{0.412869in}}%
\pgfpathlineto{\pgfqpoint{4.154322in}{0.405343in}}%
\pgfpathlineto{\pgfqpoint{4.157953in}{0.398551in}}%
\pgfpathlineto{\pgfqpoint{4.161583in}{0.392498in}}%
\pgfpathlineto{\pgfqpoint{4.165213in}{0.387187in}}%
\pgfpathlineto{\pgfqpoint{4.168843in}{0.382621in}}%
\pgfpathlineto{\pgfqpoint{4.172473in}{0.378805in}}%
\pgfpathlineto{\pgfqpoint{4.176104in}{0.375739in}}%
\pgfpathlineto{\pgfqpoint{4.179734in}{0.373426in}}%
\pgfpathlineto{\pgfqpoint{4.183364in}{0.371867in}}%
\pgfpathlineto{\pgfqpoint{4.186994in}{0.371064in}}%
\pgfpathlineto{\pgfqpoint{4.190624in}{0.371017in}}%
\pgfpathlineto{\pgfqpoint{4.194255in}{0.371726in}}%
\pgfpathlineto{\pgfqpoint{4.197885in}{0.373190in}}%
\pgfpathlineto{\pgfqpoint{4.201515in}{0.375408in}}%
\pgfpathlineto{\pgfqpoint{4.205145in}{0.378380in}}%
\pgfpathlineto{\pgfqpoint{4.208775in}{0.382103in}}%
\pgfpathlineto{\pgfqpoint{4.212406in}{0.386575in}}%
\pgfpathlineto{\pgfqpoint{4.216036in}{0.391793in}}%
\pgfpathlineto{\pgfqpoint{4.219666in}{0.397754in}}%
\pgfpathlineto{\pgfqpoint{4.223296in}{0.404454in}}%
\pgfpathlineto{\pgfqpoint{4.226926in}{0.411888in}}%
\pgfpathlineto{\pgfqpoint{4.230557in}{0.420053in}}%
\pgfpathlineto{\pgfqpoint{4.234187in}{0.428942in}}%
\pgfpathlineto{\pgfqpoint{4.237817in}{0.438551in}}%
\pgfpathlineto{\pgfqpoint{4.241447in}{0.448873in}}%
\pgfpathlineto{\pgfqpoint{4.245077in}{0.459901in}}%
\pgfpathlineto{\pgfqpoint{4.248708in}{0.471630in}}%
\pgfpathlineto{\pgfqpoint{4.252338in}{0.484050in}}%
\pgfpathlineto{\pgfqpoint{4.259598in}{0.510936in}}%
\pgfpathlineto{\pgfqpoint{4.266859in}{0.540490in}}%
\pgfpathlineto{\pgfqpoint{4.274119in}{0.572639in}}%
\pgfpathlineto{\pgfqpoint{4.281379in}{0.607301in}}%
\pgfpathlineto{\pgfqpoint{4.288640in}{0.644388in}}%
\pgfpathlineto{\pgfqpoint{4.295900in}{0.683806in}}%
\pgfpathlineto{\pgfqpoint{4.303160in}{0.725456in}}%
\pgfpathlineto{\pgfqpoint{4.310421in}{0.769231in}}%
\pgfpathlineto{\pgfqpoint{4.317681in}{0.815022in}}%
\pgfpathlineto{\pgfqpoint{4.324942in}{0.862712in}}%
\pgfpathlineto{\pgfqpoint{4.332202in}{0.912181in}}%
\pgfpathlineto{\pgfqpoint{4.343093in}{0.989444in}}%
\pgfpathlineto{\pgfqpoint{4.353983in}{1.069986in}}%
\pgfpathlineto{\pgfqpoint{4.364874in}{1.153349in}}%
\pgfpathlineto{\pgfqpoint{4.379395in}{1.268065in}}%
\pgfpathlineto{\pgfqpoint{4.393915in}{1.385790in}}%
\pgfpathlineto{\pgfqpoint{4.415697in}{1.565407in}}%
\pgfpathlineto{\pgfqpoint{4.415697in}{1.565407in}}%
\pgfusepath{stroke}%
\end{pgfscope}%
\begin{pgfscope}%
\pgfsetrectcap%
\pgfsetmiterjoin%
\pgfsetlinewidth{0.803000pt}%
\definecolor{currentstroke}{rgb}{0.000000,0.000000,0.000000}%
\pgfsetstrokecolor{currentstroke}%
\pgfsetdash{}{0pt}%
\pgfpathmoveto{\pgfqpoint{0.607800in}{0.251500in}}%
\pgfpathlineto{\pgfqpoint{0.607800in}{2.879314in}}%
\pgfusepath{stroke}%
\end{pgfscope}%
\begin{pgfscope}%
\pgfsetrectcap%
\pgfsetmiterjoin%
\pgfsetlinewidth{0.803000pt}%
\definecolor{currentstroke}{rgb}{0.000000,0.000000,0.000000}%
\pgfsetstrokecolor{currentstroke}%
\pgfsetdash{}{0pt}%
\pgfpathmoveto{\pgfqpoint{4.597025in}{0.251500in}}%
\pgfpathlineto{\pgfqpoint{4.597025in}{2.879314in}}%
\pgfusepath{stroke}%
\end{pgfscope}%
\begin{pgfscope}%
\pgfsetrectcap%
\pgfsetmiterjoin%
\pgfsetlinewidth{0.803000pt}%
\definecolor{currentstroke}{rgb}{0.000000,0.000000,0.000000}%
\pgfsetstrokecolor{currentstroke}%
\pgfsetdash{}{0pt}%
\pgfpathmoveto{\pgfqpoint{0.607800in}{0.251500in}}%
\pgfpathlineto{\pgfqpoint{4.597025in}{0.251500in}}%
\pgfusepath{stroke}%
\end{pgfscope}%
\begin{pgfscope}%
\pgfsetrectcap%
\pgfsetmiterjoin%
\pgfsetlinewidth{0.803000pt}%
\definecolor{currentstroke}{rgb}{0.000000,0.000000,0.000000}%
\pgfsetstrokecolor{currentstroke}%
\pgfsetdash{}{0pt}%
\pgfpathmoveto{\pgfqpoint{0.607800in}{2.879314in}}%
\pgfpathlineto{\pgfqpoint{4.597025in}{2.879314in}}%
\pgfusepath{stroke}%
\end{pgfscope}%
\begin{pgfscope}%
\pgfsetbuttcap%
\pgfsetmiterjoin%
\definecolor{currentfill}{rgb}{1.000000,1.000000,1.000000}%
\pgfsetfillcolor{currentfill}%
\pgfsetfillopacity{0.800000}%
\pgfsetlinewidth{1.003750pt}%
\definecolor{currentstroke}{rgb}{0.800000,0.800000,0.800000}%
\pgfsetstrokecolor{currentstroke}%
\pgfsetstrokeopacity{0.800000}%
\pgfsetdash{}{0pt}%
\pgfpathmoveto{\pgfqpoint{3.826058in}{2.156648in}}%
\pgfpathlineto{\pgfqpoint{4.541469in}{2.156648in}}%
\pgfpathlineto{\pgfqpoint{4.541469in}{2.823759in}}%
\pgfpathlineto{\pgfqpoint{3.826058in}{2.823759in}}%
\pgfpathlineto{\pgfqpoint{3.826058in}{2.156648in}}%
\pgfpathclose%
\pgfusepath{stroke,fill}%
\end{pgfscope}%
\begin{pgfscope}%
\pgfsetbuttcap%
\pgfsetroundjoin%
\pgfsetlinewidth{1.003750pt}%
\definecolor{currentstroke}{rgb}{0.000000,0.000000,0.000000}%
\pgfsetstrokecolor{currentstroke}%
\pgfsetdash{{1.000000pt}{0.000000pt}}{0.000000pt}%
\pgfpathmoveto{\pgfqpoint{3.870503in}{2.739759in}}%
\pgfpathlineto{\pgfqpoint{3.981614in}{2.739759in}}%
\pgfpathlineto{\pgfqpoint{4.092725in}{2.739759in}}%
\pgfusepath{stroke}%
\end{pgfscope}%
\begin{pgfscope}%
\definecolor{textcolor}{rgb}{0.000000,0.000000,0.000000}%
\pgfsetstrokecolor{textcolor}%
\pgfsetfillcolor{textcolor}%
\pgftext[x=4.181614in,y=2.700870in,left,base]{\color{textcolor}{\rmfamily\fontsize{8.000000}{9.600000}\selectfont\catcode`\^=\active\def^{\ifmmode\sp\else\^{}\fi}\catcode`\%=\active\def%{\%}\SI{1.0}{\hertz}}}%
\end{pgfscope}%
\begin{pgfscope}%
\pgfsetbuttcap%
\pgfsetroundjoin%
\pgfsetlinewidth{1.003750pt}%
\definecolor{currentstroke}{rgb}{0.250980,0.250980,0.250980}%
\pgfsetstrokecolor{currentstroke}%
\pgfsetdash{{2.000000pt}{1.000000pt}}{0.000000pt}%
\pgfpathmoveto{\pgfqpoint{3.870503in}{2.581314in}}%
\pgfpathlineto{\pgfqpoint{3.981614in}{2.581314in}}%
\pgfpathlineto{\pgfqpoint{4.092725in}{2.581314in}}%
\pgfusepath{stroke}%
\end{pgfscope}%
\begin{pgfscope}%
\definecolor{textcolor}{rgb}{0.000000,0.000000,0.000000}%
\pgfsetstrokecolor{textcolor}%
\pgfsetfillcolor{textcolor}%
\pgftext[x=4.181614in,y=2.542425in,left,base]{\color{textcolor}{\rmfamily\fontsize{8.000000}{9.600000}\selectfont\catcode`\^=\active\def^{\ifmmode\sp\else\^{}\fi}\catcode`\%=\active\def%{\%}\SI{2.0}{\hertz}}}%
\end{pgfscope}%
\begin{pgfscope}%
\pgfsetbuttcap%
\pgfsetroundjoin%
\pgfsetlinewidth{1.003750pt}%
\definecolor{currentstroke}{rgb}{0.501961,0.501961,0.501961}%
\pgfsetstrokecolor{currentstroke}%
\pgfsetdash{{3.000000pt}{2.000000pt}}{0.000000pt}%
\pgfpathmoveto{\pgfqpoint{3.870503in}{2.422870in}}%
\pgfpathlineto{\pgfqpoint{3.981614in}{2.422870in}}%
\pgfpathlineto{\pgfqpoint{4.092725in}{2.422870in}}%
\pgfusepath{stroke}%
\end{pgfscope}%
\begin{pgfscope}%
\definecolor{textcolor}{rgb}{0.000000,0.000000,0.000000}%
\pgfsetstrokecolor{textcolor}%
\pgfsetfillcolor{textcolor}%
\pgftext[x=4.181614in,y=2.383981in,left,base]{\color{textcolor}{\rmfamily\fontsize{8.000000}{9.600000}\selectfont\catcode`\^=\active\def^{\ifmmode\sp\else\^{}\fi}\catcode`\%=\active\def%{\%}\SI{3.0}{\hertz}}}%
\end{pgfscope}%
\begin{pgfscope}%
\pgfsetbuttcap%
\pgfsetroundjoin%
\pgfsetlinewidth{1.003750pt}%
\definecolor{currentstroke}{rgb}{0.752941,0.752941,0.752941}%
\pgfsetstrokecolor{currentstroke}%
\pgfsetdash{{4.000000pt}{3.000000pt}}{0.000000pt}%
\pgfpathmoveto{\pgfqpoint{3.870503in}{2.264425in}}%
\pgfpathlineto{\pgfqpoint{3.981614in}{2.264425in}}%
\pgfpathlineto{\pgfqpoint{4.092725in}{2.264425in}}%
\pgfusepath{stroke}%
\end{pgfscope}%
\begin{pgfscope}%
\definecolor{textcolor}{rgb}{0.000000,0.000000,0.000000}%
\pgfsetstrokecolor{textcolor}%
\pgfsetfillcolor{textcolor}%
\pgftext[x=4.181614in,y=2.225537in,left,base]{\color{textcolor}{\rmfamily\fontsize{8.000000}{9.600000}\selectfont\catcode`\^=\active\def^{\ifmmode\sp\else\^{}\fi}\catcode`\%=\active\def%{\%}\SI{4.0}{\hertz}}}%
\end{pgfscope}%
\end{pgfpicture}%
\makeatother%
\endgroup%

\caption{Example \thesection\ Plot}
\end{figure}

\section{Font Example}

Example \thesection\ shows the basics of how to use the font class.

\begin{pyblock}

print(font.roman, font.math, font.sans, font.mono)

\end{pyblock}

\begin{figure}[h!]
    \begin{center}
        \ttfamily\printpythontex
    \end{center}
\caption{Example \thesection\ Output}
\end{figure}

\section{Geometry Example}

In this example, we show how to interact with the geometry class.

\begin{pyblock}

print(geometry.em_length, geometry.ex_length, geometry.in_length, geometry.cm_length)

\end{pyblock}

\begin{figure}[h!]
    \begin{center}
        \ttfamily\printpythontex
    \end{center}
\caption{Example \thesection\ Output}
\end{figure}

\end{document}
